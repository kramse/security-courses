\documentclass[Screen16to9,17pt]{foils}
\usepackage{kea-slides}
\externaldocument{build/introduction-to-incident-response-exercises}
\selectlanguage{english}

% Input:
% https://www.threathunting.net/reading-list

% Docker image with
% https://hub.docker.com/r/threathuntproj/hunting/
% This image contains a complete threat hunting & data analysis environment built on Python, Pandas, PySpark and Jupyter notebook.

\begin{document}

\mytitlepage
{8. The Way Forward: Building an Intelligence Program}
{Introduction to Incident Response Elective, KEA}


\slide{Goals for today}

\hlkimage{6cm}{thomas-galler-hZ3uF1-z2Qc-unsplash.jpg}

\begin{list2}
\item Talk about the big picture
\item Strategic Intelligence
\item Summary of the book
\end{list2}

{\hfill \small Photo by Thomas Galler on Unsplash}

\slide{Plan for today}

\begin{list2}
\item Go through last chapters from the book
\item Strategic Intelligence
\item Building an Intelligence Program -- prepare you to implement incident response
%\item Exam subjects, questions, finishing up
\end{list2}

Exercise theme:
\begin{list2}
\item Revisit some exercises
\end{list2}


\slide{Time schedule}

\begin{list2}
\item 1) Chapter 10: Strategic Intelligence -- 45min
\item 2) Chapter 11: Building an Intelligence Program -- 45 min
\item Break 15min
\item 3) Summary and finishing up the IDIR book -- 45min
\item 4) Go through the NIST SP800-61r2 -- 45min
\end{list2}


\slide{Reading Summary}

\emph{Intelligence-Driven Incident Response} (IDIR)
 Scott Roberts. Rebekah Brown, ISBN: 9781098120689

\begin{quote}

\end{quote}

\begin{list2}
\item Chapter 10: Strategic Intelligence
\item Chapter 11: Building an Intelligence Program
\end{list2}


\slide{The Way Forward}

\emph{Intelligence-Driven Incident Response} (IDIR)
 Scott Roberts. Rebekah Brown, ISBN: 9781098120689

\begin{quote}
Intelligence-driven incident response doesn’t end when the final incident report has been delivered; it will become a part of your overall security process. Part 3 covers big-picture aspects of IDIR that are outside individual incident-response investigations. These features include strategic intelligence to continually learn and improve processes, as well as implementation of an intelligence team to support security operations as a whole.
\end{quote}

\begin{list2}
\item What do you know about the \emph{overall security process}?
\item How does this subject incident response fit in?
\end{list2}


\slide{Chapter 10: Strategic Intelligence}

%\hlkimage{}{}

\begin{quote}
Every once in while, an incident responder will start an investigation with a prickling
sensation in the back of his mind. Some call it a premonition, some call it deja vu, but
as the investigation unwinds, it will inevitably hit him: he has done this before. This.
Exact. Same. Investigation.
\end{quote}
Source: \emph{Intelligence-Driven Incident Response} (IDIR)


\begin{list2}
\item Putting out fires takes time, but sometimes you should let the current fire burn, and work on things to prevent and catch future fires
\end{list2}


\slide{What Is Strategic Intelligence?}

%\hlkimage{}{}

\begin{quote}
Strategic intelligence gets its name not only from the subjects that it covers, typically a {\bf high-level analysis of information with long-term implications}, but also from its audience. {\bf Strategic intelligence is geared toward decision makers} with the ability and authority to act, because this type of intelligence should shape policies and strategies moving forward. This doesn’t mean, however, that leadership is the only group that can benefit from these insights. Strategic intelligence is {\bf extremely useful to all levels of personnel} because it can help them understand the surrounding context of the issues that they deal with at their levels.
\end{quote}
Source: \emph{Intelligence-Driven Incident Response} (IDIR)

\begin{list2}
\item Understanding and working together makes a difference
\end{list2}


\slide{The State of Strategic Analysis}

%\hlkimage{}{}

\begin{quote}
In his paper, "The State of Strategic Analysis,” John Heidenrich wrote that “a strategy is not really a plan but the logic driving a plan.” When that logic is present and clearly communicated, analysts can approach problems in a way that supports the overarching goals behind a strategic effort rather than treating each individual situation as its own entity.
\end{quote}
Source:
\emph{The State of Strategic Analysis} John Heidenrich via
\emph{Intelligence-Driven Incident Response} (IDIR)

\begin{list2}
\item Many companies in Denmark does NOT have a clear strategic plan, mission or ideas of how to \emph{do security}
\item Most companies in Denmark consider security an after-thought, burden, cost, annoying
\item Various organisations have tried to do \emph{maturity models} for software and security
\end{list2}

\slide{CIS Controls: Incident Response}

\hlkimage{11cm}{cis-17-incident-response.png}
Source: \url{https://www.cisecurity.org/controls/incident-response-management}


\slide{Developing Target Models}

\hlkimage{12cm}{idir-hierarchial-model.png}

\begin{quote}
Hierarchical models are traditionally used to show personnel or roles, but one unique application of a hierarchical model is to use it to {\bf identify the data that is important to an organization}. A hierarchical model for data includes the broad categories of data, such as financial information, customer information, and sensitive company information.
\end{quote}
Source: \emph{Intelligence-Driven Incident Response} (IDIR)


\slide{Network Models}

\hlkimage{12cm}{idir-network-model.png}

\begin{list2}
\item Process models
\item Timelines -- various uses, tool re-use, spread of attack types, etc.
\end{list2}

\slide{Intelligence Cycle or Intelligence Process}

\hlkimage{9cm}{The_Intelligence_Process_JP_2-0.png}
Source: \link{https://en.wikipedia.org/wiki/Intelligence_cycle}

\begin{list2}
\item Chapter 10 continues applying the Intelligence Cycle/Process to the strategic level \\
-- which we consider high-level for now, we won't be allowed to this in most Danish companies
\end{list2}

\slide{Conclusion Strategic Intelligence}

%\hlkimage{}{}

\begin{quote}
{\Large\bf Conclusion}\\
We consider strategic intelligence to be the {\bf logic behind the plan}, and it is no wonder that many incident responders {\bf struggle with finding the time} to conduct this level of analysis. In many organizations, incident responders would be hard-pressed to find a plan at all, much less understand the logic behind the plan. {\bf Strategic intelligence}, when {\bf properly analyzed and adopted by leadership}, can not only inform leadership of the long-term threats to an organization, but can also provide incident responders with policies and procedures that will {\bf support their ability to meet the needs of their organization}.
\end{quote}
Source: \emph{Intelligence-Driven Incident Response} (IDIR)

\begin{list2}
\item May be hard to convince leadership, so take numbers, collect data, present data
\item ... or leave the organisation
\end{list2}


\slide{Chapter 11: Building an Intelligence Program}

%\hlkimage{}{}

\begin{quote}
Working with an {\bf intelligence team} can be a game changer for many security operations programs. However, there needs to be {\bf system in place} to get {\bf everyone} on the same page, both within the intelligence team and with the customers that the team will be supporting. A {\bf structured intelligence program} will provide the {\bf benefit of a robust intelligence support capability} while avoiding many of the struggles teams go through when they are thrown together rather than purposely built.
\end{quote}

\begin{list2}
\item Having team members also help when handling incidents over multiple days/weeks
\end{list2}

\slide{Are You Ready?}

%\hlkimage{}{}

\begin{quote}
One question that frequently gets asked is, {\bf “What are the prerequisites for forming an intelligence team?”} Many things need to be done before a formalized intelligence function will be beneficial. We are {\bf not} of the mindset that an {\bf intelligence program is the last thing} that should be created at an organization, but we do view the intelligence function as the {\bf glue that holds many other security functions together}. If you do not have those existing functions, you will just end up standing around, holding a bottle of glue.
\end{quote}

\begin{list2}
    \item Is the organisation mature enough
\end{list2}


\slide{Questions to ask}

%\hlkimage{}{}

\begin{quote}
At the far end of the spectrum of determining budget is the answer,
“We were just horribly hacked and now we have to show what we
are doing differently ASAP so that it never happens again. Go buy
things.

Here are some fundamental questions to ask before beginning to develop an intelli‐
gence program, which will require funding, time, and effort:

\begin{list2}
\item Is there a security function at the organization?
\item Is there network visibility?
\item Are there multiple teams or functions to support?
\item Is there room in the budget?
\end{list2}
\end{quote}
Source: \emph{Intelligence-Driven Incident Response} (IDIR)

\slide{Planning the Program}

%\hlkimage{}{}

\begin{quote}
Three types of planning go into the development of a solid program: conceptual planning, functional planning, and detailed planning:
\begin{list2}
\item 1. Conceptual planning sets the framework that the program should work within. Stakeholders contribute the most to conceptual planning, but it is important for them to understand what intelligence can offer them, especially if they are unfamiliar with intelligence work.
\item 2. Functional planning involves input from both stakeholders and intelligence professionals to identify requirements to complete goals, logistics such as budget and staffing needs, constraints, dependencies, and any legal concerns. Functional planning provides structure and realism to the sometimes abstract conceptual planning phase.
\item 3. Detailed planning is then conducted by the intelligence team, which will determine how the goals identified by the stakeholders will be met within the func‐
tional limits.
\end{list2}

All three phases of planning are important to ensure that all aspects have been considered, from budgeting to the metrics that will be reported to stakeholders.

\end{quote}
Source: \emph{Intelligence-Driven Incident Response} (IDIR)

\slide{Defining Stakeholders, Goals and Success Criteria}

%\hlkimage{}{}

\begin{quote}
Here are a few common stakeholders:
\begin{list2}
\item Intelligence response team
\item Security operations center/team
\item Vulnerability management teams
\item Chief information security officers
\item End users -- are most often an indirect stakeholder for intelligence
\end{list2}
...

After stakeholders have been defined, it is time to identify the goals of the program
with respect to each stakeholder.

...

Defining concrete goals gets the stakeholders and the intelligence team on the same
page by using the same definition of \emph{success}.
\end{quote}
Source: \emph{Intelligence-Driven Incident Response} (IDIR)

\slide{Identifying Requirements and Constraints}

\hlkimage{14cm}{idir-stakeholder-docs.png}

\begin{list2}
\item Probably this should be in a wiki or similar dynamic document
\item Multiple organisations maintain \emph{service documentation}
\end{list2}


\slide{Tactical Use Cases}

%\hlkimage{}{}

\begin{quote}
Tactical use cases involve intelligence that is useful on a day-to-day basis. This type of
intelligence will change rapidly but can also be some of the most directly applicable
intelligence in a security program.

\begin{list2}
\item SOC Support: Alerting and signature development, Triage, Situational awareness
\item Indicator Management: Threat-intelligence platform management, Updating indicators, Third-party intelligence and feeds management
\end{list2}
\end{quote}


\slide{Operational Use Cases}

%\hlkimage{}{}

\begin{quote}
Operational use cases for an intelligence program {\bf focus on understanding campaigns and trends in attacks}, either against your {\bf own organization} or against other organizations {\bf similar to yours}. The {\bf sooner} a campaign can be identified or a series of intrusions tied together, the {\bf more likely} it is that the activity can be {\bf identified before} the attackers are {\bf successful} in achieving their goals.
\end{quote}

\begin{list2}
\item Campaign Tracking
\item Identify the campaign focus
\item Identifying tools and tactics
\item Response support
\end{list2}

\slide{Strategic Use Cases}

%\hlkimage{}{}

\begin{quote}
{\bf Architecture Support}\\
Strategic intelligence can provide information not only on the ways an organization
should respond to intrusions or attacks, but also on the ways it can posture itself to
minimize attack surface and better detect these attacks.
\begin{list2}
\item Improve defensibility
\item Focus defenses on threats
\end{list2}

{\bf Risk Assessment/Strategic Situational Awareness}\\
\begin{list2}
\item Identify when risk changes
\item Identify mitigations
\end{list2}
\end{quote}


\slide{Crafting the InfoSec Playbook}
Maybe as a reference look into the book I suggested

\hlkimage{6cm}{book-crafting-infosec-playbook.jpg}

\emph{Crafting the InfoSec Playbook: Security Monitoring and Incident Response Master Plan}\\
 by Jeff Bollinger, Brandon Enright, and Matthew Valites ISBN: 9781491949405 - short CIP


\slide{Crafting the InfoSec Playbook}


This book will help you to answer common questions:
\begin{list2}
\item How do I find bad actors on my network?
\item How do I find persistent attackers?
\item How can I deal with the pervasive malware threat?
\item How do I detect system compromises?
\item How do I find an owner or responsible parties for systems under my protection?
\item How can I practically use and develop threat intelligence?
\item How can I possibly manage all my log data from all my systems?
\item How will I benefit from increased logging—and not drown in all the noise?
\item How can I use metadata for detection?
\end{list2}
Source: \emph{Crafting the InfoSec Playbook: Security Monitoring and Incident Response Master Plan}\\
 by Jeff Bollinger, Brandon Enright, and Matthew Valites ISBN: 9781491949405

\slide{Don't forget the templates!}

%\hlkimage{}{}

Book has some nice templates:
\begin{list2}
\item Short-Form Products
\item IOC Report
\item Event Summary Report
\item Target Package
\item Requests for Intelligence
\item Long-Form Products
\end{list2}

They are in Markdowwn format, so easily used.

\slide{Part 3: Summary and finishing up the IDIR book }

\hlkimage{3cm}{book-intelligence-driven-incident-response.jpg}

\emph{Intelligence-Driven Incident Response} \\
  Scott Roberts. Rebekah Brown, ISBN: 9781098120689 {\bf 2nd edition}- short IDIR


\begin{list2}
\item How did you like the book
\item Is it practical
\item Let's discuss a bit about learning, and preparing for something unknown
\end{list2}

\slide{Incident Response}

%\hlkimage{}{}

\begin{quote}
In the fields of computer security and information technology, {\bf computer security incident management} involves the monitoring and detection of security events on a computer or computer network, and the execution of proper responses to those events. Computer security incident management is a specialized form of incident management, the primary purpose of which is the development of a well understood and predictable response to damaging events and computer intrusions.[1]
\end{quote}
Source: \url{https://en.wikipedia.org/wiki/Computer_security_incident_management} \\
via "ISO 17799|ISO/IEC 17799:2005(E)". Information technology - Security techniques - Code of practice for\\ information security management. ISO copyright office. 2005-06-15. pp. 90–94.

\begin{list2}
\item ISO 17799 is superseeded by ISO 27001 and ISO 27002
\end{list2}



\slide{Exercises}

Virtual Machines allowed us play with tech

The following are recommended systems:
\begin{list2}
\item One VM based on Debian, running various software tools
%\item One VM based on Kali Linux, running hacking tools -- primary tool is Burp Suite and browser
\item Setup instructions and help \url{https://github.com/kramse/kramse-labs}
\end{list2}

Linux is a toolbox we will use and participants will use virtual machines, we also used Windows a few times. Did you notice that a lot of tools for \emph{processing} windows data are running on Linux.

\slide{Goals and plans}

%\hlkimage{}{}

\begin{quote}
  “A goal without a plan is just a wish.”\\
  ― Antoine de Saint-Exupéry
\end{quote}

I want this course to
\begin{list2}
\item Include everything listed in contents above
\item Be practical -- you can do something useful
\item Kickstart your journey into Incident Response\\
Getting a practical book with pointers about the subject
\item Present a lot of useful sources and tools
\item Prepare you for production use of the knowledge
\end{list2}


\slide{Sources: Network overview}

\hlkimage{15cm}{sample-ip-network.pdf}

\begin{list2}
\item Internet, routers, firewalls, switches, clients and servers (Wi-Fi not shown)
\item Without data we cannot perform Incident Response
\end{list2}


\slide{Sources: Strategy for implementing identification and detection}

We recommend that the following strategy is used for implementing identification and detection -- logging:
\begin{enumerate}
\item[\faSquareO] Enable system logging from servers
\item[\faSquareO] Enable system logging from network devices
\item[\faSquareO] Enable logging from client devices
\item[\faSquareO] Centralize logging
\item[\faSquareO] Add search facilities and dashboards
\item[\faSquareO] Perform system audits manually or automatically
\item[\faSquareO] Setup alerting and notification with procedures
\end{enumerate}


\slide{Intrusion Kill Chains}

\hlkimage{13cm}{crafting-cip-kill-chain.png}

\begin{list2}
\item See also \emph{Intelligence-Driven Computer Network Defense Informed by Analysis of Adversary Campaigns and Intrusion Kill Chains}, Eric M. Hutchins , Michael J. Cloppert, Rohan M. Amin, Ph.D. Lockheed Martin Corporation\\{\footnotesize
 \link{https://www.lockheedmartin.com/content/dam/lockheed-martin/rms/documents/cyber/LM-White-Paper-Intel-Driven-Defense.pdf}}
\end{list2}



\slide{Detection Capabilities}


Security incidents happen, but what happens. One of the actions to reduce impact of incidents are done in preparing for incidents.

\begin{itemize}
\item \emph{Preparation} for an attack, establish procedures and mechanisms for detecting and responding to attacks
\end{itemize}

Preparation will enable easy {\bf identification} of affected systems, better {\bf containment} which systems are likely to be infected, {\bf eradication} what happened -- how to do the {\bf eradication} and {\bf recovery}.

\slide{Data Analysis Skills}

\begin{quote}
Although we could spend an entire book creating an exhaustive list of skills needed to be a good security data scientist, this chapter covers the following skills/domains that a data scientist will benefit from
knowing within information security:
\begin{list2}
\item Domain expertise—Setting and maintaining a purpose to the analysis
\item Data management—Being able to prepare, store, and maintain data
\item Programming—The glue that connects data to analysis
\item Statistics—To learn from the data
\item Visualization—Communicating the results effectively
\end{list2}
It might be easy to label any one of these skills as the most important, but in reality, the whole is greater than the sum of its parts. Each of these contributes a significant and important piece to the workings of
security data science.
\end{quote}

Source: \emph{Data-Driven Security: Analysis, Visualization and Dashboards} Jay Jacobs, Bob Rudis\\
ISBN: 978-1-118-79372-5 February 2014 \url{https://datadrivensecurity.info/} - short DDS


\slide{Part 4: Book: NIST SP800-61rev2}

\hlkimage{12cm}{NIST-SP800-61r2.png}

\link{https://doi.org/10.6028/NIST.SP.800-61r2}

Let's dig a bit deeper into this resource

\slidenext{}



\end{document}
