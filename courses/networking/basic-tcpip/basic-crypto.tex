\slide{Kryptografi}

\hlkimage{18cm}{images/crypto-rot13.pdf}

\begin{list1}
\item Kryptografi er læren om, hvordan man kan kryptere data
\item Kryptografi benytter algoritmer som sammen med nøgler giver en
  ciffertekst - der kun kan læses ved hjælp af den tilhørende nøgle
\end{list1}

\slide{Public key kryptografi - 1}

\hlkimage{18cm}{images/crypto-public-key.pdf}

\begin{list1}
\item privat-nøgle kryptografi (eksempelvis AES) benyttes den samme
  nøgle til kryptering og dekryptering 
\item offentlig-nøgle kryptografi (eksempelvis RSA) benytter to
  separate nøgler til kryptering og dekryptering
\end{list1}

\slide{Public key kryptografi - 2}

\hlkimage{18cm}{images/crypto-public-key-2.pdf}

\begin{list1}

\item offentlig-nøgle kryptografi (eksempelvis RSA) bruger den private
  nøgle til at dekryptere
\item man kan ligeledes bruge offentlig-nøgle kryptografi til at
  signere dokumenter - som så verificeres med den offentlige nøgle
\end{list1}


\slide{Kryptografiske principper}

\begin{list1}
\item Algoritmerne er kendte
\item Nøglerne er hemmelige
\item Nøgler har en vis levetid - de skal skiftes ofte
\item Et successfuldt angreb på en krypto-algoritme er enhver genvej
  som kræver mindre arbejde end en gennemgang af alle nøglerne 
\item Nye algoritmer, programmer, protokoller m.v. skal gennemgås nøje!
\item Se evt. Snake Oil Warning Signs:
Encryption Software to Avoid 
\link{http://www.interhack.net/people/cmcurtin/snake-oil-faq.html}
\end{list1}

\slide{DES, Triple DES og AES}

\hlkimage{15cm}{images/AES_head.png}

\begin{list1}
\item DES kryptering baseret på den IBM udviklede Lucifer algoritme
  har været benyttet gennem mange år. 
\item Der er vedtaget en ny standard algoritme Advanced Encryption
  Standard (AES) som afløser Data Encryption Standard (DES)
\item Algoritmen hedder Rijndael og er udviklet
af Joan Daemen og Vincent Rijmen.
%\item \emph{Rijndael is available for free. You can use it for
%whatever purposes  you want, irrespective of whether
%it is accepted as AES or not.}

\item Kilde:
\link{http://csrc.nist.gov/encryption/aes/}\\
\href{http://www.esat.kuleuven.ac.be/~rijmen/rijndael/}
{http://www.esat.kuleuven.ac.be/\~{}rijmen/rijndael/}
\end{list1}


\slide{Formålet med kryptering}

\vskip 3 cm
\centerline{\hlkbig kryptering er den eneste måde at sikre:}
\vskip 3 cm
\centerline{\hlkbig fortrolighed}
\vskip 3 cm
\centerline{\hlkbig autenticitet / integritet}


\slide{e-mail og forbindelser}

\begin{list1}
\item Kryptering af e-mail
\begin{list2}
\item Pretty Good Privacy - Phil Zimmermann
\item PGP = mail sikkerhed
\end{list2}
\item Kryptering af sessioner SSL/TLS
\begin{list2}
\item Secure Sockets Layer SSL / Transport Layer Services TLS
\item krypterer data der sendes mellem webservere og klienter
\item SSL kan bruges generelt til mange typer sessioner, eksempelvis
  POP3S, IMAPS, SSH m.fl.
\end{list2}
\vskip 1 cm 
\item Sender I kreditkortnummeret til en webserver der kører uden https?
\end{list1}

\slide{MD5 message digest funktion}

\hlkimage{16cm}{images/message-digest-1.pdf}

\begin{list1}
\item HASH algoritmer giver en unik værdi baseret på input
%\item output fra algoritmerne kaldes også message digest
%\item MD5 er et eksempel på en meget brugt algoritme
%\item MD5 algoritmen har følgende egenskaber:
%  \begin{list2}
%  \item output er 128-bit "fingerprint" uanset længden af input
\item værdien ændres radikalt selv ved små ændringer i input
%  \end{list2}
\item MD5 er blandt andet beskrevet i RFC-1321: The MD5 Message-Digest
  Algorithm 
%\item Algoritmen MD5 er baseret på MD4, begge udviklet af Ronald
%  L. Rivest kendt fra blandt andet RSA Data Security, Inc
\item Både MD5 og SHA-1 undersøges nøje og der er fundet kollisioner
  som kan påvirke vores brug i fremtiden - \emph{stay tuned}
\end{list1} 

%%% Local Variables: 
%%% mode: latex
%%% TeX-master: "tcpip-security"
%%% End: 
