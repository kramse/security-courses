\documentclass[Screen16to9,17pt]{foils}
\usepackage{kea-slides}

\externaldocument{communication-and-network-security-exercises}
\selectlanguage{english}

\begin{document}

\mytitlepage
{0. Introduction}
{Communication and Network Security \the\year}

\hlkprofiluk

\slide{Goals for today}

\hlkimage{6cm}{thomas-galler-hZ3uF1-z2Qc-unsplash.jpg}

\begin{list2}
\item Welcome, course goals and expectations
\item Prepare Virtual Machines - hope you brought a laptop
\item Create a good starting point for learning
\item Concrete Expectations
\item Prepare tools for the exercises
\end{list2}

Photo by Thomas Galler on Unsplash

\slide{Plan for today}

\begin{list2}
\item Introduce lecturers and students
\item Create a good starting point for learning
\item Expectations for this course
\item Literature list walkthrough
\item Prepare tools for the exercises
\item Kali and Debian Linux introduction
\end{list2}

Exercises
\begin{list2}
\item Kali Linux installation
\item Debian Linux installation
\end{list2}
Linux is a toolbox we will use and participants will use virtual machines


\slide{Time schedule}

\begin{list2}
\item 17:00 - 18:15\\
Introduction and basics for the subject

\item 30min break\\
Eat with your family if you like, I will be around most of the break, available for questions

\item 18:45 - 19:30\\
Further teaching and exercises in the subject for the evening

\item 15min break\\
Stretch your legs, get some more water

\item 19:45 -20:30 45min\\
May contain exercises to be done on your own, with input from me
\end{list2}

\centerline{I will try to keep this plan for all evenings! So you hopefully can plan family life better}


\slide{Course Materials}

\begin{list1}
\item This material is in multiple parts:

\item Slide shows - presentation - this file
\item Exercises - PDF which is updated along the way

\item Additional resources from the internet

\end{list1}

{\Large Note: the presentation slides are not a substitute for reading the books, papers and doing exercises, many details are not shown}

\slide{Fronter Platform}

\hlkimage{9cm}{fronter.png}

We will use fronter a lot, both for sharing educational materials and news during the course.

You will also be asked to turn in deliverables through fronter

\link{https://kea-fronter.itslearning.com/}

\vskip 5mm
\centerline{If you haven't received login yet, let us know}

\slide{Overview Diploma in IT-security}

\hlkimage{17cm}{kea-diplom-oversigt.png}


\slide{Course Data}

\hlkimage{6cm}{pawel-janiak-dxFi8Ea670E-unsplash.jpg}

{\bf Course: Communication and Network Security (10 ECTS)}

Teaching dates: tuesdays and thursdays 17:00 - 20:30\\
31/1 2023, 2/2 2023, 7/2 2023, 9/2 2023, 14/2 2023, 16/2 2023, 21/2 2023, 23/2 2023, 28/2 2023, 2/3 2023, 7/3 2023, \sout{9/3 2023}, 14/3 2023, 16/3 2023

Exam: 30/3 2023 \hskip 12cm Photo by Pawel Janiak on Unsplash

{\small{\bf Den 9/3 er der konference DKNOG! Vi skal således flytte denne dato!}
Vi kan flytte til en weekend og samlet lave øvelser i forbindelse med. Jeg har også i planen afsat plads til to dage med øvelser - NWWC dage.}

\slide{Deliverables and Exam}


\begin{list2}
\item Exam
\item Individual: Oral based on curriculum
\item Graded (7 scale)
\item Draw a question with no preparation. Question covers a topic
\item Try to discuss the topic, and use practical examples
\item Exam is 30 minutes in total, including pulling the question and grading
\item Count on being able to present talk for about 10 minutes
\item Prepare material (keywords, examples, exercises, wireshark captures) for different topics so that you can use it to help you at the exam

\vskip 5mm
\item Deliverables:
\item 2 Mandatory assignments
\item Both mandatory assignments are required in order to be entitled to the exam.
\end{list2}


\slide{Course Description}

{\bf OB1 Netværks- og kommunikationssikkerhed (10 ECTS)}

Indhold:\\
Elementet går ud på at forstå og håndtere netværkssikkerhedstrusler samt implementere og
konfigurere udstyr til samme.

Elementet omhandler forskellig sikkerhedsudstyr (IDS) til monitorering. Derudover vurdering
af sikkerheden i et netværk, udarbejdelse af plan til at lukke eventuelle sårbarheder i
netværket samt gennemgang af forskellige VPN teknologier.

My translation:\\
The module is centered around network threats and implementing and configuring equipment in this area.

Module includes different security equipment like IDS for monitoring.
The evaluation of security in a network, developing plans for closing security vulnerabilities in the network and a review of various VPN technologies.

From: STUDIEORDNING Diplomuddannelse i it-sikkerhed\\{\footnotesize
\url{https://kompetence.kea.dk/studieordninger/Studieordning_Diplom_IT-sikkerhed_2022_03.pdf}}


\slide{Goals and plans}

%\hlkimage{}{}

\begin{quote}
  “A goal without a plan is just a wish.”\\
  ― Antoine de Saint-Exupéry
\end{quote}

My overall goal for this course:
\begin{list2}
\item Include everything required by studieordningen
\item Introduce networking and related security issues
\item Introduce resources, programs, people, authors, documents, sites\\
 that further your exploration into network security
\item Kickstart your journey into the subjects\\
Getting the best books with pointers
\item Present a lot of useful sources, data types, tools
\item Be practical -- you can do something useful
\end{list2}


\slide{Expectations alignment}

\hlkimage{7cm}{Shaking-hands_web.jpg}

%Form groups of 2-3 students

In groups of 2 students, brainstorm for 10 minutes on what topics you would like to have in this course

Use 5 minutes more on Agreeing on 5 topics and prioritize these 5 topics

I look forward to hearing your wishes, and hopefully we can accomodate some

\vskip 1cm
PS We will from time to time have exercises, groups dont need to be the same each time.



\slide{Exercises}

Exercise theme: Virtual Machines allows us play with tech

Hardware

Since we are going to be doing exercises, each team will need virtual machines.

The following are recommended systems:
\begin{list2}
\item One VM based on Debian, running software servers and web applications
\item Setup instructions and help \url{https://github.com/kramse/kramse-labs}
\end{list2}

Linux is a toolbox we will use and participants will use virtual machines


\slide{Prerequisites}

\begin{list1}
\item This course includes exercises and getting the most of the course requires the participants to carry out these practical exercises
\item We will use Kali Linux for the exercises but previous Linux and Unix knowledge is not needed
\item It is recommended to use virtual machines for the exercises
\item Network security and most internet related security work has the following requirements:
\begin{list2}
\item Network experience
\item TCP/IP principles - often in more detail than a common user
\item Programming is an advantage, for automating things
\item Some Linux and Unix knowledge is in my opinion a {\bf necessary skill} for infosec work\\
-- too many new tools to ignore, and lots found at sites like Github and Open Source written for Linux
\end{list2}
\end{list1}

\slide{Course Network}
.
\hlkrightpic{85mm}{-1cm}{sample-network.png}

\begin{list1}
\item I have a course network with me when needed, \\
which has the following systems:
\begin{list2}
\item OpenBSD router
\item Switches Juniper EX2200-C and small TP-Link
\item UniFi AP wireless access-point
\end{list2}
\end{list1}

This will be at my home, and due to remote teaching - we will investigate your networks and scan across the internet to \emph{my servers}!

\slide{Wifi Hardware}

Since we are going to be doing exercises, sniffing data it \\
will be an advantage to have a wireless USB network card.
\begin{list2}
\item The following are two recommended models:
\item TP-link TL-WN722N hardware version 2.0 cheap but only support 2.4GHz
\item Alfa AWUS036ACH 2.4GHz + 5GHz Dual-Band and high performing
\item Both work great in Kali Linux for our purposes.\\
Unfortunately the vendors change models often enough that the above are hard to find. I recommend using your favourite search engine and research which cards work with Kali Linux and airodump-ng.
\end{list2}

I have some available for teams if you dont buy them.


\slide{Primary literature}

Primary literature are these three books:
\begin{list2}
\item \emph{Applied Network Security Monitoring Collection, Detection, and Analysis}, 2014 Chris Sanders \\
ISBN: 9780124172081 - shortened ANSM
\item \emph{Practical Packet Analysis - Using Wireshark to Solve Real-World Network Problems}, 3rd edition 2017, \\
Chris Sanders ISBN: 9781593278021 - shortened PPA
\item \emph{Linux Basics for Hackers Getting Started with Networking, Scripting, and Security in Kali}. OccupyTheWeb, December 2018, 248 pp. ISBN-13: 978-1-59327-855-7 - shortened LBfH
\end{list2}

Price check around January 2019 - all three can be bought in hardcopy for 1.000-1.100DKK


\centerline{Problem: You probably dont have the books yet ...}

\slide{Course overview}

We will now go through a little from the Table of Contents in the books.

and the \emph{Reading plan}\\
\link{https://github.com/kramse/kea-it-sikkerhed/tree/master/net-og-komm-sikkerhed}




\slide{Book: Applied Network Security Monitoring (ANSM)}

\hlkimage{5cm}{ansm-book.png}

\emph{Applied Network Security Monitoring: Collection, Detection, and Analysis}
1st Edition

Chris Sanders, Jason Smith
eBook ISBN: 9780124172166
Paperback ISBN: 9780124172081 496 pp.
Imprint: Syngress, December 2013

{\footnotesize\link{https://www.elsevier.com/books/applied-network-security-monitoring/unknown/978-0-12-417208-1}}

\slide{Book: Practical Packet Analysis (PPA)}
\hlkimage{6cm}{PracticalPacketAnalysis3E_cover.png}

\emph{Practical Packet Analysis,
Using Wireshark to Solve Real-World Network Problems}
by Chris Sanders, 3rd Edition
April 2017, 368 pp.
ISBN-13:
978-1-59327-802-1

\link{https://nostarch.com/packetanalysis3}



\slide{Supporting literature books}
\begin{list2}
\item \emph{Linux Basics for Hackers Getting Started with Networking, Scripting, and Security in Kali}\\
OccupyTheWeb, December 2018, 248 pp. ISBN-13: 978-1-59327-855-7 - shortened LBfH
\item \emph{The Debian Administrator’s Handbook}, Raphaël Hertzog and Roland Mas\\
\url{https://debian-handbook.info/} - shortened DEB
\item \emph{Kali Linux Revealed  Mastering the Penetration Testing Distribution}\\
Raphaël Hertzog, Jim O'Gorman - shortened KLR
\end{list2}



\slide{Book: Linux Basics for Hackers (LBfH)}

\hlkimage{6cm}{LinuxBasicsforHackers_cover-front.png}

\emph{Linux Basics for Hackers
Getting Started with Networking, Scripting, and Security in Kali}
by OccupyTheWeb
December 2018, 248 pp.
ISBN-13:
9781593278557

\link{https://nostarch.com/linuxbasicsforhackers}
Not curriculum but explains how to use Linux

\slide{Book: The Debian Administrator’s Handbook (DEB)}

\hlkimage{6cm}{book-debian-administrators-handbook.jpg}

\emph{The Debian Administrator’s Handbook}, Raphaël Hertzog and Roland Mas\\
\url{https://debian-handbook.info/} - shortened DEB

Not curriculum but explains how to use Debian Linux

\slide{Book: Kali Linux Revealed (KLR)}

\hlkimage{6cm}{kali-linux-revealed.jpg}

\emph{Kali Linux Revealed  Mastering the Penetration Testing Distribution}

\link{https://www.kali.org/download-kali-linux-revealed-book/}\\
Not curriculum but explains how to install Kali Linux



\exercise{ex:downloadKLR}



%%% Break?

\slide{Hackerlab Setup}

\hlkimage{7cm}{hacklab-1.png}

\begin{list2}
\item Hardware: modern laptop CPU with virtualisation\\
Dont forget to enable hardware virtualisation in the BIOS
\item Software Host OS: Windows, Mac, Linux
\item Virtualisation software: VMware, Virtual box, HyperV pick your poison
\item {\bf Hackersoftware: Kali Virtual Machine \link{https://www.kali.org/}}
\item Soft targets: Metasploitable, Windows 2000, Windows XP, ...
\end{list2}

\centerline{Having a Debian VM will also be recommended, one pr team}



\slide{Mixed exercises}
Then we will do a mixed bag of exercises to introduce technologies, find your current knowledge level with regards to:

\begin{list2}
\item Linux
\item Linux command line
\item Git, Python and Ansible
\item Elasticsearch -- how to run a \emph{service}
\item Running Java on Linux -- environment variables?!
\item Ansible provisioning -- installing and configuring software for production
\end{list2}

{\bf Note: today we will consider all these optional, we wont be able to do them all}

Later we will return to them!

\slide{Command prompts in Unix}

\begin{list1}
\item Shells :
  \begin{list2}
    \item sh - Bourne Shell
\item bash - Bourne Again Shell, often the default in Linux
\item ksh - Korn shell, original by David Korn, but often the public domain version used
\item csh - C shell, syntax similar to C language
\item Multiple others available, zsh is very popular
  \end{list2}
\item Windows have \verb+command.com+, \verb+cmd.exe+ but PowerShell is more similar to the Unix shells
\item Used for scripting, automation and programs
\end{list1}



\slide{Command prompts}


\begin{alltt}
\small
[hlk@fischer hlk]$ id
uid=6000(hlk) gid=20(staff) groups=20(staff),
0(wheel), 80(admin), 160(cvs)
[hlk@fischer hlk]$ sudo -s
[root@fischer hlk]#
[root@fischer hlk]# id {\bf
uid=0(root) gid=0(wheel)} groups=0(wheel), 1(daemon),
20(staff), 80(admin)
[root@fischer hlk]#
\end{alltt}

Note the difference between running as root and normal user. Usually books and instructions will use a prompt of hash mark \verb+#+ when the root user is assumed and dollar sign \verb+$+ when a normal user prompt.

\slide{Command syntax}


\begin{alltt}
echo [-n] [string ...]
\end{alltt}

\begin{list1}
\item Commands are written like this:
\begin{list2}
\item Always begin with the command to execute, like \verb+echo+ above
\item Options typically short form with single dash \verb+-n+
\item or long options \verb+--version+
\item Some commands allow grouing options, \verb+tar -c -v -f+ becomes \verb+tar -cvf+\\
NOTE: some options require parameters, so \verb+tar -c -f filename.tar+ not equal to \verb+tar -fc filename.tar+
\item Optional options are in brackets \verb+[ ]+
\item Output can be saved using redirection, into new file/overwrite \verb+echo hello > file.txt+ or append \verb+echo hello >> file.txt+
\item Read from files \verb+wc -l file.txt+ or pipe output into input \verb+cat file.txt | wc -l+\\
\verb+wc+ is word count, and option l is count lines
\end{list2}
\end{list1}



\slide{Unix Manual system}

\hlkimage{7cm}{images/Unix-command-1.pdf}

\begin{quote}
 It is a book about a Spanish guy called Manual. You should read it.
       -- Dilbert
\end{quote}

\begin{list1}
\item Manual system in Unix is always there!
\item Key word search \verb+man -k+ see also \verb+apropos+
\item Different sections, can be chosen
\end{list1}

See \verb+man crontab+ the command vs the file format in section 5 \verb+man 5 crontab+



\slide{A manual page}

\begin{alltt}\footnotesize
\small
NAME
     cal - displays a calendar
SYNOPSIS
     cal [-jy] [[month]  year]
DESCRIPTION
   cal displays a simple calendar.  If arguments are not specified, the cur-
   rent month is displayed.  The options are as follows:
   -j      Display julian dates (days one-based, numbered from January 1).
   -y      Display a calendar for the current year.

The Gregorian Reformation is assumed to have occurred in 1752 on the 3rd
of September.  By this time, most countries had recognized the reforma-
tion (although a few did not recognize it until the early 1900's.)  Ten
days following that date were eliminated by the reformation, so the cal-
endar for that month is a bit unusual.
\end{alltt}

\slide{The year 1752}

\begin{alltt}\footnotesize
  user@Projects:$ cal 1752
...
         April                  May                   June
  Su Mo Tu We Th Fr Sa  Su Mo Tu We Th Fr Sa  Su Mo Tu We Th Fr Sa
            1  2  3  4                  1  2      1  2  3  4  5  6
   5  6  7  8  9 10 11   3  4  5  6  7  8  9   7  8  9 10 11 12 13
  12 13 14 15 16 17 18  10 11 12 13 14 15 16  14 15 16 17 18 19 20
  19 20 21 22 23 24 25  17 18 19 20 21 22 23  21 22 23 24 25 26 27
  26 27 28 29 30        24 25 26 27 28 29 30  28 29 30
                        31
          July                 August              September
  Su Mo Tu We Th Fr Sa  Su Mo Tu We Th Fr Sa  Su Mo Tu We Th Fr Sa
            1  2  3  4                     1  {\bf        1  2 14 15 16}
   5  6  7  8  9 10 11   2  3  4  5  6  7  8  17 18 19 20 21 22 23
  12 13 14 15 16 17 18   9 10 11 12 13 14 15  24 25 26 27 28 29 30
  19 20 21 22 23 24 25  16 17 18 19 20 21 22
  26 27 28 29 30 31     23 24 25 26 27 28 29
                        30 31
...
\end{alltt}


\slide{Linux configuration in /etc}

.
\hlkrightpic{8cm}{0cm}{Unix-vfs.pdf}
\begin{list2}
\item Command line is a requirement in the \emph{studieordningen} \smiley
\item Linux and Unix uses a single virtual file system\\
\url{https://en.wikipedia.org/wiki/Unix_filesystem}
\item No drive letters like the ones in MS-DOS and Microsoft Windows
\item Everything starts at the root of the file system tree \verb+/+ - NOTE: \emph{forward slash}
\item One special directory is \verb+/etc/+ and sub directories which usually contain a lot of configuration files
\end{list2}

\slide{Installing software in Debian -- apt}

%\hlkimage{}{}

\begin{alltt}\footnotesize
DESCRIPTION
apt provides a high-level commandline interface for the package management system. It is intended as an end user interface
and enables some options better suited for interactive usage by default compared to more specialized APT tools like apt-get(8)
and apt-cache(8).

update (apt-get(8))
  update is used to download package information from all configured sources. Other commands operate on this data to e.g.
  perform package upgrades or search in and display details about all packages available for installation.

upgrade (apt-get(8))
  upgrade is used to install available upgrades of all packages currently installed on the system from the sources configured
  via sources.list(5). New packages will be installed if required to satisfy dependencies, but existing packages will never
  be removed. If an upgrade for a package requires the removal of an installed package the upgrade for this package isn't performed.

full-upgrade (apt-get(8))
  full-upgrade performs the function of upgrade but will remove currently installed packages if this is needed to upgrade the
  system as a whole.
\end{alltt}

\begin{list2}
  \item Install a program using apt, for example \verb+apt install nmap+
\end{list2}



\slide{Ansible}

\hlkimage{2cm}{Ansible_logo.png}

\begin{quote}
From my course materials:\\
Ansible is great for automating stuff, so by running the playbooks we can get a whole lot of programs installed, files modified - avoiding the Vi editor.
\end{quote}

\begin{list2}
\item Easy to read, even if you don't know much about YAML
\item \link{https://www.ansible.com/} and \link{https://en.wikipedia.org/wiki/Ansible_(software)}
\item Great documentation\\
 \link{https://docs.ansible.com/ansible/latest/collections/ansible/builtin/apt_module.html}
\end{list2}


\slide{Ansible Dependencies}

\hlkimage{10cm}{python-logo.png}

\begin{list2}
\item Ansible based on Python, only need Python installed\\
\link{https://www.python.org/}
\item Often you use Secure Shell for connecting to servers\\
\link{https://www.openssh.com/}
\item Easy to configure SSH keys, for secure connections
\end{list2}


\slide{Ansible playbooks}

Example playbook content, installing software using APT:
\begin{alltt}\small
apt:
    name: "\{\{ packages \}\}"
    vars:
      packages:
        - nmap
        - curl
        - iperf
        ...
\end{alltt}

Running it:
\begin{minted}[fontsize=\small]{shell}
cd kramse-labs/suricatazeek
ansible-playbook -v 1-dependencies.yml 2-suricatazeek.yml 3-elasticstack.yml
\end{minted}

"YAML (a recursive acronym for "YAML Ain't Markup Language") is a human-readable data-serialization language."\\
\link{https://en.wikipedia.org/wiki/YAML}

\slide{Python and YAML -- Git}

\hlkimage{7cm}{git-logo.png}

\begin{list2}
\item We need to store configurations
\item Run playbooks
\item Problem: Remember what we did, when, how
\item Solution: use git for the playbooks
\item Not the only version control system, but my preferred one
\end{list2}

\slide{Alternative}

\hlkimage{10cm}{manual-install-es.png}

My playbooks allow installation of a whole Elastic stack in less then 10 minutes,

compare to:\\
\emph{Getting started with the Elastic Stack}\\
{\footnotesize\link{https://www.elastic.co/guide/en/elastic-stack-get-started/current/get-started-elastic-stack.html}}


\slide{Git getting started}

{\bf Hints:}\\
Browse the Git tutorials on \link{https://git-scm.com/docs/gittutorial}\\
and \link{https://guides.github.com/activities/hello-world/}

\begin{list2}
\item What is git
\item Terminology
\end{list2}

Note: you don't need an account on Github to download/clone repositories, but having an acccount allows you to save repositories yourself and is recommended.

\slide{Demo: Ansible, Python, Git!}

\begin{quote}
  Running Git will allow you to clone repositories from others easily. This is a great way to get new software packages, and share your own.

  Git is the name of the tool, and Github is a popular site for hosting git repositories.
\end{quote}


\begin{list2}
\item Go to \link{https://github.com/kramse/kramse-labs}
\item Lets explore while we talk
\end{list2}


\slide{Demo: output from running a git clone}

\begin{alltt}\footnotesize
user@Projects:tt$ {\bf git clone https://github.com/kramse/kramse-labs.git}
Cloning into 'kramse-labs'...
remote: Enumerating objects: 283, done.
remote: Total 283 (delta 0), reused 0 (delta 0), pack-reused 283
Receiving objects: 100% (283/283), 215.04 KiB | 898.00 KiB/s, done.
Resolving deltas: 100% (145/145), done.

user@Projects:tt$ {\bf cd kramse-labs/}

user@Projects:kramse-labs$ {\bf ls}
LICENSE  README.md  core-net-lab  lab-network  suricatazeek  work-station
user@Projects:kramse-labs$ git pull
Already up to date.
\end{alltt}

for reference at home later


\slide{Exercise CHAOS: Don't Panic -- have fun learning}

\hlkimage{6cm}{dont-panic.png}

\begin{quote}
“It is said that despite its many glaring (and occasionally fatal) inaccuracies, the Hitchhiker’s Guide to the Galaxy itself has outsold the Encyclopedia Galactica because it is slightly cheaper, and because it has the words ‘DON’T PANIC’ in large, friendly letters on the cover.”
\end{quote}
Hitchhiker’s Guide to the Galaxy, Douglas Adams

\slide{Your lab setup}

\begin{list2}
\item Go to GitHub, Find user Kramse, click through kramse-labs
\item Look into the instructions for the Virtual Machine -- Debian only

\item Get the lab instructions, from\\ {\footnotesize\url{https://github.com/kramse/kramse-labs/tree/master/suricatazeek}}
\end{list2}

Yes, we will reuse some instructions for the Suricata Zeek workshop - tested and working!

\exercise{ex:basicKaliVM}

\exercise{ex:sw-downloadDEB}
\exercise{ex:sw-basicDebianVM}
\exercise{ex:sw-basicLinuxetc}
\exercise{ex:debian-firewall}

\exercise{ex:git-tutorial}
\exercise{ex:basicansible}
\exercise{ex:xpack-security}


\slidenext{Buy the books! Create your VMs}


\end{document}
