\documentclass[Screen16to9,17pt]{foils}
\usepackage{zencurity-slides}

\externaldocument{communication-and-network-security-exercises}
\selectlanguage{english}

\begin{document}

\mytitlepage
{12. Building Robust Networks}
{Communication and Network Security 2019}


\slide{Plan for today}

\begin{list1}
\item Subjects
\begin{list2}
\item Design a robust network
\item Isolation and segmentation
\item (Routing Security) removed, see Running a Modern Network slide set
\item Switch and access security, port security
\item (Wireless security) removed - see Wireless Security slide set
\item (Using reputation lists) removed - see Network Intrusion Detection slide set
\end{list2}
\item Exercises We will design and build a sample network together
\begin{list2}
\item VLANs, Routing and RPF
\item Wifi, WPA and guest network
\item Monitoring - setup LibreNMS
\item IDS with Bro and Suricata - if we have time
\item Configure port security - if we have time
\end{list2}
\end{list1}


\slide{Reading Summary}

\begin{list2}
\item Read
\item {\small \link{https://nsrc.org/workshops/2018/myren-nsrc-cndo/networking/cndo/en/presentations/Campus_Security_Overview.pdf}}
\item {\small \link{https://nsrc.org/workshops/2018/tenet-nsrc-cndo/networking/cndo/en/presentations/Campus_Operations_BCP.pdf}}
\item Download, but dont read it all\\{\small \link{https://nsrc.org/workshops/2015/apricot2015/raw-attachment/wiki/Track1Agenda/01-ISP-Network-Design.pdf}}
\end{list2}





\slide{Produktionsmodning af miljøer}

\begin{list1}
\item Tænk på det miljø som servere og services skal udsættes for
\item Sørg for hærdning og tænk generel sikring:
  \begin{list2}
  \item Opdateret software - ingen kendte sikkerhedshuller eller
  sårbarheder
\item fjern {\bfseries single points of failure} - redundant strøm, ekstra enheder, to DNS servere fremfor en
\item adskilte servere - interne og eksterne til forskellige formål\\
Eksempelvis den interne postserver hvor alle e-mail opbevares og en
DMZ-postserver til ekstern post
\item lav filtre på netværket, eller på data - firewalls og proxy
  funktioner
\item begræns adgangen til at læse information
\item begræns adgangen til at skrive information - eksempelvis databaser
\item {\bfseries least privileges} - sørg for at programmer og brugere
  kun har de nødvendige rettigheder til at kunne udføre opgaver
\item følg med på områderne der har relevans for virksomheden og
  \emph{jeres} installation - Windows, UNIX, BIND, Oracle, ...
  \end{list2}
  \item Meld jer på security mailinglister for de produkter I benytter, også open source
\end{list1}


\slide{Change management}

\begin{list1}
\item Er der tilstrækkeligt med fokus på software i produktion
\item Kan en vilkårlig server nemt reetableres
\item Foretages rettelser direkte på produktionssystemer
\item Er der fall-back plan
\item Burde være god systemadministrator praksis
\end{list1}



\slide{Fundamentet skal være iorden}

\begin{list1}
\item Sørg for at den infrastruktur som I bygger på er sikker:
\begin{list2}
 \item redundans
       \item opdateret
        \item dokumenteret
        \item nem at vedligeholde
\end{list2}

\item  Husk tilgængelighed er også en sikkerhedsparameter
\end{list1}
\slide{Fokus 2019}

\begin{list2}
\item Brugerstyring
\item Asset management
\item Laptop sikkerhed
\item VPN alle steder
\item Penetration testing
\item Firewalls og segmentering
\item TLS og VPN indstillinger
\item DNS og email
\item Syslog og monitorering
\item Incident Response og reaktion
\end{list2}

Check eventuelt IT sikkerhedsupdate 2019 præsentationen:\\
{\small\link{https://github.com/kramse/security-courses/tree/master/presentations/misc/it-sikkerhedsupdate-2019}}

\slide{Design a robust network Isolation and segmentation}

\begin{list1}
\item Hvad kan man gøre for at få bedre netværkssikkerhed?
\begin{list2}
\item Bruge switche - der skal ARP spoofes og bedre performance
\item Opdele med firewall til flere DMZ zoner for at holde
      udsatte servere adskilt fra hinanden, det interne netværk og
      Internet
\item Overvåge, læse logs og reagere på hændelser
\end{list2}
\item Husk du skal også kunne opdatere dine servere
\end{list1}

\slide{Basic Network Security Pattern Isolate in VLANs}

\hlkimage{10cm}{images/demo-netvaerk.pdf}

\begin{list1}
\item Du bør opdele dit netværk i segmenter efter traffik
\item Du bør altid holde interne og eksterne systemer adskilt!
\item Du bør isolere farlige services i jails og chroots
\end{list1}


\slide{Our Networks}

We will now configure networks, using our sample router TP-Link T1500G-10PS

Use the guides from:\\
{\small \link{https://www.tp-link.com/uk/support/download/t1500g-10ps/\#Related-Documents}}

\slide{Using reputation lists}

\slide{Switch and access security, port security}


\slide{VLANs, Routing and RPF}

\slide{Monitoring - setup LibreNMS - if we have time}

\slide{IDS with Bro and Suricata - if we have time}


\slide{Configure port security - if we have time}




\exercise{ex:suricata-real-network}

\exercise{ex:port-security}



\slidenext

\end{document}
