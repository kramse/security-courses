\documentclass[Screen16to9,17pt]{foils}
\usepackage{zencurity-slides}

\externaldocument{communication-and-network-security-exercises}
\selectlanguage{english}

\begin{document}

\mytitlepage
{Running a Modern Network}
{Communication and Network Security 2019}


\slide{Plan for today}

\begin{list1}
\item Subjects
\begin{list2}
\item
\end{list2}
\item Exercises
\begin{list2}
\item
\end{list2}
\end{list1}

\slide{Infrastrukturer i praksis}

\begin{list1}
\item Vi vil nu gennemgå netværksdesign med udgangspunkt i vores setup
\item Vores setup indeholder:
\begin{list2}
\item Routere
\item Firewall
\item Wireless
\item DMZ
\item DHCPD, BIND, BGPD, OSPFD, ...
\end{list2}
\item Den kunne udvides med flere andre teknologier vi har til rådighed:
\begin{list2}
\item VLAN inkl VLAN trunking/distribution
\item WPA Enterprise
\end{list2}
\item Hvad taler for og imod - de næste slides gennemgår nogle standardsetups
\item En slags Patterns for networking
\end{list1}





\slide{Netværksdesign og sikkerhed}

\begin{list1}
\item Hvad kan man gøre for at få bedre netværkssikkerhed?
\begin{list2}
\item Bruge switche - der skal ARP spoofes og bedre performance
\item Opdele med firewall til flere DMZ zoner for at holde
      udsatte servere adskilt fra hinanden, det interne netværk og
      Internet
\item Overvåge, læse logs og reagere på hændelser
\end{list2}
\item Husk du skal også kunne opdatere dine servere
\end{list1}

\slide{basalt netværk}

\hlkimage{10cm}{images/demo-netvaerk.pdf}

\begin{list1}
\item Du bør opdele dit netværk i segmenter efter traffik
\item Du bør altid holde interne og eksterne systemer adskilt!
\item Du bør isolere farlige services i jails og chroots
\end{list1}



\slide{Intrusion Detection Systems - IDS}

\begin{list1}
  \item angrebsværktøjerne efterlader spor

\item hostbased IDS - kører lokalt på et system og forsøger at
  detektere om der er en angriber inde
\item network based IDS - NIDS - bruger netværket
\item Automatiserer netværksovervågning:
  \begin{list2}
  \item bestemte pakker kan opfattes som en signatur
\item analyse af netværkstrafik - FØR angreb
\item analyse af netværk under angreb - sender en alarm
  \end{list2}
\item \link{http://www.snort.org} - det kan anbefales at se på Snort
\end{list1}

\slide{snort}

\hlkimage{5cm}{images/snort_tm.png}

\begin{list1}
\item Snort er Open Source og derfor godt til undervisning
\item man kan se det som et antivirus system til netværket
\item forsøger at detektere \emph{angreb}, \emph{skadelig} og
  \emph{forkert} traffik
\item pakker der minder om eksempelvis:
  \begin{list2}
    \item nmap portscan
\item nmap OS detection - med underlige pakker
\item fragmenter der overlapper
\item shellcode der sendes til systemer som BIND
  \end{list2}
\end{list1}

\slide{Snort regler}

\begin{alltt}\small
alert icmp $HOME_NET any -> $EXTERNAL_NET any (msg:"ICMP Address Mask
Reply"; icode:0; itype:18; classtype:misc-activity; sid:386; rev:5;)
alert icmp $EXTERNAL_NET any -> $HOME_NET any (msg:"ICMP Address Mask
Reply undefined code"; icode:>0; itype:18; classtype:misc-activity;
sid:387; rev:7;)
alert icmp $EXTERNAL_NET any -> $HOME_NET any (msg:"ICMP Address Mask
Request"; icode:0; itype:17; classtype:misc-activity; sid:388; rev:5;)
alert icmp $EXTERNAL_NET any -> $HOME_NET any (msg:"ICMP Address Mask
Request undefined code"; icode:>0; itype:17; classtype:misc-activity;
sid:389; rev:7;)
alert icmp $EXTERNAL_NET any -> $HOME_NET any (msg:"ICMP Alternate
Host Address"; icode:0; itype:6; classtype:misc-activity; sid:390; rev:5;)
\end{alltt}

\begin{list2}
\item sid - snort rules id - identificerer en signatur
\item reference - hvor kommer reglen fra
\item icode - ICMP code
\item itype - ICMP type
\item ... se mere i snort manualen
\end{list2}

\slide{Ulemper ved IDS}

\hlkimage{5cm}{images/snort_tm.png}

\begin{list1}
\item snort er baseret på signaturer
\item mange falske alarmer - tuning og vedligehold
\item hvordan sikrer man sig at man har opdaterede signaturer for
  angreb som går verden rundt på et døgn
\end{list1}

\slide{ Planlægning af IDS miljøer}

\begin{list1}
\item Før installationen
\begin{list2}
\item Hvad er formålet - reaktion eller "statistik"
\item Hvor skal der måles - hele netværket eller specifikke dele
\item Hvad skal måles og hvilke operativsystemer og servere/services
\end{list2}
\item Implementationen
\begin{list2}
\item Er infrastrukturen iorden som den er
\item Er der gode målepunkter - monitorporte
\item Et målepunkt eller flere
\item Hvormeget trafik skal måles
\end{list2}
\item Selve idriftsættelsen
\begin{list2}
\item Ændringer af infrastrukturen
\item Installation af udstyret
\item Test af udstyret udenfor drift
\item Installation i driftsmiljøet
\item Test af udstyret i driftsmiljøet
\end{list2}
\end{list1}

\slide{ Opsætning og konfiguration af IDS miljøer}

\begin{list1}
\item Vælg en simpel installation til at starte med!
\item Undgå for alt i verden for meget information
\begin{list2}
\item Start med en enkelt sensor
\item Byg en server med database og "brugerværktøjer"
\item Start med at overvåge dele af nettet
\item Brug et specifikt regelsæt i starten - eksempelvis kun Windows eller kun UNIX
\item Lav nogle simple rapporter til at starte med
\end{list2}
\item Gør netværket mere sikkert før du lytter på hele netværket
\item Brug tcpdump/Ethereal til at se på trafik, lær IP pakker at
  kende
\item Brug Snort til at evaluere
\begin{list2}
\item husk at man kan starte med Snort og senere skifte til andre
produkter
\item Erfaring tæller, Snort tillader at man ser de fine detaljer - motoren
\end{list2}
\end{list1}

\slide{ Vedligehold og overvågning af IDS miljøer}

\begin{list1}
\item Uden vedligehold er IDS værdiløst - lad hellere være!
\begin{list2}
\item Vedligehold af software på operativsystemet
\item Vedligehold af IDS softwaren
\item Vedligehold af regelsæt
\end{list2}
\item Overvågning - kører IDS systemet, databaser og sensorer
\item Statistik og brug af IDS systemet
\begin{list2}
\item Vedligehold af rapporter - hvad er vi interesseret i
\item Automatisk rapportgenerering - daglig rapport, rapport pr måned
\item Specielle hændelser - hvad skete der onsdag mellem 11-12
\end{list2}
\item Et IDS kan også blot være en ARPwatch
\item ARPwatch advarer hvis nogen tager adressen fra default gateway
\end{list1}


\slide{Honeypots}

\begin{list1}
\item Man kan udover IDS installere en honeypot
\item En honeypot består typisk af:
  \begin{list2}
    \item Et eller flere sårbare systemer
\item Et eller flere systemer der logger traffik til og fra honeypot
  systemerne
  \end{list2}
\item Meningen med en honeypot er at den bliver angrebet og brudt ind
  i
\end{list1}

%\slide{Prelude}

%Måske Prelude i kombination med Nagios, Cricket, MRTG, RRDTool, Smokeping, ARPwatch


%\slide{Oversigt over forsvar mod sårbarheder}

\begin{list1}
\item Hvad muligheder har man
  \begin{list2}
  \item Ændre miljø
  \item forbedre systemerne
  \item undgå standardindstillinger
  \item vær opdateret på sikkerhedsområdet
  \item have retningslinier - ens sikkerhedsniveau
  \item drop kompatibilitet med usikre systemer
  \item en god infrastruktur
  \item brug kryptografi
  \item brug standardbiblioteker
  \item test af systemer
  \end{list2}
\end{list1}

\slide{Ændre miljø}

\begin{list1}
\item Ændre arkitektur sw/hw/netværkstopologi
  \begin{list2}
  \item blokere porte således at en webserver IKKE kan connecte tilbage til hackeren!
  \item blokere de services der IKKE skal tilgås udefra
  \item skifte programmeringssprog
  \end{list2}
\item Husk altid at hackeren også kan gå ind ad hovedøren
\item eksempelvis SAP Internet gateway, hvor man kunne lægge det
  bagvedliggende system ned med loginrequests
\end{list1}
\slide{Forbedre systemerne}

\begin{list1}
\item Operativsystemet
  \begin{list2}
  \item non-executable stack
  \item non-executable heap
  \end{list2}
\item Applikationsservere
  \begin{list2}
  \item filtrering af "dårlige" requests e-Eye sikret IIS
  \item mere "sikker" default opsætning
  \end{list2}
\item Jeg tror vi vil se flere implementere den slags løsninger
\item Eksempelvis:
\begin{list2}
\item Microsoft IIS web server version 6 er mere sikker i default opsætningen
\item Apache HTTPD web server version 2 er mere modulær og nemmere at bygge sikkert
\end{list2}
\end{list1}

\slide{Undgå standard indstillinger}

\begin{list1}
\item Giv jer selv mere tid til at patche og opdatere
\item Tiden der går fra en sårbarhed annonceres på bugtraq til den bliver
       udnyttet er meget kort idag!
\item Ved at undgå standard indstillinger kan der
       måske opnås en lidt længere frist - inden ormene kommer
\item NB: ingen garanti
\end{list1}



\slide{Pattern: erstat Telnet med SSH}

\begin{list1}
\item Telnet er død!
\item Brug altid Secure Shell fremfor Telnet
\item Opgrader firmware til en der kan SSH, eller køb bedre udstyr næste gang
\item Selv mine små billige Linksys switche forstår SSH!
\end{list1}

\slide{Pattern: erstat FTP med HTTP}

\begin{list1}
\item Hvis der kun skal distribueres filer kan man ofte benytte HTTP istedet for FTP
\item Hvis der skal overføres med password er SCP/SFTP fra Secure Shell at foretrække
\end{list1}


\slide{Anti-patterns}

\begin{list1}
\item Nu præsenteres et antal setups, som ikke anbefales
\item Faktisk vil jeg advare mod at bruge dem
\item Husk følgende slides er min mening
\end{list1}

\slide{Anti-pattern dobbelt NAT i eget netværk}

\hlkimage{17cm}{nat-double.pdf}

\begin{list1}
\item Det er nødvendigt med NAT for at oversætte traffik der sendes videre
ud på internet.
\item Der er ingen som helst grund til at benytte NAT indenfor eget netværk!
\end{list1}

\slide{Anti-pattern blokering af ALT ICMP}

\begin{alltt}
ipfw add allow icmp from any to any icmptypes 3,4,11,12
\end{alltt}

\begin{list1}
\item Lad være med at blokere for alt ICMP, så ødelægger du funktionaliteten i dit net
\vskip 1cm
\item \end{list1}

\slide{Anti-pattern blokering af DNS opslag på TCP}

\begin{list1}
\item Det bliver (er) nødvendigt med DNS opslag over TCP på grund af store svar. Det betyder at firewalls skal tillade DNS opslag via TCP
\vskip 1cm
\item
\item Guide:\\
Brug en caching nameserver, således at det kun er den som kan lave DNS opslag ud i verden

\end{list1}

\slide{Anti-pattern daisy-chain}

\hlkimage{15cm}{daisy-chain-server.pdf}

\begin{list1}
\item Daisy-chain af servere, erstat med firewall, switch og VLAN
\item Det giver et væld af problemer med overvågning, administration, backup og opdatering
\end{list1}

\slide{Anti-pattern WLAN forbundet direkte til LAN}

\hlkimage{10cm}{images/wlan-accesspoint-2.pdf}

\begin{list1}
\item WLAN AP'er forbundet direkte til LAN giver risiko for at sikkerheden brydes, fordi AP falder tilbage på den usikre standardkonfiguration
\item Ved at sætte WLAN direkte på LAN risikerer man at eksterne får direkte adgang
\end{list1}


\slide{At være på internet}

\begin{list1}
\item RFC-2142 Mailbox Names for Common Services, Roles and Functions
\item Du BØR konfigurere dit domæne til at modtage post for følgende adresser:
\begin{list2}
\item postmaster@domæne.dk
\item abuse@domæne.dk
\item webmaster@domæne.dk, evt. www@domæne.dk
\end{list2}
\item Du gør det nemmere at rapportere problemer med dit netværk og services
\end{list1}

\slide{E-mail best current practice}

\begin{alltt}\small
MAILBOX       AREA                USAGE
-----------   ----------------    ---------------------------
ABUSE         Customer Relations  Inappropriate public behaviour
NOC           Network Operations  Network infrastructure
SECURITY      Network Security    Security bulletins or queries
...
MAILBOX       SERVICE             SPECIFICATIONS
-----------   ----------------    ---------------------------
POSTMASTER    SMTP                [RFC821], [RFC822]
HOSTMASTER    DNS                 [RFC1033-RFC1035]
USENET        NNTP                [RFC977]
NEWS          NNTP                Synonym for USENET
WEBMASTER     HTTP                [RFC 2068]
WWW           HTTP                Synonym for WEBMASTER
UUCP          UUCP                [RFC976]
FTP           FTP                 [RFC959]
\end{alltt}

Kilde:
RFC-2142 Mailbox Names for Common Services, Roles and Functions. D.
Crocker. May 1997

\slide{Brug krypterede forbindelser}

\hlkimage{12cm}{images/dsniff-comments.pdf}

\begin{list1}
\item Især på utroværdige netværk kan det give problemer at benytte
  sårbare protokoller
\end{list1}

\slide{Mission 1: Kommunikere sikkert}

\begin{list1}
\item Du må ikke bruge ukrypterede forbindelser til at administrere
  UNIX
\item Du må ikke sende kodeord i ukrypterede e-mail beskeder
\end{list1}

\centerline{\hlkbig Telnet daemonen - telnetd må og skal dø!}

\pause
\centerline{\hlkbig FTP daemonen - ftpd må og skal dø!}

\pause
\centerline{\hlkbig POP3 daemonen port 110 må og skal dø!}

\pause
\centerline{\hlkbig IMAPD daemonen port 143 må og skal dø!}

\pause
\vskip 1cm
\centerline{\hlkbig\bf væk med alle de ukrypterede forbindelser!}


\slide{Change management}

\begin{list1}
\item Er der tilstrækkeligt med fokus på software i produktion
\item Kan en vilkårlig server nemt reetableres
\item Foretages rettelser direkte på produktionssystemer
\item Er der fall-back plan
\item Burde være god systemadministrator praksis
\end{list1}



\slide{Fundamentet skal være iorden}

\begin{list1}
\item Sørg for at den infrastruktur som I bygger på er sikker:
\begin{list2}
 \item redundans
       \item opdateret
        \item dokumenteret
        \item nem at vedligeholde
\end{list2}

\item  Husk tilgængelighed er også en sikkerhedsparameter
\end{list1}


\slide{individuel autentificering!}

\hlkimage{7cm}{images/ssh-root.pdf}

\begin{list1}
\item Mange UNIX systemer administreres fejlagtigt ved brug af
  root-login
\item Undgå direkte root-login
\item Insister på \verb+sudo+ eller \verb+su+
\item Hvorfor?
\begin{list2}
\item Sporbarheden mistes hvis brugere logger direkte ind som root
\item Hvis et kodeord til root gættes er der direkte adgang til alt!
\end{list2}
\end{list1}

\slide{Jump Host}

\slidenext

\end{document}
