\documentclass[Screen16to9,17pt]{foils}
\usepackage{zencurity-slides}

\externaldocument{communication-and-network-security-exercises}
\selectlanguage{english}

\begin{document}

\mytitlepage
{13. Running a Modern Network}
{Communication and Network Security \the\year}


\slide{Plan for today}

\begin{list1}
\item Subjects
\begin{list2}
\item BCP38 RFC2827: Network Ingress Filtering: Defeating Denial of Service Attacks\\
 which employ IP Source Address Spoofing
\item Mutually Agreed Norms for Routing Security (MANRS)
\item Testing security, evaluating and reporting
\item Hardened network device configurations
\item Jump hosts and management networks
\item DDoS protection
\item Check you network from outside RIPEstat, BGPmon
\end{list2}
\item Exercises
\begin{list2}
\item Look at your own networks, from the outside
\end{list2}
\end{list1}


Misc stuff that didn't fit in the other parts \smiley

\slide{Reading Summary}

\begin{list1}
\item Browse\\
{\small\link{https://www.manrs.org/}}\\
and read this\\ {\small\link{https://www.manrs.org/wp-content/uploads/2018/09/MANRS_PDF_Sep2016.pdf}}
\item Browse / skim this:\\
{\small\link{https://tools.ietf.org/pdf/bcp38.pdf}}\\
RFC2827: Network Ingress Filtering: Defeating Denial of Service Attacks
\end{list1}


\slide{Infrastrukturer i praksis}

\begin{list1}
\item Vi vil nu gennemgå netværksdesign med udgangspunkt i mit setup
\begin{list2}
\item Routere - router og forwarder pakker
\item Firewall - checker pakkerne før de sendes videre
\item Wireless - trådløse netværk skal man have
\item DMZ - adskilte netværks zone
\item DHCPD, BIND, BGPD, OSPFD, ... routing protkoller ofte nødvendige i større netværk
\item LLDP, VRRP, ... støtteprotokoller til diverse formål
\end{list2}
\item Den kunne udvides med flere andre teknologier vi har til rådighed:
\begin{list2}
\item VLAN inkl VLAN trunking/distribution, WPA Enterprise
\end{list2}
\item Hvad taler for og imod - de næste slides gennemgår nogle standardsetups
\item En slags patterns og
anti-patterns for networking
\item Husk følgende slides er min mening
\end{list1}





\slide{Netværksdesign og sikkerhed}

\begin{list1}
\item Hvad kan man gøre for at få bedre netværkssikkerhed?
\begin{list2}
\item Bruge switche -- managed switche med monitorering, logning og administration
\item Opdele med firewall til flere DMZ zoner for at holde
      udsatte servere adskilt fra hinanden, det interne netværk og
      Internet
\item Overvåge, læse logs og reagere på hændelser
\end{list2}
\end{list1}

\slide{Basic Network Security Pattern Isolate in VLANs}

\hlkimage{10cm}{images/demo-netvaerk.pdf}

\begin{list1}
\item Du bør opdele dit netværk i segmenter efter traffik
\item Du bør altid holde interne og eksterne systemer adskilt!
\item Du bør isolere farlige services i jails og chroots
\end{list1}



\slide{Pattern use IDS to get flow, connections and data}

\begin{list1}
\item Use Intrusion Detection Systems - IDS
\item Angrebsværktøjerne efterlader spor
\item Det anbefales at have IDS og flow opsamling som minimum
\item Hostbased IDS - kører lokalt på et system og forsøger at
  detektere om der er en angriber inde
\item Network based IDS - NIDS - bruger netværket
\item Automatiserer netværksovervågning:
  \begin{list2}
  \item bestemte pakker kan opfattes som en signatur
\item analyse af netværkstrafik - FØR angreb
\item analyse af netværk under angreb - sender en alarm
  \end{list2}
\end{list1}


\slide{Drift og Planlægning af sikre miljøer}

\begin{list1}
\item Før installationen scope
\begin{list2}
\item Hvad er formålet - reaktion eller "statistik"
\item Hvor skal der måles - hele netværket eller specifikke dele
\item Hvad skal måles og hvilke operativsystemer og servere/services
\end{list2}
\item Implementationen
\begin{list2}
\item Er infrastrukturen iorden som den er
\item Er der gode målepunkter - monitorporte
\item Et målepunkt eller flere, Hvor meget trafik skal måles
\end{list2}
\item Selve idriftsættelsen
\begin{list2}
\item Ændringer af infrastrukturen
\item Installation af udstyret og test af udstyret udenfor drift
\item Installation i driftsmiljøet
\item Test af udstyret i driftsmiljøet
\end{list2}
\end{list1}

\slide{Eksempel Opsætning og konfiguration af IDS miljøer}

\begin{list1}
\item Vælg en simpel installation til at starte med!
\item Undgå for alt i verden for meget information
\begin{list2}
\item Start med en enkelt sensor
\item Byg en server med database og "brugerværktøjer"
\item Start med at overvåge dele af nettet
\item Brug et specifikt regelsæt i starten - eksempelvis kun Windows eller kun UNIX
\item Lav nogle simple rapporter til at starte med
\end{list2}
\item Gør netværket mere sikkert før du lytter på hele netværket
\item Brug tcpdump/Wireshark til at se på trafik, lær IP pakker at
  kende
\item Brug Suricata og Zeek til at opsamle og evaluere trafikken -- baseline af netværket
\begin{list2}
\item Husk at man kan starte med vilkårligt værktøj og senere skifte til andre
produkter, evt. kommercielle
\item Praktisk erfaring med eget netværk er nødvendigt og værdifuldt
\end{list2}
\end{list1}

\slide{Honeypots}

\begin{list1}
\item Man kan udover IDS installere en honeypot
\item En honeypot består typisk af:
  \begin{list2}
    \item Et eller flere sårbare systemer
\item Et eller flere systemer der logger traffik til og fra honeypot
  systemerne
  \end{list2}
\item Meningen med en honeypot er at den bliver angrebet og brudt ind
  i
\end{list1}

\slide{Forslag undgå standard indstillinger}

\begin{list1}
\item Giv jer selv mere tid til at patche og opdatere
\item Tiden der går fra en sårbarhed annonceres på internet til den bliver
       udnyttet er meget kort idag!
\item Ved at undgå standard indstillinger kan der
       måske opnås en lidt længere frist
\item NB: ingen garanti
\end{list1}






\slide{Brug krypterede forbindelser}

\hlkimage{12cm}{images/dsniff-comments.pdf}

\begin{list1}
\item Især på utroværdige netværk kan det give problemer at benytte
  sårbare protokoller
\end{list1}

\slide{Mission 1: Kommunikere sikkert}

\begin{list1}
\item Du må ikke bruge ukrypterede forbindelser til at administrere netværk eller servere
\item Du må ikke sende kodeord i ukrypterede e-mail beskeder
\end{list1}

\centerline{\hlkbig Telnet daemonen - telnetd må og skal dø!}

\pause
\centerline{\hlkbig FTP daemonen - ftpd må og skal dø!}

\pause
\centerline{\hlkbig POP3 daemonen port 110 må og skal dø!}

\pause
\centerline{\hlkbig IMAPD daemonen port 143 må og skal dø!}

\pause
\vskip 1cm
\centerline{\hlkbig\bf væk med alle de ukrypterede forbindelser!}

\slide{Pattern: erstat Telnet med SSH}

\begin{list1}
\item Telnet er død!
\item Brug altid Secure Shell fremfor Telnet
\item Opgrader firmware til en der kan SSH, eller køb bedre udstyr næste gang
\item Selv mine små billige Linksys switche forstår SSH!
\end{list1}

\slide{Pattern: erstat FTP med HTTP}

\begin{list1}
\item Hvis der kun skal distribueres filer kan man ofte benytte HTTP istedet for FTP
\item Hvis der skal overføres med password er SCP/SFTP fra Secure Shell at foretrække
\end{list1}


\slide{FTP File Transfer Protocol}

\begin{list1}
\item File Transfer Protocol - filoverførsler
\item Bruges især til:
  \begin{list2}
    \item FTP - drivere, dokumenter, rettelser - Windows Update? er
    enten HTTP eller FTP
  \end{list2}
\item FTP sender i klartekst\\
{\bfseries USER brugernavn} og \\
{\bfseries PASS hemmeligt-kodeord}
\item Der findes varianter som tillader kryptering, men brug istedet SCP/SFTP over Secure Shell protokollen
\end{list1}


\slide{Anti-pattern dobbelt NAT i eget netværk}

\hlkimage{17cm}{nat-double.pdf}

\begin{list1}
\item Det er nødvendigt med NAT for at oversætte traffik der sendes videre
ud på internet.
\item Der er ingen som helst grund til at benytte NAT indenfor eget netværk!
\end{list1}

\slide{Anti-pattern blokering af ALT ICMP}

\begin{alltt}\small
# Allow ICMPv6 destination unreach (don't filter it out)
        $fwcmd6 add pass ipv6-icmp from any to any icmptypes 1
# Allow NS/NA/toobig (don't filter it out)
        $fwcmd6 add pass ipv6-icmp from any to any icmptypes 2
# Allow timex Time exceeded (don't filter it out)
        $fwcmd6 add pass ipv6-icmp from any to any icmptypes 3
# Allow parameter problem (don't filter it out)
        $fwcmd6 add pass ipv6-icmp from any to any icmptypes 4
# IPv6 ICMP - echo request (128) and echo reply (129)
        $fwcmd6 add pass ipv6-icmp from any to any icmptypes 128,129
# IPv6 ICMP - router solicitation (133) and router advertisement (134)
        $fwcmd6 add pass ipv6-icmp from any to any icmptypes 133,134
# IPv6 ICMP - neighbour discovery solicitation (135) and advertisement (136){\bf
        $fwcmd6 add pass ipv6-icmp from any to any icmptypes 135,136}
\end{alltt}

\begin{list2}
\item Lad være med at blokere for alt ICMP, så ødelægger du funktionaliteten i dit net
\item Eksemplet viser ICMPv6 -- hvor nogle NDP pakker er kritiske at have tilladte
\end{list2}

\slide{ICMPv4 beskedtyper}

\begin{list1}
\item Type
\begin{list2}
\item 0 = net unreachable;
\item 1 = host unreachable;
\item 2 = protocol unreachable;
\item 3 = port unreachable;
\item 4 = fragmentation needed and DF set;
\item 5 = source route failed.
\end{list2}
\item Tillad disse ICMP types:
\begin{list2}
\item 3 Destination Unreachable
\item 4 Source Quench Message
\item 11 Time Exceeded
\item 12 Parameter Problem Message
\end{list2}
\end{list1}

\begin{list1}
\item Lad være med at blokere for alt ICMP, så ødelægger du funktionaliteten i dit net
\end{list1}


\slide{Anti-pattern blokering af DNS opslag på TCP}

\begin{list1}
\item Det bliver (er) nødvendigt med DNS opslag over TCP
\vskip 1cm
\item Store svar kræver TCP
\item Det betyder at firewalls skal tillade DNS opslag via TCP
\vskip 1cm
\item De nye forslag DNS over TLS (DoT) og DNS over HTTPS (DoH)
\item DNS kryptering bliver med TCP
\vskip 1cm
\item Anbefaling i enterprise netværk:\\
Brug en caching nameserver, således at det kun er den som kan lave DNS opslag ud i verden

\end{list1}

\slide{Anti-pattern daisy-chain}

\hlkimage{15cm}{daisy-chain-server.pdf}

\begin{list1}
\item Daisy-chain af servere, erstat med firewall, switch og VLAN
\item Det giver et væld af problemer med overvågning, administration, backup og opdatering
\end{list1}

\slide{Anti-pattern WLAN forbundet direkte til LAN}

\hlkimage{8cm}{images/wlan-accesspoint-2.pdf}

\begin{list1}
\item WLAN AP'er forbundet direkte til LAN giver risiko for at sikkerheden brydes, fordi AP falder tilbage på den usikre standardkonfiguration
\item Ved at sætte WLAN direkte på LAN risikerer man at eksterne får direkte adgang
\item Tegningen skal forstås som den logiske kobliung. Typisk placeres APs ofte i eget VLAN, og dermed adskilt fra LAN
\end{list1}


\slide{Pattern individuel autentificering!}

\hlkimage{4cm}{images/ssh-root.pdf}

\begin{list1}
\item Mange systemer administreres fejlagtigt ved brug af
  root-login eller andet delt administrator login
\item Undgå direkte root-login eller administrator
\item Insister på \verb+sudo+ eller \verb+su+
\item Hvorfor?
\begin{list2}
\item Sporbarheden mistes hvis brugere logger direkte ind som root
\item Hvis et kodeord til root gættes er der direkte adgang til alt!
\end{list2}
\end{list1}

%\slide{Jump hosts and management networks}

\slide{Centralized management SSH, Jump hosts}

\hlkimage{7cm}{images/soekris-siig-4S-3.jpg}

\begin{quote}
A jump server, jump host or jumpbox is a computer on a network used to access and manage devices in a separate security zone. The most common example is managing a host in a DMZ from trusted networks or computers.
\end{quote}
Source: \link{https://en.wikipedia.org/wiki/Jump_server}



\slide{Mutually Agreed Norms for Routing Security (MANRS)}

\hlkimage{2cm}{MANRS_square.png}

\begin{quote}
  Mutually Agreed Norms for Routing Security (MANRS) is a global initiative, supported by the Internet Society, that provides crucial fixes to reduce the most common routing threats. 
\end{quote}

\begin{list1}
\item Problems related to incorrect routing information
\item Problems related to traffic with spoofed source IP addresses
\item Problems related to coordination and collaboration between network operators
\item {\small\link{https://www.manrs.org/isps/}}
\item {\small\link{https://www.manrs.org/wp-content/uploads/2018/09/MANRS_PDF_Sep2016.pdf}}
\end{list1}


\slide{Hosting og internet-udbydere}

\hlkimage{15cm}{network-bgp-asn.png}

\begin{list2}
\item Data krydser mange internetudbydere
\item Det er stadig muligt at spoofe mange steder fra
\item Din sikkerhed er afhængig af andres sikkerhed
\end{list2}

\slide{Expected Actions in MANRS for Network Operators}

\begin{list1}
\item 1. Prevent propagation of incorrect routing information\\
Clear routing policies and systems for correctness, route import filters
\item 2. Prevent traffic with spoofed source IP addresses\\
Validate source address from end-users and infrastructure, use BCP38
\item 3. Facilitate global operational communication and coordination between network operators\\
Maintain contact information in databases like Whois, PeeringDB
{\small\link{https://www.peeringdb.com/}}

\item Advanced
\item 4. Facilitate validation of routing information on a global scale\\
Use RPSL {\small\link{ https://en.wikipedia.org/wiki/Routing_Policy_Specification_Language}}
\end{list1}

\slide{At være på internet}

\begin{list1}
\item RFC-2142 Mailbox Names for Common Services, Roles and Functions
\item Du BØR konfigurere dit domæne til at modtage post for følgende adresser:
\begin{list2}
\item postmaster@domæne.dk
\item abuse@domæne.dk
\item webmaster@domæne.dk, evt. www@domæne.dk
\end{list2}
\item Du gør det nemmere at rapportere problemer med dit netværk og services
\end{list1}

\slide{E-mail best current practice}

\begin{alltt}\small
MAILBOX       AREA                USAGE
-----------   ----------------    ---------------------------
ABUSE         Customer Relations  Inappropriate public behaviour
NOC           Network Operations  Network infrastructure
SECURITY      Network Security    Security bulletins or queries
...
MAILBOX       SERVICE             SPECIFICATIONS
-----------   ----------------    ---------------------------
POSTMASTER    SMTP                [RFC821], [RFC822]
HOSTMASTER    DNS                 [RFC1033-RFC1035]
USENET        NNTP                [RFC977]
NEWS          NNTP                Synonym for USENET
WEBMASTER     HTTP                [RFC 2068]
WWW           HTTP                Synonym for WEBMASTER
UUCP          UUCP                [RFC976]
FTP           FTP                 [RFC959]
\end{alltt}

Kilde:
RFC-2142 Mailbox Names for Common Services, Roles and Functions. D.
Crocker. May 1997


\slide{RIPE NCC abuse-c}

\hlkimage{18cm}{ripe-abuse-c.png}

{\small\link{
https://www.ripe.net/manage-ips-and-asns/resource-management/abuse-c-information}}


\slide{Resource Public Key Infrastructure RPKI}

\begin{list2}
\item 1997 - AS7007 mistakenly (re)announces 72,000+ routes (becomes the poster-child for route filtering).
\item 2008 - ISP in Pakistan accidentally announces IP routes for YouTube by blackholing the video service internally to their network.
\item 2017 - Russian ISP leaks 36 prefixes for payments services owned by Mastercard, Visa, and major banks.
\item 2018 - BGP hijack of Amazon DNS to steal crypto currency.
\end{list2}
Source: \link{https://blog.cloudflare.com/rpki/}

\begin{list1}
\item RPKI \link{https://en.wikipedia.org/wiki/Resource_Public_Key_Infrastructure}
\item Authenticated routing protocols passwords, secrets etc.
\end{list1}

\slide{Use RPKI}

\begin{quote}
  Routinator is RPKI Relying Party software, also known as an RPKI Validator. It is designed to have a small footprint and great portability.

Routinator connects to the Trust Anchors of the five Regional Internet Registries (RIRs) — APNIC, AFRINIC, ARIN, LACNIC and RIPE NCC — downloads all of the certificates and ROAs in their repositories and validates the signatures. It can feed the validated information to hardware routers supporting Route Origin Validation such as Juniper, Cisco and Nokia, as well as serving software solutions like BIRD and OpenBGPD. Alternatively, Routinator can output the validated data in a number of useful formats, such as CSV, JSON and RPSL.
\end{quote}

\begin{list1}
\item Quote from {\small\link{https://www.nlnetlabs.nl/projects/rpki/routinator/}}
\item Update your records in the Whois system, read about RIPE here:\\ {\small\link{https://www.ripe.net/manage-ips-and-asns/resource-management/certification}}
\end{list1}



\slide{BCP38 RFC2827: Network Ingress Filtering}

BCP38 RFC2827: Network Ingress Filtering: Defeating Denial of Service Attacks which employ IP Source Address Spoofing

{\small\link{https://tools.ietf.org/pdf/bcp38.pdf}}\\
RFC2827: Network Ingress Filtering: Defeating Denial of Service Attacks




\slide{Testing security, evaluating and reporting}


\begin{center}
\begin{tikzpicture}[->,>=stealth',scale=0.8, transform shape]
\newlength{\boxwidth}
\setlength{\boxwidth}{0.21\paperwidth}
\newlength{\boxheight}
\setlength{\boxheight}{0.25\paperheight}
\newlength{\boxspace}
\setlength{\boxspace}{10cm}

% http://texample.net/tikz/examples/epc-flow-charts/

% https://www.overleaf.com/learn/latex/LaTeX_Graphics_using_TikZ:_A_Tutorial_for_Beginners_(Part_3)%E2%80%94Creating_Flowcharts

 % Use previously defined 'state' as layout (see above)
 % use tabular for content to get columns/rows
 % parbox to limit width of the listing
 \node[state,text width=\boxwidth,minimum height=\boxheight] (MEASURE)
 {\begin{tabular}{l}
 {\bf Measure}\\
    Enable logging\\
   Setup graphs\\
   Service monitoring
 \end{tabular}};

 %
 \node[state,    	% layout (defined above)
  text width=\boxwidth, 	% max text width
  minimum height=\boxheight,
  %yshift=2cm, 		% move 2cm in y
  right of=MEASURE, 	% Position is to the right of QUERY
  node distance=\boxspace, 	% distance to First node
  anchor=center] (STRATEGY) 	% posistion relative to the center of the 'box'
 {%
 \begin{tabular}{l}
 {\bf Strategy}\\
 Dependencies\\
 Implementation plan\\
 Inform others
 \end{tabular}};

 % STATE QUERYREP
 \node[state,
  below of=STRATEGY,
  yshift=-\boxheight,
  anchor=center,
  minimum height=\boxheight,
  text width=\boxwidth] (EXECUTE)
 {%
 \begin{tabular}{l}
 {\bf Execute}\\
  Make Changes\\
  Revert bad changes\\
  Timestamp changes
 \end{tabular}
 };

 % STATE EPC
 \node[state,
  right of=STRATEGY,
  text width=\boxwidth, 	% max text width
  minimum height=\boxheight,
  node distance=\boxspace,
  anchor=center] (SUCCESS)
 {%
 \begin{tabular}{l}
 {\bf Verify Success}\\
  Report to others\\
  Policy Compliance\\
  Result achieved?
 \end{tabular}
 };

 % draw the paths and and print some Text below/above the graph
 \path
 (MEASURE) 	edge node[anchor=north,above]{Identify} (STRATEGY)
 (MEASURE) 	edge node[anchor=south,below]{problems} (STRATEGY)
 (STRATEGY) edge node[anchor=north,above]{Review }   (SUCCESS)
 (STRATEGY) edge  node[anchor=right,left]{Planning} (EXECUTE)
 (EXECUTE)  edge[loop right]    node[anchor=right,left,text width=27mm]{multiple phases} (EXECUTE)
 (SUCCESS)   edge[bend left=110] node[anchor=south,below]{Improve Efficiency}  (MEASURE);

\end{tikzpicture}
\end{center}

Make incremental changes

\slide{Example: DDoS protection and flooding}

\hlkimage{12cm}{overview-routing-customer-2015.pdf}

\begin{list2}
\item Transport Layer Attacks TCP SYN flood TCP sequence numbers
\item High level attacks like Slowloris - keep TCP/HTTP connection for a long time.
\end{list2}

\slide{Båndbreddestyring og policy based routing}

\begin{list1}
\item Mange routere og firewalls idag kan lave båndbredde allokering til
  protokoller, porte og derved bestemte services
  \item Specielt relevant for DDoS beskyttelse
  \item Findes på F5 BigIP, Cisco, Junos osv.
\item Mest kendte er i Open Source:
\begin{list2}
\item OpenBSD - integreret i PF
\item FreeBSD har dummynet
\item Linux Traffic Control
\end{list2}
\item Det kaldes også traffic shaping
\end{list1}

\slide{hping3 packet generator}

\begin{alltt}\footnotesize
usage: hping3 host [options]
  -i  --interval  wait (uX for X microseconds, for example -i u1000)
      --fast      alias for -i u10000 (10 packets for second)
      --faster    alias for -i u1000 (100 packets for second)
      --flood      sent packets as fast as possible. Don't show replies.
...
hping3 is fully scriptable using the TCL language, and packets
can be received and sent via a binary or string representation
describing the packets.
\end{alltt}

\begin{list2}
\item Hping3 packet generator is a very flexible tool to produce simulated DDoS traffic with specific charateristics
\item Home page: \link{http://www.hping.org/hping3.html}
\item Source repository \link{https://github.com/antirez/hping}
\end{list2}

\centerline{My primary DDoS testing tool, easy to get specific rate pps}

\slide{t50 packet generator}


\begin{alltt}\footnotesize
root@cornerstone03:~# t50 -?
T50 Experimental Mixed Packet Injector Tool 5.4.1
Originally created by Nelson Brito <nbrito@sekure.org>
Maintained by Fernando Mercês <fernando@mentebinaria.com.br>

Usage: T50 <host> [/CIDR] [options]

Common Options:
    --threshold NUM        Threshold of packets to send     (default 1000)
    --flood                This option supersedes the 'threshold'
...
6. Running T50 with '--protocol T50' option, sends ALL protocols sequentially.
root@cornerstone03:~# t50 -? | wc -l
264
\end{alltt}

\begin{list2}
\item T50 packet generator, another high speed packet generator
can easily overload most firewalls by producing a randomized traffic with multiple protocols like IPsec, GRE, MIX \\
home page: \link{http://t50.sourceforge.net/resources.html}
\end{list2}

\centerline{Extremely fast and breaks most firewalls when flooding, easy 800k pps/400Mbps}

\slide{Process: monitor, attack, break, repeat}

\begin{list2}
\item Pre-test: Monitoring setup - from multiple points
\item Pre-test: Perform full Nmap scan of network and ports
\item Start small, run with delays between packets
\item Turn up until it breaks, decrease delay - until using \verb+--flood+
\item Monitor speed of attack on your router interface pps/bandwidth
\item Give it maximum speed\\
 \verb+hping3 --flood -1+ and \verb+hping3 --flood -2+
\item Have a common chat with network operators/customer to talk about symptoms and things observed
\item Any information resulting from testing is good information
\end{list2}

\vskip 1cm
\centerline{Ohh we lost our VPN into the environment, ohh the fw console is dead}


\slide{Packet processing in firewalls -- detailed view}

% Picture with pipeline from Juniper SRX
\hlkimage{21cm}{srx-firewall-flow-based-processing.png}
\emph{Traffic Processing on SRX Series Devices Overview}\\ {\scriptsize
\link{https://www.juniper.net/documentation/us/en/software/junos/flow-packet-processing/topics/topic-map/security-srx-devices-processing-overview.html}}




\slide{Scanning and Attacking -- \bf Pressure Points and Scope}

% Drawing showing "network"

\hlkimage{21cm}{generic-network-pressure-points.pdf}

\begin{list2}
\item In scope for me is everything that could adversely affect the network
\item Common scope IPv4: Link network /28 or /26 and a hosting network /24
\item Common scope IPv6: Link network /64 (bad) or /127 (RFC9099) and a hosting network /48 with subnets
%\item Domain Name Service outside of the network is ofc also attack vector, ignored today
\end{list2}



\slide{Enter MoonGen and Dedicated Hardware}

\hlkimage{6cm}{dell-precision-3240-compact.jpg}

\begin{list2}
\item Modern computer with modern CPU and PCIe x8 or better\\
I currently use a few cheap Dell devices Precision 3640 Tower / Precision 3240 Compact (not a recommendation)
\item Supported card - I am using the old Intel 82599 based 10Gbit cards
\item DPDK and software -- I use MoonGen \link{https://github.com/emmericp/MoonGen}
\item A huge thanks to Paul Emmerich emmericp for programming and publishing his works!
\item Maybe the easiest way to use DPDK currently -- "Craft all packets in user-controller Lua scripts"
\end{list2}


\slide{Running MoonGen}

\begin{alltt}\footnotesize
root@penguin01:~/projects/MoonGen# {\bf ./build/MoonGen ./examples/l3-tcp-syn-flood.lua 0 -d 192.0.2.138}
[INFO]  Initializing DPDK. This will take a few seconds...
EAL: Detected 16 lcore(s)
[INFO]  Found 1 usable devices:
   Device 0: 00:25:90:32:9F:F3 (Intel Corporation 82599ES 10-Gigabit SFI/SFP+ Network Connection)
PMD: ixgbe_dev_link_status_print():  Port 0: Link Down
[INFO]  Device 0 (00:25:90:32:9F:F3) is up: 10000 MBit/s
[INFO]  Detected an IPv4 address.
...
[Device: id=0] TX: 14.88 Mpps, 7619 Mbit/s (9999 Mbit/s with framing)
[Device: id=0] TX: 14.48 Mpps, 7414 Mbit/s (9730 Mbit/s with framing)
[Device: id=0] TX: 14.88 Mpps, 7619 Mbit/s (10000 Mbit/s with framing)
\end{alltt}

\begin{list2}
\item Installing Debian Linux went \emph{okay} -- little bit of disable secure boot, RAID/AHCI settings, ...
\item After install -- tuning and enabling Hugepages
\item Then adding a cable between the two ports on a dual card
\item Clone the repository \link{https://github.com/emmericp/MoonGen} build and run
\item {\bf Note: the full 14.8Mpps is done using a single core!}
\end{list2}

\slide{Turn down as you please}

\begin{alltt}\footnotesize
root@penguin01:~/projects/MoonGen# ./build/MoonGen ./examples/l3-tcp-syn-flood.lua 0 -r 5000 -d 192.0.2.138
[INFO]  Initializing DPDK. This will take a few seconds...
EAL: Detected 16 lcore(s)
[INFO]  Found 1 usable devices:
   Device 0: 00:25:90:32:9F:F3 (Intel Corporation 82599ES 10-Gigabit SFI/SFP+ Network Connection)
PMD: ixgbe_dev_link_status_print():  Port 0: Link Down
[INFO]  Device 0 (00:25:90:32:9F:F3) is up: 10000 MBit/s
[INFO]  Detected an IPv4 address.

[Device: id=0] TX: 9.77 Mpps, 5000 Mbit/s (6562 Mbit/s with framing)
[Device: id=0] TX: 9.68 Mpps, 4955 Mbit/s (6504 Mbit/s with framing)
[Device: id=0] TX: 9.77 Mpps, 5000 Mbit/s (6562 Mbit/s with framing)
[Device: id=0] TX: 9.77 Mpps, 5000 Mbit/s (6562 Mbit/s with framing)
\end{alltt}

IPv6 and UDP, replace tcp with udp in example:
\begin{alltt}\footnotesize
./build/MoonGen ./examples/l3-tcp-syn-flood.lua 0 -r 5000 -d 2001:DB8:ABCD:0053::138 -i 2001:DB8:ABCD:0053::1

./build/MoonGen ./examples/l3-udp-flood-hlk.lua 0 -r 5000 -d 2001:DB8:ABCD:0053::138 -i 2001:DB8:ABCD:0053::1
\end{alltt}




\slide{DDoS and network attacks}

\hlkrightimage{15cm}{network-layers-1.png}
.
\begin{list1}
\item You really should try testing
\item Investigate your existing devices\\
all of them, RTFM, upgrade firmware
\item Choose which devices does which\\
part - discard early to free resources\\
for later devices to dig deeper
\end{list1}

\vskip 3cm
\centerline{And dont forget that DDoS testing is as much a firedrill for the organisation}


\slide{Fundamentet skal være iorden}

\begin{list1}
\item Sørg for at den infrastruktur som I bygger på er sikker:
\begin{list2}
 \item redundans
       \item opdateret
        \item dokumenteret
        \item nem at vedligeholde
\end{list2}

\item  Husk tilgængelighed er også en sikkerhedsparameter
\end{list1}


\slide{Hardened network device configurations}

\begin{list1}
\item Alle services skal være konfigureret korrekt:
\begin{list2}
\item Administration kun fra jump host og egne administrator netværk, SSH og HTTPS
\item Alle protokoller med mulighed for \emph{secrets} bør evalueres for om det skal benyttes
\item Protokoller som BGP skal benytte route import filtre
\item Protokoller som OSPF bør benytte policies OG secrets
\item Brug, Router protect filter således at kun relevant adgang tillades til services på udstyret
\item Brug Reverse Path Forwarding uRPF / RPF
\end{list2}
\end{list1}



\slide{Check you network from outside}

\begin{list1}
\item How does your network look like from the outside?
\item Check your network using:
\item \link{https://stat.ripe.net/}
\item Consider:
\begin{list2}
  \item Join the NLNOG RING \link{https://ring.nlnog.net/}
\item \link{https://bgpmon.net/} - commercial tool, some alternatives exist
\item \link{https://shadowserver.org/wiki/} sign up for Shadowserver - ASN \& Netblock Alerting \& Reporting Service
\end{list2}
\end{list1}

\slide{For Next Time}

\hlkimage{2cm}{clipboard_01.png}

\begin{list1}
\item Think about the subjects from this time, write down questions

\item Next time is exam preparation
\item All books should be read by now
\item Visit web sites and download papers if needed
\item Retry the exercises to get more confident using the tools
\end{list1}

\end{document}
