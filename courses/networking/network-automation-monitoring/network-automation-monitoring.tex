\documentclass[Screen16to9,17pt]{foils}
\usepackage{zencurity-slides}

\externaldocument{communication-and-network-security-exercises}
\selectlanguage{english}

\begin{document}

\mytitlepage
{Network Automation and Monitoring Basics}
{How to manage networks}

\hlkprofiluk


\slide{Goals: Network Automation and Monitoring Basics}

\hlkimage{17cm}{ddos-after-filtering.png}

\begin{list2}
\item Introduce network management suitable for modern times
\item Introduce resources, programs, people, authors, documents, sites\\
 that further your exploration into network management
\item Starting with manual tools, move to some excellent automated ones
\item I recommend open source tools, feel free to try these and then decide to pay for commercial ones, if you like
\end{list2}



\slide{Plan for today}

A starter pack for network management

\begin{list2}
\item Network information TCP/IP
\item The basic tools for diagnosing network problems
\item Smokeping for automated monitoring
\item LibreNMS - a fully featured network monitoring system
\item Ansible for robust configuration of servere and other devices
\item Oxidized gathering network device configuration
%\item Challenges in network management
\item Syslog server - Logstash, Elasticsearch, Dashboards
\end{list2}

Duration: 4 hours - with breaks

Keywords:
IP address plans, core infrastructure, SNMP, DNS, DHCPD, , Netflow, dashboards, TCP, UDP, ICMP, routing, switching, a little BGP, Ansible for network devices with Junos, OpenBSD and Linux as examples, Netflow, sFlow, monitoring systems, LibreNMS, Oxidized, Smokeping


\slide{Time schedule}

\begin{list2}
\item 17:00 - 18:15\\
Introduction and basic manual tools
\item 30min break\\

\item 18:45 - 19:30 45min \\
Recommended tools, Nipap, Smokeping, LibreNMS
\item 15min break\\

\item 19:45 - 21:00\\
Centralized solutions, tie together\\

with break somewhere
\end{list2}


\slide{About equipment and exercises}

\begin{list2}
\item Bringing a laptop to my courses is not required, but welcome
\item Links etc. are in the slides and open source licensed, PDFs
\item Exercises booklets are available for many of my courses, see Github\\
but it is expected that participants will do any exercises on their own later or at the scheduled hacker days
\item The hacker days will be announced in various places

\item Events like BornHack are excellent places to arrange hacker days in the network warrior village, or other places
\end{list2}

\vskip 1cm

\centerline{\LARGE Invite a few friends, make a hacker day and work together!}

\slide{Course Materials}

\begin{list1}
\item This material is in multiple parts:
\begin{list2}
%\item Introduktionsmateriale med baggrundsinformation
\item Slide shows - presentation - this file
\item Exercises - PDF files in my repository
\end{list2}
\item Links
\begin{list2}
\item All materials will be released as open source at:\\
\link{https://github.com/kramse/security-courses/}
\item Additional resources from the internet linked from lecture plans:\\
\link{https://zencurity.gitbook.io/kea-it-sikkerhed/}
\end{list2}
\end{list1}

Note: slides and materials will mostly be in english, but presentation language will be danish



\slide{Hackerlab Setup}

\hlkimage{7cm}{hacklab-1.png}

\begin{list2}
\item Hardware: modern laptop CPU with virtualisation\\
Dont forget to enable hardware virtualisation in the BIOS
\item Software Host OS: Windows, Mac, Linux
\item Virtualisation software: VMware, Virtual box, HyperV pick your poison
\item Hackersoftware: Kali Virtual Machine \link{https://www.kali.org/}
\end{list2}



\slide{Networking Hardware}

If you want to do exercises and work with networks \\
It will be an advantage to have a wireless USB network card and an USB Ethernet card.

The following are two models I have used:
\begin{list2}
\item TP-link TL-WN722N hardware version 2.0 cheap
\item Alfa AWUS036ACH 2.4GHz + 5GHz Dual-Band and high performing
\item Often you need to compile drivers yourself, and research a bit
\end{list2}

Getting an USB card allows you to use the regular one for the main OS, and insert the USB into the virtual machine



\slide{Network Management}

\begin{quote}
Network management is the process of administering and managing computer networks. Services provided by this discipline include fault analysis, performance management, provisioning of networks and maintaining the quality of service. Software that enables network administrators to perform their functions is called network management software.\\
\link{https://en.wikipedia.org/wiki/Network_management}
\end{quote}


\begin{list2}
\item What are we talking about today
\item Complex modern networks

\item Hint: easier to consider your network a critical resource, and start monitoring now
\end{list2}





\slide{Internet Today}

\hlkimage{10cm}{images/server-client.pdf}

\begin{list2}
\item Clients and servers, roots in the academic world
\item Protocols are old, some more than 20 years
\item Very little is encrypted, mostly HTTPS
\end{list2}


\slide{Internet is Open Standards!}

{\hlkbig \color{titlecolor}
We reject kings, presidents, and voting.\\
We believe in rough consensus and running code.\\
-- The IETF credo Dave Clark, 1992.}

\begin{list1}
\item Request for comments - RFC - er en serie af dokumenter
\item RFC, BCP, FYI, informational\\
de første stammer tilbage fra 1969
\item Ændres ikke, men får status Obsoleted når der udkommer en nyere
  version af en standard
\item Standards track:\\
Proposed Standard $\rightarrow$ Draft Standard $\rightarrow$ Standard
\item  Åbne standarder = åbenhed, ikke garanti for sikkerhed
\end{list1}


\slide{What is the Internet}

\begin{list1}
\item Communication between humans - currently!
\item Based on TCP/IP
\begin{list2}
\item best effort
\item packet switching (IPv6 calls it packets, not datagram)
\item \emph{connection-oriented} TCP
\item \emph{connection-less} UDP
\end{list2}
\end{list1}

RFC-1958:
\begin{quote}
 A good analogy for the development of the Internet is that of
 constantly renewing the individual streets and buildings of a city,
 rather than razing the city and rebuilding it. The architectural
 principles therefore aim to provide a framework for creating
 cooperation and standards, as a small "spanning set" of rules that
 generates a large, varied and evolving space of technology.
\end{quote}

\slide{Internet historisk set -  anno 1969}
\hlkimage{6cm}{1969_4-node_map.png}
%size 2

\begin{list2}
\item Node 1: University of California Los Angeles
\item Node 2: Stanford Research Institute
\item Node 3: University of California Santa Barbara
\item Node 4: University of Utah
%\item Kilde: \link{http://www.zakon.org/robert/internet/timeline/}
\end{list2}



\slide{A switch}

\hlkimage{8cm}{switch-1.pdf}

\begin{list1}
\item Today we use switches, Don't buy a hub, not even for experimenting or sniffing
\item A switch can receive and send data on multiple ports at the same time
\item Performance only limited by the backplane and switching chips
\item Can also often route with the same speed
\item Always buy managed switches with SNMP and IEEE 802.1q, even for home
\end{list1}


\slide{Topologier og Spanning Tree Protocol}

\hlkimage{13cm}{switch-STP.pdf}

See also Radia Perlman, \emph{Interconnections: Bridges, Routers, Switches, and Internetworking Protocols}

\slide{Core, Distribution og Access net}

\hlkimage{17cm}{core-dist.pdf}

\centerline{Not always this way - but often }


\slide{Bridges and routers}

\hlkimage{17cm}{wan-network.pdf}
\centerline{Often you inherit a mess}



\slide{OSI and Internet models}

\hlkimage{11cm,angle=90}{images/compare-osi-ip.pdf}

\slide{MAC address}
%\hlkimage{10cm}{apple-oui.png}

\begin{alltt}
00-03-93   (hex)        Apple Computer, Inc.
000393     (base 16)    Apple Computer, Inc.
                        20650 Valley Green Dr.
                        Cupertino CA 95014
                        UNITED STATES
\end{alltt}
\begin{list1}
\item Network technologies use a layer 2 hardware address
\item Typically using 48-bit MAC addresses known from Ethernet MAC-48/EUI-48
\item First half is assigned to companies -- Organizationally Unique Identifier (OUI)
\item Using the OUI you can see which producer and roughly when a network chip was produced
\item \link{http://standards.ieee.org/regauth/oui/index.shtml}
\end{list1}


\slide{Common Address Space}

\vskip 2 cm
\hlkimage{13cm}{IP-address.pdf}

\begin{list2}
\item Internet is defined by the address space, one
\item Based on 32-bit addresses, example dotted decimal format 10.0.0.1
\end{list2}


\slide{CIDR Classless Inter-Domain Routing}

\hlkimage{13cm}{CIDR-aggregation.pdf}

\begin{list2}
\item Subnet mask originally inferred by the class
\item Started to allocate multiple C-class networks - save remaining B-class\\
Resulted in routing table explosion
\item A subnet mask today is a row of 1-bit
\item 10.0.0.0/24 means the network 10.0.0.0 with subnet mask 255.255.255.0
\item Supernet, supernetting
\end{list2}


\slide{Preparing an IPv6
Addressing Plan}

\hlkimage{12cm}{ipv6-address-plan-ripe.png}

{\footnotesize \link{http://www.ripe.net/training/material/IPv6-for-LIRs-Training-Course/IPv6_addr_plan4.pdf}}

See also \link{https://blog.apnic.net/2019/08/22/how-to-ipv6-address-planning/}


\slide{Example adress plan input}

\hlkimage{16cm}{ipv6-linked-to-ipv4.png}

\centerline{Easy and coupled with VLAN IDs it will work \smiley}



\slide{IP Address Management IPAM }

\hlkimage{16cm}{nipap-search.png}
\begin{list2}
\item Recommend Nipap \link{http://spritelink.github.io/NIPAP/}
\end{list2}



\slide{NIPAP example, adding firewall cluster interfaces - multiple VLANs}

%\hlkimage{}{}

\begin{minted}[fontsize=\small]{shell}
#! /bin/sh
# Add sites to NIPAP,  dont waste time doing it with mouse
addsite ()
{  SITE=$1
   SITEID=$2
  for VLAN in 100 200 300 400 500
  do
    nipap address add family ipv4 prefix 172.$SITEID.$VLAN.1 node $SITE-fw.example.net
    nipap address add family ipv4 prefix 172.$SITEID.$VLAN.2 node $SITE-fw-01.example.net
    nipap address add family ipv4 prefix 172.$SITEID.$VLAN.3 node $SITE-fw-02.example.net
  done
}

addsite dk-odense 1231
addsite dk-svendborg 1232
\end{minted}

\begin{list2}
  \item Automating saves time, and less errors from manual input!
\end{list2}

\slide{UDP User Datagram Protocol}
\hlkimage{14cm}{udp-1.pdf}
\begin{list1}
\item Connection-less RFC-768, \emph{connection-less}
\item Used for Domain Name Service (DNS)
\end{list1}

\slide{TCP Transmission Control Protocol}
\hlkimage{12cm}{tcp-1.pdf}

\begin{list1}
\item Connection oriented RFC-791 September 1981, \emph{connection-oriented}
\item Either data delivered in correct order, no data missing, checksum or an error is reported
\item Used for HTTP and others
\end{list1}


\slide{Well-known port numbers}

\hlkimage{6cm}{iana1.jpg}

\begin{list1}
\item IANA maintains a list of magical numbers in TCP/IP
\item Lists of protocl numbers, port numers etc.
\item A few notable examples:
\begin{list2}
\item Port 25/tcp Simple Mail Transfer Protocol (SMTP)
\item Port 53/udp and 53/tcp Domain Name System (DNS)
\item Port 80/tcp Hyper Text Transfer Protocol (HTTP)
\item Port 443/tcp HTTP over TLS/SSL (HTTPS)
\end{list2}
\item Source: \link{http://www.iana.org}
\end{list1}



\slide{ICMP Internet Control Message Protocol}

\begin{list1}
\item Control protocol, error messages
\item Common messages
\begin{list2}
\item ICMP ECHO, anyone there?
\item Host unreachable
\item Port unreachable
\end{list2}
\item \emph{signaling}
\item Defined in RFC-792
\end{list1}

\centerline{\bf Don't block all ICMP -- wrong!}

\slide{ICMP messages to allow -- Similar for IPv6}

\begin{list1}
\item Type
\begin{list2}
\item 0 = net unreachable;
\item 1 = host unreachable;
\item 2 = protocol unreachable;
\item 3 = port unreachable;
\item 4 = fragmentation needed and DF set;
\item 5 = source route failed.
\end{list2}
\item If you remove all ICMP you will have time outs instead
\item Allow these ICMP types:
\begin{list2}
\item 3 Destination Unreachable
\item 4 Source Quench Message
\item 11 Time Exceeded
\item 12 Parameter Problem Message
\end{list2}
\end{list1}


\slide{ IPv6 neighbor discovery protocol (NDP)}

\hlkimage{18cm}{ipv6-arp-ndp.pdf}

\begin{list1}
\item Address Resolution Protocol (ARP) is gone from IPv6
\item NDP replaces and expand, command to use \verb+arp -an+ replaced by \verb+ndp -an+
\end{list1}

\slide{ARP vs NDP}

\begin{alltt}
\small
hlk@bigfoot:basic-ipv6-new$ arp -an
? (10.0.42.1) at{\bf 0:0:24:c8:b2:4c} on en1 [ethernet]
? (10.0.42.2) at 0:c0:b7:6c:19:b on en1 [ethernet]
hlk@bigfoot:basic-ipv6-new$ ndp -an
Neighbor                      Linklayer Address  Netif Expire    St Flgs Prbs
::1                           (incomplete)         lo0 permanent R
2001:16d8:ffd2:cf0f:21c:b3ff:fec4:e1b6 0:1c:b3:c4:e1:b6 en1 permanent R
fe80::1%lo0                   (incomplete)         lo0 permanent R
fe80::200:24ff:fec8:b24c%en1 {\bf 0:0:24:c8:b2:4c}      en1 8h54m51s  S  R
fe80::21c:b3ff:fec4:e1b6%en1  0:1c:b3:c4:e1:b6     en1 permanent R
\end{alltt}



\slide{Basic test tools TCP - Ping and Traceroute}

We should all know
\begin{list2}
\item \verb+ping+ -- like sending a radar ping, anything there
\item \verb+traceroute+ (windows tracert) -- find the route packets traverse
\end{list2}

and add these!
\begin{list2}
\item \verb+Wireshark+ -- like sending a radar ping, anything there
\item \verb+Nmap+ and \verb+Nping+ -- port scan and advanced ping program!
\end{list2}


\slide{Ping}

\begin{alltt}
\small {\bfseries
$ ping 192.168.1.1}
PING 192.168.1.1 (192.168.1.1): 56 data bytes
64 bytes from 192.168.1.1: icmp_seq=0 ttl=150 time=8.849 ms
64 bytes from 192.168.1.1: icmp_seq=1 ttl=150 time=0.588 ms
64 bytes from 192.168.1.1: icmp_seq=2 ttl=150 time=0.553 ms
\end{alltt}


\begin{list1}
\item ICMP -- Internet Control Message Protocol
\item ECHO -- only ICMP message that generates another
\item ICMP ECHO request sent, and ICMP ECHO reply expected
\item Same with IPv6, \verb+ping6+
\end{list1}



\slide{traceroute}

\begin{alltt}
\small{\bfseries
hkj@bob:~$ traceroute www.kramse.dk}
traceroute to www.kramse.dk (185.129.60.130), 30 hops max, 60 byte packets
 1  10.0.42.1 (10.0.42.1)  0.365 ms  0.277 ms  0.239 ms
 2  79.142.xxx.xxx (79.142.xxx.xxx)  5.174 ms  4.979 ms  5.113 ms
 3  bgp2-dix.prod.bolignet.dk (79.142.224.2)  5.538 ms  5.057 ms  5.483 ms
 4  217.74.215.57 (217.74.215.57)  5.990 ms  5.962 ms  5.932 ms
...
 8  185.150.199.178 (185.150.199.178)  7.684 ms  7.647 ms  4.627 ms
 9  * * * // firewall here!
\end{alltt}

\begin{list1}
\item Works using the Time to live (TTL) counter
\item Sending with $TTL=1$ returns ICMP from first host/router
\item Default sends UDP on Unix, and ICMP on Windows
\item Kali has programs that can emulate, or send using any protocol
\end{list1}

\slide{Wireshark - graphical network sniffer}

\hlkimage{18cm}{images/wireshark-website.png}

\centerline{\link{http://www.wireshark.org}}

\slide{Using Wireshark}

\hlkimage{13cm}{images/wireshark-http.png}

\centerline{Capture - Options, select a network interface}

\slide{Detailed view of network traffic with Wireshark}

\hlkimage{10cm}{images/wireshark-sni-twitter.png}

\centerline{Notice also the filtering possibilities, capture and view}





\slide{Remote network debugging}

\begin{list2}
\item TShark and Tcpdump, I often use: \verb+tcpdump –nei eth0+\\
\verb+tshark -z expert -r download-slow.pcapng+

\item Remote packet dumps, \verb+tcpdump –i eth0 –w packets.pcap+

\item Story: tcpdump was originally written in 1988 by Van Jacobson, Sally Floyd, Vern Paxson and Steven McCanne who were, at the time, working in the Lawrence Berkeley Laboratory Network Research Group\\
 \link{https://en.wikipedia.org/wiki/Tcpdump}
\end{list2}

\vskip 5mm

\centerline{\Large Great network security comes from knowing networks!}





\slide{Note About Hardware IPv4 checksum offloading}

\begin{list1}
\item IPv4 checksum must be calculated for every packet received
\item IPv4 checksum must be calculated for every packet sent\\
Usually on a router the Time To Live is decremented, to need re-calculation
\vskip 1cm
\item Let an ASIC chip on the network card do the work!
\item Most server network chips today support this and more
\item Benefit for performance, but beware when using security tools
\item If every packet in wireshark has wrong checksum, its the network card doing it
\item Can be turned off, when doing security work
\end{list1}
\vskip 1cm


\slide{Nping}

\begin{alltt}\footnotesize
  root@KaliVM:~# nping --tcp -p 80 www.zencurity.com

  Starting Nping 0.7.70 ( https://nmap.org/nping ) at 2018-09-07 19:06 CEST
  SENT (0.0300s) TCP 10.137.0.24:3805 > 185.129.60.130:80 S ttl=64 id=18933 iplen=40  seq=2984847972 win=1480
  RCVD (0.0353s) TCP 185.129.60.130:80 > 10.137.0.24:3805 SA ttl=56 id=49674 iplen=44  seq=3654597698 win=16384 <mss 1460>
  SENT (1.0305s) TCP 10.137.0.24:3805 > 185.129.60.130:80 S ttl=64 id=18933 iplen=40  seq=2984847972 win=1480
  RCVD (1.0391s) TCP 185.129.60.130:80 > 10.137.0.24:3805 SA ttl=56 id=50237 iplen=44  seq=2347926491 win=16384 <mss 1460>
  SENT (2.0325s) TCP 10.137.0.24:3805 > 185.129.60.130:80 S ttl=64 id=18933 iplen=40  seq=2984847972 win=1480
  RCVD (2.0724s) TCP 185.129.60.130:80 > 10.137.0.24:3805 SA ttl=56 id=9842 iplen=44  seq=2355974413 win=16384 <mss 1460>
  SENT (3.0340s) TCP 10.137.0.24:3805 > 185.129.60.130:80 S ttl=64 id=18933 iplen=40  seq=2984847972 win=1480
  RCVD (3.0387s) TCP 185.129.60.130:80 > 10.137.0.24:3805 SA ttl=56 id=1836 iplen=44  seq=3230085295 win=16384 <mss 1460>
  SENT (4.0362s) TCP 10.137.0.24:3805 > 185.129.60.130:80 S ttl=64 id=18933 iplen=40  seq=2984847972 win=1480
  RCVD (4.0549s) TCP 185.129.60.130:80 > 10.137.0.24:3805 SA ttl=56 id=62226 iplen=44  seq=3033492220 win=16384 <mss 1460>

  Max rtt: 40.044ms | Min rtt: 4.677ms | Avg rtt: 15.398ms
  Raw packets sent: 5 (200B) | Rcvd: 5 (220B) | Lost: 0 (0.00%)
  Nping done: 1 IP address pinged in 4.07 seconds
\end{alltt}

\begin{list2}
\item The Nmap portscanner includes a nice ping utility, that allows us to \emph{ping} with TCP
\end{list2}






\slide{Challenges in network management}

\hlkimage{14cm}{dragon-drawing-6.jpg}

\vskip 2cm
\centerline{\Large Internet here be dragons}

We will jump directly into some solutions


\slide{The basic tools for monitoring network}


\centerline{\LARGE Moving from manual checking to automated}


\begin{list2}
\item Wireshark is advanced manual tool, try right-clicking different places
\item How do we continously monitor the network?
\end{list2}



\slide{Smokeping packet loss}

\hlkimage{11cm}{images/smokeping-packet-loss1}

Old skool, but very usefull \link{https://oss.oetiker.ch/smokeping/}

\slide{Smokeping latency changed}

\hlkimage{11cm}{images/smokeping-latency-change.png}




\slide{NTP Network Time Protocol}

{~}
\hlkrightpic{5cm}{0cm}{images/xclock.pdf}

\begin{list1}
\item Vigtigt at netværksenheder bruger korrekt tid, sikkerhed og drift
\item Server NTP foregår typisk i \verb+/etc/ntp.conf+ eller \verb+/etc/ntpd.conf+
\item det vigtigste er navnet på den/de servere man vil bruge som tidskilde
\item Brug enten en NTP server hos din udbyder eller en fra \link{http://www.pool.ntp.org/}
\item Eksempelvis:
\end{list1}

\begin{alltt}
server 0.dk.pool.ntp.org
server 0.europe.pool.ntp.org
server 3.europe.pool.ntp.org

\end{alltt}





\slide{SNMP and management}

\begin{list1}
\item Often we see devices in the network configured using HTTP, Telnet eller SSH
\begin{list2}
\item Where is the configuration stored?
\item Is the device functioning?
\end{list2}
\item Today we can use automation like:
\begin{list2}
\item (old RANCID \link{http://www.shrubbery.net/rancid/})
\item Ansible \link{https://www.ansible.com/}
\item Python with SSH libraries
\item Oxidized is a network device configuration backup tool. RANCID replacement!
\link{https://github.com/ytti/oxidized}
\end{list2}
\end{list1}

\slide{SNMP output -- data from devices}

%\hlkimage{}{}

\begin{minted}[fontsize=\small]{shell}
  snmpwalk -v2c -c public 172.16.1.2
  SNMPv2-MIB::sysDescr.0 = STRING: PX2 020200
  SNMPv2-MIB::sysObjectID.0 = OID: SNMPv2-SMI::enterprises.13742.6
  snmpwalk -v2c -c public 172.16.1.2 .1.3.6.1.4.1 | grep site
  SNMPv2-SMI::enterprises.13742.6.3.2.2.1.13.1 = STRING: "site-pdu-rack-a-27"
  SNMPv2-SMI::enterprises.13742.6.3.5.3.1.3.1.1 = STRING: "switch site-dist-sw-02"
  SNMPv2-SMI::enterprises.13742.6.3.5.3.1.3.1.11 = STRING: "router site-mx-01"
  snmpwalk -v2c -c public 172.16.1.2 SNMPv2-SMI::enterprises.13742.6.3.5.3.1.3.1
  Giver hele tabellen fra den PDU - så nu kan vi smide det data videre :-)
\end{minted}

\begin{list2}
  \item Typical values, packet count on ports, errors observed, speed, firmware, versions, 
\end{list2}

\slide{SNMP version 2 vs version 3}

Use Simple Network Management Protocol, but
\begin{list2}
\item SNMP versions 1 and 2c are insecure
\item SNMP version 3 created to fix this
\item Authenticity and integrity: Keys are used for
users and messages have digital signatures
generated with a hash function (MD5 or SHA)
\item Privacy: Messages can be encrypted with
secret-key (private) algorithms
\end{list2}

Example for Juniper can be found at:\\
{\small\link{https://www.juniper.net/documentation/en_US/junos/topics/example/snmpv3-configuration-junos-nm.html}}



\slide{Config example: SNMP}

\begin{alltt}
snmp \{
    description "SW-CPH-02";
    location "Teknikrum, Graested, Denmark";
    contact "noc@zencurity.com";
    community yourcommunitynotmine \{
        authorization read-only;
        clients \{
            10.1.1.1/32;
            10.1.2.2/32;
        \}
    \}
\}
\end{alltt}

SNMPv2 example

\slide{RANCID output}


\hlkimage{15cm}{images/rancid-email.png}



\slide{Step 1: configure devices properly}

\begin{slidelist}
\item You should always configure your devices properly
\item Use Secure Shell (SSH)
\item Turn on SNMP, probably SNMPv3
\item Turn on LLDP Link Layer Discovery Protocol -- vendor-neutral\\
{\small\link{http://en.wikipedia.org/wiki/Link_Layer_Discovery_Protocol}}
\item Configure centralized syslog
\vskip 1 cm
\item And updated firmware, HTTPS and SSH only etc. the usual stuff
\end{slidelist}

Then use Oxidized and LibreNMS


\slide{Oxidized}

%\hlkimage{}{}

\begin{alltt}
  gem install oxidized
  gem install oxidized-script oxidized-web
\end{alltt}

\begin{list2}
\item Oxidized is written in Ruby as a replacement for RANCID (Perl)
\item Very easy to get running
\item Fetches and imports configurations into Git repository, or others
\item RESTful API and Web Interface, and command line tools
\item \link{https://github.com/ytti/oxidized}
\end{list2}

\slide{Oxidized configuration}


\begin{alltt}
\verb+~/.config/oxidized/router.db+
  router01.example.com:ios
  switch01.example.com:procurve
  router02.example.com:ios
\end{alltt}


\begin{list1}
  \item Easy to configure new device types, added Clavister firewalls once
\end{list1}




\slide{Config example: Dell switch LLDP}
%\hlkrightimage{8cm}{images/lldp-dell-8024f.png}

\begin{alltt}\small
interface ethernet 1/xg17
mtu 9216
lldp transmit-tlv port-desc sys-name sys-desc sys-cap
lldp transmit-mgmt
exit
\end{alltt}



\slide{LLDP trick using tcpdump}

\begin{alltt}\footnotesize
[hlk@ljh ~]$ sudo tcpdump -i eth0 ether proto 0x88cc
tcpdump: verbose output suppressed, use -v or -vv for full protocol decode
listening on eth0, link-type EN10MB (Ethernet), capture size 96 bytes
.... wait for it ....
11:03:55.395064 00:1c:23:80:49:ff (oui Unknown) > 01:80:c2:00:00:0e (oui Unknown),
ethertype Unknown (0x88cc), length 60:
	0x0000:  0207 0400 1c23 8049 fd04 0705 312f 302f  .....#.I....{\bf{}1/0/}
	0x0010:  3331 0602 0078 0000 0000 0000 0000 0000  {\bf 31}...x..........
	0x0020:  0000 0000 0000 0000 0000 0000 0000       ..............

1 packets captured
3 packets received by filter
0 packets dropped by kernel
\end{alltt}

\vskip 2 cm
\centerline{I know {\bf for sure} that this server is in Unit 1 port 31!}



\slide{LibreNMS  a fully featured network monitoring system }

\begin{list2}
\item Homepage: \link{https://www.librenms.org/}
\item {\bf Suggested method:}
\link{https://docs.librenms.org/Installation/}
\item How basic information about devices are presented, from devices when added - and nothing more.\\
See how to add a device and add your own. \link{https://docs.librenms.org/Support/Adding-a-Device/}
\item How SNMP location is used to categorize devices and provice maps, see\\
\link{https://docs.librenms.org/Extensions/World-Map/}
\item How protocols like LLDP allow LibreNMS to make maps, see\\
 \link{https://docs.librenms.org/Extensions/Network-Map/}
\item How port description can be used for describing ports\\
 \link{https://docs.librenms.org/Extensions/Interface-Description-Parsing/}
\end{list2}

Most of this happens with very little effort. Just configure devices consistently and they will be presented nicely.



\slide{LibreNMS Automatic discovery}

\hlkimage{12cm}{librenms-switches.png}

Automatically discover your entire network using CDP, FDP, LLDP,
OSPF, BGP, SNMP and ARP.

\slide{LibreNMS Geo Location}
\hlkimage{12cm}{librenms-geo-location.png}

\slide{LLDP spaghetti?}
\hlkimage{\textwidth-3cm}{images/lldp-mess-censor.png}

\centerline{LLDP is needed!}

\slide{LibreNMS wireless clients}
\hlkimage{20cm}{images/librenms-wireless-clients.png}



\slide{Centralized management SSH, Jump hosts}

\begin{quote}
A jump server, jump host or jumpbox is a computer on a network used to access and manage devices in a separate security zone. The most common example is managing a host in a DMZ from trusted networks or computers.
\end{quote}

\link{https://en.wikipedia.org/wiki/Jump_server}

Advantage, you can configure this server/system to only allow Key based logins -- eliminate the possibility for brute-force attacks succeeding

\slide{OpenSSH client config with jump host}

My recommended SSH client settings, put in \verb+$HOME/.ssh/config+:
\begin{alltt}\footnotesize
Host *
    ServerAliveInterval=30
    ServerAliveCountMax=30
    NoHostAuthenticationForLocalhost yes
    HashKnownHosts yes
    UseRoaming no

Host jump-01
  Hostname 10.1.2.3
  Port 12345678

Host fw-site-01 10.1.2.5
  User hlk
  Port 34
  Hostname 10.1.2.5
  ProxyCommand ssh -q -a -x jump-01 -W %h:%p
\end{alltt}

I configure fw using both hostname and IP,\\
then I can use name, and any program using IP get this config too



\slide{Ansible}

\hlkimage{2cm}{Ansible_logo.png}

\begin{quote}
From my course materials:\\
Ansible is great for automating stuff, so by running the playbooks we can get a whole lot of programs installed, files modified - avoiding the Vi editor.
\end{quote}

\begin{list2}
\item Easy to read, even if you don't know much about YAML
\item \link{https://www.ansible.com/} and \link{https://en.wikipedia.org/wiki/Ansible_(software)}
\item Great documentation\\
 \link{https://docs.ansible.com/ansible/latest/collections/ansible/builtin/apt_module.html}
\end{list2}


\slide{Ansible Dependencies}

\hlkimage{10cm}{python-logo.png}

\begin{list2}
\item Ansible based on Python, only need Python installed\\
\link{https://www.python.org/}
\item Often you use Secure Shell for connecting to servers\\
\link{https://www.openssh.com/}
\item Easy to configure SSH keys, for secure connections
\end{list2}


\slide{How Ansible Works: inventory files}

List your hosts in one or multiple text files:
\begin{alltt}\footnotesize
[all:vars]
ansible_ssh_port=34443

[office]
fw-01 ansible_ssh_host=192.168.1.1 ansible_ssh_port=22
ansible_python_interpreter=/usr/local/bin/python

[infrastructure]
smtp-01     ansible_ssh_host=192.0.2.10
ansible_python_interpreter=/usr/local/bin/python
vpnmon-01   ansible_ssh_host=10.50.60.18

\end{alltt}

\begin{list2}
\item Inventory files specify the hosts we work with
\item Linux and OpenBSD servers shown here
\item Real inventory for a site with development and staging may be 500 lines
\item office and infrastructure are group names
\end{list2}


\slide{How Ansible Works: ad hoc commands }

Using the inventory file you can run commands with Ansible:

\begin{alltt}\footnotesize
  ansible -m ping new-server
  ansible -a "date" new-server
  ansible -m shell -a "grep a /etc/something" new-server
\end{alltt}

\begin{list2}
\item Running commands on multiple servers is easy now
\item This alone has value, you can start
\item Checking settings on servers
\item Making small changes to servers
\end{list2}


\slide{How Ansible Works: Playbooks}

The benefit comes with tasks listed in playbooks -- example playbook content, installing software using APT:
\begin{alltt}\small
apt:
    name: "\{\{ packages \}\}"
    vars:
      packages:
        - nmap
        - curl
        - iperf
        ...
\end{alltt}

Running it:
\begin{minted}[fontsize=\small]{shell}
cd kramse-labs/suricatazeek
ansible-playbook -v 1-dependencies.yml 2-suricatazeek.yml 3-elasticstack.yml
\end{minted}

"YAML (a recursive acronym for "YAML Ain't Markup Language") is a human-readable data-serialization language."\\
\link{https://en.wikipedia.org/wiki/YAML}



\slide{Python and YAML -- Git}

\hlkimage{7cm}{git-logo.png}

\begin{list2}
\item We need to store configurations
\item Run playbooks
\item Problem: Remember what we did, when, how
\item Solution: use git for the playbooks
\item Not the only version control system, but my preferred one
\end{list2}


\slide{How Ansible Works: typical execution}

\begin{alltt}\footnotesize
ansible-playbook -i hosts.cph1 -K infrastructure-firewalls.yml -t pf.conf --check --diff

ansible-playbook -i hosts.cph1 -K infrastructure-firewalls.yml -t pf.conf

ansible-playbook -i hosts.cph1 -K infrastructure-nagios.yml -t config-only

ansible-playbook -i smartboxes -K create-pf-conf.yml -l smartbox-xxx-01
\end{alltt}

\begin{list2}
\item Pro tip: check before you push out changes to production networks \smiley
\item Check will see if something needs changing
\item Diff will show the changes about to be made
\item Having configuration in Git improves things a lot!
\end{list2}




\slide{Get ready, Up and running with Ansible}

Prequisites for Ansible - you need a Linux machine:
\begin{list2}
\item python language - Ansible uses this
\item ssh keys - remote login without passwords
\item Sudo - allow regular users to do superuser tasks
\item Recommended tool: \verb+ssh-copy-id+ for getting your key on new server
\item Recommended Change: \verb+sshd_config+ - no passwords allowed, no bruteforce
\item Recommended to use: jump hosts/ProxyCommand in \verb+ssh_config+
\item Highly recommended: Git and/or github for version control
\end{list2}

Official docs:\\
\link{https://docs.ansible.com/ansible/intro_installation.html}



\slide{Ansible configuration management}

\begin{alltt}\small
- apt: name={{ item }} state=latest
  with_items:
        - unzip
        - elasticsearch
        - logstash
        - redis-server
        - nginx
- lineinfile: "dest=/etc/elasticsearch/elasticsearch.yml state=present
  regexp='network.host: localhost' line='network.host: localhost'"
- name: Move elasticsearch data into /data
  command: creates=/data/elasticsearch mv /var/lib/elasticsearch /data/
- name: Make link to /data/elasticsearch
  file: state=link src=/data/elasticsearch path=/var/lib/elasticsearch
\end{alltt}

Example playbooks:\\
\verb+git clone +\link{https://github.com/kramse/ansible-workshop}


\slide{Ansible and Junos -- Juniper Networks devices}

%\hlkimage{}{}

\begin{quote}
  Juniper Networks provides support for using Ansible to manage devices running Junos OS. Starting in Ansible Release 2.1, Ansible natively includes a number of core modules that can be used to manage devices running Junos OS. In addition, Juniper Networks provides the Juniper.junos Ansible role, which is hosted on the Ansible Galaxy website and includes additional modules for Junos OS.
\end{quote}
Source:{\footnotesize
\link{https://www.juniper.net/documentation/en_US/junos-ansible/topics/task/installation/junos-ansible-server-installing.html}}

\begin{list2}
\item Minimal configuration changes: \verb+user@host# set netconf ssh+
\end{list2}

How cool is that, centralized configuration management with source code!


\slide{Logging}

\centerline{\LARGE Logging}

\begin{quote}

\end{quote}

\begin{list2}
\item The world needs answers
\item You can only present answers, if you got them

\item Using logging you can investigate what happened
\end{list2}




\slide{Netflow -- Lets move to detailed monitoring and logging}

\begin{slidelist}
\item Cisco standard NetFlow version 5 defines a flow as a unidirectional sequence of packets that all share the following 7 values:
\begin{list2}
\item Ingress interface (SNMP ifIndex), IP protocol, Source IP address and Destination IP address
\item Source port for UDP or TCP, 0 for other protocols, IP Type of Service
\item Destination port for UDP or TCP, type and code for ICMP, or 0 for other protocols
\end{list2}
\item Today we can use Netflow version 9 or IPFIX with more fields available, or  sFlow, short for "sampled flow", is an industry standard for packet export at Layer 2 of the OSI model, \\
\link{https://en.wikipedia.org/wiki/SFlow}
\item NFSen is an old but free application
\link{http://nfsen.sourceforge.net/}
\end{slidelist}

Source: \\{\footnotesize
\link{https://en.wikipedia.org/wiki/NetFlow}\\
\link{https://en.wikipedia.org/wiki/IP_Flow_Information_Export}}

\slide{Netflow using NFSen}

\hlkimage{13cm}{images/nfsen-overview.png}


\slide{ Netflow NFSen}

\hlkimage{17cm}{nfsen-udp-flood.png}

\centerline{An extra 100k packets per second from this netflow source (source is a router)}


\slide{Netflow processing from the web interface}

\hlkimage{12cm}{images/nfsen-processing-1.png}

\centerline{Bringing the power of the command line forward}


\slide{ElastiFlow}

\hlkimage{10cm}{elastiflow.png}

\begin{quote}
  ElastiFlow™ provides network flow data collection and visualization using the Elastic Stack (Elasticsearch, Logstash and Kibana). It supports Netflow v5/v9, sFlow and IPFIX flow types (1.x versions support only Netflow v5/v9).
\end{quote}
Source: Picture and text from \link{https://github.com/robcowart/elastiflow} \\
PS I havent tried it in real life, yet -- but a friend has





\slide{View data efficiently}

\hlkimage{10cm}{logstash-search.png}

\begin{list1}
\item View data by digging into it easily - must be fast
\item Logstash and Kibana are just examples, but use indexing to make it fast!
\item Other popular examples include Graylog and Grafana
\end{list1}

\slide{Big Data tools: Elasticsearch}

\hlkimage{10cm}{kibana-basics-with-vega.jpg}

Elasticsearch is an open source distributed, RESTful search and analytics engine capable of solving a growing number of use cases.

\link{https://www.elastic.co}

\vskip 1cm
\centerline{We are all Devops now, even security people!}


\slide{Kibana}

\hlkimage{12cm}{kibanascreenshothomepagebannerbigger.jpg}

\centerline{Highly recommended for a lot of data visualisation}

Non-programmers can create, save, and share dashboards

Source:
\link{https://www.elastic.co/products/kibana}


\slide{Architecture}

\hlkimage{17cm}{elastic-stack-buffered.png}
\begin{list2}
\item Real production environments often add some buffering in between
\item Allows the ingestion to become more smooth, no lost messages
\end{list2}



\slide{How to get started}

\begin{list1}
\item How to get started searching for security events?
\item Collect basic data from your devices and networks
\begin{list2}
\item Netflow data from routers
\item Session data from firewalls
\item Logging from applications: email, web, proxy systems
\end{list2}
\item {\bf Centralize!}
\item Process data
\begin{list2}
\item Top 10: interesting due to high frequency, occurs often, brute-force attacks
\item {\it ignore}
\item Bottom 10: least-frequent messages are interesting
\end{list2}
\end{list1}



\slide{SIEM and logging systems}

%\hlkimage{}{}

\begin{quote}\small
You could {\bf buy a bunch of expensive gear}, point it all to a {\bf log management} or a {\bf security incident and event management (SIEM)} system, and let it automatically tell you what it knows. Some incident response teams may start this way, but unfortunately, many never evolve. {\bf Working only with what the SIEM tells you}, versus what you have configured it to tell you {\bf based on your contextual data}, will invariably fail. Truly demonstrating {\bf value} from security monitoring and incident response {\bf requires a major effort}. As with all projects, {\bf planning} is the most important phase. {\bf Designing} an approach that works best for you requires significant effort up front, but offers a great payout later in the form of {\bf meaningful incident response and overall improved security}.
\end{quote}
Source: Crafting the InfoSec Playbook, 4 A Data-Centric Approach to Security Monitoring (bold by me)
by Jeff Bollinger, Brandon Enright, and Matthew Valites ISBN: 9781491949405

\begin{list2}
\item I recommend pre-filtering, see what noise your devices \emph{would send}\\
"Collecting only relevant data can have a direct impact on reducing costs as well."
\item Same is recommended in CIP page 50: Just the Facts
\item Normalization -- "Data normalization for
the purposes of log mining is the process by which a portion, or field, of a log event is
transformed into its canonical form."
\end{list2}

\slide{Metadata -- enrichment}

\hlkimage{10cm}{crafting-security-playbook-metadata.png}

Source: picture from Crafting the InfoSec Playbook, CIP

Metadata + Context

\slide{Summary, what to log}

\begin{list1}
\item CIP 7 Tools of the Trade, need to know NetFlow, DNS, and HTTP proxy logs in the real-world
\begin{list2}
\item Defense in Depth -- we will never catch everything
\item Log Management: The Security Event Data Warehouse
\item Intrusion Detection Isn’t Dead
\item DNS, the One True King -- Logging and analyzing DNS transactions, Blocking DNS requests or responses
\item HTTP Is the Platform: Web Proxies -- Web proxies allow you to solve additional security problems
\item Rolling Packet Capture -- In a perfect world, we would have full packet capture everywhere
\end{list2}
\end{list1}



\slide{Detection Capabilities}


Security incidents happen, but what happens. One of the actions to reduce impact of incidents are done in preparing for incidents.

\begin{itemize}
\item \emph{Preparation} for an attack, establish procedures and mechanisms for detecting and responding to attacks
\end{itemize}

Preparation will enable easy {\bf identification} of affected systems, better {\bf containment} which systems are likely to be infected, {\bf eradication} what happened -- how to do the {\bf eradication} and {\bf recovery}.

\slide{Strategy for implementing identification and detection}

We recommend that the following strategy is used for implementing identification and detection.

We have the following recommendations and actions points for logging:
\begin{enumerate}
\item[\faSquareO] Enable system logging from servers
\item[\faSquareO] Enable system logging from network devices
\item[\faSquareO] Centralize logging
\item[\faSquareO] Add search facilities and dashboards
\item[\faSquareO] Perform system audits manually or automatically
\item[\faSquareO] Setup notification and notification procedures
\end{enumerate}

\slide{Extended Sources}
When a basic logging infrastructure is setup, it can be expanded to increase coverage, by
adding more sources:

\begin{list2}
\item DNS query logging -- will enable multiple cases to be resolved, example malware identification and tracing, when was a malware domain queried, when was the first infection
\item Session data from Firewalls, Netflow -- traffic patterns can be investigated and both attacks and cases like exfiltration can likely be seen
\end{list2}

Hint: Take the sources available first, make a proof-of-concept, expand coverage

\slide{Data Analysis Skills}

\begin{quote}
Although we could spend an entire book creating an exhaustive list of skills needed to be a good security data scientist, this chapter covers the following skills/domains that a data scientist will benefit from
knowing within information security:
\begin{list2}
\item Domain expertise—Setting and maintaining a purpose to the analysis
\item Data management—Being able to prepare, store, and maintain data
\item Programming—The glue that connects data to analysis
\item Statistics—To learn from the data
\item Visualization—Communicating the results effectively
\end{list2}
It might be easy to label any one of these skills as the most important, but in reality, the whole is greater than the sum of its parts. Each of these contributes a significant and important piece to the workings of
security data science.
\end{quote}

Source: \emph{Data-Driven Security: Analysis, Visualization and Dashboards} Jay Jacobs, Bob Rudis\\
ISBN: 978-1-118-79372-5 February 2014 \url{https://datadrivensecurity.info/} - short DDS



\slide{Baseline Skills}

\begin{list2}\small
\item Threat-Centric Security, NSM, and the NSM Cycle
\item TCP/IP Protocols
\item Common Application Layer Protocols
\item Packet Analysis
\item Windows Architecture
\item Linux Architecture
\item Basic Data Parsing (BASH, Grep, SED, AWK, etc)
\item IDS Usage (Snort, Suricata, etc.)
\item Indicators of Compromise and IDS Signature Tuning
\item Open Source Intelligence Gathering
\item Basic Analytic Diagnostic Methods
\item Basic Malware Analysis
\end{list2}

Source: \emph{Applied Network Security Monitoring Collection, Detection, and Analysis}, Chris Sanders and Jason Smith



\slide{Automated packet sniffing tools}

\hlkimage{10cm}{zeek-ids.png}

\begin{list1}
\item Zeek -- Network Security Monitor {\footnotesize\url{https://zeek.org}}
\item Suricata -- open source, mature, fast and robust network threat detection {\footnotesize\url{https://suricata-ids.org/}}
\item ntopng -- High-speed web-based traffic analysis  {\footnotesize\url{https://www.ntop.org/}}
\item Maltrail -- Malicious traffic detection system {\footnotesize\url{https://github.com/stamparm/MalTrail}}
\end{list1}

Slide included as a reference for Network SECURITY Monitoring


\slide{Side note: Zeek Security Monitor handles formats differently}

Zeek has files formatted with a header:
\begin{alltt}\footnotesize
#fields ts      uid     id.orig_h       id.orig_p       id.resp_h       id.resp_p       proto   trans_id
        rtt     query   qclass  qclass_name     qtype   qtype_name      rcode   rcode_name      AA
        TC      RD      RA      Z       answers TTLs    rejected

1538982372.416180	CD12Dc1SpQm42QW4G3	10.xxx.0.145	57476	10.x.y.141	53	udp	20383
	0.045021	www.dr.dk	1	C_INTERNET	1	A	0	NOERROR	F	F	T	T	0
   www.dr.dk-v1.edgekey.net,e16198.b.akamaiedge.net,2.17.212.93	60.000000,20409.000000,20.000000	F
\end{alltt}

Note: this show ALL the fields captured and dissected by Zeek, there is a nice utility program bro-cut which can select specific fields:

\begin{alltt}\small
root@NMS-VM:/var/spool/bro/bro# cat dns.log | bro-cut -d ts query answers | grep dr.dk
2018-10-08T09:06:12+0200	www.dr.dk	www.dr.dk-v1.edgekey.net,e16198.b.akamaiedge.net,2.17.212.93
\end{alltt}

Can also just use JSON now via Filebeat



\slide{IEEE 802.1q}

\hlkimage{16cm}{vlan-8021q.pdf}

\begin{list1}
\item Lets introduce some more advanced things, now that configurations are saved -- easy to see changes
\item Using IEEE 802.1q  VLAN tagging on Ethernet frames
\item Virtual LAN, to pass from one to another, must use a router/firewall
\item Allows separation/segmentation and protects traffic from many security issues
\end{list1}


\slide{Port Security -- Rogue DHCP servers}

\begin{list1}
\item Common problem in networks is people connecting devices with DHCPD servers
\item In general make sure to segment networks
\item Start to use port security on switches, including DHCP snooping\\
\link{https://en.wikipedia.org/wiki/DHCP_snooping}
\end{list1}

\slide{Example port security}

\begin{alltt}\small
[edit ethernet-switching-options secure-access-port]
set interface ge-0/0/1 mac-limit 4

set interface ge-0/0/2 allowed-mac 00:05:85:3A:82:80
set interface ge-0/0/2 allowed-mac 00:05:85:3A:82:81
set interface ge-0/0/2 allowed-mac 00:05:85:3A:82:83
set interface ge-0/0/2 allowed-mac 00:05:85:3A:82:85
set interface ge-0/0/2 allowed-mac 00:05:85:3A:82:88
set interface ge-0/0/2 mac-limit 4

set interface ge-0/0/1 persistent-learning
set interface ge-0/0/8 dhcp-trusted
set vlan employee-vlan arp-inspection
set vlan employee-vlan examine-dhcp
set vlan employee-vlan mac-move-limit 5
\end{alltt}

Source: Overview of Port Security, Juniper\\ {\small\link{https://www.juniper.net/documentation/en_US/junos/topics/example/overview-port-security.html}}




\slide{BGP intro}

\begin{list1}
\item What is BGP Border Gateway Protocol
\item Dynamic routing protocol, BGPv4 used on whole internet
\item Networks identified using AS numbers ASNs
\item Autonomous System (AS) can be small or very big, world wide
\item BGP version 4 RFC-4271 uses TCP connections
\emph{peering}
\item \link{http://en.wikipedia.org/wiki/Border_Gateway_Protocol}
\end{list1}


\slide{Hosting and internet providers}

\hlkimage{17cm}{network-bgp-asn.png}

\begin{list2}
\item BGP networks are used for all of the Internet
\item New standards like Resource Public Key Infrastructure (RPKI) are underway
\end{list2}

\slide{Routing and BGP Solutions }

\begin{list2}
\item Filtrering, ingress / egress:\\
"reject external packets that claim to be from the local net"
\item See also Reverse Path forwarding \link{https://en.wikipedia.org/wiki/Reverse-path_forwarding}
\item Routers and routing protocols must be more skeptical\\
Routing filters implemented everywhere, auth on routing protocols OSPF/BGP etc.
\item Has been recommended for some years, but not done in all organisations
\item BGP routing Resource Public Key Infrastructure RPKI
\item BCP38 is RFC2827: \emph{Network Ingress Filtering: Defeating Denial of Service Attacks which employ IP Source Address Spoofing}\\
\link{http://www.bcp38.info/}
\item \emph{Mutually Agreed Norms for Routing Security}, \link{https://www.manrs.org/}
\end{list2}




 \slide{Exercise at home -- Your lab setup}

 \begin{list2}
 \item Go to GitHub, Find user Kramse, click through security-courses, courses, suricatazeek and download the PDF files for the slides and exercises:\\  {\footnotesize \url{https://github.com/kramse/security-courses/tree/master/courses/networking/suricatazeek-workshop}}

 \item Get the lab instructions, from\\ {\footnotesize\url{https://github.com/kramse/kramse-labs/tree/master/suricatazeek}}
 \end{list2}


\myquestionspage

\slide{Recommended further reading}

\begin{list2}
\item Campus Network Security: High Level Overview , Network Startup Resource Center
\link{https://nsrc.org/workshops/2018/myren-nsrc-cndo/networking/cndo/en/presentations/Campus_Security_Overview.pdf}

\item Campus Operations Best Current Practice, Network Startup Resource Center
\link{https://nsrc.org/workshops/2018/tenet-nsrc-cndo/networking/cndo/en/presentations/Campus_Operations_BCP.pdf}

\item Mutually Agreed Norms for Routing Security (MANRS)
\link{https://www.manrs.org/wp-content/uploads/2018/09/MANRS_PDF_Sep2016.pdf}

\item RFC2827: Network Ingress Filtering: Defeating Denial of Service Attacks
\link{https://tools.ietf.org/html/rfc2827}
\end{list2}

\slide{Book: Linux Basics for Hackers (LBhf)}

\hlkimage{6cm}{LinuxBasicsforHackers_cover-front.png}

\emph{Linux Basics for Hackers
Getting Started with Networking, Scripting, and Security in Kali}
by OccupyTheWeb
December 2018, 248 pp.
ISBN-13:
9781593278557

\link{https://nostarch.com/linuxbasicsforhackers}
Explains how to use Linux

\slide{Book: Kali Linux Revealed (KLR)}

\hlkimage{6cm}{kali-linux-revealed.jpg}

\emph{Kali Linux Revealed  Mastering the Penetration Testing Distribution}

\link{https://www.kali.org/download-kali-linux-revealed-book/}\\
Explains how to install Kali Linux



\slide{Book: Practical Packet Analysis (PPA)}

\hlkimage{6cm}{PracticalPacketAnalysis3E_cover.png}

\emph{Practical Packet Analysis,
Using Wireshark to Solve Real-World Network Problems}
by Chris Sanders, 3rd Edition
April 2017, 368 pp.
ISBN-13:
978-1-59327-802-1

\link{https://nostarch.com/packetanalysis3}


\slide{Book: Applied Network Security Monitoring (ANSM)}

\hlkimage{5cm}{ansm-book.png}

\emph{Applied Network Security Monitoring: Collection, Detection, and Analysis}
1st Edition

Chris Sanders, Jason Smith
eBook ISBN: 9780124172166
Paperback ISBN: 9780124172081 496 pp.
Imprint: Syngress, December 2013

{\footnotesize\link{https://www.elsevier.com/books/applied-network-security-monitoring/unknown/978-0-12-417208-1}}


\slide{Network Security Through Data Analysis}

\hlkimage{4cm}{network-security-through-data-analysis.png}

Low page count, but high value! Recommended.

Network Security through Data Analysis, 2nd Edition
By Michael S Collins
Publisher: O'Reilly Media
2015-05-01: Second release, 348 Pages

New Release Date: August 2017



\end{document}
