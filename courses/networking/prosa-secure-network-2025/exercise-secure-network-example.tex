\documentclass[a4paper,11pt,notitlepage,landscape]{report}
% Henrik Kramselund  , February 2001
% hlk@security6.net,
% My standard packages
\usepackage{zencurity-one-page}
\usepackage{alltt}
%\usepackage[outputdir=build]{minted2}

\begin{document}

\rm
\selectlanguage{english}

\newcommand{\subject}[1]{PROSA: Hvad er et sikkert netværk?}

%\mytitle{SIEM and Log Analysis}{exercises}
%{\LARGE Kickstart: SIEM and Log Analysis}{}
\lhead{\fancyplain{}{\color{titlecolor}\bfseries\LARGE Exercise: \subject}}

\normal

Dette materiale er en del af kurset
\emph{Hvad er et sikkert netværk? (Onlinemodul 1 af 3)} hos PROSA, som afholdes online i efteråret 2025 af Henrik Kramselund fra Zencurity Aps.

{\bf Objective:}\\
Investigate a secure network, what does it look like. Learn what a generic network looks like on the Internet.

{\bf Purpose:}\\
See that we can quickly get an overview of a network from the outside, using easily accessible tools.

{\bf Suggested method:}\\
We will look at the external network for Zencurity Aps which is AS57860.

Instructor will create groups of 5-6 persons, you will discuss and look at the network. Run tools, collect and share data about what you found. See if you can get an overview, help each other!


The network is a basic network with the usual services:
\begin{list1}
\item[\faSquareO] Multiple domains -- main one \verb+zencurity.com+
\item[\faSquareO] DNS servers -- basic TCP and UDP port 53
\item[\faSquareO] Mail server -- mail.kramse.org, services SMTP and IMAP
\item[\faSquareO] Web servers -- multiple www.zencurity.com HTTP and HTTPS/TLS
\end{list1}

These are \emph{public services} -- without these you cannot receive emails or guide customers into your web sites and services.

\eject
{\bf Hints:}\\
Use your Debian Linux VM and the tools:

\begin{list2}
\item Ping \verb+ping 185.129.63.1+ and Traceroute \verb+traceroute 185.129.63.1+
\item sudo nmap a router \verb+sudo nmap -A 185.129.63.1+
\item sudo nmap a DNS server \verb+sudo nmap -A -p 50-55 185.129.63.141+ and  \verb+sudo nmap -A -sU -p 50-55 185.129.63.141+
\item sudo nmap a Mail server mail.kramse.org \verb+sudo nmap -A -p 25,587,993 185.129.63.250+ -- btw what are those ports?!
\item sudo nmap web servers \verb+sudo nmap -A -p 80,443 www.zencurity.com garden.kramse.org+
\end{list2}

You are welcome to port scan the whole of 185.129.63.0/24 -- if you have time.

If you do not have a Linux VM with tools, you can use web sites for some things:

\begin{list2}
\item \url{https://internet.nl/} Generic checker
\item \url{https://www.hardenize.com/} Generic checker
\item \url{https://stat.ripe.net/} RIPEstat -- information about internet prefixes and routing
\item \url{https://dnsviz.net/} DNS zone check
%\item \url{https://www.wormly.com/test_ssl} Test TLS
\item \url{https://observatory.mozilla.org/} Web site headers check
%\item \url{https://rpki.cloudflare.com/} Check RPKI - route validator
\end{list2}

{\bf Solution:}\\
When you have used at least a few tools in the group and seen some results.

{\bf Discussion:}\\
Was it hard running the tools in Linux?

Which tools do you prefer? GUI tools, web sites or command line -- what if you want to test 100 domains!


\end{document}
