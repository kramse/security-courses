\documentclass[a4paper,11pt,notitlepage,landscape]{report}
% Henrik Kramselund  , February 2001
% hlk@security6.net,
% My standard packages
\usepackage{zencurity-one-page}
\usepackage{graphicx}
%\usepackage{lscape}

\begin{document}

\rm
\selectlanguage{english}

\newcommand{\subject}[1]{PROSA Secure Network}

%\mytitle{SIEM and Log Analysis}{exercises}
%{\LARGE Kickstart: SIEM and Log Analysis}{}
\lhead{\fancyplain{}{\color{titlecolor}\bfseries\LARGE Kickstart 2: GL.inet Opal SFT1200 router}}


\normal

This material is prepared for use in \emph{\subject} and was prepared by
Henrik Kramselund, \url{hkj@zencurity.com}.

I have recommended buying a small router from GL.inet, which contains a lot of features.

\hlkimage{55mm}{opal-sft1200.jpg}

\begin{list2}
\item When learning and investigating it is nice to have a \emph{lab network} -- make changes, play with settings, break things
\item If you live alone, and are not in a remote meeting -- play with you own network!
\item I recommended the small GL-Inet Opal (GL-SFT1200) Wireless Travel Router\\
\url{https://store.gl-inet.com/products/opal-gigabit-wireless-pocket-sized-openwrt-ipv6-sft1200}
\item It has 2 LAN ports for connecting, 1 WAN port for Internet or can act as a Wi-Fi client. All powered by USB-C etc.
\item Manual and documentation \url{https://docs.gl-inet.com/router/en/4/user_guide/gl-sft1200/}
\end{list2}

The following pages show screenshots with comments. The screenshots show some of the features that I find very interesting with the Opal router.

\eject

Opal router contains a modern web interface
\hlkimage{17cm}{opal-overview.png}
\eject

Firmware upgrade is easy to find and perform:
\hlkimage{17cm}{opal-firmware.png}
\eject

Guest wi-fi can be added on 2.4GHz or 5GHz or both:
\hlkimage{17cm}{opal-guest-wifi.png}
\eject

Router can be configured as a router, or just an Access Point:
\hlkimage{17cm}{opal-network-mode.png}
\eject

Currently running as a client on my home Wi-Fi:
\hlkimage{17cm}{opal-wifi-repeater.png}
\eject

Router includes IP version 6:
\hlkimage{17cm}{opal-ipv6.png}
\eject

When Secure Shell (ssh) is enabled you have a command line available:
\hlkimage{17cm}{opal-ssh-login.png}

Note: the current config did not agree with my modern OpenSSH client, so needed to add a small script/option:\\
\verb;ssh -oHostKeyAlgorithms=+ssh-rsa root@192.168.8.1;

\eject


There are a lot of packages that can be installed with web interface or \verb+opkg+:
\hlkimage{17cm}{opal-packages.png}
\hlkimage{17cm}{opal-opkg-nmap.png}



\eject

\section*{Virtual Private Network (VPN)}

You can enable VPN functionality as a client or server.

Client would be connecting you to your \emph{home network} when travelling:
\hlkimage{15cm}{opal-wireguard-client.png}
\eject

Server would be if this was your home router:
\hlkimage{15cm}{opal-wireguard-server.png}

We recommend Wireguard as a modern alternative to OpenVPN

\section*{Advanced Features -- through the LuCI interface}

There is a more advanced web interface named LuCI that can be accessed with the Advanced Settings menu -- directing you to \url{http://192.168.8.1/cgi-bin/luci}:

\hlkimage{15cm}{opal-luci-connections.png}
\eject

This includes small network diagnostics tools:
\hlkimage{15cm}{opal-luci-diag.png}

Various firewall administration options:
\hlkimage{15cm}{opal-luci-firewall-easy.png}
\hlkimage{15cm}{opal-luci-firewall.png}
\eject

You can change network configuration, show and add static routes:
\hlkimage{15cm}{opal-luci-routes.png}
\eject

Reconfigure ports for various VLANs:

Add the WAN to become just another LAN port, or add VLAN tagging -- and perhaps some switch connected to the router.

\hlkimage{15cm}{opal-luci-vlans.png}
\eject

Even review changes before saving them:
\hlkimage{15cm}{opal-config-changes.png}



\end{document}
