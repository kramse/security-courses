\documentclass[Screen16to9,17pt]{foils}
\usepackage{zencurity-slides}

\externaldocument{prosa-secure-network-2025-exercises}
\selectlanguage{english}

% SIEM er et bredt udtryk for systemer der kan samle information om
% sikkerhedsevents og netværksdelen er særligt interessant. I dette
% foredrag vil vi præsentere grundsten som Netflow og værktøjer til at
% generere, opsamle og præsentere disse data effektivt med grafiske
% værktøjer der kan tilgås via browser og måske i visse tilfælde kunne
% automatiseres med Machine Learning -- som ikke er en del af foredraget.

\begin{document}



\misp


\data driven vs intelligence


https://www.crowdsec.net/ log4j 


\mytitlepage
{Network Monitoring and SIEM}
{PROSA September \the\year{}}


\hlkprofiluk

\slide{Goals for today}

\hlkimage{6cm}{bornhack-camp-2024.jpg}

\begin{list2}
\item
\end{list2}

Photo is NWWC camp at BornHack 2024, looks about the same every year
--come by and say hi


\slide{Time schedule}

\begin{list2}
\item 45 min Introduction and basics

\item Rest of the time:
 Connect to the network, play with TCP/IP and routers
\end{list2}

Note: even though I talk a lot about Unix and Linux, you can definitely run a lot of tools on Windows and Mac OS X. The basic tools are available like the built-in ones and Nmap

Command line tools are sometimes used in the slides, as they only show text where a GUI screenshot can be cluttered with a lot of information, feel free to find GUI tools and web sites with same functionality

\slide{Exercises}

Exercises are completely optional

\hlkimage{5cm}{eugen-str-CrhsIRY3JWY-unsplash.jpg}
\begin{list2}
\item Try ping and traceroute
\item See your own IP settings
\item Borrow a USB Ethernet and connect to a switch or router
\end{list2}

Linux is a toolbox I will use and participants are free to use whatever they feel like
\hfill Photo by Eugen Str on Unsplash



\slide{Course Materials}

\begin{list2}
\item This material is in multiple parts:

\item Slide show - presentation - this file
\item Exercises - PDF which is used for this and other workshops
\item Additional resources from the internet are linked throughout
\item Wikipedia has a LOT of nice pages about IP protocols, for example:
\end{list2}

\begin{quote}\small
Transport Layer Security (TLS) is a cryptographic protocol designed to provide communications security over a computer network. The protocol is widely used in applications such as email, instant messaging, and voice over IP, but its use in securing HTTPS remains the most publicly visible.
\end{quote}
Source: \url{https://en.wikipedia.org/wiki/Transport_Layer_Security}



\slide{Prerequisites}

If you are interested in TCP/IP you are welcome

If you want to be an expert in IP and network security I recommend doing exercises

\begin{list1}
\item Network security and most internet related security work has the following requirements:
\begin{list2}
\item Network experience
\item TCP/IP principles - often in more detail than a common user
\item Programming is an advantage, for automating things
\item Some Linux and Unix knowledge is in my opinion a {\bf necessary skill} for infosec work\\
-- too many new tools to ignore, and lots found at sites like Github and Open Source written for Linux
\end{list2}
\item It is recommended to use virtual machines for the exercises
\end{list1}



\slide{Wifi Hardware}

If you want to do sniffing of wireless it will be an advantage to have a wireless USB network card. Make sure to play nice, and dont abuse knowledge!

\begin{list2}
\item The following are two recommended models:
\item TP-link TL-WN722N hardware version 2.0 cheap but only support 2.4GHz
\item Alfa AWUS036ACH 2.4GHz + 5GHz Dual-Band and high performing
\item Both work great in Kali Linux for our purposes, but are older models by now
\end{list2}

Unfortunately the vendors change models often enough that the above are hard to find. I recommend using your favourite search engine and research which cards work with Kali Linux and airodump-ng.

I have some available you can borrow


\slide{Book: Practical Packet Analysis (PPA)}


\hlkimage{6cm}{PracticalPacketAnalysis3E_cover.png}

\emph{Practical Packet Analysis,
Using Wireshark to Solve Real-World Network Problems}
by Chris Sanders, 3rd Edition
April 2017, 368 pp.
ISBN-13:
978-1-59327-802-1
\link{https://nostarch.com/packetanalysis3}

I recommend this book for people new to networking, it has been in HumbleBundle book bundles multiple times



\slide{Internet Today}

\hlkimage{10cm}{images/server-client.pdf}

\begin{list1}
\item Clients and servers, roots in the academic world
\item Protocols are old, some more than 20 years
\item Very few protocols where encrypted, today a lot has switched to HTTPS and TLS
\end{list1}


\slide{Opsummering}

\vskip 3 cm

\begin{list1}
\item Husk følgende:
\begin{list2}
\item UNIX og Linux er blot eksempler - navneservice eller HTTP
  server kører fint på Windows
\item DNS er grundlaget for Internet
\item Sikkerheden på internet er generelt dårlig, brug SSL!
\item Procedurerne og vedligeholdelse er essentiel for alle
  operativsystemer!
\item Man skal \emph{hærde} operativsystemer \emph{før} man sætter dem på
  Internet
\item Husk: IT-sikkerhed er ikke kun netværkssikkerhed!
\item God sikkerhed kommer fra langsigtede intiativer\\
\end{list2}
\item Jeg håber I har lært en masse om netværk og kan bruge det i praksis :-)
\end{list1}

\slide{Spørgsmål?}


\vskip 4cm

\begin{center}
\hlkbig

\myname

\myweb
\vskip 2 cm

I er altid velkomne til at sende spørgsmål på e-mail
\end{center}



\slide{Referencer: netværksbøger}

\begin{list2}
\item Stevens, Comer,
\item Network Warrior
\item TCP/IP bogen på dansk
\item KAME bøgerne
\item O'Reilly generelt IPv6 Essentials og IPv6 Network Administration
\item O'Reilly cookbooks: Cisco, BIND og Apache HTTPD
\item Cisco Press og website
\item Firewall bøger, Radia Perlman: IPsec,
\end{list2}

\slide{Bøger om IPv6}

\begin{list1}
\item \emph{IPv6 Network Administration}
af David Malone og Niall Richard Murphy
 - god til real-life admins, typisk
O'Reilly bog
\item \emph{IPv6 Essentials} af Silvia Hagen, O'Reilly 2nd edition (May 17, 2006)
	god reference om emnet
\item \emph{IPv6 Core Protocols Implementation}
af Qing Li, Tatuya Jinmei og Keiichi Shima
\item \emph{IPv6 Advanced Protocols Implementation}
af Qing Li, Jinmei Tatuya og Keiichi Shima
\item - flere andre
\end{list1}




\end{document}
