\documentclass[Screen16to9,17pt]{foils}
\usepackage{zencurity-slides}

\externaldocument{prosa-secure-network-2025-exercises}
\selectlanguage{english}

% SIEM er et bredt udtryk for systemer der kan samle information om
% sikkerhedsevents og netværksdelen er særligt interessant. I dette
% foredrag vil vi præsentere grundsten som Netflow og værktøjer til at
% generere, opsamle og præsentere disse data effektivt med grafiske
% værktøjer der kan tilgås via browser og måske i visse tilfælde kunne
% automatiseres med Machine Learning -- som ikke er en del af foredraget.

\begin{document}


\mytitlepage
{Network Monitoring and SIEM}
{PROSA September \the\year{}}


\hlkprofiluk


\slide{Code of Conduct }

I subscribe to having a Code of Conduct for events, we need them still! Usually I say the BornHack code of conduct apply whenever I teach! \url{https://bornhack.dk/conduct/}

Today we talk about networking, so I recommend this also:
RIPE Code of Conduct
Publication date: 05 Oct 2021

\begin{quote}
Rationale
Our goals in having this Code of Conduct are:
\begin{list2}
\item {\bf To help everyone feel safe and included.} Many people will be new to our community. Some may have had negative experiences in other communities. We want to set a clear expectation that harassment and related behaviours are not tolerated here. If people do have an unpleasant experience, they will know that this is neither the norm nor acceptable to us as a community.

\item {\bf To make everyone aware of expected behaviour.} We are a diverse community; a CoC sets clear expectations in terms of how people should behave.
\end{list2}
\end{quote}
Source: {\small
\link{https://www.ripe.net/publications/docs/ripe-766}}

\slide{Time schedule}

\begin{list2}
\item 17:00 - 17:40 Introduction and basics for the subject

\item 17:40 - 18:05 Exercises

\item 30min break Eat with your family if you like, I will be around most of the break, available for questions

\item 18:45 - 19:30 Further teaching and exercises in the subject for the evening

\item 15min break Stretch your legs, get some more water

\item 19:45 - 20:30
Further teaching and exercises in the subject for the evening, questions and more

\item 20:30 - 21:00 May contain exercises to be done on your own, with input from me
\end{list2}

\centerline{I will try to keep this plan for all evenings! So you hopefully can plan family life better}

Will also try to make smaller breaks/exercises during the slidesshows, check for questions etc.


\slide{Modul 3: Netværksovervågning og SIEM (Onlinemodul 3 af 3)}

%\hlkimage{}{}


\begin{quote}
{\bf Hvordan kan vi overvåge netværk effektivt?}

SIEM er et bredt udtryk for systemer, der kan samle information om sikkerhedsevents og netværksdelen er særligt interessant. I dette modul får du præsenteret grundsten som Netflow samt værktøjer til at generere, opsamle og præsentere disse data effektivt med grafiske værktøjer, som kan tilgås via browser. Og måske i visse tilfælde kunne automatiseres med Machine Learning, som ikke er en del af oplægget.


\vskip 5mm
Keywords: CIA modellen, CVE sårbarheder, switch, router, firewall, ACL, DoS/DDoS, VLAN, segmentering, {\bf logning, monitoring, Netflow, Zeek, Suricata, \bf Elasticsearch}, Nmap, IEEE 802.1x, IPv4, IPv6, NTP, DNS
\end{quote}

\begin{list2}
\item Vi skal prøve at få et overblik over hvad vi har lært hidtil
\item Hvordan kan vi bruge den viden vi samler op effektivt
\item Alle værktøjer der præsenteres er veldokumenterede mange steder -- inkl videoer
\end{list2}




\slide{Goals for today}

\hlkimage{10cm}{incident-response-life-cycle.png}

\begin{list2}
\item Make sure we have an overview of SIEM terms, including NSM
\item Have a big picture of how a network can be more secure after applying the methods in these modules
\item Put the icing on the cake -- see beautiful pictures
\end{list2}



\slide{Exercises}

Exercises are completely optional

\hlkimage{5cm}{eugen-str-CrhsIRY3JWY-unsplash.jpg}
\begin{list2}
\item See example systems
\item Try SELKS -- if you have docker installed
\item
\end{list2}

Linux is a toolbox I will use and participants are free to use whatever they feel like
\hfill Photo by Eugen Str on Unsplash




\slide{What is a Secure Network}

%\hlkimage{}{}

\begin{quote}
A controlled environment with a purpose and goal which is designed, implemented and monitored to be sufficiently secure -- according to the policies and wishes of the owner and operator
\end{quote}

Example networks
\begin{list2}
\item Home network -- should support a \emph{family typically}
\item Factory network -- should support machines, robots, production of things
\item Office network -- should be available for employees and without malware and data leaks
\end{list2}

\slide{Network Security as a Holistic Approach}

%\hlkimage{10cm}{holistic-approach.png }
\begin{quote}
{\bf\Large holistic} adjective

\begin{list2}
\item[1]: of or relating to holism
\item[2] : relating to or concerned with wholes or with complete systems rather than with the analysis of, treatment of, or dissection into parts\\
holistic medicine attempts to treat both the mind and the body\\
holistic ecology views humans and the environment as a single system
\end{list2}
\end{quote}
Source: \url{https://www.merriam-webster.com/dictionary/holistic}

\begin{list2}
\item The network spans the whole organisation and we use \emph{the network} -- the Internet for many things
\item Network security affects the whole organisation
\item When improving network security, we often improve overall security
\end{list2}

\slide{Example plot 6-17 }

\hlkimage{9cm}{jay-service-discovery.png}
Source: DDS 6. Visualizing Security Data

\begin{list2}
\item Interesting graph, and interesting results Changing away from standard ports delay attackers!
\item Attackers are constantly scanning -- internet tinnitus
\end{list2}

\slide{Applied Security Visualization examples}


\hlkimage{10cm}{applied-security-visualization-flow.png}

Source: Network Flow Data in \emph{Applied security visualization}, Rafael Marty, 2009

\slide{CIP 1 Incident Response Fundamentals}

%\hlkimage{}{}

\begin{quote}
\begin{list2}
\item Keeping an organization safe from attack, as well as having a {\bf talented team} available to {\bf respond quickly, minimizes damage} to your reputation and business.
\item Fostering and developing {\bf relationships} with IT, HR, legal, executives, and others is critical to the success of a CSIRT.
\item {\bf Sharing incident and threat data} with external groups improves everyone’s security and gives your organization credibility and trust with groups that might be able to help in the future.
\item A good team relies on good tools, and a great team optimizes their operations.
\item A solid and well-socialized InfoSec policy gives the incident response team the authority and charter to protect networks and data.
\end{list2}
\end{quote}
Source:
 \emph{Crafting the InfoSec Playbook: Security Monitoring and Incident Response Master Plan}\\
 by Jeff Bollinger, Brandon Enright, and Matthew Valites ISBN: 9781491949405

\slide{CIP 2 What Are You Trying to Protect?}

%\hlkimage{}{}

\begin{quote}
\begin{list2}
\item You can’t properly protect your network if you don’t know what to protect.
\item Define and understand your critical assets and what’s most important to your
  organization.
\item Ensure that you can attribute ownership or responsibility for all systems on your
  network.
\item Understand and leverage the log data that can help you determine host owner‐
  ship.
\item A complex network is difficult to protect, unless you understand it well.
\end{list2}
\end{quote}

Source:
 \emph{Crafting the InfoSec Playbook: Security Monitoring and Incident Response Master Plan}\\
 by Jeff Bollinger, Brandon Enright, and Matthew Valites ISBN: 9781491949405


\slide{SIEM}

%\hlkimage{}{}

\begin{quote}
{\bf Security information and event management (SIEM)} is a subsection within the field of computer security, where software products and services combine security information management (SIM) and security event management (SEM). They provide real-time analysis of security alerts generated by applications and network hardware.

  Vendors sell SIEM as software, as appliances, or as managed services; these products are also used to log security data and generate reports for compliance purposes.[1]

  The term and the initialism SIEM was coined by Mark Nicolett and Amrit Williams of Gartner in 2005.[2]
\end{quote}
Source: \link{https://en.wikipedia.org/wiki/Security_information_and_event_management}

\begin{list2}
  \item Note: there are alerting examples towards the bottom of the page, with sources
  \item Closely related to log management, incident response
\end{list2}

\slide{SOC}

%\hlkimage{}{}

\begin{quote}
An information security operations center (ISOC or SOC) is a facility where enterprise information systems (web sites, applications, databases, data centers and servers, networks, desktops and other endpoints) are monitored, assessed, and defended.

...

A security operations center (SOC) can also be called a security defense center (SDC), security analytics center (SAC), network security operations center (NSOC),[3] security intelligence center, cyber security center, threat defense center, security intelligence and operations center (SIOC). In the Canadian Federal Government the term, infrastructure protection center (IPC), is used to describe a SOC.
\end{quote}
Source: \link{https://en.wikipedia.org/wiki/Information_security_operations_center}

\begin{list2}
  \item We have a whole book about SOCs, but I skipped the introductory chapters!
  \item If you need to build a SOC, that is great source of information
\end{list2}


\slide{Data Driven vs Intelligence Driven}

What is the difference? Data Driven, Intelligence Driven, Network Security Monitoring

Short description
\begin{list2}
\item Network Security Monitoring is what we do, with tools like Zeek, Suricata, logging tools
\item Data Driven is when we use that data: start blocking, investigate incidents based on alerting
\item It becomes intelligence when we share it with other: sharing a list of systems that tried brute-forcing our services
\item They become Intelligence Driven when they use process sources: block lists, IoC etc.
\item We become Intelligence Driven when we use data sources from others
\end{list2}

TL;DR Don't worry, use tools, resources and anything that you can!

\slide{Network security monitoring (NSM)}

%\hlkimage{}{}

\begin{quote}
Network security monitoring (NSM) is the collection and analysis of network data such as logs, traffic patterns, and anomalies. Security professionals use this data to discover and respond to potential intrusions and malicious activity.

\end{quote}
Source: \url{https://corelight.com/resources/glossary/network-security-monitoring-nsm}

\begin{list2}
\item Note: they are selling commercial product Open NDR based on Zeek, Suricata and Sigma SIEM rules
\end{list2}

\slide{Data-Driven}

%\hlkimage{}{}

\begin{quote}
{\large\bf  data-driven} /ˈdeɪtəˌdrɪvn,ˈdɑːtəˌdrɪvn/
adjective\\
determined by or dependent on the collection or analysis of data.
"decisions are data-driven and made by committee"
\end{quote}
Source:  Oxford Languages

\begin{list2}
\item Was the preferred term a few years back
\item Today everything is \emph{intelligence} -- see also Artificial Intelligence (AI)
\item You need data to be able to make evidence-based decisions and perform actions
\end{list2}

\slide{Intelligence in network security}

%\hlkimage{}{}

\begin{quote}{\bf
Intelligence is derived from a process of collecting, processing, and analyzing data.}
Once it has been analyzed, it {\bf must be disseminated} in order to be useful. Intelligence
that does not get to the right audience is wasted intelligence. Wilhelm Agrell, a Swedish writer and historian who studied peace and conflict, once famously said, “Intelligence analysis combines the dynamics of journalism with the problem solving of science.”
\end{quote}
Source:  IDIR 2. Basics of Intelligence\\
\emph{Intelligence-Driven Incident Response} (IDIR)\\
 Scott Roberts. Rebekah Brown

\begin{list2}
\item Sharing data helps us and others
\item We can use many sources of data to enable quicker response
\end{list2}


\slide{Intelligence-driven computer network defense}

\begin{quote}
Intelligence-driven computer network defense is a risk management strategy that addresses the threat
component of risk, incorporating analysis of adversaries, their capabilities, objectives, doctrine and limitations. This is necessarily a continuous process, leveraging indicators to discover new activity with yet more indicators to leverage. It requires a new understanding of the intrusions themselves, not as singular events, but rather as phased progressions. This paper presents a new intrusion kill chain model to analyze intrusions and drive defensive courses of action.

\end{quote}
Source: \emph{Intelligence-Driven Computer Network Defense Informed by Analysis of Adversary Campaigns and Intrusion Kill Chains}, Eric M. Hutchins , Michael J. Cloppert, Rohan M. Amin, Ph.D. Lockheed Martin Corporation\\{\footnotesize
 \link{https://www.lockheedmartin.com/content/dam/lockheed-martin/rms/documents/cyber/LM-White-Paper-Intel-Driven-Defense.pdf}}

\url{https://www.lockheedmartin.com/en-us/capabilities/cyber/cyber-kill-chain.html}

\slide{Intrusion Kill Chains}

\hlkimage{10cm}{crafting-cip-kill-chain.png}


\begin{list2}
\item  Source: figure from \emph{Crafting the InfoSec Playbook: Security Monitoring and Incident Response Master Plan}\\ by Jeff Bollinger, Brandon Enright, and Matthew Valites ISBN: 9781491949405.
\item Framework from \emph{Intelligence-Driven Computer Network Defense Informed by Analysis of Adversary Campaigns and Intrusion Kill Chains}, Eric M. Hutchins , Michael J. Cloppert, Rohan M. Amin, Ph.D. Lockheed Martin Corporation\\{\footnotesize
 \link{https://www.lockheedmartin.com/content/dam/lockheed-martin/rms/documents/cyber/LM-White-Paper-Intel-Driven-Defense.pdf}}
\end{list2}



\slide{Detection Capabilities}


Security incidents happen, but what happens. One of the actions to reduce impact of incidents are done in preparing for incidents.

\begin{itemize}
\item \emph{Preparation} for an attack, establish procedures and mechanisms for detecting and responding to attacks
\end{itemize}

Preparation will enable easy {\bf identification} of affected systems, better {\bf containment} which systems are likely to be infected, {\bf eradication} what happened -- how to do the {\bf eradication} and {\bf recovery}.



\slide{Indicators of Compromise and Signatures}

\begin{quote}
An IOC is any piece of information that can be used to objectively describe a network intrusion, expressed in a platform-independent manner. This could include a simple indicator such as the IP address of a command and control (C2) server or a complex set of behaviors that indicate that a mail server is being used as a malicious SMTP relay.

When an IOC is taken and used in a platform-specific language or format, such as a Snort Rule or a Zeek-formatted file, it becomes part of a signature. A signature can contain one or more IOCs.
\end{quote}

Source: Applied Network Security Monitoring Collection, Detection, and Analysis, 2014 Chris Sanders


\slide{Data Analysis Skills}

\begin{quote}
Although we could spend an entire book creating an exhaustive list of skills needed to be a good security data scientist, this chapter covers the following skills/domains that a data scientist will benefit from
knowing within information security:
\begin{list2}
\item Domain expertise—Setting and maintaining a purpose to the analysis
\item Data management—Being able to prepare, store, and maintain data
\item Programming—The glue that connects data to analysis
\item Statistics—To learn from the data
\item Visualization—Communicating the results effectively
\end{list2}
It might be easy to label any one of these skills as the most important, but in reality, the whole is greater than the sum of its parts. Each of these contributes a significant and important piece to the workings of
security data science.
\end{quote}

Source: \emph{Data-Driven Security: Analysis, Visualization and Dashboards} Jay Jacobs, Bob Rudis\\
ISBN: 978-1-118-79372-5 February 2014 \url{https://datadrivensecurity.info/} - short DDS




\slide{Data-Driven Security: Analysis, Visualization ...}

\hlkimage{7cm}{jay-data-science-workflow.png}

\begin{list2}
\item Find and Collect Relevant Data
\item Learn through Iteration
\item Find Statistics
\end{list2}
Source: \emph{Data-Driven Security: Analysis, Visualization and Dashboards} Jay Jacobs, Bob Rudis
ISBN: 978-1-118-79372-5 February 2014 \url{https://datadrivensecurity.info/}



\slide{Strategy for implementing identification and detection}

We recommend that the following strategy is used for implementing identification and detection.

We have the following recommendations and actions points for logging:
\begin{enumerate}
\item[\faSquareO] Enable system logging from servers
\item[\faSquareO] Enable system logging from network devices
\item[\faSquareO] Centralize logging
\item[\faSquareO] Add search facilities and dashboards
\item[\faSquareO] Perform system audits manually or automatically
\item[\faSquareO] Setup notification and notification procedures
\end{enumerate}

\slide{Extended Sources}
When a basic logging infrastructure is setup, it can be expanded to increase coverage, by
adding more sources:

\begin{list2}
\item DNS query logging -- will enable multiple cases to be resolved, example malware identification and tracing, when was a malware domain queried, when was the first infection
\item Session data from Firewalls, Netflow -- traffic patterns can be investigated and both attacks and cases like exfiltration can likely be seen
\end{list2}

Hint: Take the sources available first, make a proof-of-concept, expand coverage


\slide{Data overview JSON}

\begin{quote}
JavaScript Object Notation (JSON, pronounced /ˈdʒeɪsən/; also /ˈdʒeɪˌsɒn/[note 1]) is an open-standard file format or data interchange format that uses {\bf human-readable text} to transmit data objects consisting of attribute–value pairs and array data types (or any other serializable value). It is a very common data format, with a diverse range of applications, such as serving as replacement for XML in AJAX systems.[6]
\end{quote}
Source: \url{https://en.wikipedia.org/wiki/JSON}

\begin{list2}
\item I like JSON much better than XML
\item Many web services can supply data in JSON format
\end{list2}

\slide{The Zeek Network Security Monitor}

Together with firewalls -- The Zeek Network Security Monitor is not a single tool, more of a powerful network analysis framework

\hlkimage{8cm}{zeek-ids.png}

\begin{quote}
While focusing on network security monitoring, Zeek provides a comprehensive platform for more general network traffic analysis as well. Well grounded in more than 15 years of research, Zeek has successfully bridged the traditional gap between academia and operations since its inception.
\end{quote}

Zeek is the tool formerly known as Bro, changed name in 2018. \link{https://www.zeek.org/}



\slide{Suricata IDS/IPS/NSM}
\hlkimage{6cm}{suricata.png}

\begin{quote}
Together with firewalls -- Suricata is a high performance Network IDS, IPS and Network Security Monitoring engine.
\end{quote}

\link{https://suricata.io}
\link{https://openinfosecfoundation.org}



\slide{Zeek, Suricata, JSON and jq }

\begin{minted}[fontsize=\footnotesize]{json}
{
  "timestamp": "2008-07-22T03:51:08.386060+0200",
  "flow_id": 1376641579994488,
  "pcap_cnt": 67,
  "event_type": "dns",  "proto": "UDP",
  "src_ip": "192.168.1.64",  "src_port": 27440,
  "dest_ip": "192.168.1.254",  "dest_port": 53,
  "pkt_src": "wire/pcap",
  "dns": {
    "version": 2,
    "type": "query",
    "id": 11992,
    "rrname": "ssl-google-analytics.l.google.com",
    "rrtype": "AAAA",
    "tx_id": 0,
    "opcode": 0
  }
}
\end{minted}

\begin{list2}
\item
\verb+hlk@debian-lab:~/suri$ cat eve.json | jq | head -21+
jq is a lightweight and flexible command-line JSON processor \url{https://jqlang.org/}
\end{list2}




\slide{Commercial Support}

You can and should use updated rulesets for Suricata.

I Recommend the Emerging Threats ET Pro ruleset



\slide{Metadata -- enrichment}

\hlkimage{10cm}{crafting-security-playbook-metadata.png}

Source: picture from Crafting the InfoSec Playbook, CIP

Metadata + Context


\slide{Reputation-Based Detection}

\hlkimage{4cm}{switch-1.pdf}

\begin{list2}
\item The most basic form of intrusion detection is reputation-based detection
\item Similar concept to block lists for SMTP spam relays
\item I often recommend \link{https://github.com/stamparm/maltrail} as a source of lists
\item Other sources are lists like RIPE NCC delegated, which IP prefixes are handed out in different countries\\
\link{https://ftp.ripe.net/pub/stats/ripencc/delegated-ripencc-extended-latest}\\
\verb+ripencc|DK|ipv4|185.129.60.0|1024|20151130|allocated|+
\item Should we trust all danish network companies?\\
Probably not, but we can easily get into contact with them and report \emph{bad servers}
\end{list2}

\slide{IP reputation}

\begin{list1}
\item Zeek documentation Intel framework\\
\link{https://docs.zeek.org/en/stable/frameworks/intel.html}\\
\item Suricata reputation support\\
\link{https://suricata.readthedocs.io/en/latest/reputation/index.html}
\end{list1}

\exercise{ex:ip-address-research}

\exercise{siem:ip-reputation}

\slide{CrowdSec -- Crowd sourced security information}

%\hlkimage{}{}

\begin{quote}
{\bf Detect and block Log4j exploitation attempts with CrowdSec}\\
If you work in Infosec, you had a very lousy weekend. And that’s because of the Log4j zero-day vulnerability (CVE-2021-44228) that was discovered. We had no choice but to roll up our sleeves to help our community before things got messier than they already were.

As a result, we have released a scenario that will help you detect and block exploitation attempts of the vulnerability. This new scenario can be directly downloaded from our Hub and installed in a blink of an eye. Check this quick video to see the plugin in action:
\end{quote}

\url{https://www.crowdsec.net/blog/detect-block-log4j-exploitation-attempts}


\begin{list2}
\item Log4j is a popular software library in the Java world
\item CrowdSec quickly provided a list of systems scanning the internet for this vulnerability
\item Full disclosure they gave me a hoodie at an event \smiley
\end{list2}




\slide{Research MISP Project 30min}

\hlkimage{9cm}{misp-project-visualization.png}

Demo and Research the MISP Project. Running MISP Project is  will allow you to fetch reputation lists easily and analyse logs better

{\bf Suggested method if you want to try it:} \url{https://www.misp-project.org/download/}\\
I use the \emph{Production ready docker images for MISP and MISP-modules }

\exercise{ex:misp-install}

\slide{Dashboards and Searching}

\hlkimage{14cm}{selks-17.png}

\begin{list2}
\item Some of the most important parts of a SIEM is searching and dashboards
\end{list2}



\slide{SIEM Architecture and Storage platform ElasticSearch!}

\hlkimage{14cm}{elastic-logstash-queue-publish.png}

Note: Kibana makes it easy to use sample data, feel free to experiment!

Elasticsearch and Kibana are \emph{services} which open a listening socket/port. So access ES via \link{https://127.0.0.1:9200} and Kibana via \link{https://127.0.0.1:5601} on your Debian using a browser or Postman

Using the SELKS Docker images are the easiest way


\slide{Example data store: Elasticsearch}

\begin{quote}
Elasticsearch is a search engine based on the Lucene library. It provides a distributed, multitenant-capable full-text search engine with an HTTP web interface and schema-free JSON documents. Elasticsearch is developed in Java.
\end{quote}

Source: Wikipedia \link{https://en.wikipedia.org/wiki/Elasticsearch}

\begin{list2}
\item Initial release	8 February 2010
\item Open core means parts of the software are licensed under various open-source licenses (mostly the Apache License)
\item Various browser tools and plugins for ES exist, to make life easier
\item I often use ES for storing Log Messages and Events from multiple systems, a SIEM Security information and event management.
\end{list2}


\slide{Elasticsearch}

%\hlkimage{}{}

\begin{quote}\small
{\bf ElasticSearch consumes practically anything you give it} and provides straightforward ways to ask it questions and get data out of it. You just need to feed it {\bf semi- or unstructured data} and fold in some domain intelligence to enable smart indexing. It works its multi-node NoSQL magic in conjunction with {\bf a layer of full-text searching} to give you {\bf almost instantaneous query results even for large amounts of data.}\\
Source: DDS 8. Breaking Up with Your Relational Database
\end{quote}

\begin{list2}
\item Elasticsearch SIEM -- from Elastic
\item Wazuh -- agent for clients, log events, integrity protection etc.
\item HELK -- all-in one hunting system
\item ElastiFlow -- netflow system
\item Arkime (renamed recently from Moloch) -- packet capture
\end{list2}

Lots of commercial systems, and lots of companies providing cloud logging platform

Microsoft Azure promotes Sentinel -- cloud based SIEM\\ {\footnotesize
\link{https://azure.microsoft.com/da-dk/services/azure-sentinel/}}


\slide{Elasticsearch SIEM}

%\hlkimage{}{}

\begin{quote}{\bf
Elastic Common Schema (ECS)}\\
The Elastic Common Schema (ECS) defines a common set of fields for ingesting data into Elasticsearch. A common schema helps you correlate data from sources like logs and metrics or IT operations analytics and security analytics.
\end{quote}

\begin{list2}

\item Some structure is useful, Elastic Common Schema (ECS)\\
  \link{https://github.com/elastic/ecs}
\item I would use their schemas for a green field deployment,\\
  as they have been expanded and developed over some time
\item Correlation becomes implicit in every search!\\
-- hint/fact from {\small\link{https://www.elastic.co/webinars/introducing-the-elastic-common-schema}}
\end{list2}



\slide{Architecture for packet capture}

\hlkimage{5cm}{network-horiz-onion.png}

\begin{quote}
Security Onion is a free and open platform built by defenders for defenders. It includes network visibility, host visibility, intrusion detection honeypots, log management, and case management.
\end{quote}
Source:  \url{https://docs.securityonion.net/en/2.4/introduction.html}


\slide{Lets design a SIEM Infrastructure Proof of Concept}

\hlkimage{16cm}{demo-siem-setup.pdf}


\slide{Summary Processing of Data in the SIEM world}

Let's look at some processing

\begin{list2}
\item Processing includes normalizing collected data into uniform formats for analysis
\item Indexing -- Large volumes of data need to be made searchable
\item Translation -- for our course we might get multiple input formats but need JSON or XML
\item Enrichment -- Providing additional metadata for a piece of information is important. For example, domain addresses need to be resolved to IP addresses, and {\bf WHOIS registration data fetched}
\item Filtering --
Not all data provides equal value, and analysts can be overwhelmed when presented with endless streams of irrelevant data
\item Prioritization --
The data that has been collected may need to be ranked so that analysts can allo‐
cate resources to the most important items\\
Note: this relates to a \emph{baseline}, what errors are normal in your environment
\item Visualization -- Data visualization has advanced significantly and the human eye and brain can often see patterns
\end{list2}


\slide{SELKS -- now Clear NDR}

%\hlkimage{}{}

\begin{quote}
SELKS™ is a free, open-source, and turn-key Suricata network intrusion detection/protection system (IDS/IPS), network security monitoring (NSM) and threat hunting implementation created and maintained by Stamus Networks.
\end{quote}
Source: \url{https://www.stamus-networks.com/blog/selks-10-the-next-big-leap-for-open-source-network-security}

\begin{list2}
\item Suricata - Ready to use Suricata
\item Elasticsearch - Search engine
\item Logstash - Log injection
\item Kibana - Custom dashboards and event exploration
\item Stamus C.E. (formerly Scirius) - Suricata ruleset management and Suricata threat hunting interface
\end{list2}
\url{https://github.com/StamusNetworks/SELKS}

\slide{SELKS 10 (now Clear NDR - Community)}

\hlkimage{12cm}{selks-16.png}

\begin{list2}
\item Description for this setup is in the Kickstart 2 document
\item Using Docker we can turn up a full installation of Elasticsearch with data in minuts!
\item \url{https://github.com/StamusNetworks/SELKS}
\end{list2}


\slide{Packetbeat as an alternative}

\hlkimage{10cm}{demo-siem-setup-packetbeat.pdf}


\begin{list2}
\item By installing packetbeat and doing network mirroring from the network switch, we can gather a lot of information
\item Packetbeat supports Elastic Common Schema (ECS) \link{https://www.elastic.co/beats/packetbeat}
\item ICMP (v4 and v6)
DHCP (v4)
DNS
HTTP
AMQP 0.9.1
Cassandra
Mysql
PostgreSQL
Redis
Thrift-RPC
MongoDB
Memcache
NFS
TLS
SIP/SDP (beta)
\end{list2}


\slide{Running SELKS}

%\hlkimage{}{}

{\Large I will run this first, then we will discuss the tools and related questions}

\slide{Baseline}

\hlkimage{10cm}{nfsen-ddos-profile-1.png}

\begin{list2}
\item Picture from NFsen running a specific profile to catch attacks
\item When you have a running system, it will start to gather a baseline
\item Comparing data from various times become possible, and usefull
\item The best baseline is from running the actual systems and services for an extended \emph{learning} period
\end{list2}


\slide{Exposure, Attack surfaces, and reducing them}

\begin{list2}
\item Incident prevention
\item Real-time intrusion detection systems (IDS/IPS)
\item {\bf Definition 27-7} An \emph{attack surface} is the set of entry points and data that attackers can u
se to compromise a system.
\item Reducing the chance of success also helps, randomization
\item Use stack and heap protection
\item Address space layout randomization (ASLR) is a host-level moving target defense.
\item OpenBSD even randomizes the kernel on install -- kernel address randomized link (KARL)
\item Limit number of listening services, change insecure defaults, implement access control and firewalls
\item Remove anything but the necessary request methods on web servers \verb+GET+, \verb+HEAD+ and \verb+POST+
\item Restrict access to administrative interfaces
\item Implement network segmentation
\end{list2}


\slide{MITRE Adversary Emulation Plans}

\hlkimage{10cm}{Mitre-APT3_phase_diagram.png}

\begin{quote}
To showcase the practical use of ATT\&CK for offensive operators and defenders, MITRE created Adversary Emulation Plans. These are prototype documents of what can be done with publicly available threat reports and ATT\&CK.
\end{quote}
Source: \url{https://attack.mitre.org/resources/adversary-emulation-plans/}


\slide{Sample reports from ENISA}

\begin{quote}
{\bf ENISA Consolidated Annual Activity Report 2023}\\
This publication presents the annual activity report of ENISA for 2023. The report is based on the 2023 work programme as approved by the agency's Management Board.\\
\url{https://www.enisa.europa.eu/publications/corporate-documents/enisa-consolidated-annual-activity-report-2023}

{\bf ENISA Threat Landscape 2023}\\
This is the eleventh edition of the ENISA Threat Landscape (ETL) report, an annual report on the status of the cybersecurity threat landscape. It identifies the top threats, major trends observed with respect to threats, threat actors and attack techniques, as well as impact and motivation analysis. It also describes relevant mitigation measures. This year’s work has again been supported by ENISA’s ad hoc Working Group on Cybersecurity Threat Landscapes (CTL).\\
\url{https://www.enisa.europa.eu/publications/enisa-threat-landscape-2023}
\end{quote}
%Example: ENISA \emph{Trust Services Security Incidents 2019 Annual Analysis Report}\\
%\link{https://www.enisa.europa.eu/publications/trust-services-security-incidents-2019-annual-analysis-report}


\slide{Staffing your security team}

%\hlkimage{}{}

\begin{quote}

With regard to analysts and staffing, your options essentially boil down to:
\begin{list2}
\item Paying a managed security service a regular subscription fee to “do your security,”
  with little to no context about your network; the service might, however, handle a
  broad spectrum of security beyond incident response (e.g., vulnerability scan‐
  ning)
\item Tasking a part-time “security person” to work on a best-effort security monitor‐
  ing system (e.g., a SIEM) when they have time
\item Hiring a sufficient number of security analysts and tailoring your security opera‐
tions to your business requirements
\item Calling in an emergency response team after your organization has been compromised
\end{list2}
\end{quote}
Source: CIP 6 Operationalize!

\begin{list2}
  \item Hard truth
\end{list2}

\slide{Buy or DIY?}

%\hlkimage{}{}

\begin{quote}
DNSDB is a database that stores and indexes both the passive DNS data available via Farsight Security’s Security Information Exchange as well as the authoritative DNS data that various zone operators make available.
\end{quote}
Source: from \link{https://docs.dnsdb.info/}
\begin{list2}
  \item Excellent services can be bought, have used \link{https://team-cymru.com/}
\item Compare using \link{https://docs.dnsdb.info/}
  Farsight DNSDB API documentation
\item \link{https://nullsecure.org/building-your-own-passivedns-feed/}
\item Lots of examples for adding functionality, building and expanding SIEM and log systems
\item I usually go to Github and have found a lot of useful tools
\end{list2}

\slide{Team Cymru}

%\hlkimage{}{}

\begin{quote}
  We operate as our own ISP and are part of the fabric of the internet. We’ve amassed an unmatched number of data sharing partnerships with operators worldwide, in addition to gathering threat intelligence from a global grid of sensors, honeypots, darknets and crawlers. We give you our visibility via our Pure Signal™ platform, Augury™.

\begin{list2}
\item Trace threat actors through dozens of proxies and VPNs.
\item Map the extended infrastructure.
\item Preemptively block associated IPs.
\item Then monitor these threats to defend against them indefinitely.
\end{list2}
\end{quote}
Source: from \link{https://team-cymru.com/}

\begin{list2}
\item Often you need sources that are hard to get
\item Many vendors integrate sources into other products too
\item Firewalls and Load balancing products that include reputation lists
\end{list2}

\slide{The Spamhaus Don't Route Or Peer Lists}

\begin{quote}
The Spamhaus Don't Route Or Peer Lists

DROP (Don't Route Or Peer) and EDROP are advisory "drop all traffic" lists, consisting of stolen 'hijacked' netblocks and netblocks controlled entirely by criminals and professional spammers. DROP and EDROP are a tiny subset of the SBL designed for use by firewalls and routing equipment.
\end{quote}

\link{http://www.spamhaus.org/drop/}


\begin{list2}
\item When your SIEM alerts you, you need tools to block and restrict
\item Recommend adding empty blocking access control lists etc. to your network infrastructure
\item Add premade blocking to your name servers, proxy servers, recursive servers
\item Recommend implementing country lists
\end{list2}



\slide{Incident Handling, phases}

The procedures developed for incident response must cover the complete life-cycle

\begin{list2}
\item  Preparation for an attack, establish procedures and mechanisms for detecting and responding to attacks
\item  Identification of an attack, notice the attack is ongoing
\item  Containment (confinement) of the attack, limit effects of the attack as much as possible
\item  Eradication of the attack, stop attacker, block further similar attacks
\item  Recovery from the attack, restore system to a secure state
\item  Follow-up to the attack, include lessons learned  improve environment
\end{list2}
Source: NIST-SP800-61r2.png

\link{https://doi.org/10.6028/NIST.SP.800-61r2}

\slide{Incident Response Life cycle}

\hlkimage{18cm}{incident-response-life-cycle.png}
Source: \emph{Computer Security Incident Handling Guide}, NIST SP 800-61 Rev. 2



\slide{Case management}

%\hlkimage{}{}

\begin{quote}
There are a number of open source and commercial case management tools available on the market, most sharing a set of common features. Most coordinate the end-to-end response, investigation, and reporting of security incidents. Most provide a secure web-based collaboration platform that allows for multiple parties to work together to investigate incident reports and manage incidents. Most provide the ability to report on individual incidents and provide trending data for longer-term analysis. Most provide some level of integration with other systems to streamline investigations and response, particularly integration with SIEMs, forensics platforms, and enterprise ticketing systems. Some also support compliance and security incidents, providing for anonymous incident reporting for ethics violations.
\end{quote}
Source:  SOC 11. Reacting to events and Incidents

\slide{Conclusion}


I hope you learnt something about firewalls, network security, protocols -- how it helps keep the network secure, and monitoring for intrusions that are bound to happen

\begin{list2}
\item Implement firewalls – take control over network packets, flows, protocols, services etc.
also to reduce noise!
\item Start monitoring with available data feed, sources, netflow, firewall logging, DNS queries etc.
\item Start from the bottom and from client ports, or from server ports if you like

\item UNIX and Linux are example systems -- and often found in appliances\\
Learn some Linux and use open source projects, really, will save you thosands of USD/EUR/DKK\\
or you can decide to outsource this, your choice, but make it a well reasoned choice

\end{list2}


\slide{Primary literature used in the course SIEM and Log Analysis}

\hlkrightpic{5cm}{0cm}{old_book_lumen_design_st_01.png}
Primary literature:
\begin{list2}
\item \emph{Data-Driven Security: Analysis, Visualization and Dashboards} Jay Jacobs, Bob Rudis\\
ISBN: 978-1-118-79372-5 February 2014 \url{https://datadrivensecurity.info/} - short DDS
\item \emph{Crafting the InfoSec Playbook: Security Monitoring and Incident Response Master Plan}\\
 by Jeff Bollinger, Brandon Enright, and Matthew Valites ISBN: 9781491949405 - short CIP
\item \emph{Intelligence-Driven Incident Response} \\
 Scott Roberts. Rebekah Brown, ISBN: 9781098120689 {\bf 2nd edition}- short IDIR

\item \emph{Modern Security Operations Center, The}\\
ISBN: 978-0135619858 Joseph Muniz - short SOC
\end{list2}


\slide{Data-Driven Security: Analysis, Visualization and Dashboards}

\hlkimage{6cm}{book-data-driven-security.jpg}
\emph{Data-Driven Security: Analysis, Visualization and Dashboards} Jay Jacobs, Bob Rudis\\
ISBN: 978-1-118-79372-5 February 2014 \url{https://datadrivensecurity.info/} - short DDS

Our main book for this course. We will read a lot from this one. From basic data processing to dashboards

\slide{Crafting the InfoSec Playbook}

\hlkimage{6cm}{book-crafting-infosec-playbook.jpg}

\emph{Crafting the InfoSec Playbook: Security Monitoring and Incident Response Master Plan}\\
 by Jeff Bollinger, Brandon Enright, and Matthew Valites ISBN: 9781491949405 - short CIP

\emph{Develop your own threat intelligenceand incident detection strategy}

\slide{Intelligence-Driven Incident Response}

\hlkimage{6cm}{book-intelligence-driven-incident-response.jpg}

\emph{Intelligence-Driven Incident Response} \\
  Scott Roberts. Rebekah Brown, ISBN: 9781098120689 {\bf 2nd edition}- short IDIR



\slide{Security Operations Center}

\hlkimage{6cm}{ book-security-operations-center.jpg}

\emph{Modern Security Operations Center, The}\\
ISBN: 978-0135619858 Joseph Muniz - short SOC

\slide{Supporting literature books}
\begin{list2}
\item \emph{The Linux Command Line: A Complete Introduction}, 2nd Edition\\
 by William Shotts
\item \emph{The Debian Administrator’s Handbook}, Raphaël Hertzog and Roland Mas\\
\url{https://debian-handbook.info/} - shortened DEB
\end{list2}

\slide{Book: The Linux Command Line}

\hlkimage{4cm}{lcl2_front_new.png}

\emph{The Linux Command Line: A Complete Introduction }, 2nd Edition
by William Shotts

Print: \link{https://nostarch.com/tlcl2}\\
Download -- internet edition \link{https://sourceforge.net/projects/linuxcommand}\\
Not curriculum but explains how to use Linux

\slide{ The Debian Administrator’s Handbook (DEB)}

\hlkimage{6cm}{book-debian-administrators-handbook.jpg}

\emph{The Debian Administrator’s Handbook}, Raphaël Hertzog and Roland Mas\\
\url{https://debian-handbook.info/} - shortened DEB

\slide{Primary literature used in the course Communication and Network Security}

Primary literature are these three books:
\begin{list2}
\item \emph{Applied Network Security Monitoring Collection, Detection, and Analy
sis}, 2014 Chris Sanders \\
ISBN: 9780124172081 - shortened ANSM
\item \emph{Practical Packet Analysis - Using Wireshark to Solve Real-World Network Problems}, 3rd edition 2017, \\
Chris Sanders ISBN: 9781593278021 - shortened PPA
\end{list2}


\slide{Book: Applied Network Security Monitoring (ANSM)}

\hlkimage{5cm}{ansm-book.png}

\emph{Applied Network Security Monitoring: Collection, Detection, and Analysis}
1st Edition

Chris Sanders, Jason Smith
eBook ISBN: 9780124172166
Paperback ISBN: 9780124172081 496 pp.
Imprint: Syngress, December 2013

{\footnotesize\link{https://www.elsevier.com/books/applied-network-security-monitoring/unknown/978-0-12-417208-1}}

\slide{Book: Practical Packet Analysis (PPA)}
\hlkimage{6cm}{PracticalPacketAnalysis3E_cover.png}

\emph{Practical Packet Analysis,
Using Wireshark to Solve Real-World Network Problems}
by Chris Sanders, 3rd Edition
April 2017, 368 pp.
ISBN-13:
978-1-59327-802-1

\link{https://nostarch.com/packetanalysis3}



\end{document}
