\documentclass[Screen16to9,17pt]{foils}
\usepackage{zencurity-slides}

\externaldocument{prosa-secure-network-2025-exercises}
\selectlanguage{english}

% SIEM er et bredt udtryk for systemer der kan samle information om
% sikkerhedsevents og netværksdelen er særligt interessant. I dette
% foredrag vil vi præsentere grundsten som Netflow og værktøjer til at
% generere, opsamle og præsentere disse data effektivt med grafiske
% værktøjer der kan tilgås via browser og måske i visse tilfælde kunne
% automatiseres med Machine Learning -- som ikke er en del af foredraget.

\begin{document}


\mytitlepage
{Network Monitoring and SIEM}
{PROSA September \the\year{}}


\hlkprofiluk



\slide{Husk}

%\hlkimage{}{}

\begin{quote}

\end{quote}


misp



\url{https://www.crowdsec.net/} log4j


\begin{list2}
    \item
\end{list2}



\slide{Code of Conduct }

I subscribe to having a Code of Conduct for events, we need them still! Usually I say the BornHack code of conduct apply whenever I teach! \url{https://bornhack.dk/conduct/}

Today we talk about networking, so I recommend this also:
RIPE Code of Conduct
Publication date: 05 Oct 2021

\begin{quote}
Rationale
Our goals in having this Code of Conduct are:
\begin{list2}
\item {\bf To help everyone feel safe and included.} Many people will be new to our community. Some may have had negative experiences in other communities. We want to set a clear expectation that harassment and related behaviours are not tolerated here. If people do have an unpleasant experience, they will know that this is neither the norm nor acceptable to us as a community.

\item {\bf To make everyone aware of expected behaviour.} We are a diverse community; a CoC sets clear expectations in terms of how people should behave.
\end{list2}
\end{quote}
Source: {\small
\link{https://www.ripe.net/publications/docs/ripe-766}}

\slide{Time schedule}

\begin{list2}
\item 17:00 - 17:40 Introduction and basics for the subject

\item 17:40 - 18:05 Exercises

\item 30min break Eat with your family if you like, I will be around most of the break, available for questions

\item 18:45 - 19:30 Further teaching and exercises in the subject for the evening

\item 15min break Stretch your legs, get some more water

\item 19:45 - 20:30
Further teaching and exercises in the subject for the evening, questions and more

\item 20:30 - 21:00 May contain exercises to be done on your own, with input from me
\end{list2}

\centerline{I will try to keep this plan for all evenings! So you hopefully can plan family life better}

Will also try to make smaller breaks/exercises during the slidesshows, check for questions etc.


\slide{Modul 3: Netværksovervågning og SIEM (Onlinemodul 3 af 3)}

%\hlkimage{}{}


\begin{quote}
{\bf Hvordan kan vi overvåge netværk effektivt?}

SIEM er et bredt udtryk for systemer, der kan samle information om sikkerhedsevents og netværksdelen er særligt interessant. I dette modul får du præsenteret grundsten som Netflow samt værktøjer til at generere, opsamle og præsentere disse data effektivt med grafiske værktøjer, som kan tilgås via browser. Og måske i visse tilfælde kunne automatiseres med Machine Learning, som ikke er en del af oplægget.


\vskip 5mm
Keywords: CIA modellen, CVE sårbarheder, switch, router, firewall, ACL, DoS/DDoS, VLAN, segmentering, {\bf logning, monitoring, Netflow, Zeek, Suricata, \bf Elasticsearch}, Nmap, IEEE 802.1x, IPv4, IPv6, NTP, DNS
\end{quote}

\begin{list2}
\item Vi skal prøve at få et overblik over hvad vi har lært hidtil
\item Hvordan kan vi bruge den viden vi samler op effektivt
\item Alle værktøjer der præsenteres er veldokumenterede mange steder -- inkl videoer
\end{list2}




\slide{Goals for today}

\hlkimage{10cm}{incident-response-life-cycle.png}

\begin{list2}
\item Make sure we have an overview of SIEM terms, including NSM
\item Have a big picture of how a network can be more secure after applying the methods in these modules
\item Put the icing on the cake -- see beautiful pictures
\end{list2}



\slide{Exercises}

Exercises are completely optional

\hlkimage{5cm}{eugen-str-CrhsIRY3JWY-unsplash.jpg}
\begin{list2}
\item See example systems
\item Try SELKS -- if you have docker installed
\item
\end{list2}

Linux is a toolbox I will use and participants are free to use whatever they feel like
\hfill Photo by Eugen Str on Unsplash




\slide{What is a Secure Network}

%\hlkimage{}{}

\begin{quote}
A controlled environment with a purpose and goal which is designed, implemented and monitored to be sufficiently secure -- according to the policies and wishes of the owner and operator
\end{quote}

Example networks
\begin{list2}
\item Home network -- should support a \emph{family typically}
\item Factory network -- should support machines, robots, production of things
\item Office network -- should be available for employees and without malware and data leaks
\end{list2}

\slide{Network Security as a Holistic Approach}

%\hlkimage{10cm}{holistic-approach.png }
\begin{quote}
{\bf\Large holistic} adjective

\begin{list2}
\item[1]: of or relating to holism
\item[2] : relating to or concerned with wholes or with complete systems rather than with the analysis of, treatment of, or dissection into parts\\
holistic medicine attempts to treat both the mind and the body\\
holistic ecology views humans and the environment as a single system
\end{list2}
\end{quote}
Source: \url{https://www.merriam-webster.com/dictionary/holistic}

\begin{list2}
\item The network spans the whole organisation and we use \emph{the network} -- the Internet for many things
\item Network security affects the whole organisation
\item When improving network security, we often improve overall security
\end{list2}

\slide{SIEM}

%\hlkimage{}{}

\begin{quote}
{\bf Security information and event management (SIEM)} is a subsection within the field of computer security, where software products and services combine security information management (SIM) and security event management (SEM). They provide real-time analysis of security alerts generated by applications and network hardware.

  Vendors sell SIEM as software, as appliances, or as managed services; these products are also used to log security data and generate reports for compliance purposes.[1]

  The term and the initialism SIEM was coined by Mark Nicolett and Amrit Williams of Gartner in 2005.[2]
\end{quote}
Source: \link{https://en.wikipedia.org/wiki/Security_information_and_event_management}

\begin{list2}
  \item Note: there are alerting examples towards the bottom of the page, with sources
  \item Closely related to log management, incident response
\end{list2}


\slide{Data Driven vs Intelligence Driven}

What is the difference? Data Driven, Intelligence Driven, Network Security Monitoring

Short description
\begin{list2}
\item Network Security Monitoring is what we do, with tools like Zeek, Suricata, logging tools
\item Data Driven is when we use that data: start blocking, investigate incidents based on alerting
\item It becomes intelligence when we share it with other: sharing a list of systems that tried brute-forcing our services
\item They become Intelligence Driven when they use process sources: block lists, IoC etc.
\item We become Intelligence Driven when we use data sources from others
\end{list2}

TL;DR Don't worry, use tools, resources and anything that you can!

\slide{Network security monitoring (NSM)}

%\hlkimage{}{}

\begin{quote}
Network security monitoring (NSM) is the collection and analysis of network data such as logs, traffic patterns, and anomalies. Security professionals use this data to discover and respond to potential intrusions and malicious activity.

\end{quote}
Source: \url{https://corelight.com/resources/glossary/network-security-monitoring-nsm}

\begin{list2}
\item Note: they are selling commercial product Open NDR based on Zeek, Suricata and Sigma SIEM rules
\end{list2}

\slide{Data-Driven}

%\hlkimage{}{}

\begin{quote}
{\large\bf  data-driven} /ˈdeɪtəˌdrɪvn,ˈdɑːtəˌdrɪvn/
adjective\\
determined by or dependent on the collection or analysis of data.
"decisions are data-driven and made by committee"
\end{quote}
Source:  Oxford Languages

\begin{list2}
\item Was the preferred term a few years back
\item Today everything is \emph{intelligence} -- see also Artificial Intelligence (AI)
\item You need data to be able to make evidence-based decisions and perform actions
\end{list2}

\slide{Intelligence-driven computer network defense}

\begin{quote}
Intelligence-driven computer network defense is a risk management strategy that addresses the threat
component of risk, incorporating analysis of adversaries, their capabilities, objectives, doctrine and limitations. This is necessarily a continuous process, leveraging indicators to discover new activity with yet more indicators to leverage. It requires a new understanding of the intrusions themselves, not as singular events, but rather as phased progressions. This paper presents a new intrusion kill chain model to analyze intrusions and drive defensive courses of action.

\end{quote}
Source: \emph{Intelligence-Driven Computer Network Defense Informed by Analysis of Adversary Campaigns and Intrusion Kill Chains}, Eric M. Hutchins , Michael J. Cloppert, Rohan M. Amin, Ph.D. Lockheed Martin Corporation\\{\footnotesize
 \link{https://www.lockheedmartin.com/content/dam/lockheed-martin/rms/documents/cyber/LM-White-Paper-Intel-Driven-Defense.pdf}}

\url{https://www.lockheedmartin.com/en-us/capabilities/cyber/cyber-kill-chain.html}

\slide{Intrusion Kill Chains}

\hlkimage{10cm}{crafting-cip-kill-chain.png}


\begin{list2}
\item  Source: figure from \emph{Crafting the InfoSec Playbook: Security Monitoring and Incident Response Master Plan}\\ by Jeff Bollinger, Brandon Enright, and Matthew Valites ISBN: 9781491949405.
\item Framework from \emph{Intelligence-Driven Computer Network Defense Informed by Analysis of Adversary Campaigns and Intrusion Kill Chains}, Eric M. Hutchins , Michael J. Cloppert, Rohan M. Amin, Ph.D. Lockheed Martin Corporation\\{\footnotesize
 \link{https://www.lockheedmartin.com/content/dam/lockheed-martin/rms/documents/cyber/LM-White-Paper-Intel-Driven-Defense.pdf}}
\end{list2}



\slide{Detection Capabilities}


Security incidents happen, but what happens. One of the actions to reduce impact of incidents are done in preparing for incidents.

\begin{itemize}
\item \emph{Preparation} for an attack, establish procedures and mechanisms for detecting and responding to attacks
\end{itemize}

Preparation will enable easy {\bf identification} of affected systems, better {\bf containment} which systems are likely to be infected, {\bf eradication} what happened -- how to do the {\bf eradication} and {\bf recovery}.



\slide{Crafting the InfoSec Playbook}


This book will help you to answer common questions:
\begin{list2}
\item How do I find bad actors on my network?
\item How do I find persistent attackers?
\item How can I deal with the pervasive malware threat?
\item How do I detect system compromises?
\item How do I find an owner or responsible parties for systems under my protection?
\item How can I practically use and develop threat intelligence?
\item How can I possibly manage all my log data from all my systems?
\item How will I benefit from increased logging—and not drown in all the noise?
\item How can I use metadata for detection?
\end{list2}
Source: \emph{Crafting the InfoSec Playbook: Security Monitoring and Incident Response Master Plan}\\
 by Jeff Bollinger, Brandon Enright, and Matthew Valites ISBN: 9781491949405


\slide{Mitre ATT\&CK framework}

\hlkimage{10cm}{mitre-attack.png}
Source: \link{https://attack.mitre.org/} Great resource for attack categorization

\begin{list2}
\item Most if not all attacks leave a trace, if you collect data. DNS requests, TCP connections, HTTPS certificates, session length -- time and data transmitted
\item Mitre ATT\&CK framework recommends Application log and Network Traffic for Detection
\end{list2}

\slide{Indicators of Compromise and Signatures}

\begin{quote}
An IOC is any piece of information that can be used to objectively describe a network intrusion, expressed in a platform-independent manner. This could include a simple indicator such as the IP address of a command and control (C2) server or a complex set of behaviors that indicate that a mail server is being used as a malicious SMTP relay.

When an IOC is taken and used in a platform-specific language or format, such as a Snort Rule or a Zeek-formatted file, it becomes part of a signature. A signature can contain one or more IOCs.
\end{quote}

Source: Applied Network Security Monitoring Collection, Detection, and Analysis, 2014 Chris Sanders


\slide{Data Analysis Skills}

\begin{quote}
Although we could spend an entire book creating an exhaustive list of skills needed to be a good security data scientist, this chapter covers the following skills/domains that a data scientist will benefit from
knowing within information security:
\begin{list2}
\item Domain expertise—Setting and maintaining a purpose to the analysis
\item Data management—Being able to prepare, store, and maintain data
\item Programming—The glue that connects data to analysis
\item Statistics—To learn from the data
\item Visualization—Communicating the results effectively
\end{list2}
It might be easy to label any one of these skills as the most important, but in reality, the whole is greater than the sum of its parts. Each of these contributes a significant and important piece to the workings of
security data science.
\end{quote}

Source: \emph{Data-Driven Security: Analysis, Visualization and Dashboards} Jay Jacobs, Bob Rudis\\
ISBN: 978-1-118-79372-5 February 2014 \url{https://datadrivensecurity.info/} - short DDS




\slide{Data-Driven Security: Analysis, Visualization ...}

\hlkimage{7cm}{jay-data-science-workflow.png}

\begin{list2}
\item Find and Collect Relevant Data
\item Learn through Iteration
\item Find Statistics
\end{list2}
Source: \emph{Data-Driven Security: Analysis, Visualization and Dashboards} Jay Jacobs, Bob Rudis
ISBN: 978-1-118-79372-5 February 2014 \url{https://datadrivensecurity.info/}




\slide{Data overview JSON}

\begin{quote}
JavaScript Object Notation (JSON, pronounced /ˈdʒeɪsən/; also /ˈdʒeɪˌsɒn/[note 1]) is an open-standard file format or data interchange format that uses {\bf human-readable text} to transmit data objects consisting of attribute–value pairs and array data types (or any other serializable value). It is a very common data format, with a diverse range of applications, such as serving as replacement for XML in AJAX systems.[6]
\end{quote}
Source: \url{https://en.wikipedia.org/wiki/JSON}

\begin{list2}
\item I like JSON much better than XML
\item Many web services can supply data in JSON format
\end{list2}



\slide{Zeek JSON and jq }

\begin{minted}[fontsize=\footnotesize]{json}
{
  "timestamp": "2008-07-22T03:51:08.386060+0200",
  "flow_id": 1376641579994488,
  "pcap_cnt": 67,
  "event_type": "dns",  "proto": "UDP",
  "src_ip": "192.168.1.64",  "src_port": 27440,
  "dest_ip": "192.168.1.254",  "dest_port": 53,
  "pkt_src": "wire/pcap",
  "dns": {
    "version": 2,
    "type": "query",
    "id": 11992,
    "rrname": "ssl-google-analytics.l.google.com",
    "rrtype": "AAAA",
    "tx_id": 0,
    "opcode": 0
  }
}
\end{minted}

\begin{list2}
\item
\verb+hlk@debian-lab:~/suri$ cat eve.json | jq | head -21+
jq is a lightweight and flexible command-line JSON processor \url{https://jqlang.org/}
\end{list2}


\slide{Scirius Security Platform}

\hlkimage{12cm}{selks-16.png}

\begin{list2}
\item Description for this setup is in the Kickstart 2 document
\item Using Docker we can turn up a full installation of Elasticsearch with data in minuts!
\item \url{https://github.com/StamusNetworks/SELKS}
\end{list2}


\slide{Dashboards and Searching}

\hlkimage{14cm}{selks-17.png}

\begin{list2}
\item Some of the most important parts of a SIEM is searching and dashboards
\end{list2}




\slide{Metadata -- enrichment}

\hlkimage{10cm}{crafting-security-playbook-metadata.png}

Source: picture from Crafting the InfoSec Playbook, CIP

Metadata + Context


\slide{IP reputation}

\begin{list1}
\item Zeek documentation Intel framework\\
\link{https://docs.zeek.org/en/stable/frameworks/intel.html}\\
\item Suricata reputation support\\
\link{https://suricata.readthedocs.io/en/latest/reputation/index.html}
\end{list1}


\slide{Research MISP Project 45min}

\hlkimage{9cm}{misp-project-visualization.png}

Demo and Research the MISP Project. Running MISP Project is  will allow you to fetch reputation lists easily and analyse logs better

{\bf Suggested method if you want to try it:} \url{https://www.misp-project.org/download/}\\
I use the \emph{Production ready docker images for MISP and MISP-modules }


\slide{}


\begin{quote}

\end{quote}

\begin{list2}
    \item
\end{list2}



\slide{Opsummering}

\vskip 3 cm

\begin{list1}
\item Husk følgende:
\begin{list2}
\item UNIX og Linux er blot eksempler - navneservice eller HTTP
  server kører fint på Windows
\item DNS er grundlaget for Internet
\item Sikkerheden på internet er generelt dårlig, brug SSL!
\item Procedurerne og vedligeholdelse er essentiel for alle
  operativsystemer!
\item Man skal \emph{hærde} operativsystemer \emph{før} man sætter dem på
  Internet
\item Husk: IT-sikkerhed er ikke kun netværkssikkerhed!
\item God sikkerhed kommer fra langsigtede intiativer\\
\end{list2}
\item Jeg håber I har lært en masse om netværk og kan bruge det i praksis :-)
\end{list1}

\slide{Spørgsmål?}


\vskip 4cm

\begin{center}
\hlkbig

\myname

\myweb
\vskip 2 cm

I er altid velkomne til at sende spørgsmål på e-mail
\end{center}





\end{document}
