\documentclass[Screen16to9,17pt]{foils}
\usepackage{zencurity-slides}

\externaldocument{prosa-secure-network-2025-exercises}
\selectlanguage{english}

% Hvordan kan vi overvåge netværk effektivt. Det får I svaret på i
% foredraget ved at vi starter med at se på nogle simple eksempler som TCP
% forbindelse, HTTPS request og DNS. Dernæst peger jeg på
% eksempelværktøjer som igennem mere end 25 år har automatiseret
% netværksovervågning som Zeek.

% Vi vil også tale om sessionslogning, firewall logning, Netflow

% Der vil også være præsentation af Suricata IDS som eksempel på et andet
% værktøj der kan bruges til Intrusion Detection eller som central rolle
% i blokering af DNS traffik, IP negativ lister m.v.

\begin{document}

\mytitlepage
{Network Traffic Inspection}
{PROSA September \the\year{}}


\hlkprofiluk


\slide{Code of Conduct }

I subscribe to having a Code of Conduct for events, we need them still! Usually I say the BornHack code of conduct apply whenever I teach! \url{https://bornhack.dk/conduct/}

Today we talk about networking, so I recommend this also:
RIPE Code of Conduct
Publication date: 05 Oct 2021

\begin{quote}
Rationale
Our goals in having this Code of Conduct are:
\begin{list2}
\item {\bf To help everyone feel safe and included.} Many people will be new to our community. Some may have had negative experiences in other communities. We want to set a clear expectation that harassment and related behaviours are not tolerated here. If people do have an unpleasant experience, they will know that this is neither the norm nor acceptable to us as a community.

\item {\bf To make everyone aware of expected behaviour.} We are a diverse community; a CoC sets clear expectations in terms of how people should behave.
\end{list2}
\end{quote}
Source: {\small
\link{https://www.ripe.net/publications/docs/ripe-766}}

\slide{Time schedule}

\begin{list2}
\item 17:00 - 17:40 Introduction and basics for the subject

\item 17:40 - 18:05 Exercises

\item 30min break Eat with your family if you like, I will be around most of the break, available for questions

\item 18:45 - 19:30 Further teaching and exercises in the subject for the evening

\item 15min break Stretch your legs, get some more water

\item 19:45 - 20:30
Further teaching and exercises in the subject for the evening, questions and more

\item 20:30 - 21:00 May contain exercises to be done on your own, with input from me
\end{list2}

\centerline{I will try to keep this plan for all evenings! So you hopefully can plan family life better}

Will also try to make smaller breaks/exercises during the slidesshows, check for questions etc.


\slide{Modul 2: Traffic inspection (Onlinemodul 2 af 3)}

%\hlkimage{}{}


\begin{quote}
{\bf Hvordan kan vi overvåge netværk effektivt?}

    Det får I svaret på i dette oplæg. Vi starter med at se på nogle simple eksempler som TCP forbindelse, HTTPS request og DNS. Dernæst peger underviseren på eksempelværktøjer som igennem mere end 25 år har automatiseret netværksovervågning som Zeek. Vi vil også tale om sessionslogning, firewall logning, Netflow. Og du får en præsentation af Suricata IDS som eksempel på et andet værktøj, der kan bruges til Intrusion Detection eller som central rolle i blokering af DNS traffik, IP negativ lister m.v.
\vskip 5mm
Keywords: CIA modellen, CVE sårbarheder, switch, router, firewall, ACL, DoS/DDoS, VLAN, segmentering, {\bf logning, monitoring, Netflow, Zeek, Suricata}, Nmap, Elasticsearch, IEEE 802.1x, IPv4, IPv6, NTP, DNS
\end{quote}

\begin{list2}
\item Vi skal dykke ned i netværkspakker
\item Man kan godt følge med uden at lave øvelser
\item Alle værktøjer der præsenteres er veldokumenterede mange steder -- inkl videoer
\end{list2}

\slide{Planlægning}

\begin{quote}
Ved gennemgangen af evalueringerne af ”Modul 1: Hvad er et sikkert netværk” afholdt 25.8: har vi lyttet til jer - om at der mangler nogle praktiske opgaver. Vi vil derfor meget gerne invitere til øvelsesaftener, hvor der udelukkende laves opgaver og kan stilles spørgsmål. Vi vil arbejde med den lille opal router, opgaver fra øvelseshæftet og der vil være lejlighed til at stille spørgsmål omkring materialerne fra Henrik.
\end{quote}

Planen er således:
\begin{list2}
\item Mandag 8. september kl. 17-21: ”Modul 2: Traffic Inspection”
\item Mandag 15. september kl. 17-21: Åbent hus online med opgaver
\item Mandag 22. september kl 17-21: ”Modul 3: Netværksovervågning og SIEM”
\item Mandag 29. september kl. 17-21: Åbent hus online med opgaver
\end{list2}

På åbent hus aftener vil Henrik Kramselund være online og guide mht. spørgsmål, opgaver, løsning og input. Der vil ikke være foredrag. Mødeinvitation til alle aftener er samme Zoom-link!

\slide{Goals}

%\hlkimage{6cm}{}

Goals for course
\begin{list2}
\item Multiple days -- longer than one day
\item Larger with preparation, exercises and perhaps a course certificate
\end{list2}

Goals for today
\begin{list2}
\item Be able to process packets
\item Understand enough about network packet data from at least a few common protocols DNS and HTTPS
\item Know a few tool names Zeek and Suricata
\item Run Zeek on the web, and run Zeek on the Debian VM
\end{list2}

We have added exercise days, if you can join great -- we will still do exercises today. Also I have limited the number of slides today!


\slide{What is a Secure Network}

%\hlkimage{}{}

\begin{quote}
A controlled environment with a purpose and goal which is designed, implemented and monitored to be sufficiently secure -- according to the policies and wishes of the owner and operator
\end{quote}

Example networks
\begin{list2}
\item Home network -- should support a \emph{family typically}
\item Factory network -- should support machines, robots, production of things
\item Office network -- should be available for employees and without malware and data leaks
\end{list2}

\slide{Network Security as a Holistic Approach}

%\hlkimage{10cm}{holistic-approach.png }
\begin{quote}
{\bf\Large holistic} adjective

\begin{list2}
\item[1]: of or relating to holism
\item[2] : relating to or concerned with wholes or with complete systems rather than with the analysis of, treatment of, or dissection into parts\\
holistic medicine attempts to treat both the mind and the body\\
holistic ecology views humans and the environment as a single system
\end{list2}
\end{quote}
Source: \url{https://www.merriam-webster.com/dictionary/holistic}

\begin{list2}
\item The network spans the whole organisation and we use \emph{the network} -- the Internet for many things
\item Network security affects the whole organisation
\item When improving network security, we often improve overall security
\end{list2}


\slide{Best Current Practice }

%\hlkimage{}{}

\begin{quote}
Lets get this out of the way immediately, you should already be doing
\end{quote}

\begin{list2}
\item Network segmentation and filtering -- we could write a book about this! {\myalert}
\item {\bf Monitor your network -- both bandwidth, error, netflow etc.} {\myalert}
\item Take control of your network, no more admin/admin logins on core devices {\myalert}
\item Turn on authentication for protocols -- routing protocols but also any http service within your org {\myalert}
\item Configure host-based firewalls {\myalert}
\item  {\bf Control DNS -- internally and externally, recursive, authoritative etc.} {\myalert}
\end{list2}

\centerline{This goes for IPv4-only, IPv6-only, and mixed networks!}


\slide{SIEM}

%\hlkimage{}{}

\begin{quote}
{\bf Security information and event management (SIEM)} is a subsection within the field of computer security, where software products and services combine security information management (SIM) and security event management (SEM). They provide real-time analysis of security alerts generated by applications and network hardware.

  Vendors sell SIEM as software, as appliances, or as managed services; these products are also used to log security data and generate reports for compliance purposes.[1]

  The term and the initialism SIEM was coined by Mark Nicolett and Amrit Williams of Gartner in 2005.[2]
\end{quote}
Source: \link{https://en.wikipedia.org/wiki/Security_information_and_event_management}

\begin{list2}
  \item Note: there are alerting examples towards the bottom of the page, with sources
  \item Closely related to log management, incident response
\end{list2}


\slide{Crafting the InfoSec Playbook}


This book will help you to answer common questions:
\begin{list2}
\item How do I find bad actors on my network?
\item How do I find persistent attackers?
\item How can I deal with the pervasive malware threat?
\item How do I detect system compromises?
\item How do I find an owner or responsible parties for systems under my protection?
\item How can I practically use and develop threat intelligence?
\item How can I possibly manage all my log data from all my systems?
\item How will I benefit from increased logging—and not drown in all the noise?
\item How can I use metadata for detection?
\end{list2}
Source: \emph{Crafting the InfoSec Playbook: Security Monitoring and Incident Response Master Plan}\\
 by Jeff Bollinger, Brandon Enright, and Matthew Valites ISBN: 9781491949405


\slide{Mitre ATT\&CK framework}

\hlkimage{10cm}{mitre-attack.png}
Source: \link{https://attack.mitre.org/} Great resource for attack categorization

\begin{list2}
\item Most if not all attacks leave a trace, if you collect data. DNS requests, TCP connections, HTTPS certificates, session length -- time and data transmitted
\item Mitre ATT\&CK framework recommends Application log and Network Traffic for Detection
\end{list2}

\slide{Indicators of Compromise and Signatures}

\begin{quote}
An IOC is any piece of information that can be used to objectively describe a network intrusion, expressed in a platform-independent manner. This could include a simple indicator such as the IP address of a command and control (C2) server or a complex set of behaviors that indicate that a mail server is being used as a malicious SMTP relay.

When an IOC is taken and used in a platform-specific language or format, such as a Snort Rule or a Zeek-formatted file, it becomes part of a signature. A signature can contain one or more IOCs.
\end{quote}

Source: Applied Network Security Monitoring Collection, Detection, and Analysis, 2014 Chris Sanders


\slide{Networks are Built from Components}

\hlkimage{15cm}{eugen-str-CrhsIRY3JWY-unsplash.jpg}

\hfill Photo by Eugen Str on Unsplash


\slide{The basic tools for countering threats}

{\Large Knowledge and insight}
\begin{list2}
\item Networks have end-points and conversations on multiple layers
\item Wireshark is advanced, try right-clicking different places
\item Name resolution includes low level MAC addresses, and IP - names
\end{list2}

\begin{list2}
\item Tcpdump format, built-in to many network devices
\item Remote packet dumps, like \verb+tcpdump –i eth0 –w packets.pcap+
\item Story: tcpdump was originally written in 1988 by Van Jacobson, Sally Floyd, Vern Paxson and Steven McCanne who were, at the time, working in the Lawrence Berkeley Laboratory Network Research Group\\
 \link{https://en.wikipedia.org/wiki/Tcpdump}
\end{list2}

\vskip 5mm

\centerline{\Large Great network security comes from knowing networks!}


\slide{Exercises}

Exercises are completely optional

\hlkimage{5cm}{eugen-str-CrhsIRY3JWY-unsplash.jpg}
\begin{list2}
\item Try Zeek and Suricata
\item See your own IP traffic -- from your virtual machine
\item Decode packets and see data from DNS, HTTPS and TCP
\end{list2}

Linux is a toolbox I will use and participants are free to use whatever they feel like
\hfill Photo by Eugen Str on Unsplash




\slide{Course Materials}

\begin{list2}
\item This material is in multiple parts:
\item Kickstart document -- basic information
\verb+kickstart-prosa-secure-network.pdf+
\item Slide show - the presentation - this file
\verb+prosa-network-traffic-inspection-2025.pdf+
\item Exercise for today: example secure network \verb+exercise-secure-network-example.pdf+
\item Exercise/inspiration for today: \verb+kickstart-2-opal-router.pdf+
\item Exercise booklet -- large PDF with many exercises, stuff to do if you want to learn networks on your own
\verb+prosa-secure-network-2025-exercises.pdf+
\item \verb+prosa-secure-network-2025.pdf+ Modul 1
\item Additional resources from the internet like \verb+firewall-book-10-DRAFT-PROSA.pdf+
\item Later, next time:
\verb+prosa-network-monitoring-SIEM-2025.pdf+
\end{list2}


\slide{Baseline Skills}

\begin{list2}\small
\item Threat-Centric Security, NSM, and the NSM Cycle
\item TCP/IP Protocols
\item Common Application Layer Protocols
\item Packet Analysis
\item Windows Architecture
\item Linux Architecture
\item Basic Data Parsing (BASH, Grep, SED, AWK, etc)
\item IDS Usage (Snort, Suricata, etc.)
\item Indicators of Compromise and IDS Signature Tuning
\item Open Source Intelligence Gathering
\item Basic Analytic Diagnostic Methods
\item Basic Malware Analysis
\end{list2}

Source: \emph{Applied Network Security Monitoring Collection, Detection, and Analysis}, Chris Sanders and Jason Smith


\slide{Data Analysis Skills}

\begin{quote}
Although we could spend an entire book creating an exhaustive list of skills needed to be a good security data scientist, this chapter covers the following skills/domains that a data scientist will benefit from
knowing within information security:
\begin{list2}
\item Domain expertise—Setting and maintaining a purpose to the analysis
\item Data management—Being able to prepare, store, and maintain data
\item Programming—The glue that connects data to analysis
\item Statistics—To learn from the data
\item Visualization—Communicating the results effectively
\end{list2}
It might be easy to label any one of these skills as the most important, but in reality, the whole is greater than the sum of its parts. Each of these contributes a significant and important piece to the workings of
security data science.
\end{quote}

Source: \emph{Data-Driven Security: Analysis, Visualization and Dashboards} Jay Jacobs, Bob Rudis\\
ISBN: 978-1-118-79372-5 February 2014 \url{https://datadrivensecurity.info/} - short DDS




\slide{Anatomy of an Auditing System}


Sample logs from login with Secure Shell (SSH) and performing \verb+sudo su -+
\begin{alltt}\footnotesize
Jun  5 11:53:15 pumba sshd[64505]: Accepted publickey for hlk from 79.142.233.18 port 43902
 ssh2: ED25519 SHA256:l8OJMcywyBcraJiCWJ06uZ2yzHfu0VuiArqVvlVyfEI

Jun  5 11:53:19 pumba sudo:      hlk : TTY=ttyp2 ; PWD=/home/hlk ; USER=root ; COMMAND=/usr/bin/su -
\end{alltt}

\begin{list1}
\item Example systems: Unix syslog, IBM main frame RACF and Windows Event Logs service
\item \emph{swatchdog} is an old skool, but simple tool that works
\item Logs should be protected and considered confidential information
\end{list1}



\slide{Anatomy of an Auditing System}

When data has been gathered it should be analyzed.

\begin{itemize}
\item {\bf Logger functions} - collect
\item {\bf Analyzer} - analyze it, creating dashboard can provide some insights
\item {\bf Notifier} - report results by email or other means
\item Example systems Windows Event Logs service can inform of successful and failed logins, both should be collected
\item Logs should be protected and considered confidential information, by sending it to a centralized system with a high security level protects it
\end{itemize}

Modern systems exist to take all data from logging and provide high capacity storage, searching and sorting.



\slide{Your Network}
.
\hlkrightpic{85mm}{-1cm}{sample-network.png}

\begin{list1}
\item I have a home network which has the following systems:
\begin{list2}
\item OpenBSD router
\item Juniper and small TP-Link switches
\item UniFi wireless access-point
\end{list2}
\end{list1}

Due to online remote teaching - we will investigate other networks and scan across the internet to \emph{my servers}!


\slide{Data-Driven Security: Analysis, Visualization ...}

\hlkimage{7cm}{jay-data-science-workflow.png}

\begin{list2}
\item Find and Collect Relevant Data
\item Learn through Iteration
\item Find Statistics
\end{list2}
Source: \emph{Data-Driven Security: Analysis, Visualization and Dashboards} Jay Jacobs, Bob Rudis
ISBN: 978-1-118-79372-5 February 2014 \url{https://datadrivensecurity.info/}



\slide{Data-Driven Security, continued}

%\hlkimage{}{}

{\bf Building a Real-Life Security Data Science Team}\\
... a clear goal: Given an IP address (or IP/Port combination), {\bf search across all our perimeter devices in less than five minutes.}

Three core principles focused the team.
\begin{list2}
\item First, explore the open source versions of tools before engaging vendors. If you don’t
know how the sausage is being made, you really have no idea what’s being done, and
this is vital when working with real data.
\item Second, follow the mantra of “no single tool; no single database; and, no single approach
to solving a problem.” Do not put blinders on because you are either comfortable with
certain technologies or have an affinity for a certain tool.
\item Third, failure is expected, but you must learn from each journey down the wrong path.
Continuous adaptation and adjustment is the name of the game.
\end{list2}


Source: \emph{Data-Driven Security: Analysis, Visualization and Dashboards} Jay Jacobs, Bob Rudis
ISBN: 978-1-118-79372-5 February 2014 \url{https://datadrivensecurity.info/}



\slide{Lab Networks}

\hlkimage{55mm}{opal-sft1200.jpg}

\begin{list2}
\item When learning and investigating it is nice to have a \emph{lab network} -- make changes, play with settings, break things
\item If you live alone, and are not in a remote meeting -- play with you own network!
\item I recommended the small GL-Inet Opal (GL-SFT1200) Wireless Travel Router\\
\url{https://store.gl-inet.com/products/opal-gigabit-wireless-pocket-sized-openwrt-ipv6-sft1200}
\item It has 2 LAN ports for connecting, 1 WAN port for Internet or can act as a Wi-Fi client. All powered by USB-C etc.
\item Manual and documentation \url{https://docs.gl-inet.com/router/en/4/user_guide/gl-sft1200/}
\end{list2}



\slide{Well-Known Port Numbers}

\hlkimage{6cm}{iana1.jpg}

\begin{list1}
\item IANA maintains a list of magical numbers in TCP/IP
\item Lists of protocl numbers, port numers etc.
\item A few notable examples:
\begin{list2}
\item Port 25/tcp Simple Mail Transfer Protocol (SMTP)
\item Port 53/udp and 53/tcp Domain Name System (DNS)
\item Port 80/tcp Hyper Text Transfer Protocol (HTTP)
\item Port 443/tcp HTTP over TLS/SSL (HTTPS)
\end{list2}
\item Source: \link{http://www.iana.org}
\end{list1}



\slide{Unencrypted data protocols }

Examples
\begin{list2}
\item TFTP use UDP and is unencrypted
\item TFTP still used for configuration files and firmwares
\item FTP sends data in cleartext\\
{\bfseries USER username}\\
{\bfseries PASS password}\\
Stop using FTP on the internet!
\item DNS is by default sending unencrypted on UDP and TCP\\
Use DNS over HTTPS (DoH) or DNS over TLS (DoT) to encrypt your DNS queries
\end{list2}

\slide{Person in the middle attacks}

\begin{list1}
\item ARP spoofing, ICMP redirects, the classics
\item Used to be called Man in The Middle (MiTM)
\begin{list2}
\item ICMP redirect
\item ARP spoofing
\item Wireless listening and spoofing higher levels like  airpwn-ng \link{https://github.com/ICSec/airpwn-ng}
\end{list2}
\item Usually aimed at unencrypted protocols or redirecting clients to wrong sites
\end{list1}


\slide{Recommended Reading}

So to get started in network security I recommend learning the basics:
\begin{list2}
\item Chapter 1: Packet Analysis and Network Basics
\item Chapter 2: Tapping into the Wire
\item Chapter 3: Introduction to Wireshark
\end{list2}
\emph{Practical Packet Analysis,
Using Wireshark to Solve Real-World Network Problems}
by Chris Sanders, 3rd Edition



\slide{Using Wireshark}

\hlkimage{13cm}{images/wireshark-http.png}

\centerline{Capture - Options}

\slide{What about encrypted traffic}

\hlkimage{10cm}{images/wireshark-sni-twitter.png}

\centerline{Current TLS version 1.2 used in HTTPS show the name!}


\slide{The Zeek Network Security Monitor}

Together with firewalls -- The Zeek Network Security Monitor is not a single tool, more of a powerful network analysis framework

\hlkimage{8cm}{zeek-ids.png}

\begin{quote}
While focusing on network security monitoring, Zeek provides a comprehensive platform for more general network traffic analysis as well. Well grounded in more than 15 years of research, Zeek has successfully bridged the traditional gap between academia and operations since its inception.
\end{quote}

Zeek is the tool formerly known as Bro, changed name in 2018. \link{https://www.zeek.org/}



\exercise{ex:wireshark-install}
\exercise{ex:wireshark-capture}

\exercise{ex:zeek-on-debian}



\slide{Ping and port sweep}

\begin{list1}
\item Scans across the network are named sweeps
\item Ping sweeps using ICMP Ping probes
\item Port sweep trying to find a specific service, like port 80 web
\item Quite easy to see in network traffic:
\begin{list2}
\item Selecting two IP-adresser not in use
\item Should not see any traffic, but if it does, its being scanned
\item If traffic is received on both addresses, its a sweep -- if they are a bit apart it is even better, like 10.0.0.100 and 10.0.0.200
  \end{list2}

\vskip 2cm
Pro tip: a Great network intrusion detection engine (IDS), is Suricata \link{suricata-ids.org}
\end{list1}


\slide{Nmap port sweep for web servers}

\begin{alltt}\small
root@cornerstone:~#{\bfseries  nmap -p80,443 172.29.0.0/24}

Starting Nmap 6.47 ( http://nmap.org ) at 2015-02-05 07:31 CET
Nmap scan report for 172.29.0.1
Host is up (0.00016s latency).
PORT    STATE    SERVICE
{\color{darkgreen}80/tcp  open     http}
443/tcp filtered https
MAC Address: 00:50:56:C0:00:08 (VMware)

Nmap scan report for 172.29.0.138
Host is up (0.00012s latency).
PORT    STATE  SERVICE
{\color{darkgreen}80/tcp  open   http}
443/tcp closed https
MAC Address: 00:0C:29:46:22:FB (VMware)

\end{alltt}


\slide{Network Data is Data}

\hlkimage{13cm}{ethernet-frame-1.pdf}

We can find existing programs that decode packets
\begin{list2}
\item Wireshark -- GUI program
\item Tcpdump -- command line
\item Zeek -- engine for decoding network packets into data
\item Suricata -- Intrusion Detection System (IDS)
\end{list2}

What output format do we want


\slide{Data overview JSON}

\begin{quote}
JavaScript Object Notation (JSON, pronounced /ˈdʒeɪsən/; also /ˈdʒeɪˌsɒn/[note 1]) is an open-standard file format or data interchange format that uses {\bf human-readable text} to transmit data objects consisting of attribute–value pairs and array data types (or any other serializable value). It is a very common data format, with a diverse range of applications, such as serving as replacement for XML in AJAX systems.[6]
\end{quote}
Source: \url{https://en.wikipedia.org/wiki/JSON}

\begin{list2}
\item I like JSON much better than XML
\item Many web services can supply data in JSON format
\end{list2}

\slide{JSON example}

\begin{minted}[fontsize=\footnotesize]{json}
{
  "first name": "John",
  "last name": "Smith",
  "age": 25,
  "address": {
    "street address": "21 2nd Street",
    "city": "New York",
    "state": "NY",
    "postal code": "10021"
  },
  "phone numbers": [
    {
      "type": "home",
      "number": "212 555-1234"
    },
  ],
}
\end{minted}

\begin{list2}
\item This is a basic JSON document, new data attribute-value pairs can be added\\
Source: \url{https://en.wikipedia.org/wiki/JSON}
\end{list2}


\slide{JSON and jq }

\begin{minted}[fontsize=\footnotesize]{json}
{
  "timestamp": "2008-07-22T03:51:08.386060+0200",
  "flow_id": 1376641579994488,
  "pcap_cnt": 67,
  "event_type": "dns",  "proto": "UDP",
  "src_ip": "192.168.1.64",  "src_port": 27440,
  "dest_ip": "192.168.1.254",  "dest_port": 53,
  "pkt_src": "wire/pcap",
  "dns": {
    "version": 2,
    "type": "query",
    "id": 11992,
    "rrname": "ssl-google-analytics.l.google.com",
    "rrtype": "AAAA",
    "tx_id": 0,
    "opcode": 0
  }
}
\end{minted}

\begin{list2}
\item
\verb+hlk@debian-lab:~/suri$ cat eve.json | jq | head -21+
jq is a lightweight and flexible command-line JSON processor \url{https://jqlang.org/}
\end{list2}




\slide{Start programming}

%\hlkimage{}{}
Example program, reading from HTTP, into Python, out using JSON
\inputminted{python}{programs/rest-1.py}

\begin{list2}
\item Think of programing in this course as as reading, processing and outputting data
\item Read in any format
\item Process with any function
\item Output in any format
\item And reuse existing software
\end{list2}




\slide{Suricata IDS/IPS/NSM}
\hlkimage{6cm}{suricata.png}

\begin{quote}
Together with firewalls -- Suricata is a high performance Network IDS, IPS and Network Security Monitoring engine.
\end{quote}

\link{https://suricata.io}
\link{https://openinfosecfoundation.org}




\slide{Commercial Support}

You can and should use updated rulesets for Suricata.

I Recommend the Emerging Threats ET Pro ruleset




\slide{Processing data with Zeek}

Get Started with Zeek requires the tool installed, and a packet capture

\begin{list2}
\item To run in “base” mode:\\
 \verb+zeek -r traffic.pcap+
\item To run in a “near zeekctl” mode:\\
\verb+zeek -r traffic.pcap local+
\item To add extra scripts:\\
\verb+zeek -r traffic.pcap myscript.zeek+
\end{list2}

When you have the tool installed you can then configure live capture, deploy the tool in your network.

\slide{Zeek demo: Run}

// Use the deploy command to initialize and start zeek first
\begin{minted}[fontsize=\footnotesize]{shell}
debian:~ root# zeekctl

Welcome to ZeekControl 1.5
Type "help" for help.

[ZeekControl] > install
creating policy directories ...
installing site policies ...
generating standalone-layout.zeek ...
generating local-networks.zeek ...
generating zeekctl-config.zeek ...
generating zeekctl-config.sh ...
...
debian:etc root# grep eth0 node.cfg
interface=eth0
\end{minted}

\centerline{Our Zeek node.cfg is in/opt/zeek/etc}

\slide{Zeek demo: Run Zeek}

\begin{minted}[fontsize=\footnotesize]{shell}
[ZeekControl] > start
... starting zeek
// Exit using ctrl-d and then look at logs
debian:zeek root# cd /opt/zeek/logs/current
debian:zeek root# pwd
/opt/zeek/logs/current
debian:current root# tail -f dns.log
\end{minted}

More examples at:\\
\url{https://www.zeek.org/sphinx/script-reference/log-files.html}



\slide{Example tools and graphs}

\hlkimage{12cm}{etherape-2018.png}


\begin{list2}
\item Etherape shown, see \url{https://etherape.sourceforge.io/}
\item A portscan captured would show one node contacting all the rest
\end{list2}


\slide{Parallel coordinate plots}

\hlkimage{16cm}{collberg-parallel-plot.png}
Source: image from Network Security Visualization Keith Fligg and Genevieve Max\\{\footnotesize
\link{https://www2.cs.arizona.edu/~collberg/Teaching/466-566/2013/Resources/presentations/2012/topic13-final/report.pdf}}

\begin{list2}
\item \link{https://en.wikipedia.org/wiki/Parallel_coordinates}
\item Nice for explaining connections, but not used much in real life systems
\end{list2}




\slide{DNS logging}

Since most malware uses DNS today, to be able to switch to new command and control endpoints, we can leverage that to our advantage.

Domain Name System (DNS) depends on a query from the client, and a server that resolves this to a value.

\begin{list2}
\item We can log any DNS traffic into a database
\item We can look up if any clients have done a lookup for a specific name or IP during incident handling
\item This can confirm if a client has ever \emph{visited} a malicious site, because first it needs to lookup the name to IP address before it can make the TCP/HTTP connection, or send data
\end{list2}




\slide{Unbound and NSD}

\begin{quote}
Unbound is a validating, recursive, caching DNS resolver. It is designed to be fast and lean and incorporates modern features based on open standards.

To help increase online privacy, Unbound supports DNS-over-TLS which allows clients to encrypt their communication. In addition, it supports various modern standards that limit the amount of data exchanged with authoritative servers.
\end{quote}

\link{https://www.nlnetlabs.nl/projects/unbound/about/}

My preferred local DNS server.

Also check out uncensored DNS and his DNS over TLS setup!\\
Even has pinning information available:\\ {\small\link{https://blog.censurfridns.dk/blog/32-dns-over-tls-pinning-information-for-unicastcensurfridnsdk/}}



\slide{Demo and exercises }

\exercise{ex:zeekweb}

\exercise{ex:zeekdnsbasic}
\exercise{ex:zeektlsbasic}



\slide{Discussion}

Where do we want to log DNS?

From the network directly with Zeek and Suricata?


From the DNS servers -- query log?




\slide{Collect Network Evidence from the network}

\begin{list1}
\item Network Flows -- Netflow and Session Logging
\item Netflow sampling is vital information - 123Mbit, but what kind of traffic
\item Detecting DoS/DDoS and problems is essential
\item Cisco standard NetFlow version 5 defines a flow as a unidirectional sequence of packets that all share the following 7 values:
\begin{list2}
\item Ingress interface (SNMP ifIndex)
\item IP protocol, Source IP address and Destination IP address
\item Source port for UDP or TCP, 0 for other protocols
\item Destination port for UDP or TCP, type and code for ICMP, or 0 for other protocols
\item IP Type of Service
\end{list2}
\end{list1}


\slide{Netflow using NFSen}

\hlkimage{10cm}{images/nfsen-overview.png}

I do not recommend using NfSen anymore, but netflow processing has been around for decades!\\
\url{https://nfsen.sourceforge.net/}

\slide{ Netflow NFSen}

\hlkimage{17cm}{nfsen-udp-flood.png}

\centerline{An extra 100k packets per second from this netflow source (source is a router)}


\slide{Netflow processing from the web interface}

\hlkimage{10cm}{images/nfsen-processing-1.png}

\begin{list2}
\item Bringing the power of the command line forward
\item Extremely easy to get top 10 lists pr destination, packets per second etc.
\end{list2}

\slide{ElastiFlow -- Elasticsearch based}

\hlkimage{10cm}{elastiflow.png}

\begin{quote}
  ElastiFlow™ provides network flow data collection and visualization using the Elastic Stack (Elasticsearch, Logstash and Kibana). It supports Netflow v5/v9, sFlow and IPFIX flow types (1.x versions support only Netflow v5/v9).
\end{quote}
Source: Picture and text from \link{https://github.com/robcowart/elastiflow} \\

\slide{Akvorado: flow collector, enricher and visualizer}

\hlkimage{8cm}{akvorado-timeseries.png}

\begin{quote}
This program receives flows (currently Netflow/IPFIX and sFlow), enriches them with interface names (using SNMP), geo information (using IPinfo.io), and exports them to Kafka, then ClickHouse. It also exposes a web interface to browse the collected data.
\end{quote}
Source: Picture and text from \url{https://github.com/akvorado/akvorado}


\slide{Cloud Network Security: Cilium overview}

\hlkimage{12cm}{cilium-overview.png}

\begin{quote}
Kubernetes provides Network Policies for controlling traffic going in and out of the pods. Cilium implements the Kubernetes Network Policies for L3/L4 level and extends with L7 policies for granular API-level security for common protocols such as HTTP, Kafka, gRPC, etc
\end{quote}
Source: picture and text from \link{https://cilium.io/blog/2018/09/19/kubernetes-network-policies/}

\slide{Cloud Network Security: Cilium Hubble}

\hlkimage{8cm}{hubble_arch.png}

\begin{quote}
The Linux kernel technology eBPF is enabling visibility into systems and applications at a granularity and efficiency that was not possible before. It does so in a completely transparent way, without requiring the application to change or for the application to hide information.
\end{quote}
Source: picture and text from \link{https://github.com/cilium/hubble/}



\slide{Cloud Network Security: Cilium overview}

\hlkimage{12cm}{network_and_tcp.png}
\begin{quote}
The metrics and monitoring functionality provides an overview of the state of systems and allow to recognize patterns indicating failure and other scenarios that require action. The following is a short list of example metrics, for a more detailed list of examples, see the Metrics Documentation.
\end{quote}


Source: picture and text from \link{https://github.com/cilium/hubble/}




\slide{Big Data tools: Elasticsearch and Kibana}

\hlkimage{10cm}{kibana-basics-with-vega.jpg}

Elasticsearch is an open source distributed, RESTful search and analytics engine capable of solving a growing number of use cases.

\link{https://www.elastic.co}



\slide{Scirius Security Platform}

\hlkimage{12cm}{selks-16.png}

\begin{list2}
\item Description for this setup is in the Kickstart 2 document
\item Using Docker we can turn up a full installation of Elasticsearch with data in minuts!
\item \url{https://github.com/StamusNetworks/SELKS}
\end{list2}


\slide{Dashboards and Searching}

\hlkimage{14cm}{selks-17.png}

\begin{list2}
\item Some of the most important parts of a SIEM is searching and dashboards
\end{list2}

\exercise{ex:suricata-read}

\slide{Reading Summary, False Positives}

\begin{list2}
\item True Positive (TP). An alert that has correctly identified a specific activity. If a signature was designed to detect a certain type of malware, and an alert is generated when that malware is launched on a system, this would be a true positive, which is what we strive for with every deployed signature.Indicators of Compromise and Signatures
\item False Positive (FP). An alert has incorrectly identified a specific activity. If a signature was designed to detect a specific type of malware, and an alert is generated for an instance in which that malware was not present, this would be a false positive.
\item True Negative (TN). An alert has correctly not been generated when a specific activity has not occurred. If a signature was designed to detect a certain type of malware, and no alert is generated without that malware being launched, then this is a true negative, which is also desirable. This is difficult, if not impossible, to quantify in terms of NSM detection.
\item False Negative (FN). An alert has incorrectly not been generated when a specific activity has occurred.
\end{list2}

Source: Applied Network Security Monitoring Collection, Detection, and Analysis, 2014 Chris Sanders

\slide{Reputation-Based Detection}

\hlkimage{4cm}{switch-1.pdf}

\begin{list2}
\item The most basic form of intrusion detection is reputation-based detection
\item Similar concept to block lists for SMTP spam relays
\item I often recommend \link{https://github.com/stamparm/maltrail} as a source of lists
\item Other sources are lists like RIPE NCC delegated, which IP prefixes are handed out in different countries\\
\link{https://ftp.ripe.net/pub/stats/ripencc/delegated-ripencc-extended-latest}\\
\verb+ripencc|DK|ipv4|185.129.60.0|1024|20151130|allocated|+
\item Tool often mentioned are Argus and SiLK \link{https://tools.netsa.cert.org/silk/}\\
If we end up having time today, or another day, we should look into this tool chain also!
\item Old and mature tools have been proven to work
\end{list2}


\slide{Metadata -- enrichment}

\hlkimage{10cm}{crafting-security-playbook-metadata.png}

Source: picture from Crafting the InfoSec Playbook, CIP

Metadata + Context


\slide{IP reputation}

\begin{list1}
\item Zeek documentation Intel framework\\
\link{https://docs.zeek.org/en/stable/frameworks/intel.html}\\
\item Suricata reputation support\\
\link{https://suricata.readthedocs.io/en/latest/reputation/index.html}
\end{list1}


\slide{Conclusion}

% \hlkrightimage{15cm}{network-layers-1.png}

\begin{list2}
\item Implement firewalls -- take control over network packets, flows, protocols, services etc.\\
also to reduce noise!
\item Start monitoring with available data feed, sources, netflow, firewall logging, DNS queries etc.
\item Start from the bottom and from client ports, or from server ports if you like
\item Learn some Linux and use open source projects, really, will save you thosands of USD/EUR/DKK
\end{list2}


\end{document}
