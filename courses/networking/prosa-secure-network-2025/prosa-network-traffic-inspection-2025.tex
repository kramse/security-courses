\documentclass[Screen16to9,17pt]{foils}
\usepackage{zencurity-slides}

\externaldocument{prosa-secure-network-2025-exercises}
\selectlanguage{english}

% Hvordan kan vi overvåge netværk effektivt. Det får I svaret på i
% foredraget ved at vi starter med at se på nogle simple eksempler som TCP
% forbindelse, HTTPS request og DNS. Dernæst peger jeg på
% eksempelværktøjer som igennem mere end 25 år har automatiseret
% netværksovervågning som Zeek.

% Vi vil også tale om sessionslogning, firewall logning, Netflow

% Der vil også være præsentation af Suricata IDS som eksempel på et andet
% værktøj der kan bruges til Intrusion Detection eller som central rolle
% i blokering af DNS traffik, IP negativ lister m.v.

\begin{document}

\mytitlepage
{Network Traffic Inspection}
{PROSA September \the\year{}}


\hlkprofiluk


\slide{Code of Conduct }

I subscribe to having a Code of Conduct for events, we need them still! Usually I say the BornHack code of conduct apply whenever I teach! \url{https://bornhack.dk/conduct/}

Today we talk about networking, so I recommend this also:
RIPE Code of Conduct
Publication date: 05 Oct 2021

\begin{quote}
Rationale
Our goals in having this Code of Conduct are:
\begin{list2}
\item {\bf To help everyone feel safe and included.} Many people will be new to our community. Some may have had negative experiences in other communities. We want to set a clear expectation that harassment and related behaviours are not tolerated here. If people do have an unpleasant experience, they will know that this is neither the norm nor acceptable to us as a community.

\item {\bf To make everyone aware of expected behaviour.} We are a diverse community; a CoC sets clear expectations in terms of how people should behave.
\end{list2}
\end{quote}
Source: {\small
\link{https://www.ripe.net/publications/docs/ripe-766}}

\slide{Time schedule}

\begin{list2}
\item 17:00 - 17:40 Introduction and basics for the subject

\item 17:40 - 18:05 Exercise in groups:

\item 30min break Eat with your family if you like, I will be around most of the break, available for questions

\item 18:45 - 19:30 Further teaching and exercises in the subject for the evening

\item 15min break Stretch your legs, get some more water

\item 19:45 - 20:30
Further teaching and exercises in the subject for the evening, questions and more

\item 20:30 - 21:00 May contain exercises to be done on your own, with input from me
\end{list2}

\centerline{I will try to keep this plan for all evenings! So you hopefully can plan family life better}

Will also try to make smaller breaks/exercises during the slidesshows, check for questions etc.


\slide{Modul 2: Traffic inspection (Onlinemodul 2 af 3)}

%\hlkimage{}{}


\begin{quote}
{\bf \large Hvordan kan vi overvåge netværk effektivt?}

    Det får I svaret på i dette oplæg. Vi starter med at se på nogle simple eksempler som TCP forbindelse, HTTPS request og DNS. Dernæst peger underviseren på eksempelværktøjer som igennem mere end 25 år har automatiseret netværksovervågning som Zeek. Vi vil også tale om sessionslogning, firewall logning, Netflow. Og du får en præsentation af Suricata IDS som eksempel på et andet værktøj, der kan bruges til Intrusion Detection eller som central rolle i blokering af DNS traffik, IP negativ lister m.v.
\vskip 5mm
Keywords: CIA modellen, CVE sårbarheder, switch, router, firewall, ACL, DoS/DDoS, VLAN, segmentering, {\bf logning, monitoring, Netflow, Zeek, Suricata}, Nmap, Elasticsearch, IEEE 802.1x, IPv4, IPv6, NTP, DNS
\end{quote}

\begin{list2}
\item Vi skal dykke ned i netværkspakker
\item Man kan godt følge med uden at lave øvelser
\item Alle værktøjer der præsenteres er veldokumenterede mange steder -- inkl videoer
\end{list2}



\slide{Goals for today}

%\hlkimage{6cm}{}

\begin{list2}
\item
\end{list2}






\slide{What is a Secure Network}

%\hlkimage{}{}

\begin{quote}
A controlled environment with a purpose and goal which is designed, implemented and monitored to be sufficiently secure -- according to the policies and wishes of the owner and operator
\end{quote}

Example networks
\begin{list2}
\item Home network -- should support a \emph{family typically}
\item Factory network -- should support machines, robots, production of things
\item Office network -- should be available for employees and without malware and data leaks
\end{list2}

\slide{Network Security as a Holistic Approach}

%\hlkimage{10cm}{holistic-approach.png }
\begin{quote}
{\bf\Large holistic} adjective

\begin{list2}
\item[1]: of or relating to holism
\item[2] : relating to or concerned with wholes or with complete systems rather than with the analysis of, treatment of, or dissection into parts\\
holistic medicine attempts to treat both the mind and the body\\
holistic ecology views humans and the environment as a single system
\end{list2}
\end{quote}
Source: \url{https://www.merriam-webster.com/dictionary/holistic}

\begin{list2}
\item The network spans the whole organisation and we use \emph{the network} -- the Internet for many things
\item Network security affects the whole organisation
\item When improving network security, we often improve overall security
\end{list2}


\slide{Best Current Practice }

%\hlkimage{}{}

\begin{quote}
Lets get this out of the way immediately, you should already be doing
\end{quote}

\begin{list2}
\item Network segmentation and filtering -- we could write a book about this! {\myalert}
\item {\bf Monitor your network -- both bandwidth, error, netflow etc.} {\myalert}
\item Take control of your network, no more admin/admin logins on core devices {\myalert}
\item Turn on authentication for protocols -- routing protocols but also any http service within your org {\myalert}
\item Configure host-based firewalls {\myalert}
\item  {\bf Control DNS -- internally and externally, recursive, authoritative etc.} {\myalert}
\end{list2}

\centerline{This goes for IPv4-only, IPv6-only, and mixed networks!}



\slide{Networks are Built from Components}

\hlkimage{15cm}{eugen-str-CrhsIRY3JWY-unsplash.jpg}

\hfill Photo by Eugen Str on Unsplash


\slide{The basic tools for countering threats}

{\Large Knowledge and insight}
\begin{list2}
\item Networks have end-points and conversations on multiple layers
\item Wireshark is advanced, try right-clicking different places
\item Name resolution includes low level MAC addresses, and IP - names
\end{list2}

\begin{list2}
\item Tcpdump format, built-in to many network devices
\item Remote packet dumps, like \verb+tcpdump –i eth0 –w packets.pcap+
\item Story: tcpdump was originally written in 1988 by Van Jacobson, Sally Floyd, Vern Paxson and Steven McCanne who were, at the time, working in the Lawrence Berkeley Laboratory Network Research Group\\
 \link{https://en.wikipedia.org/wiki/Tcpdump}
\end{list2}

\vskip 5mm

\centerline{\Large Great network security comes from knowing networks!}


\slide{Exercises}

Exercises are completely optional

\hlkimage{5cm}{eugen-str-CrhsIRY3JWY-unsplash.jpg}
\begin{list2}
\item Try ping and traceroute
\item See your own IP settings
\item Connect to a switch or router -- most have web interfaces
\end{list2}

Linux is a toolbox I will use and participants are free to use whatever they feel like
\hfill Photo by Eugen Str on Unsplash




\slide{Course Materials}

\begin{list2}
\item This material is in multiple parts:
\item Kickstart document -- basic information
\verb+kickstart-prosa-secure-network.pdf+
\item Slide show - the presentation - this file
\verb+prosa-secure-network-2025.pdf+
\item Exercise for today: example secure network \verb+exercise-secure-network-example.pdf+
\item Exercise/inspiration for today: \verb+kickstart-2-opal-router.pdf+
\item Exercise booklet -- large PDF with many exercises, stuff to do if you want to learn networks on your own
\verb+prosa-secure-network-2025-exercises.pdf+
\item \verb+prosa-network-traffic-inspection-2025.pdf+
\item Additional resources from the internet like \verb+firewall-book-10-DRAFT-PROSA.pdf+
\item Later, next times:
%\verb+prosa-network-monitoring-SIEM-2025.pdf+
\end{list2}

\slide{Prerequisites}

If you are interested in TCP/IP you are welcome

If you want to be an expert in IP and network security I recommend doing exercises

\begin{list1}
\item Network security and most internet related security work has the following requirements:
\begin{list2}
\item Network experience
\item TCP/IP principles - often in more detail than a common user
\item Programming is an advantage, for automating things
\item Some Linux and Unix knowledge is in my opinion a {\bf necessary skill} for infosec work\\
-- too many new tools to ignore, and lots found at sites like Github and Open Source written for Linux
\end{list2}
\item It is recommended to use virtual machines for the exercises
\end{list1}





\slide{Data on an Internet }

TCP/IP basic packets,
\begin{list2}
\item
\end{list2}

insert slide


\slide{Your Network}
.
\hlkrightpic{85mm}{-1cm}{sample-network.png}

\begin{list1}
\item I have a home network which has the following systems:
\begin{list2}
\item OpenBSD router
\item Juniper and small TP-Link switches
\item UniFi wireless access-point
\end{list2}
\end{list1}

Due to online remote teaching - we will investigate other networks and scan across the internet to \emph{my servers}!

\slide{Lab Networks}

\hlkimage{55mm}{opal-sft1200.jpg}

\begin{list2}
\item When learning and investigating it is nice to have a \emph{lab network} -- make changes, play with settings, break things
\item If you live alone, and are not in a remote meeting -- play with you own network!
\item I recommended the small GL-Inet Opal (GL-SFT1200) Wireless Travel Router\\
\url{https://store.gl-inet.com/products/opal-gigabit-wireless-pocket-sized-openwrt-ipv6-sft1200}
\item It has 2 LAN ports for connecting, 1 WAN port for Internet or can act as a Wi-Fi client. All powered by USB-C etc.
\item Manual and documentation \url{https://docs.gl-inet.com/router/en/4/user_guide/gl-sft1200/}
\end{list2}



\slide{Well-Known Port Numbers}

\hlkimage{6cm}{iana1.jpg}

\begin{list1}
\item IANA maintains a list of magical numbers in TCP/IP
\item Lists of protocl numbers, port numers etc.
\item A few notable examples:
\begin{list2}
\item Port 25/tcp Simple Mail Transfer Protocol (SMTP)
\item Port 53/udp and 53/tcp Domain Name System (DNS)
\item Port 80/tcp Hyper Text Transfer Protocol (HTTP)
\item Port 443/tcp HTTP over TLS/SSL (HTTPS)
\end{list2}
\item Source: \link{http://www.iana.org}
\end{list1}



\slide{Unencrypted data protocols }

Examples
\begin{list2}
\item TFTP use UDP and is unencrypted
\item TFTP still used for configuration files and firmwares
\item FTP sends data in cleartext\\
{\bfseries USER username}\\
{\bfseries PASS password}\\
Stop using FTP on the internet!
\item DNS sending unencrypted on UDP and TCP\\
Use DNS over HTTPS (DoH) or DNS over TLS (DoT)
\end{list2}



\slide{Person in the middle attacks}

\begin{list1}
\item ARP spoofing, ICMP redirects, the classics
\item Used to be called Man in The Middle (MiTM)
\begin{list2}
\item ICMP redirect
\item ARP spoofing
\item Wireless listening and spoofing higher levels like  airpwn-ng \link{https://github.com/ICSec/airpwn-ng}
\end{list2}
\item Usually aimed at unencrypted protocols or redirecting clients to wrong sites
\end{list1}


\slide{Recommended Reading}

So to get started in network security I recommend learning the basics:
\begin{list2}
\item Chapter 1: Packet Analysis and Network Basics
\item Chapter 2: Tapping into the Wire
\item Chapter 3: Introduction to Wireshark
\end{list2}
\emph{Practical Packet Analysis,
Using Wireshark to Solve Real-World Network Problems}
by Chris Sanders, 3rd Edition



\slide{Using Wireshark}

\hlkimage{13cm}{images/wireshark-http.png}

\centerline{Capture - Options}

\slide{What about encrypted traffic}

\hlkimage{10cm}{images/wireshark-sni-twitter.png}

\centerline{Current TLS version 1.2 used in HTTPS show the name!}



\slide{Ping and port sweep}

\begin{list1}
\item Scans across the network are named sweeps
\item Ping sweeps using ICMP Ping probes
\item Port sweep trying to find a specific service, like port 80 web
\item Quite easy to see in network traffic:
\begin{list2}
\item Selecting two IP-adresser not in use
\item Should not see any traffic, but if it does, its being scanned
\item If traffic is received on both addresses, its a sweep -- if they are a bit apart it is even better, like 10.0.0.100 and 10.0.0.200
  \end{list2}

\vskip 2cm
Pro tip: a Great network intrusion detection engine (IDS), is Suricata \link{suricata-ids.org}
\end{list1}


\slide{Nmap port sweep for web servers}

\begin{alltt}\small
root@cornerstone:~#{\bfseries  nmap -p80,443 172.29.0.0/24}

Starting Nmap 6.47 ( http://nmap.org ) at 2015-02-05 07:31 CET
Nmap scan report for 172.29.0.1
Host is up (0.00016s latency).
PORT    STATE    SERVICE
{\color{darkgreen}80/tcp  open     http}
443/tcp filtered https
MAC Address: 00:50:56:C0:00:08 (VMware)

Nmap scan report for 172.29.0.138
Host is up (0.00012s latency).
PORT    STATE  SERVICE
{\color{darkgreen}80/tcp  open   http}
443/tcp closed https
MAC Address: 00:0C:29:46:22:FB (VMware)

\end{alltt}


\slide{Network Data is Data}

 We can find existing programs that decode packets
\begin{list2}
\item Wireshark -- GUI program
\item Tcpdump -- command line
\item Zeek -- engine for decoding network packets into data
\item Suricata -- Intrusion Detection System (IDS)
\end{list2}



\slide{The Zeek Network Security Monitor}

Together with firewalls -- The Zeek Network Security Monitor is not a single tool, more of a powerful network analysis framework

\hlkimage{8cm}{zeek-ids.png}

\begin{quote}
While focusing on network security monitoring, Zeek provides a comprehensive platform for more general network traffic analysis as well. Well grounded in more than 15 years of research, Zeek has successfully bridged the traditional gap between academia and operations since its inception.
\end{quote}

Zeek is the tool formerly known as Bro, changed name in 2018. \link{https://www.zeek.org/}



\slide{Suricata IDS/IPS/NSM}
\hlkimage{6cm}{suricata.png}

\begin{quote}
Together with firewalls -- Suricata is a high performance Network IDS, IPS and Network Security Monitoring engine.
\end{quote}

\link{https://suricata.io}
\link{https://openinfosecfoundation.org}


\slide{JSON and jq }

\begin{list2}
\item jq is a lightweight and flexible command-line JSON processor \url{https://jqlang.org/}
\end{list2}






\slide{Processing data with Zeek}

Simple example of reading a file with a few packets

\begin{list2}
\item
\end{list2}





\slide{Graphics }

Example graphs plots from earlier
\begin{list2}
\item
\end{list2}


\slide{Ettercap}

port scan with graph



\slide{Logging and Monitoring}

%\hlkimage{}{}


\begin{list2}
\item
\end{list2}


\slide{Netflow and Session Logging}

\begin{list2}
\item Netflow is getting more important, more data share the same links
\item Accounting is important
\item Detecting DoS/DDoS and problems is essential
\item Netflow sampling is vital information - 123Mbit, but what kind of traffic
\item NFSen is old and not recommended anymore
\link{http://nfsen.sourceforge.net/}
\item sFlow, short for "sampled flow", is an industry standard for packet export at Layer 2 of the OSI model, \\
\link{https://en.wikipedia.org/wiki/SFlow}
\end{list2}




\slide{Netflow using NFSen}

\hlkimage{10cm}{images/nfsen-overview.png}

I do not recommend using NfSen anymore, but netflow processing has been around for decades!\\
\url{https://nfsen.sourceforge.net/}

\slide{ Netflow NFSen}

\hlkimage{17cm}{nfsen-udp-flood.png}

\centerline{An extra 100k packets per second from this netflow source (source is a router)}


\slide{Netflow processing from the web interface}

\hlkimage{10cm}{images/nfsen-processing-1.png}

\begin{list2}
\item Bringing the power of the command line forward
\item Extremely easy to get top 10 lists pr destination, packets per second etc.
\end{list2}

\slide{ElastiFlow -- Elasticsearch based}

\hlkimage{10cm}{elastiflow.png}

\begin{quote}
  ElastiFlow™ provides network flow data collection and visualization using the Elastic Stack (Elasticsearch, Logstash and Kibana). It supports Netflow v5/v9, sFlow and IPFIX flow types (1.x versions support only Netflow v5/v9).
\end{quote}
Source: Picture and text from \link{https://github.com/robcowart/elastiflow} \\

\slide{Akvorado: flow collector, enricher and visualizer}

\hlkimage{8cm}{akvorado-timeseries.png}

\begin{quote}
This program receives flows (currently Netflow/IPFIX and sFlow), enriches them with interface names (using SNMP), geo information (using IPinfo.io), and exports them to Kafka, then ClickHouse. It also exposes a web interface to browse the collected data.
\end{quote}
Source: Picture and text from \url{https://github.com/akvorado/akvorado}



\slide{Cloud Network Security: Cilium overview}

\hlkimage{12cm}{cilium-overview.png}

\begin{quote}
Kubernetes provides Network Policies for controlling traffic going in and out of the pods. Cilium implements the Kubernetes Network Policies for L3/L4 level and extends with L7 policies for granular API-level security for common protocols such as HTTP, Kafka, gRPC, etc
\end{quote}
Source: picture and text from \link{https://cilium.io/blog/2018/09/19/kubernetes-network-policies/}

\slide{Cloud Network Security: Cilium Hubble}

\hlkimage{8cm}{hubble_arch.png}

\begin{quote}
The Linux kernel technology eBPF is enabling visibility into systems and applications at a granularity and efficiency that was not possible before. It does so in a completely transparent way, without requiring the application to change or for the application to hide information.
\end{quote}
Source: picture and text from \link{https://github.com/cilium/hubble/}



\slide{Cloud Network Security: Cilium overview}

\hlkimage{12cm}{network_and_tcp.png}
\begin{quote}
The metrics and monitoring functionality provides an overview of the state of systems and allow to recognize patterns indicating failure and other scenarios that require action. The following is a short list of example metrics, for a more detailed list of examples, see the Metrics Documentation.
\end{quote}


Source: picture and text from \link{https://github.com/cilium/hubble/}




\slide{Big Data tools: Elasticsearch and Kibana}

\hlkimage{10cm}{kibana-basics-with-vega.jpg}

Elasticsearch is an open source distributed, RESTful search and analytics engine capable of solving a growing number of use cases.

\link{https://www.elastic.co}

\slide{DNS logging}

Since most malware uses DNS today, to be able to switch to new command and control endpoints, we can leverage that to our advantage.

Domain Name System (DNS) depends on a query from the client, and a server that resolves this to a value.

\begin{list2}
\item We can log any DNS traffic into a database
\item We can look up if any clients have done a lookup for a specific name or IP during incident handling
\item This can confirm if a client has ever \emph{visited} a malicious site, because first it needs to lookup the name to IP address before it can make the TCP/HTTP connection, or send data
\end{list2}



\slide{Unbound and NSD}

\begin{quote}
Unbound is a validating, recursive, caching DNS resolver. It is designed to be fast and lean and incorporates modern features based on open standards.

To help increase online privacy, Unbound supports DNS-over-TLS which allows clients to encrypt their communication. In addition, it supports various modern standards that limit the amount of data exchanged with authoritative servers.
\end{quote}

\link{https://www.nlnetlabs.nl/projects/unbound/about/}

My preferred local DNS server.

Also check out uncensored DNS and his DNS over TLS setup!\\
Even has pinning information available:\\ {\small\link{https://blog.censurfridns.dk/blog/32-dns-over-tls-pinning-information-for-unicastcensurfridnsdk/}}



\slide{Demo and exercises }



\slide{Discussion}

Where do we want to log DNS?

From the network directly with Zeek and Suricata?


From the DNS servers -- query log?



\slide{Opsummering}

\vskip 3 cm

\begin{list1}
\item Husk følgende:
\begin{list2}
\item UNIX og Linux er blot eksempler - navneservice eller HTTP
  server kører fint på Windows
\item DNS er grundlaget for Internet
\item Sikkerheden på internet er generelt dårlig, brug SSL!
\item Procedurerne og vedligeholdelse er essentiel for alle
  operativsystemer!
\item Man skal \emph{hærde} operativsystemer \emph{før} man sætter dem på
  Internet
\item Husk: IT-sikkerhed er ikke kun netværkssikkerhed!
\item God sikkerhed kommer fra langsigtede intiativer\\
\end{list2}
\item Jeg håber I har lært en masse om netværk og kan bruge det i praksis :-)
\end{list1}

\slide{Spørgsmål?}


\vskip 4cm

\begin{center}
\hlkbig

\myname

\myweb
\vskip 2 cm

I er altid velkomne til at sende spørgsmål på e-mail
\end{center}



\slide{Referencer: netværksbøger}

\begin{list2}
\item Stevens, Comer,
\item Network Warrior
\item TCP/IP bogen på dansk
\item KAME bøgerne
\item O'Reilly generelt IPv6 Essentials og IPv6 Network Administration
\item O'Reilly cookbooks: Cisco, BIND og Apache HTTPD
\item Cisco Press og website
\item Firewall bøger, Radia Perlman: IPsec,
\end{list2}

\slide{Bøger om IPv6}

\begin{list1}
\item \emph{IPv6 Network Administration}
af David Malone og Niall Richard Murphy
 - god til real-life admins, typisk
O'Reilly bog
\item \emph{IPv6 Essentials} af Silvia Hagen, O'Reilly 2nd edition (May 17, 2006)
	god reference om emnet
\item \emph{IPv6 Core Protocols Implementation}
af Qing Li, Tatuya Jinmei og Keiichi Shima
\item \emph{IPv6 Advanced Protocols Implementation}
af Qing Li, Jinmei Tatuya og Keiichi Shima
\item - flere andre
\end{list1}




\end{document}
