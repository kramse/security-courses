\documentclass[a4paper,11pt,notitlepage,landscape]{report}
% Henrik Kramselund  , February 2001
% hlk@security6.net,
% My standard packages
\usepackage{zencurity-one-page}
\usepackage{tabularx}
%\usepackage{lscape}

\begin{document}

\rm
\selectlanguage{english}

\newcommand{\subject}[1]{Networking and TCP/IP for beginners BornHack 2025}

%\mytitle{SIEM and Log Analysis}{exercises}
%{\LARGE Kickstart: SIEM and Log Analysis}{}
\lhead{\fancyplain{}{\color{titlecolor}\bfseries\LARGE Worksheet: Networking and TCP/IP for beginners}}

\normal

This material is prepared for use in \emph{\subject} and was prepared by Henrik Kramselund, \url{hlk@zencurity.dk}.\\
\link{https://codeberg.org/kramse/security-courses/tree/master/courses/networking/basic-tcpip}

\begin{tabularx}{\textwidth-5cm}{|p{4cm}|p{2cm}|p{7cm}|X|} \hline
{\bf Device } & {\bf Address family} & {\bf Address and Prefixes} & {\bf Default gateway} \\\hline
Laptop & IPv4 & & \\\hline
Laptop & IPv6 & & \\\hline
Phone / Tablet & IPv4 & & \\\hline
Phone / Tablet & IPv6 & & \\\hline
Examplev6  & IPv6 & 2001:DB8:ABCD:0053::/64 & 2001:DB8:ABCD:0053::1 \\\hline
Examplev4  & IPv4 & 192.0.2.0/24 & 192.0.2.1 \\\hline
\end{tabularx}

\begin{tabularx}{\textwidth-5cm}{|p{4cm}|p{2cm}|X|} \hline
{\bf DNS server } & {\bf Address family} & {\bf Description} \\\hline
& IPv4 & \\\hline
& IPv6 & \\\hline
9.9.9.9 149.112.112.112 & IPv4 & Example quad9  \url{https://quad9.net/} \\\hline
2620:fe::fe 2620:fe::9 & IPv6 & Example quad9 \url{https://quad9.net/}\\\hline
\end{tabularx}

Special Addresses \url{https://datatracker.ietf.org/doc/html/rfc5735} and \url{https://www.rfc-editor.org/rfc/rfc5156.html}:\\
\begin{tabularx}{\textwidth-2cm}{|p{3cm}|X|} \hline
{\bf Prefix } & {\bf Description}\\\hline
   10.0.0.0/8, 192.168.0.0/16, 172.16.0.0/12   & These blocks are set aside for use in private networks.
   Its intended use is documented in [RFC1918] \\\hline

127.0.0.1 from 127.0.0.0/8 & This block is assigned for use as the Internet host loopback address. \\\hline

169.254.0.0/16 & The "link local" block.  As described in
   [RFC3927], it is allocated for communication between hosts on a
   single link.  Hosts obtain these addresses by auto-configuration,
   such as when a DHCP server cannot be found. \\\hline
192.0.2.0/24 & This block is assigned as "TEST-NET-1" for use in
   documentation and example code.  It is often used in conjunction with
   domain names example.com or example.net in vendor and protocol
   documentation. \\\hline
::1/128 & IPv6 loopback address [RFC4291] \\\hline
 fe80::/10 & IPv6 link-local unicast [RFC4291] addresses.  Addresses
   within this block should not appear on the public Internet. \\\hline
fc00::/7 & unique-local addresses [RFC4193].  Addresses within
   this block should not appear by default on the public Internet.
   Procedures for advertising these addresses are further described in
   [RFC4193].  \\\hline
   2001:db8::/32 & IPv6 documentation addresses [RFC3849].  They
   are used for documentation purposes such as user manuals, RFCs, etc. \\\hline
\end{tabularx}


\end{document}
