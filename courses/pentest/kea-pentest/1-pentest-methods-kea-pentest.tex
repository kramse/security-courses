\documentclass[Screen16to9,17pt]{foils}
\usepackage{zencurity-slides}

\externaldocument{kea-pentest-exercises}
\selectlanguage{english}

%VF 3 Netværkspenetrationstest (5 ECTS)
% Indhold
%Den studerende lærer om hvordan en penetration test udføres, samt kan indhente oplysninger om de seneste sårbarheder, og kan benytte sig af de relevante værktøjer til dette formål.
%Viden
%Den studerende viden om og forståelse for:
%* Etiske samt kontraktuelle forhold omkring en penetrationstest.
%* Standardiseringorganisationers og myndigheders krav til og om penetrationstesting
%Færdigheder
%Den studerende kan:
%Tage højde for sikkerhedsaspekter ved at:
%* Anvende relevante metoder ved udførsel af en penetrationstest
%* Udarbejde en angrebsplan ud fra indsamlede oplysninger om et mål
%* Finde sårbarheder i et givet system
% * Dokumentere og rapportere fundne sårbarheder
% Kompetencer
%Den studerende kan:
% * Planlægge en penetration test, samt eksekvere den både ved brug af værktøjer og manuelt.


\begin{document}

\mytitlepage
{1. Introduction to hacking and pentest methods}
{KEA Kompetence Penetration Testing 2019}


\slide{Plan for today}

\begin{list1}
\item Subjects
\begin{list2}
\item Terminology and methods

\end{list2}
\item Exercises
\begin{list2}
\item
\item
\end{list2}
\item  Reading Curriculum:
\begin{list2}
\item Grayhat chapters 1 and 6-9
\end{list2}
\item  Reading Related resources:
\begin{list2}
\item
\end{list2}
\end{list1}

\vskip 1cm
\centerline{Do you like the books?}

\slide{Goals for today}
\vskip 2 cm

%{\hlkbig En 3 dages workshop, hvor du lærer at angribe dit netværk!}
\hlkimage{3cm}{dont-panic.png}
\centerline{\color{titlecolor}\LARGE Don't Panic!}


\begin{list1}
\item Introduce the term penetration testing and basic pentest methods
\item Introduce some of the basic tools in this genre of hacker tools
\item Give an insight into the process of doing security testing
\item Create an understanding of hacker tools
\item Show a hacker lab
\end{list1}


\slide{Hackertools}

\begin{list1}
\item \emph{Improving the Security of Your Site by Breaking Into it}\\ by
Dan Farmer and Wietse Venema in 1993
\item Later in 1995 release the software SATAN\\
\emph{Security Administrator Tool for Analyzing Networks}
\item Caused some commotion, panic and discussions, every script kiddie can hack, the internet will melt down!
\vskip 5mm
\begin{quote}
We realize that SATAN is a two-edged sword -- like
many tools, it can be used for good and for evil
purposes. We also realize that intruders (including
wannabees) have much more capable (read intrusive)
tools than offered with SATAN.
\end{quote}
\end{list1}

\vskip 1cm
Source:
\link{http://www.fish2.com/security/admin-guide-to-cracking.html}


\slide{Use hacker tools!}

\begin{list1}
\item Port scan can reveal holes in your defense
\item Web testing tools can crawl through your site and find problems
\item Pentesting is a verification and proactively finding problems
\item Its not a silverbullet and mostly find known problems in existing systems
\item Combined with honeypots they may allow better security
\end{list1}


\slide{Hacker -- cracker}

{\bfseries Short answer -- dont discuss this}

%Det lidt længere svar:\\
Yes, originally there was another meaning to hacker, but the media has perverted it and today, and since early 1990s it has meant breaking into stuff for the public

{\color{red}\hlkbig Today a hacker breaks into systems!}

Reference. Spafford, Cheswick, Garfinkel, Stoll, \ldots
- wrote about this and it was lost

Story is interesting and the old meaning is ALSO used in smaller communities, like hacker spaces full of hackers - doing fun and interesting stuff
\begin{list2}
\item \emph{Cuckoo's Egg: Tracking a Spy Through the Maze of Computer
 Espionage},  Clifford Stoll
\item \emph{Hackers: Heroes of the Computer Revolution},
Steven Levy
\item \emph{Practical Unix and Internet Security},
Simson Garfinkel, Gene Spafford, Alan Schwartz
\end{list2}

\slide{Agreements for testing networks}

\begin{quote}\small
Danish Criminal Code\\
Straffelovens paragraf 263 Stk. 2. Med bøde eller fængsel indtil 1 år og 6 måneder straffes den, der uberettiget skaffer sig adgang til en andens oplysninger eller programmer, der er bestemt til at bruges i et informationssystem.
\end{quote}

Hacking can result in:
\begin{list2}
\item Getting your devices confiscated by the police
\item Paying damages to persons or businesses
\item If older getting a fine and a record -- even jail perhaps
\item Getting a criminal record, making it hard to travel to some countries and working in security
\item Fear of terror has increased the focus -- so dont step over bounds!
\end{list2}

Asking for permission and getting an OK before doing invasive tests, always!

\slide{ISC2 code of ethics}

\hlkimage{23cm}{isc2-code-of-ethics.png}

CISSP certified people sign papers to this extent.\\
\link{https://www.isc2.org/ethics/default.aspx}


\slide{Why even do security testing?}

\begin{list1}
\item Lots of security problems
\item Pentesting may be a requirement from external partners -- example VISA PCI standard
\end{list1}

\begin{list2}
\item Boss asking: should we do a security test?
\item CIO: hmm, okay
\item IT Admins: *sigh* -- I know the security sucks in places!
\item Its not your systems -- dont take the criticism personal, but as an opportunity to get things improved
\end{list2}

\vskip 2cm
\centerline{\Large Many see the benefits after doing a pentest, so try it!}


\slide{Introduction -- terms and technologies}

\begin{list1}
\item Sikkerhedstest / penetrationstest\\
Afprøvning af sikkerhedsforanstaltninger og evaluering af
sikkerhedsniveau ved hjælp af IT systemer og \emph{hackerværktøjer}
\item Kaldes tillige sårbarhedstest, sårbarhedsanalyse m.v.
\item Ekstern -- udføres fra internet, typisk over WAN
\item Intern, inside, on-site -- udføres hos kunden, typisk over LAN og
  bag firewall
\end{list1}

\link{https://www.google.com/search?q=sikkerhedstest}

\slide{Blackbox, greybox og whitebox}

\begin{list2}
\item Forudsætninger og forudgående kendskab til miljøet
\item Black Box testen involverer en sikkerhedstestning af et netværk uden
nogen form for insider viden om systemet udover den IP-adresse, der
ønskes testet. Dette svarer til den situation en fjendtlig hacker vil
stå i og giver derfor det mest realistiske billede af netværkets
sårbarhed overfor angreb udefra. Men er dårlig ressourceudnyttelse.
\item I den anden ende  af skalaen har vi White Box testen. I dette tilfælde
har sikkerhedsspecialisten både før og under testen fuld adgang til
alle informationer om det scannede netværk. Analysen vil derfor kunne
afsløre sårbarheder, der ikke umiddelbart er synlige for en almindelig
angriber. En White Box test er typisk mere omfattende end en Black Box
test og forudsætter en højere grad af deltagelse fra kundens side, men
giver en meget detaljeret og tilbundsgående undersøgelse.

\item En Grey Box test er som navnet siger et kompromis mellem en White Box
og en Black Box test. Typisk vil sikkerhedsspecialisten udover en
IP-adresse være i besiddelse af de mest grundlæggende
systemoplysninger: Hvilken type af server der er tale om (mail-,
webserver eller andet), operativsystemet og eventuelt om der er
opstillet en firewall foran serveren.
\end{list2}


\slide{Benefits of having a planned security test done}

\begin{quote}
Goal of testing is to reduce risk for the systems and secure the organisation\\ from unexpected loss of data, image and increased costs.
\end{quote}

\begin{list1}
\item Målgrupper:
\begin{list2}
\item IT-afdeling og teknisk personale
\item Ledelse, koncernledelse
\item Eksterne revisorer, VISA PCI, offentligheden
\end{list2}
\item Afleveringer:
\begin{list2}
\item Rapport med tekniske anbefalinger og opsummering/checklister
\item Executive summary
\end{list2}
\end{list1}

Goal is not to find a scape goat to blame -- management allocates resources

If security is below in places more resources may be needed.


\slide{Persongalleri, Godkendelse og tilladelse}


\begin{list1}
\item Sikkerhedskonsulent -- den konsulent der kommer ud til kunden
\item Inden en test kan udføres skal der indhentes tilladelser fra:
\begin{list2}
\item Systemejer -- den ansvarlige for et bestemt system
\item Netværksejer -- den ansvarlige for netværk hos kunden
\item Driftorganisation -- dem der driver systemerne
\item Sikkerhedsansvarlig -- den ansvarlige for sikkerheden hos kunden
\item Kontaktperson udpeges -- kundens ansatte som kan hjælpe med praktiske
  spørgsmål og skabe kontakt til de rette personer i kundens organisation
\end{list2}
\end{list1}

\slide{Planlægning af sikkerhedstest}

\begin{list1}
\item Sårbarhedsanalysens omfang aftales på forhånd
\begin{list2}
\item Scope -- hvad skal testes
\item Hvornår skal testes -- indenfor et aftalt tidsrum, wall clock time
\item Hvor testes fra -- logfilerne vil afsløre IP-adresser
\item Kan overskrides delvist -- eksempelvis ved port 80 scan på samme
  subnet eller tilsvarende
\item Skal der forsøges ude af drift angreb -- DoS
\item Se endvidere slide om Rules of engagement senere
\end{list2}
\item {\bf Sårbarhedsanalysen omfatter (targets):}
\begin{list2}
\item 192.168.1.1 -- firewall/router
\item 192.168.1.2 -- mailserver
\item 192.168.1.3 -- webserver
\item Testen udføres i tidsrummet mandag 1. til fredag 5.
\item Testere udfører \emph{angreb} fra 192.0.2.0/28
\end{list2}
\end{list1}


\slide{Før konsulenten ankommer -- forberedelse}

\begin{list1}
\item Testplan med oversigt over targets og IP-adresser
\item Netværkstegninger og anden information som er aftalt oplyst
\item Hvor skal sikkerhedskonsulenten placeres ved insidetest -- ikke i serverrum, tak :-)
\item Kabling af netværksstik
\item Gæstekort -- til test over flere dage
\item Kantine, toiletter osv.
\end{list1}
\vskip 1cm
\centerline{Betragt det som en ny kollega -- med tidsbegrænset kontrakt}

\slide{Udvælgelse af systemer til test}

\hlkimage{11cm}{overview-routing-customer-2015.png}

\begin{list2}
\item Routere på netværksvejen til kritiske systemer og netværk -
  tilgængelighed
\item Firewall -- begrænses trafikken tilstrækkeligt
\item Mailservere -- tillades relaying udefra
\item Webservere -- kan der afvikles kode på systemet, downloades data
\end{list2}


\slide{Scannerudstyr på insidetest}

\vskip 2 cm
\begin{quote}
Scannersystemer, hardware og software kræver en del ekspertice og
opsætning. Det er tidskrævende at foretage denne opsætning og
konsulenten har på forhånd udvalgt og konfigureret udstyr til testen.
Det skal derfor accepteres at konsulenten tilslutter eget udstyr til
de pågældende netværk og dette sker naturligvis under strenge krav til
konsulentens udstyr.
\end{quote}
\vskip 2 cm
\centerline{\bf Det er ikke en mulighed at bruge kundens udstyr!}

\slide{Testens udførelse}

\begin{list1}
\item Testen udføres ved samarbejde mellem konsulent og virksomhed
\item Først og fremmest skal testen startes
\begin{list2}
\item Når konsulenten ankommer kontaktes kontaktpersonen
\item Konsulenten vises til rette og pakker ud/stiller op
\item Såfremt det ønskes inspiceres og godkendes udstyret
\item Konsulenten tilslutter sig netværket og test er officielt igang
\item Konsulenten verificerer adgangen til netværk og melder klar,
  begynder test
\end{list2}
\item ... tiden går ... testen udføres ...
\item Kontaktpersonen er hele tiden til rådighed på mobiltelefon
\item Testen afsluttes og der pakkes ned i modsat rækkefølge
\end{list1}


\slide{Afbrydelse af testen -- kompromitterede maskiner}

\begin{list1}
\item Der kan være årsager der medfører at testen skal indstilles
\item Sikkerhedskonsulenten afbryder testen
\begin{list2}
\item Det anses for uforsvarligt at fortsætte, der er fundet
  kompromitterede systemer eller beviser der kan ødelægges
\item Netværket er dårligt, mulighederne for udførelse er forringet
\end{list2}
\item Kunden ønsker at afbryde testen
\begin{list2}
\item Der opleves for store problemer under udførelsen
\item Systemnedbrud på forretningskritiske systemer
\item Andre kriser der gør det valgte tidspunkt uegnet
\end{list2}
\item NB: Eksempler! -- man afbryder altid når kunden ønsker det!
\end{list1}

\slide{Oprydning efter testen}

\begin{list1}
\item Sikkerhedskonsulenten er ansvarlig for:
\begin{list2}
\item Fjerne data fra systemerne
\item Fjerne brugerkonti, få fjernet brugeroplysninger og
  loginmuligheder
\item Fjerne software som ikke skal benyttes mere
\end{list2}
\item Driftsorganisationen er ansvarlig for:
\begin{list2}
\item Undersøgelse af systemerne
\item Eventuel genstart af systemer, der kan være nedsat effektivitet
\item Fjerne patchkabler for stik der er kablet speciet til konsulenten
\end{list2}
\end{list1}

\slide{Afrapportering -- resultater}

\begin{list1}
\item Hvad indeholder en sikkerhedstest rapport:
\begin{list2}
\item Titel, indholdsfortegnelse, firmanavne -- ca. 15-30 sider for 5 hosts
\item Fortrolighedserklæring -- det er fortrolige oplysninger
\item Executive summary -- ofte i større virksomheder
\item Information om den udførte scanning
\item Omfang/scope
\item Gennemgang af targets -- detaljeret information og med anbefalinger
\item Konklusion -- ofte mere teknisk
\item Bilag -- detaljerede oplysninger og oversigter, checklister
\end{list2}
\item Det er organisationen der selv vælger hvilke anbefalinger der følges
\end{list1}


\slide{Rules of engagement -- regler og etik for sikkerhedstest}

\begin{list2}
\item NB: Stor forskel på Danmark og udlandet!
\item Sikkerhedskonsulenten må ikke give anledning til nye sårbarheder
  som følge af testen
\item Sikkerhedskonsulenten må ikke installere ny software på
  systemer uden forudgående aftale
\item Sikkerhedskonsulenten efterlader ikke usikre
  systemadministratorkonti eller tilsvarende efter testen
\item Sikkerhedskonsulenten tager altid kontakt til kunden ved
  høj-risiko sårbarheder
\item Er man hyret til netværkssikkerhed kan man godt \emph{snuse}
  lidt rundt om systemerne under test -- der kan være et sårbart
  testsystem lige ved siden af
\item Min holdning er at ved opdagelse af åbenlyse sikkerhedsrisici
  dokumenteres disse i rapporten, uanset scope for opgaven ellers
\end{list2}

\centerline{Det er en balancegang}



\slide{Konsulentens udstyr -- vil du være sikkerhedskonsulent}

\begin{list1}

\item Laptops, gerne flere, men én er nok til at lære!
\begin{list2}
\item Sikkerhedskonsulenterne bruger typisk Open Source værktøjer på Linux og
enkelte systemer med Windows -- jeg bruger helst Windows 7 i dag
\item Netværkserfaring \emph{TCP/IP protocol suite} -- TCP, UDP, ICMP osv. i detaljer
\item Programmmeringserfaring er en fordel
\item Linux/Unix kendskab er ofte en {\bfseries nødvendighed}\\
- fordi de nyeste værktøjer er skrevet til Unix i form af Linux og BSD
\item \emph{A Hands-On Introduction to Hacking
by Georgia Weidman}, June 2014\\
 \link{http://www.nostarch.com/pentesting}
\item Metasploit Unleashed -- gratis kursus i Metasploit\\
\link{https://www.offensive-security.com/metasploit-unleashed/}
\end{list2}
\end{list1}


\slide{Hackerværktøjer}
% måske til reference afsnit?
\hlkimage{3cm}{hackers_JOLIE+1995.jpg}

\begin{list2}
\item Alle bruger nogenlunde de samme værktøjer, se også \link{http://www.sectools.org/}
\item Portscanner Nmap, Nping -- tester porte, godt til firewall admins \link{https://nmap.org}
\item Generel sårbarhedsscanner Metasploit Framework \link{https://www.metasploit.com/}
\item Specielle scannere -- wifi Aircrack-ng, web Burpsuite, Nikto, Skipfish \link{http://portswigger.net/burp/}
\item Wireshark avanceret netværkssniffer -- \link{https://www.wireshark.org/}
\item og scripting, PowerShell, Unix shell, Perl, Python, Ruby, \ldots
\end{list2}

Billedet: Angelina Jolie fra Hackers 1995


\slide{Hvad skal der ske?}

\begin{list1}
\item Tænk som en hacker
\item Rekognoscering
\begin{list2}
\item ping sweep, port scan
\item OS detection -- TCP/IP eller banner grab
\item Servicescan -- rpcinfo, netbios, ...
\item telnet/netcat interaktion med services
\end{list2}
\item Udnyttelse/afprøvning: Metasploit, Nikto, exploit programs
\item Oprydning/hærdning vises måske ikke, men I bør i praksis:
\begin{list2}
\item Lav en rapport
\item Ændre, forbedre og hærde systemer
\item Gennemgå rapporten, registrer ændringer
\item Opdater programmer, konfigurationer, arkitektur, osv.
\end{list2}
\item I skal jo også VISE andre at I gør noget ved sikkerheden.
\end{list1}


\slide{Hackerlab opsætning}

\hlkimage{8cm}{hacklab-1.png}

\begin{list2}
\item Hardware: en moderne laptop med CPU der kan bruge virtualisering\\
Husk at slå virtualisering til i BIOS
\item Software: dit favoritoperativsystem, Windows, Mac, Linux
\item Virtualiseringssoftware: VMware, Virtual box, vælg selv
\item Hackersoftware: Kali som Virtual Machine \link{https://www.kali.org/}
\item Soft targets: Metasploitable, Windows 2000, Windows XP, ...
\end{list2}


\slide{Teknisk hvad er hacking}

\hlkimage{12cm}{buffer-overflow-3.pdf}


\slide{Internet i dag}

\hlkimage{10cm}{images/server-client.pdf}

\begin{list1}
\item Klienter og servere
\item Rødder i akademiske miljøer
\item Protokoller der er op til 20 år gamle
\item Meget lidt kryptering, mest på http til brug ved e-handel
\end{list1}

\slide{Trinity breaking in}

\hlkimage{14cm}{trinity-nmapscreen-hd-cropscale-418x250.jpg}
Meget realistisk - sådan foregår det næsten:\\
\link{https://nmap.org/movies/}\\
\link{https://youtu.be/51lGCTgqE_w}



\slide{Hacking er magi}

\hlkimage{5cm}{wizard_in_blue_hat.png}

\vskip 1 cm

\centerline{Hacking ligner indimellem  magi}


\slide{Hacking er ikke magi}

\hlkimage{15cm}{ninjas.png}

\vskip 1 cm
\centerline{Hacking kræver blot lidt ninja-træning}

\slide{Hacking eksempel -- det er ikke magi}

\begin{list1}
\item MAC filtrering på trådløse netværk
\item Alle netkort har en MAC adresse -- BRÆNDT ind i kortet fra fabrikken
\item Mange trådløse Access Points kan filtrere MAC adresser
\item Kun kort som er på listen over godkendte adresser tillades adgang til netværket

\item Det virker dog ikke \smiley
\item De fleste netkort tillader at man overskriver denne adresse midlertidigt
\item og man kan aflæse de godkendte når de er aktive på netværket
\item Derudover har der ofte været fejl i implementeringen af MAC filtrering
\end{list1}

\slide{Myten om MAC filtrering}

\begin{list1}
\item Eksemplet med MAC filtrering er en af de mange myter
\item Hvorfor sker det?
\item Marketing -- producenterne sætter store mærkater på æskerne
\item Manglende indsigt -- forbrugerne kender reelt ikke koncepterne
\item Hvad \emph{er} en MAC adresse egentlig
\item Relativt få har forudsætningerne for at gennemskue dårlig sikkerhed
\item Løsninger?

\item Udbrede viden om usikre metoder til at sikre data og computere
\item Udbrede viden om sikre metoder til at sikre data og computere
\end{list1}

\slide{MAC filtrering}

\hlkimage{12cm}{stupid-security.jpg}


\slide{OSI og Internet modellerne}

\hlkimage{10cm,angle=90}{images/compare-osi-ip.pdf}

\slide{Kali Linux the pentest toolbox}

\hlkimage{14cm}{kali-linux.png}

\begin{list1}
\item  Kali \link{http://www.kali.org/}
\item 100.000s of videos on youtube alone, searching for kali and \$TOOL
\item Also versions for Raspberry Pi, mobile and other small computers
\end{list1}

\slide{Really do Nmap your world}

\hlkimage{8cm}{nmap-zenmap.png}

\begin{list2}
\item Nmap is a port scanner, but does more
\item Finding your own infrastructure available from the guest network?
\item See your printers having all the protocols enabled AND a wireless?
\end{list2}


\slide{Getting to your data: Google for it}

\hlkimage{10cm}{images/googledorks-1.pdf}

\begin{list2}
\item Google as a hacker tool? oprindeligt beskrevet af Johnny Long
\item Concept named googledorks when google indexes information not supposed to be public
\item \link{http://www.exploit-db.com/google-dorks/}
\end{list2}


\slide{Security devops}

\begin{list1}
\item We need devops skillz in security
\item automate, security is also big data
\item integrate tools, transfer, sort, search, pattern matching, statistics, ...
\item tools, languages, databases, protocols, data formats
\item Use Github! Der er så mange biblioteker og programmer, noget eksisterende løser måske dit problem 90%
\item Example introductions:
\begin{list2}
\item Seven languages/database/web frameworks in Seven Weeks
\item Elasticsearch the definitive guide
\end{list2}
\end{list1}

\centerline{We are all Devops now, even security people!}


\exercise{ex:pentest-report}


\slidenext{}

\end{document}
