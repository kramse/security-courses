\documentclass[a4paper,11pt,notitlepage]{report}
% Henrik Lund Kramshoej, February 2001
% hlk@security6.net,
% My standard packages
\usepackage{zencurity-network-exercises}

\begin{document}

\rm
\selectlanguage{english}

\newcommand{\emne}[1]{hacker workshop}
\newcommand{\kursus}[1]{ethical hacker workshop}
\newcommand{\kursusnavn}[1]{ethical hacker workshop\\ exercises}

\mytitle{Nmap Hackerworkshop}{exercises}

\pagenumbering{roman}


\setcounter{tocdepth}{0}

\normal

{\color{titlecolor}\tableofcontents}
%\listoffigures - not used
%\listoftables - not used

\normal
\pagestyle{fancyplain}
\chapter*{\color{titlecolor}Preface}
\markboth{Preface}{}

This material is prepared for use in \emph{\kursus} and was prepared by
Henrik Lund Kramshoej, \link{http://www.zencurity.com} .
It describes the networking setup and
applications for trainings and workshops where hands-on exercises are needed.

\vskip 1cm
Further a presentation is used which is available as PDF from kramse@Github\\
Look for \jobname in the repo security-courses.

These exercises are expected to be performed in a training setting with network connected systems. The exercises use a number of tools which can be copied and reused after training. A lot is described about setting up your workstation in the repo

\url{https://github.com/kramse/kramse-labs}



\section*{\color{titlecolor}Prerequisites}

This material expect that participants have a working knowledge of
TCP/IP from a user perspective. Basic concepts such as web site addresses and email should be known as well as IP-addresses and common protocols like DHCP.

\vskip 1cm
Have fun and learn
\eject

% =================== body of the document ===============
% Arabic page numbers
\pagenumbering{arabic}
\rhead{\fancyplain{}{\bf \chaptername\ \thechapter}}

% Main chapters
%---------------------------------------------------------------------
% gennemgang af emnet
% check questions


\chapter*{\color{titlecolor}Introduction to networking}
%\markboth{Introduktion til netværk}{}
\label{chap:intro}

\section*{\color{titlecolor}IP - Internet protocol suite}

It is extremely important to have a working knowledge about IP to implement
secure and robust infrastructures. Knowing about the alternatives while doing
implementation will allow the selection of the best features.

\section*{\color{titlecolor}ISO/OSI reference model}
A very famous model used for describing networking is the ISO/OSI model
of networking which describes layering of network protocols in stacks.

This model divides the problem of communicating into layers which can
then solve the problem as smaller individual problems and the solution
later combined to provide networking.

Having layering has proven also in real life to be helpful, for instance
replacing older hardware technologies with new and more efficient technologies
without changing the upper layers.

In the picture the OSI reference model is shown along side with
the Internet Protocol suite model which can also be considered to have different layers.


\begin{figure}[H]
\label{fig:osi}
\begin{center}
\colorbox{white}{\includegraphics[width=8cm,angle=90]{images/compare-osi-ip.pdf}}
\end{center}
\caption{OSI og Internet Protocol suite}
\end{figure}


\chapter*{\color{titlecolor}Exercise content}
\markboth{Exercise content}{}

Most exercises follow the same procedure and has the following content:
\begin{itemize}
\item {\bf Objective:} What is the exercise about, the objective
\item {\bf Purpose:} What is to be the expected outcome and goal of doing this exercise
\item {\bf Suggested method:} suggest a way to get started
\item {\bf Hints:} one or more hints and tips or even description how to
do the actual exercises
\item {\bf Solution:} one possible solution is specified
\item {\bf Discussion:} Further things to note about the exercises, things to remember and discuss
\end{itemize}

Please note that the method and contents are similar to real life scenarios and does not detail every step of doing the exercises. Entering commands directly from a book only teaches typing, while the exercises are designed to help you become able to learn and actually research solutions.



\chapter{Wireshark installation}
\label{ex:wireshark-install}

\hlkimage{10cm}{wireshark-http.png}


{\bf Objective:}\\
Install the program Wireshark locally your workstation

If you already have Kali installed you have Wireshark. Done.

{\bf Purpose:}\\
Installing Wireshark will allow you to analyse packets and protocols

{\bf Suggested method:}\\
Download and install the program, either download from web server locally or from \link{http://www.wireshark.org}\\
Wireshark requires a packet capture library to be installed

{\bf Hints:}\\
PCAP is a packet capture library allowing you to read packets from the network. Wireshark is a graphical application to allow you to browse through traffic, packets and protocols.

{\bf Solution:}\\
When Wireshark is installed sniff some packets, also see next exercise.

{\bf Discussion:}\\
Wireshark is just an example other packet analyzers exist, some commercial and some open source like Wireshark


\chapter{Nmap installation}

\label{ex:nmap-install}

\hlkimage{10cm}{wireshark-http.png}


{\bf Objective:}\\
Install the package of programs locally your workstation

If you already have Kali installed you have Wireshark. Done

{\bf Purpose:}\\
Installing the Nmap package will allow you to use various tools on a daily basis

{\bf Suggested method:}\\
Download and install the program, either download from web server locally or from \link{http://www.nmap.org}\\

{\bf Hints:}\\
Nmap includes more than just the nmap tool. Nping is an awesome tool to test connectivity using multiple protocols

{\bf Solution:}\\
When the package is installed we are ready for the next steps

{\bf Discussion:}\\
There are other port scanners, some stateless and others stateful like Nmap

\chapter{Lookup Whois data}
\label{ex:whois}


{\bf Objective:} \\
Learn to use Whois databases

{\bf Purpose:}\\
Knowing who to contact in case of problems on the internet is important, and also verifying before starting scanning is required.

{\bf Suggested method:}\\
Use the website of RIPE NCC \url{https://www.ripe.net/} or their other site

\url{https://stat.ripe.net/}

{\bf Hints:}\\
Whois databases are distributed to Regional Internet Registries such as ARIN, AfriNIC, RIPE, LACNIC and APNIC.

{\bf Solution:}\\
If you are using Linux or Mac you have a command line tool too:\\
Use the command whois with an IP address, \verb+whois 185.129.60.130+.

{\bf Discussion:}\\
The whois system was implemented after the Morris Worm affected the internet in November 1988, because it was realized that the internet had grown to a size that required more management.





\chapter{Sniffing network packets}
\label{ex:wireshark-sniff}

\hlkimage{6cm}{tcp-three-way.pdf}


{\bf Objective:}\\
Sniff packets and dissect them using Wireshark

{\bf Purpose:}\\
See real network traffic, also know that a lot of information is available and not encrypted.

Note the three way handshake between hosts

{\bf Suggested method:}\\
Open Wireshark and start a capture\\
Then in another window execute the ping program while sniffing

or perform a Telnet connection while Wireshark sniff data

{\bf Hints:}\\
When running on Linux the network cards are usually named eth0 for the first Ethernet and wlan0 for the first Wireless network card. In Windows the names of the network cards are long and if you cannot see which cards to use then try them one by one.

{\bf Solution:}\\
When you have collected some packets you are done.

{\bf Discussion:}
Is it ethical to collect packets from an open wireless network?

Also note the TTL values in packets from different operating systems


\chapter{Nmap basic Quick Scan}

Quick scan

Formål,
* Se Zenmap
* Opdage Tabs med hosts og services
* Se farvekodning
* Forskellen mellem 10.0.45.123/32 og 10.0.45.0/24 og 10.0.45.0/25 og 10.0.45.1-10
* Se de indbyggede profiler


\chapter{Discover active systems ping sweep}
\label{ex:nmap-pingsweep}
\hlkimage{10cm}{nmap-zenmap.png}

{\bf Objective:}\\
Use nmap to discover active systems

{\bf Purpose:}\\
Know how to use nmap to scan networks for active systems.

{\bf Suggested method:}\\
Try different scans,
\begin{itemize}
\item Ping sweep to find active systems
\item Port sweeps to find active systems with specific ports
\end{itemize}

{\bf Hints:} \\
Try nmap in sweep mode

{\bf Solution:}\\
Use the command below as examples:
\begin{itemize}
\item Ping sweep \verb+nmap -sP 10.0.45.*+
\item Port sweeps \verb+nmap -p 80 10.0.45.*+
\end{itemize}

{\bf Discussion:}\\

{\bf You can also use the graphical interface to nmap called Zenmap.}


\chapter{Execute nmap TCP and UDP port scan}
\label{ex:nmap-synscan}


{\bf Objective:} \\
Use nmap to discover open ports on active systems

{\bf Purpose:}\\
Finding open ports will allow you to find vulnerabilities on these ports.

{\bf Suggested method:}\\
Use \verb+nmap -p 1-1024 server+ to scan the first 1024 TCP
ports

Try to use \verb+nmap -sU+ to scan using UDP ports, not really possible if a firewall is in place.

If a firewall blocks ICMP you might need to add \verb+-P0+
or even \verb+-PN+ to make nmap scan even if there are no Ping responses

{\bf Hints:} \\
Sample command: \verb+nmap -P0 -sU -p1-1024 server+ UDP port scanning
1024 ports without doing a Ping first

{\bf Solution:}\\
Discover some active systems and you are done.

{\bf Discussion:}\\
There is a lot of documentation about the nmap portscanner, even a book by the author
of nmap. Make sure to visit \link{http://www.nmap.org}

TCP and UDP is very different when scanning. TCP is connection/flow oriented and requires a handshake which is very easy to identify. UDP does not have a handshake and most applications will not respond to probes from nmap. If there is no firewall the operating system will respond to UDP probes on closed ports - and the ones that do not respond must be open.

When doing UDP scan on the internet you will almost never get a response, so you cannot tell open (not responding services) from blocked ports (firewall drop packets). Instead try using specific service programs for the services, sample program could be \verb+nsping+ which sends DNS packets, and will often get a response from a DNS server running on UDP port 53.

\chapter{Perform nmap OS detection}
\label{ex:nmap-os}

{\bf Objective:} \\
Use nmap OS detection and see if you can guess the devices on the network

{\bf Purpose:}\\
Getting the operating system of a system will allow you to focus your next attacks.

{\bf Suggested method:}\\
Look at the list of active systems, or do a ping sweep.

Then add the OS detection using the option \verb+-O+

{\bf Hints:} \\
Use the manual page

The nmap can send a lot of packets that will get different responses, depending on the operating system.

{\bf Solution:}\\
Use a command like \verb+nmap -O -p1-100 10.0.45.45+


{\bf Discussion:}\\
nmap OS detection is not a full proof way of knowing the actual operating system, but in most cases in can detect the family and in some cases it can identify the exact patch level of the system.

Another tool which does the same is Xprobe.

\chapter{Perform nmap service scan}
\label{ex:nmap-service}

{\bf Objective:} \\
Use more advanced features in nmap to discover services.

{\bf Purpose:}\\
Getting more intimate with the system will allow more precise discovery of the vulnerabilities and also allow you to select the next tools to run.

{\bf Suggested method:}\\
Use \verb+nmap -A+ option for enabling service detection

{\bf Hints:} \\
Look into the manual page of nmap or the web site book about nmap scanning

{\bf Solution:}\\
Run nmap and get results.

{\bf Discussion:}\\

Some services will show software versions allowing an attacker easy lookup at web sites to known vulnerabilities and often exploits that will have a high probability of success.

Make sure you know the difference between a vulnerability which is discovered, but not really there, a false positive, and a vulnerability not found due to limitations in the testing tool/method, a false negative.

A sample false positive might be reporting that a Windows server has a vulnerability that you know only to exist in Unix systems.


\chapter{Lav en specifik Nmap profil}

Prøv at scanne en række porte, 1-1024     et antal udvalgte, 22,23,80,443,8080



\chapter{E) UDP scan på LAN}

Lav UDP scan på lokalnet, closed = ICMP aktivt svar, open = open|filtered

Bonus:
Hvorfor virker det ikke ude i verden, firewalls
Koblet med Pingscan, hvad siger det så

Efter
The six port states recognized by Nmap

\chapter{Fuld scan}
, hvad betyder det

Formål
* Vise at Nmap / portscan er en process, mere end

1) quick scan
2) Advanced, alle porte TCP -p 1-65535 med -A
3) -p 100 -sU UDP scan
4) ...
Husk
-P0
-Pn: Treat all hosts as online -- skip host discovery


5) Specific source ports -g -g/--source-port <portnum>: Use given port number FTP 20, DNS 53

Hvornår bruges fuld scan versus enkelte scans på tværs af subnets

Bonus
-iL input liste af prefixes
 -iL <inputfilename>: Input from list of hosts/networks

Vi designer sammen en process og taler om resultaterne

\chapter{Hvad indeholder Nmap idag}

Vi gennemgår seneste release notes, 7.70 pt.
http://seclists.org/nmap-announce/2018/0

Bonus tal om hvordan man submitter fingerprints til Nmap

\chapter{Reporting HTML}

Få resultaterne pænt ud af Nmap

Brug option -oA - alle formater

Formål
* Forstå fordele ved GNMAP og XML

xsltproc xml html

			 --stylesheet <path/URL>: XSL stylesheet to transform XML output to HTML
			 --webxml: Reference stylesheet from Nmap.Org for more portable XML
Ndiff


\chapter{Find systems with SNMP}
\label{ex:find-snmp}

{\bf Objective:}\\
Use snmpwalk to research SNMP systems

{\bf Purpose:}\\
Learn that gathering information can help an attacker.

{\bf Suggested method:} \\
Log into the Unix server provided and run snmpwalk which is using UDP port 161.

{\bf Hints:} \\
We are running in a LAN environment with less firewalls, so doing nmap UDP scan is possible.

When discovering an IP then use the \verb+snmpwalk+
  program to show a lot of information.


{\bf Solution:}\\
\begin{itemize}
\item Use the command \verb+snmpwalk -v 2c -c public 10.0.45.34 | less +\\
\end{itemize}

The command less will show output one screen at a time.

{\bf Discussion:}\\
In real networks SNMP is being used a lot, but new equipment is starting NOT to allow access using the community string public.



\chapter{Nmap Scripting Engine NSE scripts}

Lav en profil til Heartbleed, og SSL version 2 og version 3 scan
Formål
* Der dukker hele tiden systemer op med de gamle SSL versioner :-(
* Forstå hvad et NSE script er
* Forstå hvad NSE scripts er, banner grabbing, Advanced scan

Eksempler på real-life scenarioer
 nmap -sU -p 53 --script=dns-recursion -iL targets -oA dns-recursive

\begin{alltt}
bgpscan.gnmap:# Nmap 7.70 scan initiated Thu May 31 12:22:49 2018 as: nmap -A -p 179 -oA bgpscan -iL targets
dns-recursive.gnmap:# Nmap 7.70 scan initiated Mon Sep  3 12:24:48 2018 as: nmap -sU -p 53 --script=dns-recursion -iL targets -oA dns-recursive
php-scan.gnmap:# Nmap 7.70 scan initiated Mon Jul 30 15:17:50 2018 as: nmap -sV --script=http-php-version -p80,443 -oA php-scan -iL targets
scan-vtep-tcp.gnmap:# Nmap 7.70 scan initiated Wed May 16 12:05:03 2018 as: nmap -A -p 1-65535 -oA scan-vtep-tcp 109.105.96.77 109.105.96.78
snmp-109.105.96.0.gnmap:# Nmap 7.70 scan initiated Thu Jun 28 09:34:24 2018 as: nmap -sV -A -p 161 -sU --script=snmp-info -oA snmp-109.105.96.0 109.105.96.0/19
snmp-185.174.116.0.gnmap:# Nmap 7.70 scan initiated Thu Jun 28 09:34:24 2018 as: nmap -sV -A -p 161 -sU --script=snmp-info -oA snmp-185.174.116.0 185.174.116.0/22
snmp-192.36.171.0.gnmap:# Nmap 7.70 scan initiated Thu Jun 28 09:34:24 2018 as: nmap -sV -A -p 161 -sU --script=snmp-info -oA snmp-192.36.171.0 192.36.171.0/24
snmp-193.10.252.0.gnmap:# Nmap 7.70 scan initiated Thu Jun 28 09:34:24 2018 as: nmap -sV -A -p 161 -sU --script=snmp-info -oA snmp-193.10.252.0 193.10.252.0/24
snmp-193.10.254.0.gnmap:# Nmap 7.70 scan initiated Thu Jun 28 09:34:24 2018 as: nmap -sV -A -p 161 -sU --script=snmp-info -oA snmp-193.10.254.0 193.10.254.0/24
snmp-193.10.68.0.gnmap:# Nmap 7.70 scan initiated Thu Jun 28 09:34:24 2018 as: nmap -sV -A -p 161 -sU --script=snmp-info -oA snmp-193.10.68.0 193.10.68.0/24
snmp-193.10.94.0.gnmap:# Nmap 7.70 scan initiated Thu Jun 28 09:34:24 2018 as: nmap -sV -A -p 161 -sU --script=snmp-info -oA snmp-193.10.94.0 193.10.94.0/24
snmp-193.10.95.0.gnmap:# Nmap 7.70 scan initiated Thu Jun 28 09:34:24 2018 as: nmap -sV -A -p 161 -sU --script=snmp-info -oA snmp-193.10.95.0 193.10.95.0/24
snmp-193.11.3.0.gnmap:# Nmap 7.70 scan initiated Thu Jun 28 09:34:24 2018 as: nmap -sV -A -p 161 -sU --script=snmp-info -oA snmp-193.11.3.0 193.11.3.0/24
snmp-194.68.13.0.gnmap:# Nmap 7.70 scan initiated Thu Jun 28 09:34:26 2018 as: nmap -sV -A -p 161 -sU --script=snmp-info -oA snmp-194.68.13.0 194.68.13.0/24
snmpscan.gnmap:# Nmap 7.70 scan initiated Thu May 24 14:16:12 2018 as: nmap -sU -p 161 -oA snmpscan --script=snmp-interfaces -iL targets
sshscan.gnmap:# Nmap 7.70 scan initiated Thu May 24 13:30:02 2018 as: nmap -A -p 22 -oA sshscan -iL targets
vncscan.gnmap:# Nmap 7.70 scan initiated Thu May 24 13:52:56 2018 as: nmap -A -p 5900-5905 -oA vncscan -iL targets
webscan-110.gnmap:# Nmap 7.70 scan initiated Wed May 16 13:16:37 2018 as: nmap -A -oA webscan-110 109.105.110.0/24
\end{alltt}


Bonus
Sjove scripts / andre scripts
sudo nmap --traceroute --script traceroute-geolocation.nse -p 80 hackertarget.com

nmap --script http-enum 192.168.10.55

Til dem som har Windows med :-D og glemt at slå firewall til
nmap -p 445 --script smb-os-discovery 192.168.1.0/24


\chapter{Kopier et NSE script ind}
Find NSE script, kopier dette ind, kør Nmap med dette, enten via CLI eller Zenmap

nmap --script "http-*"

		   nmap --script "default or safe"
			   This is functionally equivalent to nmap --script "default,safe". It loads all scripts that are in the default category or the safe category or both.

		   nmap --script "default and safe"
			   Loads those scripts that are in both the default and safe categories.


nmap -script-help http-vuln-cve2013-0156.nse

\chapter{Skriv vores eget NSE script}

Vi starter en process med et bestemt indhold, laver et NSE script til detektion af dette

Måske bruge DNS signeret som eksempel?
Findes der til den der LDAP UDP ting?

Lettere at modificere et eksisterende måske

https://nmap.org/nsedoc/scripts/http-default-accounts.html

--script-trace

https://nmap.org/book/nse-usage.html
\begin{alltt}
  nmap -sC --script-args 'user=foo,pass=",{}=bar",paths={/admin,/cgi-bin},xmpp-info.server_name=localhost'
\end{alltt}

Eksempel, lav vores egen der leder efter bestemt version af OpenSSH, ref
https://isc.sans.edu/forums/diary/OpenSSH+user+enumeration+CVE201815473/24004/
https://blog.nviso.be/2018/08/21/openssh-user-enumeration-vulnerability-a-close-look/

og

CVE-2018-15919 en anden, som der ikke er fix til pt.

Eller lede efter PHP 5.5, som havde EOL 21 Jul 2016 !

https://nmap.org/nsedoc/scripts/http-php-version.html


\chapter{Try the bind-version shell script}
\label{ex:bind-version-script}

{\bf Objective:}
Try to use a shell script to automate lookups

{\bf Purpose:}\\
When doing actual security testing you should automate as much as possible.

{\bf Suggested method:}
Login to the Unix server provided and run the bind-version script

{\bf Hints:}
Unix files with \verb+#!+ as the first line will be executed using the command specified.


{\bf Solution:}\\
Run the script provided

{\bf Discussion:}
The script only does a few DNS lookups, but more elaborate scripts are being used daily by administrators, security consultants and hackers.

The script available on the system is:

\begin{alltt}\small
\input{bindversion}
\end{alltt}


\chapter{Try the dns-timecheck Perl program}
\label{ex:dns-timecheck}

{\bf Objective:}
Try to use a Perl script to communicate with a binary protocol

{\bf Purpose:}\\
See that programming languages such as Perl often include a lot of libraries which allow efficient implementation of ideas.

{\bf Suggested method:}
Login to the Unix server provided and run the dns-timecheck script

{\bf Hints:}
Perl can be a bit difficult to read, but a lot of tutorials exist

{\bf Solution:}\\

{\bf Discussion:}
While Perl has been around for lots of years it seems that security tools are often implemented using newer languages like Ruby and Python

The script available on the system is:
\begin{alltt}\small
  #!/usr/bin/perl
  # modified from original by Henrik Kramshoej, hlk@kramse.dk
  # 2004-08-19
  #
  # Original from:
  # http://www.rfc.se/fpdns/timecheck.html

  use Net::DNS;

  my $resolver = Net::DNS::Resolver->new;
  $resolver->nameservers($ARGV[0]);

  my $query = Net::DNS::Packet->new;
  $query->sign_tsig("n","test");

  my $response = $resolver->send($query);
  foreach my $rr ($response->additional) \{
  	print "localtime vs nameserver $ARGV[0] time difference: ";
  	print$rr->time_signed - time() if $rr->type eq "TSIG";
  \}
\end{alltt}



\chapter{Nping øvelser}

Vigtige!


\chapter{Bonus: Try Nmap from Metasploit}


\chapter{Bonus: Try masscan}

https://github.com/robertdavidgraham/masscan

This is the fastest Internet port scanner. It can scan the entire Internet in under 6 minutes, transmitting 10 million packets per second.

Formål:
* Tale om stateless vs stateful scanning




\chapter{Bonus: Network scripting using ncat}
\label{ex:netcat-1}

{\bf Objective:} \\
Learn how to use the netcat program for scripting

{\bf Purpose:}\\
Learn that a lot of protocols on the internet are easy read and create tools for.

{\bf Suggested method:} \\
Login to the Unix server - look at the manualen \verb+man nc+.
Then create a textfile named headh.sh using this content
\begin{alltt}
\input{scripts/head.sh}
\end{alltt}

Then use the command \verb#chmod +x head.sh# to make it executable and run it

{\bf Hints:}\\
The netcat program is a swiss army-knife for network data, and allows you to forward data to various ports and connect programs.

{\bf Solution:}\\
Run the program: \verb+./head.sh www.pentest.dk 80+


{\bf Discussion:}\\

Sometime the program will seem to hang, use ctrl-c to break it.


\appendix
\rhead{\fancyplain{}{\bf \leftmark}}
%\setlength{\parskip}{5pt}

\normal

\chapter{\color{titlecolor}Host information}

\begin{itemize}
\item You should note the IP-addresses used for servers and devices
\item The web server for installing programs:\\
http:// \hskip 15mm .\hskip 15mm .\hskip 15mm .\hskip 15mm
/public/windows/
\item Server used for team login: \hskip 15mm .\hskip 15mm .\hskip 15mm .\hskip 15mm \\
Available usernames: team1, team2, ... team10
password: \verb+team+
\item You can obtain root access using: \verb+sudo -s+
\end{itemize}

\section*{\color{titlecolor}Available servers and devices:}
\begin{itemize}
\item IP: \hskip 15mm .\hskip 15mm .\hskip 15mm .\hskip 15mm - OpenBSD router
\item IP: \hskip 15mm .\hskip 15mm .\hskip 15mm .\hskip 15mm - Your laptop
\item IP: \hskip 15mm .\hskip 15mm .\hskip 15mm .\hskip 15mm - Your laptop VM
\item IP: \hskip 15mm .\hskip 15mm .\hskip 15mm .\hskip 15mm -
\item IP: \hskip 15mm .\hskip 15mm .\hskip 15mm .\hskip 15mm -
\end{itemize}


\bibliographystyle{alpha}
%\bibliography{../ipv6-reference/security6-net.bib,../ipv6-reference/rfc.bib,../ipv6-reference/std.bib,../ipv6-reference/fyi.bib}
\bibliography{kramse.bib,rfc.bib,std.bib,fyi.bib}
%,internet.bib}


%\printindex

\end{document}

%%% Local Variables:
%%% mode: latex
%%% TeX-master: t
%%% End:
