\documentclass[a4paper,11pt,notitlepage]{report}
% Henrik Lund Kramshoej, February 2001
% hlk@security6.net,
% My standard packages
\usepackage{zencurity-network-exercises}

\begin{document}

\rm
\selectlanguage{english}

\newcommand{\emne}[1]{Software Security course}
\newcommand{\kursus}[1]{Software Security course}
\newcommand{\kursusnavn}[1]{Software Security course\\ exercises}

\mytitle{Software Security}{exercises}

\pagenumbering{roman}


\setcounter{tocdepth}{0}

\normal

{\color{titlecolor}\tableofcontents}
%\listoffigures - not used
%\listoftables - not used

\normal
\pagestyle{fancyplain}
\chapter*{\color{titlecolor}Preface}
\markboth{Preface}{}

This material is prepared for use in \emph{\kursus} and was prepared by
Henrik Lund Kramshoej, \link{http://www.zencurity.com} .
It describes the networking setup and
applications for trainings and courses where hands-on exercises are needed.

Further a presentation is used which is available as PDF from kramse@Github\\
Look for \jobname in the repo security-courses.

These exercises are expected to be performed in a training setting with network connected systems. The exercises use a number of tools which can be copied and reused after training. A lot is described about setting up your workstation in the repo

\link{https://github.com/kramse/kramse-labs}



\section*{\color{titlecolor}Prerequisites}

This material expect that participants have a working knowledge of
TCP/IP from a user perspective. Basic concepts such as web site addresses and email should be known as well as IP-addresses and common protocols like DHCP.

\vskip 1cm
Have fun and learn
\eject

% =================== body of the document ===============
% Arabic page numbers
\pagenumbering{arabic}
\rhead{\fancyplain{}{\bf \chaptername\ \thechapter}}

% Main chapters
%---------------------------------------------------------------------
% gennemgang af emnet
% check questions

\chapter*{\color{titlecolor}Exercise content}
\markboth{Exercise content}{}

Most exercises follow the same procedure and has the following content:
\begin{itemize}
\item {\bf Objective:} What is the exercise about, the objective
\item {\bf Purpose:} What is to be the expected outcome and goal of doing this exercise
\item {\bf Suggested method:} suggest a way to get started
\item {\bf Hints:} one or more hints and tips or even description how to
do the actual exercises
\item {\bf Solution:} one possible solution is specified
\item {\bf Discussion:} Further things to note about the exercises, things to remember and discuss
\end{itemize}

Please note that the method and contents are similar to real life scenarios and does not detail every step of doing the exercises. Entering commands directly from a book only teaches typing, while the exercises are designed to help you become able to learn and actually research solutions.


\chapter{Download Kali Linux Revealed (KLR) Book 10 min}
\label{ex:sw-downloadKLR}


\hlkimage{3cm}{kali-linux-revealed.jpg}

\emph{Kali Linux Revealed  Mastering the Penetration Testing Distribution}


{\bf Objective:}\\
We need a Kali Linux for running tools during the course. This is open source, and the developers have released a whole book about running Kali Linux.

This is named Kali Linux Revealed (KLR)

{\bf Purpose:}\\
We need to install Kali Linux in a few moments, so better have the instructions ready.

{\bf Suggested method:}\\
Create folders for educational materials. Go to \link{https://www.kali.org/download-kali-linux-revealed-book/}
Read and follow the instructions for downloading the book.

{\bf Solution:}\\
When you have a directory structure for download for this course, and the book KLR in PDF you are done.

{\bf Discussion:}\\
Linux is free and everywhere. The tools we will run in this course are made for Unix, so they run great on Linux.

Kali Linux is a free pentesting platform, and probably worth more than \$10.000

The book KLR is free, but you can buy/donate, and I recommend it.

\chapter{Check your Kali VM, run Kali Linux 30 min}
\label{ex:sw-basicVM}

\hlkimage{10cm}{kali-linux.png}

{\bf Objective:}\\
Make sure your virtual machine is in working order.

We need a Kali Linux for running tools during the course.

{\bf Purpose:}\\
If your VM is not installed and updated we will run into trouble later.

{\bf Suggested method:}\\
Go to \link{https://github.com/kramse/kramse-labs/}

Read the instructions for the setup of a Kali VM.

{\bf Hints:}\\
If you allocate enough memory and disk you wont have problems.

{\bf Solution:}\\
When you have a updated virtualisation software and Kali Linux, then we are good.

{\bf Discussion:}\\
Linux is free and everywhere. The tools we will run in this course are made for Unix, so they run great on Linux.

Kali Linux includes many hacker tools and should be known by anyone working in infosec.

\chapter{Check your Debian VM 10 min}
\label{ex:sw-basicDebianVM}

\hlkimage{3cm}{debian-9.png}

{\bf Objective:}\\
Make sure your virtual Debian 9 machine is in working order.

We need a Debian 9 Linux for running a few extra tools during the course.

{\Large \bf This is a bonus exercise - only one Debian is needed per team.}

{\bf Purpose:}\\
If your VM is not installed and updated we will run into trouble later.

{\bf Suggested method:}\\
Go to \link{https://github.com/kramse/kramse-labs/}

Read the instructions for the setup of a Kali VM.

{\bf Hints:}\\

{\bf Solution:}\\
When you have a updated virtualisation software and Kali Linux, then we are good.

{\bf Discussion:}\\
Linux is free and everywhere. The tools we will run in this course are made for Unix, so they run great on Linux.



\chapter{Investigate /etc 10 min}
\label{ex:sw-basicLinuxetc}


{\bf Objective:}\\
We will investigate the /etc directory on Linux. We need a Debian 9 Linux and a Kali Linux, to compare

{\bf Purpose:}\\
Start seeing example configuration files, including:
\begin{itemize}
  \item User database \verb+/etc/passwd+ and \verb+/etc/group+
  \item The password database \verb+/etc/shadow+
\end{itemize}

{\bf Suggested method:}\\
Boot your Linux VMs, log in

Investigate permissions for the user database files \verb+passwd+ and \verb+shadow+

{\bf Hints:}\\
Linux has many tools for viewing files, the most efficient would be less.

\begin{alltt}
hlk@debian:~$ cd /etc
hlk@debian:/etc$ ls -l shadow passwd
-rw-r--r-- 1 root root   2203 Mar 26 17:27 passwd
-rw-r----- 1 root shadow 1250 Mar 26 17:27 shadow
hlk@debian:/etc$ ls
... all files and directories shown, investigate more if you like
\end{alltt}

Showing a single file: \verb+less /etc/passwd+ and press q to quit

Showing multiple files: \verb+less /etc/*+ then :n for next and q for quit

\begin{alltt}
Trying reading the shadow file as your regular user:
user@debian-9-lab:/etc$ cat /etc/shadow
cat: /etc/shadow: Permission denied
\end{alltt}

Why is that? Try switching to root, using su or sudo, and redo the command.

{\bf Solution:}\\
When you have seen the most basic files you are done.

{\bf Discussion:}\\
Linux is free and everywhere. The tools we will run in this course are made for Unix, so they run great on Linux.

Sudo is a tool often used for allowing users to perform certain tasks as the super user. The tool is named from superuser do! \link{https://en.wikipedia.org/wiki/Sudo}



\chapter{Run OWASP Juice Shop}
\label{ex:sw-startjuice}

\hlkimage{3cm}{JuiceShop_Logo_100px.png}

{\bf Objective:}\\
Lets try starting the OWASP Juice Shop

{\bf Purpose:}\\
We will be doing some web hacking where you will be the hacker. There
will be an application we try to hack, designed to
optimise your learning.

It is named JuiceShop which is written in JavaScript

{\bf Suggested method:}\\
Go to \link{https://github.com/bkimminich/juice-shop}

Read the instructions for running juice-shop - docker is a simple way.

What you need

You need to have browsers and a proxy, plus a basic knowledge of HTTP.

If you could install Firefox it would be great, and we will use the
free version of Burp Suite, so please make sure you can run Java and
download the free version from Portswigger from:

\link{https://portswigger.net/burp/communitydownload}


{\bf Hints:}\\
The application is very modern, very similar to real applications.

The Burp proxy is an advanced tool! Dont be scared, we will use small parts at different times.

{\bf Solution:}\\
When you have a running Juice Shop web application in your team, then we are good.

{\bf Discussion:}\\
It has lots of security problems which can be used for learning
hacking, and thereby how to secure your applications. It is  related
to the OWASP.org Open Web Application Security Project which also has a
lot of resources.

Sources:\\
\url{https://github.com/bkimminich/juice-shop}\\
\url{https://www.owasp.org/index.php/Category:OWASP_WebGoat_Project}

It is recommended to buy the \emph{Pwning OWASP Juice Shop Official companion guide to the OWASP Juice Shop} from \link{https://leanpub.com/juice-shop} - suggested price USD 5.99



\chapter{Setup JuiceShop environment, app and proxy}
\label{ex:js-burp}

\chapter{Run small programs: Python, Shell script}
\label{ex:small-python}

\chapter{Run parts of a Django tutorial}
\label{ex:django-intro}

\chapter{SSL/TLS scanners 15 min}
\label{ex:sslscan}

{\bf Objective:}\\
Try the Online Qualys SSLLabs scanner \link{https://www.ssllabs.com/}
Try the command line tool sslscan checking servers - can check both HTTPS and non-HTTPS protocols!

{\bf Purpose:}\\
Learn how to efficiently check TLS settings on remote services.

{\bf Suggested method:}\\
Run the tool against a couple of sites of your choice.

\begin{alltt}\small
root@kali:~# sslscan --ssl2 web.kramse.dk
Version: 1.10.5-static
OpenSSL 1.0.2e-dev xx XXX xxxx

Testing SSL server web.kramse.dk on port 443
...
  SSL Certificate:
Signature Algorithm: sha256WithRSAEncryption
RSA Key Strength:    2048

Subject:  *.kramse.dk
Altnames: DNS:*.kramse.dk, DNS:kramse.dk
Issuer:   AlphaSSL CA - SHA256 - G2
\end{alltt}

Also run it without \verb+--ssl2+ and against SMTPTLS if possible.

{\bf Hints:}\\
Originally sslscan is from \link{http://www.titania.co.uk} but use the version on Kali, install with apt if not installed.

{\bf Solution:}\\
When you can run and understand what the tool does, you are done.

{\bf Discussion:}\\
SSLscan can check your own sites, while Qualys SSLLabs only can test from hostname


\chapter{Real Vulnerabilities}
\label{ex:real-vulns}

Choose a few real vulnerabilities, prioritize them


\chapter{JuiceShop Attacks}
\label{ex:juiceshop-attack}

 Try a few attacks in the JuiceShop with web proxy

\chapter{Try running brute force and fuzzing}
\label{ex:brute-force-fuzzing}

Try running brute force and fuzzing

\chapter{Try American fuzzy lop}
\label{ex:american-fuzzy-lop}

Try American fuzzy lop http://lcamtuf.coredump.cx/afl/


\chapter{Writing and exploiting a small buffer overflow}
\label{ex:writing-exploit-buffer}

Writing and exploiting a small buffer overflow

\chapter{Run debugger}
\label{ex:debugger-101}

\chapter{Django String Handling}
\label{ex:django-string}

Recommendations for handling strings, how does Python help, how does Django handle strings, and input validation

\chapter{SYN flooding exercise}
\label{ex:syn-flood}

\chapter{Securing the JuiceShop}
\label{ex:secure-juiceshop}

Layout a plan for securing the Juice Shop

\chapter{How should software be designed today}
\label{ex:software-design-today}




\end{document}
