\documentclass[Screen16to9,17pt]{foils}
\usepackage{kea-slides}
\externaldocument{siem-log-analysis-exercises}
\selectlanguage{english}


\begin{document}

\mytitlepage
{}
{Webinar: Opdag sårbarheder -- softwaresikkerhed og penetrationstest}

\hlkprofiluk

\slide{Overview Diploma in IT-security}

\hlkimage{17cm}{kea-diplom-oversigt.png}

\begin{quote}{\bf
Dit og virksomhedens udbytte}\\
Du bliver en medarbejder, der i høj grad er i stand til at analysere, planlægge og vurdere it-sikkerhed i forbindelse med drift, kontrol og udvikling af it-systemer i både private og offentlige virksomheder. Dette på et strategisk, taktisk såvel som operativt niveau på en reflekterende og handlingsorienteret måde.
\end{quote}

\slide{Course Data: VF1 Softwaresikkerhed (10 ECTS)}

\hlkimage{4cm}{pawel-janiak-dxFi8Ea670E-unsplash.jpg}

\begin{list2}
%\item Introduktionsmateriale med baggrundsinformation
\item Slide shows - presentation -- like this file and exercise booklet
\item Books listed in the lecture plan and here -- expect 1.000 - 1.500DKK
\item Additional resources from the internet
\end{list2}

Teaching dates - fall 2024 17:00 - 20:30 with Henrik Kramselund \\
29/8, 3/9, 10/9, 12/9, 17/9, 19/8, 24/9, 26/9, 1/10, 3/10, 8/10, 10/10, 22/10, 24/10

Exam: 5/11 2024 \hskip 12cm Photo by Pawel Janiak on Unsplash


\slide{Course Description: VF1 Softwaresikkerhed (10 ECTS)}

Indhold\\
Modulet fokuserer på sikkerhedsperspektivet i software, blandt andet programkvalitet og
fejlhåndterings samt datahåndterings betydning for en software arkitekturs sårbarheder.
Elementet introducerer også til forskellige designprincipper, herunder ”security by design”.

{\bf Læringsmål}\\
Viden -- Den studerende har viden om og forståelse for:
\begin{list2}
\item Hvilken betydning programkvalitet har for it-sikkerhed ift.:
\item Trusler mod software
\item Kriterier for programkvalitet
\item Fejlhåndtering i programmer
\item Forståelse for security design principles, herunder:
\item security by design
\item privacy by design.
\end{list2}

Færdigheder -- Den studerende kan:
\begin{list2}
\item Tage højde for sikkerhedsaspekter ved at:
\item Programmere håndtering af forventede og uventede fejl
\item Definere lovlige og ikke-lovlige input data, bl.a. til test
\item Bruge et Application Programming Interface (API) og/eller standard biblioteker
\item Opdage og forhindre sårbarheder i programkoder
\item Sikkerhedsvurdere et givet software arkitektur.
\end{list2}

Kompetencer -- Den studerende kan:
\begin{list2}
\item Håndtere risikovurdering af programkode for sårbarheder.
\item Håndtere udvalgte krypteringstiltag.
\end{list2}


\slide{Some keywords relating to this course}

%\hlkimage{}{}

\begin{quote}\Large
Buffer overflow  Common Vulnerabilities and Exposures (CVE)

String handling  Format String   C code   buffers  shell code

Common Vulnerability Scoring System (CVSS)  Unicode  frameworks

Address space layout randomization (ASLR) libraries input validation

Common Weakness Enumeration (CWE)  Stack protection

Static and dynamic application security testing

Software composition analysis    fuzzing   design and testing

Web Security and Defense   secure software development
\end{quote}

\begin{list2}
\item Lots of new terms, technologies and tools
\end{list2}


\slide{What is Infrastructure -- Software}


\hlkimage{10cm}{alexander-schimmeck-SeeM4AnkEHE-unsplash.jpg}

\begin{list2}
\item Enterprises today have a lot of computing systems supporting the business needs
\item These are very diverse and often discrete systems
\end{list2}

\hfill Photo by Alexander Schimmeck on Unsplash

\slide{Business Challenges}

\hlkimage{7cm}{adam-bignell-9tI2z5VZIZg-unsplash.jpg}

\begin{list2}
\item Accumulation of software
\item Legacy systems
\item Partners
\item Various types of data
\item Employee churn, replacement \hfill Photo by Adam Bignell on Unsplash
\end{list2}


\slide{Software Challenges}

\hlkimage{7cm}{john-barkiple-l090uFWoPaI-unsplash.jpg}

\begin{list2}
\item Complexity
\item Various languages
\item Various programming paradigms, client server, monolith, Model View Controller
\item Conflicting data types and available structures
\item Steam train vs electric train \hfill Photo by John Barkiple on Unsplash

\end{list2}


\slide{Primary literature}

\hlkrightpic{5cm}{0cm}{old_book_lumen_design_st_01.png}
Primary literature:
\begin{list2}
\item \emph{The Art of Software Security Testing Identifying Software Security Flaws},\\
Chris Wysopal, ISBN: 9780321304865, AoST or the Green Book
\item \emph{Web Application Security}, Andrew Hoffman, 2020, ISBN: 9781492053118 called WAS
\item Pwning OWASP Juice Shop Official companion guide to the OWASP Juice Shop
Can be found online for free, but recommend buying the PDF from \link{https://leanpub.com/juice-shop} - suggested price USD 5.99


\end{list2}


\slide{Book: The Art of Software Security Testing}

\hlkimage{5cm}{art-of-security-testing.jpeg}

\emph{The Art of Software Security Testing Identifying Software Security Flaws}\\
Chris Wysopal ISBN: 9780321304865, AoST or the Green Book

\slide{Web Application Security}

\hlkimage{7cm}{hoffman-web-application-security.jpg}
\emph{Web Application Security}, Andrew Hoffman, 2020, ISBN: 9781492053118 called WAS


\slide{Book: Gray Hat Hacking  (Grayhat)}

\hlkimage{6cm}{9781264268955-gray-hat.jpg}

\emph{Gray Hat Hacking: The Ethical Hacker's Handbook}, sixth edition
by Allen Harper, Ryan Linn, Stephen Sims, Michael Baucom, Huascar Tejeda, Daniel Fernandez, Moses Frost, Published: March 2022, ISBN: 9781264268955

Note: has some programming introduction which are very useful.
Also this book is used in the KEA Network Pentest course



\slide{Exercises: Hackerlab Setup}

Exercise theme: Virtual Machines allows us play with tech

\hlkimage{6cm}{hacklab-1.png}

\begin{list2}
\item Hardware: modern laptop CPU with virtualisation
\item Virtualisation software: VMware, Virtual box, HyperV pick your poison
\item Linux server system: Debian amd64 64-bit \link{https://www.debian.org/}
\item Setup instructions can be found at \link{https://github.com/kramse/kramse-labs}
\end{list2}

\centerline{It is enough if these VMs are pr team}


\slide{Systems Development Lifecycle (SDLC)}

\hlkimage{4cm}{sdlc_diag.png}

\begin{quote}\small
{\bf The Systems Development Lifecycle (SDLC)} is often depicted as a 6 part cyclical process where every step builds on top of the previous ones. In a similar fashion, security can be embedded in a SDLC by building on top of previous steps with policies, controls, designs, implementations and tests making sure that the product only performs the functions it was designed to and nothing more.

However, modern Agile practitioners often find themselves at an impasse, there is a wealth of competing projects, standards and vendors who all claim to be the best solution in the field.
\end{quote}
Source: picture and text from \link{https://owasp.org/www-project-integration-standards/writeups/owasp_in_sdlc/}

%Note: S can mean system or software or secure, many different but similar acronyms, SDL, SDLC, SSDLC etc.

\slide{Hacking and Intrusion Kill Chains}

\hlkimage{13cm}{crafting-cip-kill-chain.png}

\begin{list2}
\item See also \emph{Intelligence-Driven Computer Network Defense Informed by Analysis of Adversary Campaigns and Intrusion Kill Chains}, Eric M. Hutchins , Michael J. Cloppert, Rohan M. Amin, Ph.D. Lockheed Martin Corporation\\{\footnotesize
 \link{https://www.lockheedmartin.com/content/dam/lockheed-martin/rms/documents/cyber/LM-White-Paper-Intel-Driven-Defense.pdf}}
\end{list2}

\slide{OWASP Web Security Testing Guide}

%\hlkimage{}{}

\begin{quote}
The Web Security Testing Guide (WSTG) Project produces the premier cybersecurity testing resource for web application developers and security professionals.

The WSTG is a comprehensive guide to testing the security of web applications and web services. Created by the collaborative efforts of cybersecurity professionals and dedicated volunteers, the WSTG provides a framework of best practices used by penetration testers and organizations all over the world.
\end{quote}
Source: \url{https://owasp.org/www-project-web-security-testing-guide/} \\
and \url{https://github.com/OWASP/wstg/releases/download/v4.2/wstg-v4.2.pdf}

\begin{list2}
    \item OWASP Web Security Testing Guide version 4.2 is a 465 page PDF!
\end{list2}


\slide{Course Data: Netværkspenetrationstest (5 ECTS)}

\hlkimage{4cm}{pawel-janiak-dxFi8Ea670E-unsplash.jpg}

\begin{list2}
%\item Introduktionsmateriale med baggrundsinformation
\item Slide shows - presentation -- like this file and exercise booklet
\item Books listed in the lecture plan and here -- expect 1.000 - 1.500DKK
\item Additional resources from the internet
\end{list2}

Teaching dates - fall 2024 17:00 - 20:30 with Thomas Bach\\
4/9, 11/9, 25/9, 2/10, 9/10, 23/10, 30/10

Exam: 6/11 2024 \hskip 12cm Photo by Pawel Janiak on Unsplash


\slide{Course Description: Netværkspenetrationstest (5 ECTS)}

Den studerende lærer om hvordan en penetration test udføres, samt kan indhente oplysninger om de
seneste sårbarheder, og kan benytte sig af de relevante værktøjer til dette formål.

\begin{list2}
\item[] Viden -- Den studerende har viden om og forståelse for:
\item Etiske samt kontraktuelle forhold omkring en penetrationstest.
\item Standardiseringorganisationers og myndigheders krav til og om penetrationstesting
\item[] Færdigheder -- Den studerende kan:
\item[] Tage højde for sikkerhedsaspekter ved at:
\item Anvende relevante metoder ved udførsel af en penetrationstest
\item Udarbejde en angrebsplan ud fra indsamlede oplysninger om et mål
\item Finde sårbarheder i et givet system
\item Dokumentere og rapportere fundne sårbarheder
\item[] Kompetencer -- Den studerende kan:
\item Planlægge en penetration test, samt eksekvere den både ved brug af værktøjer og manuelt.
\end{list2}

\slide{Some keywords relating to this course}

%\hlkimage{}{}

\begin{quote}\Large
Buffer overflow  Common Vulnerabilities and Exposures (CVE)

Format String   C code   buffers  shell code

Pentest execution   hacker tools  discovery tools   port scan

Exploits  Metasploit  Nmap   OWASP Security Testing

Common Vulnerability Scoring System (CVSS)   brute force

Address space layout randomization (ASLR)    password attacks

Common Weakness Enumeration (CWE)   Kali Linux

Web Hacking and Attack tools   Burp Suite
\end{quote}

\begin{list2}
\item Lots of new terms, technologies and tools
\end{list2}

\slide{Demo: Hackerlab Setup}

\hlkimage{6cm}{hacklab-1.png}

\begin{list2}
\item Hardware: modern laptop CPU with virtualisation\\
Dont forget to enable hardware virtualisation in the BIOS
\item Virtualisation software: VMware, Virtual box, HyperV pick your poison
\item Hackersoftware: Kali Virtual Machine amd64 64-bit \link{https://www.kali.org/}
\item Linux server system: Debian amd64 64-bit \link{https://www.debian.org/}
\item Setup instructions can be found at \link{https://github.com/kramse/kramse-labs}
\end{list2}



\slide{OWASP Juice Shop Project}

\hlkimage{3cm}{JuiceShop_Logo_100px.png}

We will also use the OWASP Juice Shop Tool Project as a running example. This is an application which is modern AND designed to have security flaws.

Read more about this project at: \link{https://www.owasp.org/index.php/OWASP_Juice_Shop_Project}\\ \link{https://github.com/bkimminich/juice-shop}

It is recommended to buy the Pwning OWASP Juice Shop Official companion guide to the OWASP Juice Shop from \link{https://leanpub.com/juice-shop} - suggested price USD 5.99



\myquestionspage

\end{document}
