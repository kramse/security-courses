\documentclass[Screen16to9,17pt]{foils}
\usepackage{zencurity-slides}
\externaldocument{system-security-exercises}
\selectlanguage{english}

% Systemintegration

\begin{document}

\mytitlepage
{0. Introduction}
{KEA System Integration F2020}

\hlkprofiluk

\slide{Plan for today}

\begin{list2}
\item Create a good starting point for learning
\item Introduce lecturer and students
\item Expectations for this course
\item Literature list walkthrough
\item Prepare tools for the exercises
\item Kali and Debian Linux introduction
\end{list2}

Exercises
\begin{list2}
\item Kali Linux installation
\item Debian Linux installation
\end{list2}
Linux is a toolbox we will use and participants will use virtual machines

\slide{Course Materials}

\begin{list1}
\item This material is in multiple parts:
\begin{list2}
%\item Introduktionsmateriale med baggrundsinformation
\item Slide shows - presentation - this file
\item Exercises - PDF which is updated along the way
\end{list2}
\item Additional resources from the internet
\item Note: the presentation slides are not a substitute for reading the books, papers and doing exercises, many details are not shown
\end{list1}

Note: parts of this material are quotes from the book we use, and similar courses. See the README in the Github reposity  in the repo security-courses for this course \jobname\ kramse@Github

A special thanks to William D. (Bill) Young
Associate Professor of Instruction and Research Scientist,
The University of Texas at Austin

When asked if I could borrow parts from his CS361 \emph{Introduction to Computer Security} he graciously wrote:\\
"You are welcome to use them freely.  You can credit me at the beginning." 


\slide{Fronter Platform}

\hlkimage{9cm}{fronter.png}

We will use fronter a lot, both for sharing educational materials and news during the course.

You will also be asked to turn in deliverables through fronter

\link{https://fronter.com/kea/main.phtml}

\vskip 5mm
\centerline{If you haven't received login yet, let us know}

\slide{Overview Diploma in IT-security}

\hlkimage{17cm}{kea-diplom-oversigt.png}


\slide{Course Data}

{\Large\bf Course: Computer Systems Security\\
VF 3 Systemsikkerhed (10 ECTS)}

Teaching dates: tuesdays and thursdays 17:00 - 20:30\\
28/01 2020, 30/01 2020, 04/02 2020, 06/02 2020, 11/02 2020, 13/02 2020, 18/02 2020, 20/02 2020, 25/02 2020, 27/02 2020, 03/03 2020, 05/03 2020, 10/03 2020, 12/03 2020

Exam: 31/03 2020

\slide{Deliverables and Exam}

\begin{list2}
\item Exam
\item Individual: Oral based on curriculum
\item Graded (7 scale)
\item Draw a question with no preparation. Question covers a topic
\item Try to discuss the topic, and use practical examples
\item Exam is 30 minutes in total, including pulling the question and grading
\item Count on being able to present talk for about 10 minutes
\item Prepare material (keywords, examples, exercises, wireshark captures) for different topics so that you can use it to help you at the exam

\vskip 5mm
\item Deliverables:
\item 2 Mandatory assignments
\item Both mandatory assignments are required in order to be entitled to the exam.
\end{list2}


\slide{Course Description}

From: STUDIEORDNING Diplomuddannelse i it-sikkerhed August 2018\\
Indhold: Den studerende kan udføre, udvælge, anvende, og implementere praktiske
tiltag til sikring af firmaets udstyr og har viden og færdigheder der supportere dette.

{\bf Viden}

Den studerende har viden om:
\begin{list2}
\item Generelle governance principper / sikkerhedsprocedurer
\item Væsentlige forensic processer
\item Relevante it-trusler
\item Relevante sikkerhedsprincipper til systemsikkerhed
\item OS roller ift. sikkerhedsovervejelser
\item Sikkerhedsadministration i DBMS.
\end{list2}

\slide{Færdigheder}

{\bf Færdigheder}

Den studerende kan:
\begin{list2}
\item Udnytte modforanstaltninger til sikring af systemer
\item Følge et benchmark til at sikre opsætning af enhederne
\item Implementere systematisk logning og monitering af enheder
\item Analysere logs for incidents og følge et revisionsspor
\item Kan genoprette systemer efter en hændelse.
\end{list2}

\slide{Kompetencer}

{\bf Kompetencer}

Den studerende kan:
\begin{list2}
\item håndtere enheder på command line-niveau
\item håndtere værktøjer til at identificere og fjerne/afbøde forskellige typer af endpoint trusler
\item håndtere udvælgelse, anvendelse og implementering af praktiske mekanismer til at forhindre, detektere og reagere over for specifikke it-sikkerhedsmæssige hændelser
\item håndtere relevante krypteringstiltag
\end{list2}

Final word is the Studieordning which can be downloaded from\\
{\footnotesize \link{https://kompetence.kea.dk/uddannelser/it-digitalt/diplom-i-it-sikkerhed}\\
\link{Studieordning_for_Diplomuddannelsen_i_IT-sikkerhed_Aug_2018.pdf}}

\slide{Expectations alignment}

\hlkimage{7cm}{Shaking-hands_web.jpg}

Form groups of 2-3 students

In groups of 2 students, brainstorm for 5 minutes on what topics you would like to have in this course

Use 5 minutes more on Agreeing on 5 topics and prioritize these 5 topics

\vskip 1cm
PS We will from time to time have exercises, groups dont need to be the same each time.

\slide{Primary literature}

\hlkrightpic{5cm}{0cm}{old_book_lumen_design_st_01.png}
Primary literature:
\begin{list2}
\item \emph{Computer Security: Art and Science}, 2nd edition 2019! Matt Bishop ISBN: 9780321712332
\item \emph{Defensive Security Handbook: Best Practices for Securing Infrastructure}, Lee Brotherston, Amanda Berlin ISBN: 978-1-491-96038-7
\end{list2}
Supporting literature:
\begin{list2}
\item \emph{Linux Basics for Hackers Getting Started with Networking, Scripting, and Security in Kali}. OccupyTheWeb, December 2018, 248 pp. ISBN-13: 978-1-59327-855-7 - shortened LBfH
\item \emph{Kali Linux Revealed  Mastering the Penetration Testing Distribution}
Raphael Hertzog, Jim O'Gorman - shortened KLR
\end{list2}


\slide{Book: Computer Security: Art and Science}
\hlkimage{6cm}{computer-security-art-and-science.jpg}

\emph{Computer Security: Art and Science}, Matt Bishop ISBN: 9780321712332

{\footnotesize\link{https://www.pearson.com/us/higher-education/program/Bishop-Computer-Security-2nd-Edition/PGM25107.html}}

\slide{Book: Defensive Security Handbook (DSH)}

\hlkimage{6cm}{defensive-security-handbook.jpg}

\emph{Defensive Security Handbook: Best Practices for Securing Infrastructure}, Lee Brotherston, Amanda Berlin ISBN: 978-1-491-96038-7

\slide{Book: Forensics Discovery (FD)}

\hlkimage{6cm}{forensic-discovery.jpg}

\emph{Forensics Discovery}, Dan Farmer, Wietse Venema 2004, Addison-Wesley.

Can be found at http://www.porcupine.org/forensics/forensic-discovery/ but recommend buying it - to support and also better formatted for reading


\slide{Book: Linux Basics for Hackers (LBfH)}

\hlkimage{6cm}{LinuxBasicsforHackers_cover-front.png}

\emph{Linux Basics for Hackers
Getting Started with Networking, Scripting, and Security in Kali}
by OccupyTheWeb
December 2018, 248 pp.
ISBN-13:
9781593278557

\link{https://nostarch.com/linuxbasicsforhackers}
Not curriculum but explains how to use Linux

\slide{Book: Kali Linux Revealed (KLR)}

\hlkimage{6cm}{kali-linux-revealed.jpg}

\emph{Kali Linux Revealed  Mastering the Penetration Testing Distribution}

\link{https://www.kali.org/download-kali-linux-revealed-book/}\\
Not curriculum but explains how to install Kali Linux

\exercise{ex:downloadKLR}



%%% Break?

\slide{Hackerlab Setup}

\hlkimage{6cm}{hacklab-1.png}

\begin{list2}
\item Hardware: modern laptop CPU with virtualisation\\
Dont forget to enable hardware virtualisation in the BIOS
\item Virtualisation software: VMware, Virtual box, HyperV pick your poison
\item Hackersoftware: Kali Virtual Machine amd64 64-bit\link{https://www.kali.org/}
\item Linux server system: Debian 9 Stretch amd64 64-bit\link{https://www.debian.org/}
\item Setup instructions can be found at \link{https://github.com/kramse/kramse-labs}
\end{list2}

\centerline{It is enough if these VMs are pr team}

\slide{Aftale om test af netværk}

\vskip 1cm
{\bfseries Straffelovens paragraf 263 Stk. 2. Med bøde eller fængsel
  indtil 6 måneder
straffes den, som uberettiget skaffer sig adgang til en andens
oplysninger eller programmer, der er bestemt til at bruges i et anlæg
til elektronisk databehandling.}

Hacking kan betyde:
\begin{list2}
\item At man skal betale erstatning til personer eller virksomheder
\item At man får konfiskeret sit udstyr af politiet
\item At man, hvis man er over 15 år og bliver dømt for hacking, kan
  få en bøde - eller fængselsstraf i alvorlige tilfælde
\item At man, hvis man er over 15 år og bliver dømt for hacking, får
en plettet straffeattest. Det kan give problemer, hvis man skal finde
et job eller hvis man skal rejse til visse lande, fx USA og
Australien
\item Frit efter: \link{http://www.stophacking.dk} lavet af Det
  Kriminalpræventive Råd
\item Frygten for terror har forstærket ovenstående - så lad være!
\end{list2}



\exercise{ex:basicVM}

\exercise{ex:basicDebianVM}


\slide{Command prompt}

We will use Unix/Linux systems, and you need to use the command line a bit:

\begin{alltt}
\small
[hlk@fischer hlk]$ id
uid=6000(hlk) gid=20(staff) groups=20(staff),
0(wheel), 80(admin), 160(cvs)
[hlk@fischer hlk]$

[root@fischer hlk]# id
uid=0(root) gid=0(wheel) groups=0(wheel), 1(daemon),
2(kmem), 3(sys), 4(tty), 5(operator), 20(staff),
31(guest), 80(admin)
[root@fischer hlk]#
\end{alltt}

\begin{list1}
\item \verb+$+ is commonly used for showing a user login, while a \verb+#+ is for root logins
\item Change from user to root using the command \verb+sudo+ like \verb+sudo -s+
\end{list1}

\slide{Command Syntax}

A common syntax for commands are described like this:
\begin{alltt}
echo [-n] [string ...]
\end{alltt}

\begin{list2}
\item The command is the first thing on the command line, you cannot write \verb+henrik echo+
\item Options are prefixed with dash \verb+-n+, optional ones are in brackets  \verb+[]+
\item Multiple options can be combined into one group like, \verb+tar -cvf+ eller \verb+tar cvf+
\item Some options require arguments, like \verb+tar -cf filename+ where \verb+-f+ needs a filename
\end{list2}



\slide{Manual System}

\hlkimage{7cm}{images/unix-command-1.pdf}

\begin{quote}
 It is a book about a Spanish guy called Manual. You should read it.
       -- Dilbert
\end{quote}

\begin{list1}
\item The Unix/Linux manual system is where you find the options, commands and file formats
\item Manuals must be installed, if not install them immediately
\item Very similar across Unix variants, OpenBSD is known for having an excellent manual pages
\item \verb+man -k+ allows keyword search similar can be done using \verb+apropos+
\end{list1}

Try \verb+man crontab+ and \verb+man 5 crontab+



\slide{Example Manual Page}

\begin{alltt}\footnotesize
\small
NAME
     cal - displays a calendar
SYNOPSIS
     cal [-jy] [[month]  year]
DESCRIPTION
   cal displays a simple calendar.  If arguments are not specified, the cur-
   rent month is displayed.  The options are as follows:
   -j      Display julian dates (days one-based, numbered from January 1).
   -y      Display a calendar for the current year.

The Gregorian Reformation is assumed to have occurred in 1752 on the 3rd
of September.  By this time, most countries had recognized the reforma-
tion (although a few did not recognize it until the early 1900's.)  Ten
days following that date were eliminated by the reformation, so the cal-
endar for that month is a bit unusual.
\end{alltt}

\slide{Unix Command Line Shells}


  \begin{list2}
    \item sh - Bourne Shell
\item bash - Bourne Again Shell, often the default in Linux
\item ksh - Korn shell, originally by David Korn, popular version \verb+pdksh+ public domain ksh
\item csh - C shell, syntax close to the C programming language
\item multiple others exist: zsh, tcsh
  \end{list2}
\begin{list1}
\item Comparable to command.com, cmd.exe and powershell in Windows
\item Also commonly used for small programs, scripts
\item When writing scripts use the characters number sign and exclamation mark (\verb+#!+) in the beginning
\end{list1}

See more in \url{https://en.wikipedia.org/wiki/Shell_(computing)}\\
\url{https://en.wikipedia.org/wiki/Shebang_(Unix)}

\slide{Linux file system and konfiguration}

.
\hlkrightpic{8cm}{0cm}{unix-vfs.pdf}
\begin{list2}

\item Unix/Linux uses a virtual filesystem\\
\url{https://en.wikipedia.org/wiki/Unix_filesystem}
\item No drive letters, just disks mounted in a common tree
\item Everything starts with the file system root \verb+/+ - forward
\item An important directory is \verb+/etc/+ which includes a lot of configuration for the system and applications
\end{list2}



\exercise{ex:basicLinuxetc}

\slidenext{Buy the books!}



\end{document}
