\documentclass[Screen16to9,17pt]{foils}
\usepackage{zencurity-slides}
\externaldocument{system-integration-exercises}
\selectlanguage{english}

% Systemintegration

\begin{document}

\mytitlepage
{11. Integration examples and standards}
{KEA System Integration F2020 10 ECTS}


\slide{This weeks Agenda in system integration}

\begin{list2}
\item Follow the plan:\\
\url{https://zencurity.gitbook.io/kea-it-sikkerhed/system-integration/lektionsplan}
\item Plan for May 18.\\
I will go through the subjects from the books
\item
\end{list2}

\slide{Goals for today}

\hlkimage{6cm}{thomas-galler-hZ3uF1-z2Qc-unsplash.jpg}

Todays goals:
\begin{list2}
\item
\end{list2}

Photo by Thomas Galler on Unsplash


\slide{Time schedule}
\begin{list2}
\item 08:30 2x 45 min with 10min break\\
EIP Chapter 13 and 14\\
Finish Enterprise Integration Patterns
\item 10:15 2x 45 min with 10min break\\
Camel chapter 14-15
\item 12:30 2x 45min with 10min break \\
Camel chapter 16
\item 14:15 45 min\\
Exercises in course so far. Repeat Ansible, how to setup production systems
\end{list2}




\slide{Plan for today}

\begin{list2}
\item Integration examples and standards
\item Running Camel integration Management, Logs, Monitoring, Security
\item - using systems we have used in this course as examples
\end{list2}

Exercises
\begin{list2}
\item Exercises in course so far.
\item Repeat Ansible
\item How to setup production systems
\end{list2}



\slide{Reading Summary}

\begin{list1}
\item EIP 13-14
\item Camel ch 14-16
\item SOA Appendix A
\end{list1}

\slide{EIP chapter 13: Integration Patterns in Practice}

This chapter covers
\begin{list2}
\item Case Study: Bond Trading System
\item Architecture with Patterns
\item Problem Solving With Patterns
\end{list2}

Source: {\footnotesize\\
\emph{Enterprise Integration Patterns}, Gregor Hohpe and Bobby Woolf, 2004\\
ISBN: 978-0-321-20068-6}

Note: Chapter content is available at:\\
\link{https://www.enterpriseintegrationpatterns.com/patterns/messaging/BondTradingCaseStudy.html}

\slide{EIP chapter 14: Concluding Remarks}

This chapter covers
\begin{list2}
\item
\end{list2}

Source: {\footnotesize\\
\emph{Enterprise Integration Patterns}, Gregor Hohpe and Bobby Woolf, 2004\\
ISBN: 978-0-321-20068-6}



Note: Chapter content is available at:\\
\link{https://www.enterpriseintegrationpatterns.com/patterns/messaging/Future.html}



\slide{Camel chapter 14: Securing Camel}

This chapter covers
\begin{list2}
\item Securing your Camel configuration
\item Web service security
\item Transport security
\item Encryption and decryption
\item Signing messages
\item Authentication and authorization
\end{list2}

Source: {\footnotesize\\
\emph{Camel in action}, Claus Ibsen and Jonathan Anstey, 2018, 2nd edition
ISBN: 978-1-61729-293-4}

\slide{Camel chapter 15: Running and deploying Camel}

This chapter covers

\begin{list2}
\item Starting and stopping Camel safely
\item Adding and removing routes at runtime
\item Deploying Camel
\item Running standalone
\item Running in web containers
\item Running in Java EE servers
\item Running with OSGi
\item Running with CDI
\end{list2}

Source: {\footnotesize\\
\emph{Camel in action}, Claus Ibsen and Jonathan Anstey, 2018, 2nd edition
ISBN: 978-1-61729-293-4}

\slide{Camel chapter 16: Management and Monitoring}

This chapter covers
\begin{list2}
\item Monitoring Camel instances
\item Tracking application activities
\item Using notifications
\item Managing Camel applications with JMX and REST
\item Understanding and using the Camel management API
\item Gathering runtime performance statistics
\item Using Dropwizard metrics with Camel
\item Developing custom components for management
\end{list2}

Source: {\footnotesize\\
\emph{Camel in action}, Claus Ibsen and Jonathan Anstey, 2018, 2nd edition
ISBN: 978-1-61729-293-4}



\slide{SOA Appendix A: Service-Orientation Principles Reference}

\begin{quote}
This appendix provides profile tables for the patterns referenced throughout this
book. As explained in Chapter 1, each pattern reference is suffixed with the page
number of its corresponding profile table in this appendix.
\end{quote}

Design patterns are helpful because they:
\begin{list2}
\item Represent field-tested solutions to common design problems
\item Organize design intelligence into a standardized and easily “referenceable” format
\item Are generally repeatable by most IT professionals involved with design
\item Can be used to ensure consistency in how systems are designed and built
\item Can become the basis for design standards
\item Are usually fl exible and optional (and openly document the impacts of their appli-
cation and even suggest alternative approaches)
\item Can be used as educational aids by documenting specific aspects of system design
(regardless of whether they are applied)
\item Can sometimes be applied prior and subsequent to the implementation of a system
\item Can be supported via the application of other design patterns that are part of the
same collection
\item Enrich the vocabulary of a given IT field because each pattern is given a
meaningful name
\end{list2}
Similar to Appendix B: SOA Design Patterns Reference

Source: {\footnotesize\\
\emph{Service‑Oriented Architecture: Analysis and Design for Services and Microservices},\\ Thomas Erl, 2017
ISBN: 978-0-13-385858-7}

\slide{Part I 08:30 2x 45 min}

EIP Chapter 13 and 14
Finish Enterprise Integration Patterns

\slide{Case Study: Bond Trading System}

This chapter covers
\begin{list2}
\item Case Study: Bond Trading System
\item Architecture with Patterns
\item Problem Solving With Patterns
\end{list2}

Source: {\footnotesize\\
\emph{Enterprise Integration Patterns}, Gregor Hohpe and Bobby Woolf, 2004\\
ISBN: 978-0-321-20068-6}

Note: We will now continue at the book site:\\
{\footnotesize\link{https://www.enterpriseintegrationpatterns.com/patterns/messaging/BondTradingCaseStudy.html}}

\slide{Concluding Remarks}


\begin{list2}
\item Emerging Standards and Futures in Enterprise Integration
\end{list2}

Note: We will now continue at the book site:\\
\link{https://www.enterpriseintegrationpatterns.com/patterns/messaging/Future.html}


\slide{Further patterns}

\begin{quote}

\end{quote}

\begin{list2}
  \item Further patterns, newer data and articles are available from the authors:\\
  \link{https://www.enterpriseintegrationpatterns.com/ramblings.html}
\end{list2}


\slide{Modern Examples for Enterprise Integration Patterns}

\hlkimage{10cm}{eip-modern-examples.png}

Modern Examples for Enterprise Integration Patterns\\
{\footnotesize\link{https://www.enterpriseintegrationpatterns.com/ramblings/eip1_examples_updated.html}}



\slide{What Products Implement or Use Enterprise Integration Patterns?}




The patterns are not tied to a specific implementation. They help you design better solutions, whether you use any of the following platforms:
\begin{list2}
  \item {\bf EAI and SOA platforms}, such as IBM WebSphere MQ, TIBCO, Vitria, Oracle Service Bus, WebMethods (now Software AG), Microsoft BizTalk, or Fiorano.
\item {\bf Open source ESB's} like Mule ESB, JBoss Fuse, Open ESB, WSo2, Spring Integration, or Talend ESB
\item {\bf Message Brokers} like ActiveMQ, Apache Kafka, or RabbitMQ
\item {\bf Web service- or REST-based integration}, including Amazon Simple Queue Service (SQS) or Google Cloud Pub/Sub
\item {\bf JMS-based messaging systems}
\item {\bf Microsoft technologies} like MSMQ or Windows Communication Foundation (WCF)
\end{list2}

\slide{Stencils for EIP}

\hlkimage{8cm}{eip-stencils.png}

\begin{quote}
Document. You can create design documents using our icon language by downloading the Visio stencil or using the OmniGraffle stencil created by one of our readers.
\end{quote}

\begin{list2}
\item \link{https://www.enterpriseintegrationpatterns.com/patterns/messaging/downloads.html}
\item \link{http://www.graffletopia.com/stencils/137}
\end{list2}

\slide{Part II 10:15 2x 45 min}

Camel chapter 14-15

\slide{Securing your Camel configuration}


\slide{Web service security}


\slide{Transport security}


\slide{Encryption and decryption}


\slide{Signing messages}


\slide{Authentication and authorization}


\slide{Starting and stopping Camel safely}


\slide{Adding and removing routes at runtime}


\slide{Deploying Camel}


\slide{Running standalone}


\slide{Running in web containers}


\slide{Running in Java EE servers}


\slide{Running with OSGi}


\slide{Running with CDI}






\slide{Part III 12:30 2x 45min}

Camel chapter 16


\slide{Monitoring Camel instances}


\slide{Tracking application activities}


\slide{Using notifications}


\slide{Managing Camel applications with JMX and REST}


\slide{Understanding and using the Camel management API}


\slide{Gathering runtime performance statistics}


\slide{Using Dropwizard metrics with Camel}

\slide{Developing custom components for management}

\slide{Part IV 12:30 2x 45min}

Exercises in course so far. Repeat Ansible, how to setup production systems

\slidenext

\end{document}
