\documentclass[Screen16to9,17pt]{foils}
\usepackage{zencurity-slides}
\externaldocument{system-integration-exercises}
\selectlanguage{english}

% Systemintegration

\begin{document}

\mytitlepage
{3. SOAP, WSDL, XML data}
{KEA System Integration F2020 10 ECTS}

\slide{Plan for today}

\begin{list2}
\item Data overview
\item XML data, JSON
\item Data Transformation
\item WebServices
\item SOAP, WSDL
\item Investigate examples
\end{list2}

Exercises
\begin{list2}
\item
\item
\end{list2}



\slide{Reading Summary}

\begin{list1}
\item Camel chapter 3: Transforming data with Camel
\item Contains a lot of different formats
\end{list1}


\slide{Data overview XML data, JSON}

\slide{XML data}

\begin{quote}
  Extensible Markup Language (XML) is a markup language that defines a set of rules for encoding documents in a format that is both human-readable and machine-readable. The World Wide Web Consortium's XML 1.0 Specification[2] of 1998[3] and several other related specifications[4]—all of them free open standards—define XML.[5]

  The design goals of XML emphasize simplicity, generality, and usability across the Internet.[6] It is a textual data format with strong support via Unicode for different human languages. Although the design of XML focuses on documents, the language is widely used for the representation of arbitrary data structures[7] such as those used in web services.
\end{quote}
Source: \url{https://en.wikipedia.org/wiki/XML}

\begin{list2}
\item We have seen XML used for configuration in Apache Tomcat and Camel
\item Perfect for computers, less for humans
\end{list2}

\slide{XML data example - Nmap output}

\begin{alltt}\footnotesize
  <?xml version="1.0" encoding="UTF-8"?>
  <!DOCTYPE nmaprun>
  <?xml-stylesheet href="file:///usr/bin/../share/nmap/nmap.xsl" type="text/xsl"?>
  <!-- Nmap 7.70 scan initiated Sat Feb 22 23:35:53 2020 as: nmap -oA router -sP 10.0.42.1 -->
  <nmaprun scanner="nmap" args="nmap -oA router -sP 10.0.42.1" start="1582410953"
   startstr="Sat Feb 22 23:35:53 2020" version="7.70" xmloutputversion="1.04">
  <verbose level="0"/>
  <debugging level="0"/>
  <host><status state="up" reason="echo-reply" reason_ttl="62"/>
  <address addr="10.0.42.1" addrtype="ipv4"/>
  <hostnames>
  </hostnames>
  <times srtt="2235" rttvar="5000" to="100000"/>
  </host>
  <runstats><finished time="1582410953" timestr="Sat Feb 22 23:35:53 2020" elapsed="0.32"
   summary="Nmap done at Sat Feb 22 23:35:53 2020; 1 IP address (1 host up)
   scanned in 0.32 seconds" exit="success"/><hosts up="1" down="0" total="1"/>
  </runstats>
  </nmaprun>

\end{alltt}


\slide{XML data - documents}

\begin{quote}
Hundreds of document formats using XML syntax have been developed,[8] including RSS, Atom, SOAP, SVG, and XHTML. XML-based formats have become the default for many office-productivity tools, including Microsoft Office (Office Open XML), OpenOffice.org and LibreOffice (OpenDocument), and Apple's iWork[citation needed]. XML has also provided the base language for communication protocols such as XMPP. Applications for the Microsoft .NET Framework use XML files for configuration, and property lists are an implementation of configuration storage built on XML.[9]
\end{quote}
Source: \url{https://en.wikipedia.org/wiki/XML}

\begin{list2}
\item Document formats using XML may still be proprietary!
\item Documents using XML can be validated, are they well-formed according to the Document Type Definition (DTD)
\end{list2}

\slide{XML data - standards}

\begin{quote}
Many industry data standards, such as Health Level 7, OpenTravel Alliance, FpML, MISMO, and National Information Exchange Model are based on XML and the rich features of the XML schema specification. Many of these standards are quite complex and it is not uncommon for a specification to comprise several thousand pages.[citation needed] In publishing, Darwin Information Typing Architecture is an XML industry data standard. XML is used extensively to underpin various publishing formats.
\end{quote}
Source: \url{https://en.wikipedia.org/wiki/XML}

\begin{list2}
\item
\item
\end{list2}


\slide{XML data for Service-oriented architecture (SOA)}

\begin{quote}
XML is widely used in a Service-oriented architecture (SOA). Disparate systems communicate with each other by exchanging XML messages. The message exchange format is standardised as an XML schema (XSD). This is also referred to as the canonical schema. XML has come into common use for the interchange of data over the Internet. IETF RFC:3023, now superseded by RFC:7303, gave rules for the construction of Internet Media Types for use when sending XML. It also defines the media types application/xml and text/xml, which say only that the data is in XML, and nothing about its semantics.
\end{quote}
Source: \url{https://en.wikipedia.org/wiki/XML}

\begin{list2}
\item We will talk more about SOA later
\end{list2}



\slide{Transforming XML using XSLT}
\begin{quote}


XSLT (Extensible Stylesheet Language Transformations) is a language for transforming XML documents into other XML documents,[1] or other formats such as HTML for web pages, plain text or XSL Formatting Objects, which may subsequently be converted to other formats, such as PDF, PostScript and PNG.[2] XSLT 1.0 is widely supported in modern web browsers.[3]
\end{quote}
Source: \url{https://en.wikipedia.org/wiki/XSLT}

\begin{list2}
\item
\item
\end{list2}


\slide{xsltproc example using Nmap}

\begin{alltt}\footnotesize
$ su -
# apt install nmap xsltproc
# nmap -sP -oA /tmp/router 91.102.91.18
# exit
$ xsltproc /tmp/router.xml > /tmp/router.html
$ firefox /tmp/router.html
\end{alltt}


\begin{list2}
\item We can use the command line tool \verb+xlstproc+ for transforming documents
\item \verb+apt install xsltproc+
\item Its part of the package Libxslt \url{https://en.wikipedia.org/wiki/Libxslt}
\end{list2}





\slide{Data oveview JSON}

\begin{quote}

\end{quote}
Source: \url{}

\begin{list2}
\item
\item
\end{list2}




\slide{3.1 Data transformation overview}

\slide{3.2 Transforming data by using EIPs and Java}

Using the Message Translator EIP,
Using the Content Enricher EIP


\slide{3.3 Transforming XML}

Transforming XML with XSLT,
Transforming XML with object marshaling


\slide{3.4 Transforming with data formats}

Data formats provided with Camel, Using Camel’s CSV data
format, Using Camel’s Bindy data format, Using
Camel’s JSON data format, Configuring Camel data formats

\slide{3.5 Transforming with templates}
Using Apache Velocity

\slide{3.6: Understanding Camel type converters}

How the Camel type-converter mechanism works,
Using Camel type converters, Writing your own
type converter

\slide{3.7 Summary and best practices}



\slide{Web services SOAP, WSDL, }



\slide{Web Services}

\begin{quote}

\end{quote}

\begin{list2}
\item Today a generic name for services using the internet
\item Web servers such as Apache HTTPD, Nginx etc. provide a service to thew internet allowing access using HTTP
\item Source for some parts on this slide, \url{https://en.wikipedia.org/wiki/Web_service}
\end{list2}





\slide{SOAP - Simple Object Access Protocol}

\begin{quote}
SOAP (abbreviation for Simple Object Access Protocol) is a messaging protocol specification for exchanging structured information in the implementation of web services in computer networks. Its purpose is to provide extensibility, neutrality, verbosity and independence. It uses XML Information Set for its message format, and relies on application layer protocols, most often Hypertext Transfer Protocol (HTTP), although some legacy systems communicate over Simple Mail Transfer Protocol (SMTP), for message negotiation and transmission.
\end{quote}
Source: \url{https://en.wikipedia.org/wiki/SOAP}

\begin{list2}
\item
\item
\end{list2}


\slide{WSDL - Web Services Description Language}

\begin{quote}
  The Web Services Description Language (WSDL /ˈwɪz dəl/) is an XML-based interface description language that is used for describing the functionality offered by a web service. The acronym is also used for any specific WSDL description of a web service (also referred to as a WSDL file), which provides a machine-readable description of how the service can be called, what parameters it expects, and what data structures it returns. Therefore, its purpose is roughly similar to that of a type signature in a programming language.

  The current version of WSDL is WSDL 2.0. The meaning of the acronym has changed from version 1.1 where the "D" stood for "Definition".
\end{quote}
Source: \url{https://en.wikipedia.org/wiki/Web_Services_Description_Language}

\begin{list2}
\item
\item
\end{list2}





\slide{}

\begin{list2}
\item
\item
\end{list2}


\slide{}

\begin{list2}
\item
\item
\end{list2}


\slide{}

\begin{list2}
\item
\item
\end{list2}


\slide{}

\begin{list2}
\item
\item
\end{list2}


\slide{}

\begin{list2}
\item
\item
\end{list2}


\slide{}

\begin{list2}
\item
\item
\end{list2}



\slidenext

\end{document}
