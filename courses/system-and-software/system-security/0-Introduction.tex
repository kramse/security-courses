\documentclass[Screen16to9,17pt]{foils}
\usepackage{zencurity-slides}

\externaldocument{system-security-exercises}
\selectlanguage{english}


% VF1 Systemsikkerhed (10 ECTS)
% -----------------------------
% System Security. Den studerende kan udføre, udvælge, anvende, og implementere praktiske
% tiltag til sikring af firmaets udstyr og har viden og færdigheder der supportere dette.

% Viden
% Den studerende har viden om:
% * Generelle governance principper / sikkerhedsprocedurer
% * Væsentlige forensic processer
% * Relevante it-trusler
% * Relevante sikkerhedsprincipper til systemsikkerhed
% * OS roller ift. sikkerhedsovervejelser
% * Sikkerhedsadministration i DBMS.


% Færdigheder
% Den studerende kan:
% * Udnytte modforanstaltninger til sikring af systemer
% * Følge et benchmark til at sikre opsætning af enhederne
% * Implementere systematisk logning og monitering af enheder
% * Analysere logs for incidents og følge et revisionsspor
% * Kan genoprette systemer efter en hændelse.

% Kompetencer
% Den studerende kan:
% * håndtere enheder på command line-niveau
% * håndtere værktøjer til at identificere og fjerne/afbøde forskellige typer af endpoint trusler
% * håndtere udvælgelse, anvendelse og implementering af praktiske mekanismer til at forhindre, detektere og reagere over for specifikke it-sikkerhedsmæssige hændelser
% * håndtere relevante krypteringstiltag


\begin{document}

\mytitlepage
{0. Introduction}
{KEA Kompetence Computer Systems Security 2019}

\hlkprofiluk

\slide{Course Data}

{\Large\bf Course: Computer Systems Security\\
VF 3 Systemsikkerhed (10 ECTS)}

Teaching dates: tuesdays and thursdays
23/04, 25/04, 30/04, 02/05, 07/05, 09/05, 14/05, 16/05, 21/05, 23/05, 28/05, 04/06, 06/06, 11/06, 13/06

Exam: tuesday 25/06 exam

% {\bf Changes: the dates  and  will be moved!\\
% And since we are here, lets try to agree on best dates}

\slide{Course Materials}

\begin{list1}
\item This material is in multiple parts:
\begin{list2}
%\item Introduktionsmateriale med baggrundsinformation
\item Slide shows - presentation - this file
\item Exercises - PDF which is updated along the way
\end{list2}
\item Additional resources from the internet
\item Note: the presentation slides are not a substitute for reading the books, papers and doing exercises, many details are not shown
\end{list1}

\slide{Deliverables and Exam}

\begin{list2}
\item Exam
\item Individual: Oral based on curriculum
\item Graded (7 scale)
\item Draw a question with no preparation. Question covers a topic
\item Try to discuss the topic, and use practical examples
\item Exam is 30 minutes in total, including pulling the question and grading
\item Count on being able to present talk for about 10 minutes
\item Prepare material (keywords, examples, exercises, wireshark captures) for different topics so that you can use it to help you at the exam

\vskip 5mm
\item Deliverables:
\item 2 Mandatory assignments
\item Both mandatory assignments are required in order to be entitled to the exam.
\end{list2}

\slide{Fronter Platform}

\hlkimage{11cm}{fronter.png}

We will use fronter a lot, both for sharing educational materials and news during the course.

You will also be asked to turn in deliverables through fronter

\link{https://fronter.com/kea/main.phtml}

\vskip 5mm
\centerline{If you haven't received login yet, let us know}

\slide{Course Description}

From: STUDIEORDNING Diplomuddannelse i it-sikkerhed August 2018

Indhold:\\
Den studerende kan udføre, udvælge, anvende, og implementere praktiske
tiltag til sikring af firmaets udstyr og har viden og færdigheder der supportere dette.

Viden\\
Den studerende har viden om:
\begin{list2}
\item Generelle governance principper / sikkerhedsprocedurer
\item Væsentlige forensic processer
\item Relevante it-trusler
\item Relevante sikkerhedsprincipper til systemsikkerhed
\item OS roller ift. sikkerhedsovervejelser
\item Sikkerhedsadministration i DBMS.
\end{list2}

Færdigheder\\
Den studerende kan:
\begin{list2}
\item Udnytte modforanstaltninger til sikring af systemer
\item Følge et benchmark til at sikre opsætning af enhederne
\item Implementere systematisk logning og monitering af enheder
\item Analysere logs for incidents og følge et revisionsspor
\item Kan genoprette systemer efter en hændelse.

\end{list2}

Kompetencer\\
Den studerende kan:
\begin{list2}
\item håndtere enheder på command line-niveau
\item håndtere værktøjer til at identificere og fjerne/afbøde forskellige typer af endpoint trusler
\item håndtere udvælgelse, anvendelse og implementering af praktiske mekanismer til at forhindre, detektere og reagere over for specifikke it-sikkerhedsmæssige hændelser
\item håndtere relevante krypteringstiltag
\end{list2}

Final word is the Studieordning which can be downloaded from\\
{\footnotesize \link{https://kompetence.kea.dk/uddannelser/it-digitalt/diplom-i-it-sikkerhed}\\
\link{Studieordning_for_Diplomuddannelsen_i_IT-sikkerhed_Aug_2018.pdf}}

\slide{Expectations alignment}

In groups of 2 students, brainstorm for 5 minutes on what topics you would like to have in this course

Use 5 minutes more on Agreeing on 5 topics and prioritize these 5 topics

\slide{Primary literature}

Primary literature:
\begin{list2}
\item \emph{Computer Security: Art and Science}, Matt Bishop ISBN: 9780321712332
\end{list2}
Supporting literature:
\begin{list2}
\item \emph{Linux Basics for Hackers Getting Started with Networking, Scripting, and Security in Kali}. OccupyTheWeb, December 2018, 248 pp. ISBN-13: 978-1-59327-855-7 - shortened LBfH
\end{list2}


\slide{Book: Computer Security: Art and Science}
\hlkimage{6cm}{computer-security-art-and-science.jpg}

\emph{Computer Security: Art and Science}, Matt Bishop ISBN: 9780321712332

\link{https://nostarch.com/packetanalysis3}

\slide{Book: Linux Basics for Hackers (LBhf)}

\hlkimage{6cm}{LinuxBasicsforHackers_cover-front.png}

\emph{Linux Basics for Hackers
Getting Started with Networking, Scripting, and Security in Kali}
by OccupyTheWeb
December 2018, 248 pp.
ISBN-13:
9781593278557

\link{https://nostarch.com/linuxbasicsforhackers}


\slide{Book: Kali Linux Revealed (KLR)}

\hlkimage{6cm}{kali-linux-revealed.jpg}

\emph{Kali Linux Revealed  Mastering the Penetration Testing Distribution}

\link{https://www.kali.org/download-kali-linux-revealed-book/}\\
Not curriculum but explains how to install Kali Linux

\exercise{ex:downloadKLR}



%%% Break?

\slide{Hackerlab Setup}

\hlkimage{6cm}{hacklab-1.png}

\begin{list2}
\item Hardware: modern laptop CPU with virtualisation\\
Dont forget to enable hardware virtualisation in the BIOS
\item Virtualisation software: VMware, Virtual box, HyperV pick your poison
\item Hackersoftware: Kali Virtual Machine amd64 64-bit\link{https://www.kali.org/}
\item Linux server system: Debian 9 Stretch amd64 64-bit\link{https://www.debian.org/}
\item Setup instructions can be found at \link{https://github.com/kramse/kramse-labs}
\end{list2}

\centerline{It is enough if these VMs are pr team}


\exercise{ex:basicVM}

\exercise{ex:basicDebianVM}



\slide{Manualsystemet}

\hlkimage{7cm}{images/unix-command-1.pdf}

\begin{quote}
 It is a book about a Spanish guy called Manual. You should read it.
       -- Dilbert
\end{quote}

\begin{list1}
\item Manualsystemet i UNIX er utroligt stærkt!
\item Det SKAL altid installeres sammen med værktøjerne!
\item Det er næsten identisk på diverse UNIX varianter!
\item \verb+man -k+ søger efter keyword, se også \verb+apropos+
\end{list1}

Prøv \verb+man crontab+ og \verb+man 5 crontab+



\slide{En manualside}

\begin{alltt}\footnotesize
\small
NAME
     cal - displays a calendar
SYNOPSIS
     cal [-jy] [[month]  year]
DESCRIPTION
   cal displays a simple calendar.  If arguments are not specified, the cur-
   rent month is displayed.  The options are as follows:
   -j      Display julian dates (days one-based, numbered from January 1).
   -y      Display a calendar for the current year.

The Gregorian Reformation is assumed to have occurred in 1752 on the 3rd
of September.  By this time, most countries had recognized the reforma-
tion (although a few did not recognize it until the early 1900's.)  Ten
days following that date were eliminated by the reformation, so the cal-
endar for that month is a bit unusual.
\end{alltt}

\slide{Kommandolinien på UNIX}

\begin{list1}
\item Shells kommandofortolkere:
  \begin{list2}
    \item sh - Bourne Shell
\item bash - Bourne Again Shell, ofte default på Linux
\item ksh - Korn shell, lavet af David Korn
\item csh - C shell, syntaks der minder om C sproget
\item flere andre, zsh, tcsh
  \end{list2}
\item Svarer til command.com og cmd.exe på Windows
\item Kan bruges som komplette programmeringssprog
\end{list1}

\slide{Kommandoprompten}


\begin{alltt}
\small
[hlk@fischer hlk]$ id
uid=6000(hlk) gid=20(staff) groups=20(staff),
0(wheel), 80(admin), 160(cvs)
[hlk@fischer hlk]$

[root@fischer hlk]# id
uid=0(root) gid=0(wheel) groups=0(wheel), 1(daemon),
2(kmem), 3(sys), 4(tty), 5(operator), 20(staff),
31(guest), 80(admin)
[root@fischer hlk]#
\end{alltt}

\begin{list1}
\item typisk viser et dollartegn at man er logget ind som almindelig bruger
\item mens en havelåge at man er root - superbruger
\end{list1}

\slide{Kommandoliniens opbygning}


\begin{alltt}
echo [-n] [string ...]
\end{alltt}

\begin{list1}
\item Kommandoerne der skrives på kommandolinien skrives sådan:
\begin{list2}
\item Starter altid med kommandoen, man kan ikke skrive \verb+henrik echo+
\item Options skrives typisk med bindestreg foran, eksempelvis \verb+-n+
\item Flere options kan sættes sammen, \verb+tar -cvf+ eller \verb+tar cvf+
\item I manualsystemet kan man se valgfrie options i firkantede
  klammer \verb+[]+
\item Argumenterne til kommandoen skrives typisk til sidst (eller der
  bruges redirection)
\end{list2}
\end{list1}


\slidenext{Buy the books!}



\end{document}
