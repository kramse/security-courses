\documentclass[Screen16to9,17pt]{foils}
\usepackage{zencurity-slides}

\externaldocument{system-security-exercises}
\selectlanguage{english}

\begin{document}

\mytitlepage
{13. Benchmarking and Auditing Recap}
{KEA Kompetence Computer Systems Security 2019}


\slide{Plan for today}

\begin{list1}
\item Subjects
\begin{list2}
\item Using CIS controls and Visa PCI
\end{list2}
\item Exercises
\begin{list2}
\item Evaluate our network, quick gap analysis for becoming PCI compliant
\end{list2}
\end{list1}



\slide{Reading Summary}

\begin{list1}
\item CIS controls
\item PCI Best Practices for Maintaining PCI DSS Compliance v2.0 Jan 2019\\
Payment Card Industry Data Security Standard
\end{list1}



\slide{Building Secure Infrastructures}

\begin{list1}
\item We did an exercise last time, starting to build a DMZ for servers
\item A real-life setup of an infrastructure from scratch can be daunting!
\item You need:
\begin{list2}
\item Policies
\item Procedures
\item Incident Response
\end{list2}
\item Running systems which require
\begin{list2}
\item Configurations
\item Settings
\item Supporting infrastructure -- networks
\item Supporting infrastructure -- logging, dashboarding, monitoring
\end{list2}
\item Building something \emph{secure} is {\bf hard work!}
\end{list1}



\slide{Existing infrastructures}

\begin{list1}
\item or even worse you inherited an infrastructure
\item Multiple networks, with different vendors, rules
\item Multiple generations of services, applications, technologies
\item Built over decades
\item Varying to no documentation
\item Organizational challenges
\item Ingrained culture -- frozen in time
\end{list1}

How do you get started improving security?


\slide{Security Controls and Frameworks}

\begin{list1}
\item Multiple exist
\vskip 1cm
\item CIS controls Center for Internet Security (CIS) \link{https://www.cisecurity.org}
\item PCI Best Practices for Maintaining PCI DSS Compliance v2.0 Jan 2019
\item NIST Cybersecurity Framework (CSF)\\
Framework for Improving
Critical Infrastructure Cybersecurity\\ \link{https://www.nist.gov/cyberframework}
\end{list1}


\slide{Risk management defined}

\hlkimage{24cm}{shon-harris-risk-management.png}

Source: Shon Harris \emph{CISSP All-in-One Exam Guide}


\slide{First advice use the modern operating systems}

\begin{list1}
\item Newer versions of Microsoft Windows, Mac OS X and Linux
\begin{list2}
\item Buffer overflow protection
\item Stack protection, non-executable stack
\item Heap protection, non-executable heap
\item \emph{Randomization of parameters} stack gap m.v.
\end{list2}
\item Note: these still have errors and bugs, but are better than older versions
\item OpenBSD has shown the way in many cases\\ \link{http://www.openbsd.org/papers/}
\end{list1}

\vskip 1cm

\centerline{Always try to make life worse and more costly for attackers}


\slide{Good security}

\hlkimage{15cm}{god-sikkerhed.pdf}

\begin{list1}
\item You always have limited resources for protection - use them as best as possible
\end{list1}


\slide{First advice}

\begin{list1}
\item Use technology
\item Learn the technology - read the freaking manual
\item Think about the data you have, upload, facebook license?! WTF!
\item Think about the data you create - nude pictures taken, where will they show up?
\begin{list2}
\item Turn off features you don't use
\item Turn off network connections when not in use
\item Update software and applications
\item Turn on encryption: IMAP{\bf S}, POP3{\bf S},
  HTTP{\bf S} also for data at rest, full disk encryption, tablet encryption
\item Lock devices automatically when not used for 10 minutes
\item Dont trust fancy logins like fingerprint scanner or face recognition on cheap devices
\end{list2}
\end{list1}


\slide{Spearphishing - targetted attacks}


Spearphishing - targetted attacks directed at specific individuals or companies

\begin{list2}
\item Use 0-day vulnerabilities only in a few places
\item Create backdoors and mangle them until not recognized by Anti-virus software
\item Research and send to those most likely to activate program, open file, visit page
\item Stuxnet is an example of a targeted attack using multiple 0-day vulns
\end{list2}


\slide{Defense in depth - flere lag af sikkerhed}

\hlkimage{6cm}{security-layers-1-uk.pdf}

\centerline{\hlkbig Defense using multiple layers is stronger!}


\slide{Integrate or develop?}

\begin{list1}
\item Dont:
\begin{list2}
\item Reinvent the wheel - too many times, unless you can maintain it afterwards
\item Never invent cryptography yourself
\item No copy paste of functionality, harder to maintain in the future
\end{list2}
\item Do:
\begin{list2}
\item Integrate with existing solutions
\item Use existing well-tested code: cryptography, authentication, hashing
\item Centralize security in your code and organization
\end{list2}
\end{list1}


\slide{Balanced security}

\hlkimage{21cm}{afbalanceret-sikkerhed.pdf}

\begin{list1}
\item Better to have the same level of security
\item If you have bad security in some part - guess where attackers will end up
\item Hackers are not required to take the hardest path into the network
\item Realize there is no such thing as 100\% security
\end{list1}



\slide{Work together}

\hlkimage{10cm}{Shaking-hands_web.jpg}

\begin{list1}
\item Team up!
\item We need to share security information freely
\item We often face the same threats, so we can work on solving these together
\end{list1}



\slide{How to become secure}

\begin{list1}
\item Dont use computers at all, data about you is still processed by computers :-(
\item Dont use a single device for all types of data
\item Dont use a single server for all types of data, mail server != web server
\item Configure systems to be secure by default, or change defaults
\item Use secure protocols and VPN solutions
\item Some advice can be found in these places
\begin{list2}
\item \link{http://csrc.nist.gov/publications/PubsSPs.html}
\item \link{http://www.nsa.gov/research/publications/index.shtml}
\item \link{http://www.nsa.gov/ia/guidance/security_configuration_guides/index.shtml}
\end{list2}
\end{list1}



\slide{Center for Internet Security CIS Controls}

\begin{quote}
  The CIS ControlsTM are a prioritized set of actions that collectively form a defense-in-depth set
of best practices that mitigate the most common attacks against systems and networks. The
CIS Controls are developed by a community of IT experts who apply their first-hand experience
as cyber defenders to create these globally accepted security best practices. The experts who
develop the CIS Controls come from a wide range of sectors including retail, manufacturing,
healthcare, education, government, defense, and others.
\end{quote}

\begin{list1}
\item
\item
\item
\item
\end{list1}




\slide{CIS RAM}

\begin{quote}
CIS Risk Assessment Method is a free information security risk assessment method that helps organizations implement and assess their security posture against the CIS Controls™ cybersecurity best practices. CIS RAM provides instructions, examples, templates, and exercises for conducting a cyber risk assessment.
\end{quote}

\begin{list1}
\item
\item
\item
\item
\end{list1}



\slide{}

\begin{list1}
\item
\item
\item
\item
\end{list1}



\slide{Payment Card Industry Data Security Standard}

\begin{list1}
\item PCI Best Practices grew out of credit card leaks becoming a huge problem
\item Partnership between Master Card, VISA and others
\item Version  1.0 release in December, 2004
\item Version 3.2.1 was released in May 2018 \link{https://en.wikipedia.org/wiki/Payment_Card_Industry_Data_Security_Standar}
\end{list1}



\slide{PCI DSS Control Objectives}

\begin{list1}
\item High level objectives:
\begin{list2}
\item Build and Maintain a Secure Network and Systems
\item Protect Cardholder Data
\item Maintain a Vulnerability Management Program
\item Implement Strong Access Control Measures
\item Regularly Monitor and Test Networks
\item Maintain an Information Security Policy
\end{list2}
\end{list1}



\slide{Requirements for Building and Maintaining Security}

\begin{list1}
\item Installing and maintaining a firewall configuration to protect cardholder data. The purpose of a firewall is to scan all network traffic, block untrusted networks from accessing the system.
\item Changing vendor-supplied defaults for system passwords and other security parameters. These passwords are easily discovered through public information and can be used by malicious individuals to gain unauthorized access to systems.
\item Protecting stored cardholder data. Encryption, hashing, masking and truncation are methods used to protect card holder data.
\item Encrypting transmission of cardholder data over open, public networks. Strong encryption, including using only trusted keys and certifications reduces risk of being targeted by malicious individuals through hacking.

\end{list1}

\slide{Requirements for Building and Maintaining Security, cont}

\begin{list1}
\item Protecting all systems against malware and performing regular updates of anti-virus software. Malware can enter a network through numerous ways, including Internet use, employee email, mobile devices or storage devices. Up-to-date anti-virus software or supplemental anti-malware software will reduce the risk of exploitation via malware.
\item Developing and maintaining secure systems and applications. Vulnerabilities in systems and applications allow unscrupulous individuals to gain privileged access. Security patches should be immediately installed to fix vulnerability and prevent exploitation and compromise of cardholder data.

\end{list1}

\slide{Requirements for Building and Maintaining Security, cont}

\begin{list1}
  \item Restricting access to cardholder data to only authorized personnel. Systems and processes must be used to restrict access to cardholder data on a “need to know” basis.
  \item Identifying and authenticating access to system components. Each person with access to system components should be assigned a unique identification (ID) that allows accountability of access to critical data systems.
  \item Restricting physical access to cardholder data. Physical access to cardholder data or systems that hold this data must be secure to prevent the unauthorized access or removal of data.
\item Tracking and monitoring all access to cardholder data and network resources. Logging mechanisms should be in place to track user activities that are critical to prevent, detect or minimize impact of data compromises.
\end{list1}



\slide{Requirements for Building and Maintaining Security, cont}


\begin{list1}
\item Testing security systems and processes regularly. New vulnerabilities are continuously discovered. Systems, processes and software need to be tested frequently to uncover vulnerabilities that could be used by malicious individuals.
\item Maintaining an information security policy for all personnel. A strong security policy includes making personnel understand the sensitivity of data and their responsibility to protect it.
\end{list1}




\slide{Benefits of PCI DSS}

\begin{list1}
\item My opinion:
\item Before PCI theres was a LOT of breaches
\item Minimum requirements for credit card companies, should be minimum requirements for personal data
\item Good requirements, a library of tested requirements

\end{list1}




\slidenext

\end{document}
