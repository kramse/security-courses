\documentclass[Screen16to9,17pt]{foils}
\usepackage{kea-slides}

\externaldocument{system-security-exercises}
\selectlanguage{english}

\begin{document}

\mytitlepage
{9. Breaking Out}
{KEA Kompetence Computer Systems Security \the\year}


\slide{Goals for today}

\hlkimage{6cm}{thomas-galler-hZ3uF1-z2Qc-unsplash.jpg}

Todays goals:
\begin{list2}
\item
\end{list2}

  Photo by Thomas Galler on Unsplash

\slide{Plan for today}

\begin{list1}
\item Subjects
\begin{list2}
\item
\end{list2}
\item Exercises
\begin{list2}
\item
\end{list2}
\end{list1}



\slide{Reading Summary}

\begin{list1}
\item MLSH 11: Kernel Hardening and Process Isolation
\item DSH chapter 16: Vulnerability Management
\item Browse: Using Memory Errors to Attack a Virtual Machine paper, An Experimental Study of DRAM Disturbance Errors, Exploiting the DRAM rowhammer bug to gain kernel privileges \url{https://en.wikipedia.org/wiki/Row_hammer}
\end{list1}






\slide{Availability Policies}

\begin{list1}
\item An availability policy ensures that a resource or service can be accessed in some way in a timely fashion
\item Often expressed as \emph{quality of service}
\item Denial of service occurs when this resource or service becomes unavailable
\end{list1}




\slide{Fairness and starvation}

\begin{list1}
\item Fairness policy prevents starvation, often rephrased as - process will make progress
\item If one process gets all resources, memory, cpu, network the others will starve - not have enough resources to progress
\item Compare to old operating systems Windows 3 / Mac OS 9\\
Cooperative multitasking vs pre-emptive multitasking
\end{list1}

\slide{Availability and Network flooding attacks}

\begin{list2}
\item Our book spends some time on SYN and other flooding attacks
\item SYN flood is the most basic and very common on the internet towards 80/tcp and 443/tcp
\item ICMP and UDP flooding are the next targets
\item Supporting litterature is TCP Synfloods - an old yet current problem, and improving pf's response to it, Henning Brauer, BSDCan 2017
\item All of them try to use up some resources
\begin{list2}
\item Memory space in specific sections of the kernel, TCP state, firewalls state, number of concurrent sessions/connections
\item interrupt processing of packets - packets per second
\item CPU processing in firewalls, pps
\item CPU processing in server software
\item Bandwidth - megabits per second mbps
\end{list2}
\end{list2}

There is a presentation about DDoS protection with low level technical measures to implement at\\
{\footnotesize \link{https://github.com/kramse/security-courses/tree/master/presentations/network/introduction-ddos-testing}}


\exercise{ex:syn-flood-101}


\slidenext

\end{document}
