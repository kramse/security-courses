\documentclass[a4paper,11pt,notitlepage,landscape]{report}
% Henrik Kramselund  , February 2001
% hlk@security6.net,
% My standard packages
\usepackage{zencurity-one-page}
\usepackage{alltt}
%\usepackage{lscape}

\begin{document}

%\rm
\selectlanguage{english}

\lhead{\fancyplain{}{\color{titlecolor}\bfseries\LARGE Curriculum: KEA System Security F2022 course}}

\normal

Below are the required reading for the course KEA System Security F2022.

Maybe compare to the exam subjects list, and keywords.


{\bf Primary literature}

%\hlkrightpic{5cm}{0cm}{old_book_lumen_design_st_01.png}
Primary literature - not all chapters are part of the curriculum:
\begin{list2}
\item \emph{Computer Security: Art and Science}, 2nd edition 2019! Matt Bishop ISBN: 9780321712332 1440 pages
\item \emph{Defensive Security Handbook: Best Practices for Securing Infrastructure}, Lee Brotherston, Amanda Berlin ISBN: 978-1-491-96038-7 284 pages
\item \emph{Forensics Discovery}, Dan Farmer, Wietse Venema 2004, Addison-Wesley 240 pages. Can be found at http://www.porcupine.org/forensics/forensic-discovery/ but recommend buying it. Referenced below as FD
\end{list2}

The following chapters are curriculum:

\begin{list2}
\item \emph{Computer Security: Art and Science}: chapters 1,2,4,6,7,8, 10, 11 until and including 11.4, 12 until and including 12.5.3.10, 13 until and including 13.5, 14-16, 18, 23, 25, 26, 27
\\
Note: skip/skim some policy language examples, skip 4.7, skip 5.2.3 and similar math parts throughout the book!\\
Skim read if you can: chapters 5, 24, 28, 29, 30 appendix G


\item \emph{Defensive Security Handbook}: chapters 1-8, 19-20\\
Skim read if you can: chapters 10-12
\item \emph{Forensics Discovery}, basic concepts of computer forensics must be known, the book as a whole is not curriculum, skim reading chapters 1-6, and appendix B are highly recommended
\end{list2}

The following documents are curriculum:
\begin{list2}
\item Smashing The Stack For Fun And Profit, Aleph One\\
\verb+stack_smashing.pdf+
\item NCSC IT Security Guidelines for Transport Layer Security\\
\verb_IT+Security+Guidelines+for+Transport+Layer+Security+v2.1.pdf_
\item TCP Synfloods - an old yet current problem, BSDCan slides\\
\verb+http://quigon.bsws.de/papers/2017/bsdcan/+
\item Campus Operations Best Current Practices, NSRC\\
\verb+Campus_Operations_BCP.pdf+
\item Mutually Agreed Norms for Routing Security (MANRS)\\
\verb+MANRS_PDF_Sep2016.pdf+
\item Blog post about Mitre ATT\&CK\\
\verb+Mitre-ATTACK-101-Medium.pdf+
\end{list2}


The following concepts are to be known, as concepts at least, read introduction and table of contents:
\begin{list2}
\item CIS controls, Center of Internet Security
\item PCI Best Practices for Maintaining PCI DSS Compliance v2.0 Jan 2019
\item NIST Special Publication 800-63B, concept here are also NIST Special Publications in general
\item Enterprise Survival Guide for Ransomware Attacks, Shafqat Mehmoon, SANS Reading room
\item Incident Handler's Handbook, Patrick Kral, SANS Reading room
\item Download and browse the ENISA papers listed under Computer Forensics in the reading list, ENISA as a concept is also required
\end{list2}





\end{document}
