\documentclass[20pt,landscape,a4paper,footrule]{foils}
\usepackage{crypto-slides}


% Program

% 16.40 - 17.00
% FDE
% Bitlocker (Kramse)
% FDE på iPhone (Freja)
% (længde på kode, tjek at den er slået til, forklare hvad den gør)


% 17.30 - 17.50
% På nettet (Kramse)
% HTTPS everywhere
% Flere browsere
% Tor (IP-adresse osv)



\begin{document}

%\slide{}

\mytitlepage
{Fuld Disk Kryptering}

%\vskip 2cm
\centerline{\footnotesize
 PDF available kramshoej@Github}

\LogoOn

%

\slide{Fuld Disk Kryptering}

\begin{list1}
\item Data skal beskyttes, men hvor er data?
\item Vi benytter begreberne:
\begin{list2}
\item Data in transit - undervejs
\item Data at rest - i hvile
\end{list2}
\item Det er nemt at flytte en harddisk fra en computer til en anden
\item Fuld Disk Kryptering sikrer data-at-rest
\item Anbefales fremfor løsninger der kun krypterer dele af harddisken/enkeltdokumenter
\item NB: for optimal sikkerhed skal disken krypteres FØR der opbevares fortrolige data
\end{list1}

Kilder:\\
\link{https://en.wikipedia.org/wiki/Data_at_rest}\\
\link{https://en.wikipedia.org/wiki/Data_in_transit}

\slide{Kryptografi}

\hlkimage{18cm}{images/crypto-rot13.pdf}

\begin{list1}
\item Kryptografi er læren om, hvordan man kan kryptere data
\item Kryptografi benytter algoritmer som sammen med nøgler giver en
  ciffertekst - der kun kan læses ved hjælp af den tilhørende nøgle
\end{list1}

\slide{Public key kryptografi - 1}

\hlkimage{18cm}{images/crypto-public-key.pdf}

\begin{list1}
\item privat-nøgle kryptografi (eksempelvis AES) benyttes den samme
  nøgle til kryptering og dekryptering
\item offentlig-nøgle kryptografi (eksempelvis RSA) benytter to
  separate nøgler til kryptering og dekryptering
\end{list1}

\slide{Public key kryptografi - 2}

\hlkimage{18cm}{images/crypto-public-key-2.pdf}

\begin{list1}
\item offentlig-nøgle kryptografi (eksempelvis RSA) bruger den private
  nøgle til at dekryptere
\item man kan ligeledes bruge offentlig-nøgle kryptografi til at
  signere dokumenter\\ - som så verificeres med den offentlige nøgle
\end{list1}


\slide{Fuld Disk Kryptering: Bitlocker}

\begin{list2}
\item Microsoft tilbyder Bitlocker fuld disk kryptering
\item Åbnes med dit Windows kodeord
\item Meget transparent - data krypteres når det skrives ned
\item Nedsætter ikke hastigheden mærkbart, ofte forbedres den endda
\item Genetableringsnøgle - er slået til på FT computere\\
Giver mulighed for at IT-afd kan åbne din computer hvis du glemmer koden
\item Fungerer på både roterende diske og SSD, \\
men pas på SSD kan have data fra før kryptering slået til
\end{list2}

Kilde: mere information om Bitlocker\\
{\footnotesize \link{http://windows.microsoft.com/en-us/windows-vista/bitlocker-drive-encryption-overview}}

\slide{Bitlocker}

\hlkimage{18cm}{bitlocker-ms.jpg}

Kilde: {\small
\link{https://technet.microsoft.com/en-us/library/cc512654.aspx}}

\slide{Angreb mod disk kryptering}

\hlkimage{17cm}{warm-boot-attack.png}

\begin{list1}
\item Fælles for mange angreb er at Computeren skal være tændt eller i dvale,\\
så er nøglen er i hukommelsen og kan måske fiskes ud
\item Det anbefales derfor at du lukker computeren helt ned, hvis den forlades
i længere tid
\end{list1}

Kilde:
\link{https://citp.princeton.edu/research/memory/}

\slide{Bonus: Full Disk Encryption Mac OS X}

\hlkimage{16cm}{apple-filevault-enabled.png}

\centerline{Indbygget, gratis, stærk - slå det til når I kommer hjem}


% 17:30 20 minutter
\mytitlepage
{Sikker på nettet}


\slide{Internet set fra Zencurity AS57860}

\hlkimage{20cm}{asn-cloud-2016.png}

Small excerpt from:\\
\link{https://stat.ripe.net/185.129.62.0/22\#tabId=routing}

\slide{Drive-by-download}

\hlkimage{27cm}{drive-by-download-wikipedia.png}

Kilde: Wikipedia

\slide{Brug flere browsere}

\hlkimage{24cm}{multi-browser-strategy.png}


\slide{Fordele ved flere browsere}
\begin{list1}
\item Flere browsere giver højere sikkerhed
\item Data kan ikke flyde mellem flere browsere, cookies m.m.
\item Mit forslag:
\begin{list2}
\item En browser til \emph{sikre sites} banken, intranet
\item En browser til generel internet surfing
\item En browser med alle mulige plugins, web udvikling eksempelvis
\end{list2}
\item Installer gerne plugins til højere sikkerhed i allesammen:\\
HTTPS Everywhere, NoScript/ScriptBlock m.fl.

\end{list1}

\vskip 1cm
\centerline{Det anbefales at disse installeres og vedligeholdes fra IT-afdelingen}


\vskip 1cm
\centerline{\bf\Large Alle browsere har mange fejl!}

\slide{Safari på Mac eller Internet Explorer Windows}

\hlkimage{9cm}{tast-selv-krav.png}
\begin{list2}
\item Den indbyggede browser testes af banker, skat, m.fl.
\item Disse sites kræver oftest javascript, cookies, Flash og tidligere endda Java
\item {\bf Java} benyttes stadig visse steder - {\bf højrisiko}
\item Kan vise video og andet aktivt indhold, eksempelvis Netflix
\end{list2}

\slide{Chrome en rimeligt sikker browser}

\hlkimage{10cm}{clicktoflash.png}

\begin{list1}
\item Generelt er internet browsing en risikofyldt aktivitet
\item Drive-by-download hacking er reel trussel
\item {\bf Opdaterer sig selv løbende}
\item Egen Sand-box til Flash
\item Denne browser kan indstilles rimeligt sikkert
\end{list1}

\slide{Firefox}

\begin{list2}
\item Firefox er en god generel browser med mange plugins
\item Plugins kan indeholde mange fejl, læs indeholder mange fejl
\item Firefox er et godt værktøj til webudvikling og andre formål

\end{list2}

\centerline{Jeg anbefaler derfor Firefox til de opgaver hvor du skal bruge mange plugins}


\slide{Generelt indstillinger for browsere}

\begin{list1}
\item Skal være indstillet på den sikre browser til generel surf
\begin{list2}
\item Slå JavaScript fra generelt med NoScript/ScriptBlock
\item Slå click-to-play til for aktivt indhold
\item Slå "Do Not Track" til
\item Slå Java helt fra, afinstaller evt. Java helt fra computeren
\item Installer en AdBlocker - jeg bruger AdBlock\\
Vigtigt: servere der viser reklamer er ofte mål for hacking
\end{list2}
\end{list1}

\slide{Hvor ændrer man indstillingerne}

\hlkimage{20cm}{firefox-settings.png}

De fleste findes under:
\begin{list2}
\item Chrome \link{chrome://settings/} og \link{chrome://extensions/}
\item Firefox Indstillingerne og for enkelte ting: \link{about:config}
\end{list2}

\centerline{Kig også gerne på Safari eller Internet Explorer indstillingerne}



\slide{HTTPS Everywhere}

\hlkimage{5cm}{HTTPS_Everywhere_new_logo.jpg}
\begin{quote}
HTTPS Everywhere is a Firefox extension produced as a collaboration between The Tor Project and the Electronic Frontier Foundation. It encrypts your communications with a number of major websites.
\end{quote}

\centerline{\link{http://www.eff.org/https-everywhere}}

Also in Chrome web store!

\slide{NoScript Firefox and ScriptBlock Chrome}

\hlkimage{18cm}{scriptblock.png}

\vskip 2cm
NoScripts for Firefox eller ScriptBlock for Chrome\\
Tillader kun JavaScript på sider hvor det er OK


\slide{Tor project anonym webbrowsing}

\hlkimage{23cm}{tor-project.png}

\centerline{\link{https://www.torproject.org/}}

\vskip 2cm
\centerline{Der findes alternativer, men Tor er mest kendt}

\slide{Tor project - how it works 1}

\hlkimage{21cm}{how-tor-works-1.png}

\centerline{pictures from \link{https://www.torproject.org/about/overview.html.en}}

\slide{Tor project - how it works 2}

\hlkimage{21cm}{how-tor-works-2.png}

\centerline{pictures from \link{https://www.torproject.org/about/overview.html.en}}

\slide{Tor project - how it works 3}

\hlkimage{21cm}{how-tor-works-3.png}

\centerline{pictures from \link{https://www.torproject.org/about/overview.html.en}}


%\slide{Torbrowser - outdated}

%\hlkimage{20cm}{torbrowser-outdated.png}

%\centerline{\color{red}Hov den mangler opdatering!}

\slide{Torbrowser - anonym browser}

\hlkimage{20cm}{torbrowser-main-window.png}

\centerline{\color{titlecolor} Mere anonym browser - Firefox i forklædning}


\slide{Torbrowser - hidden service web site}

\hlkimage{18cm}{sample-tor-site.png}

\centerline{\color{titlecolor} .onion er Tor adresser - hidden sites}

%\centerline{Den viste side er SecureDrop hos Radio24syv}\\
\link{http://www.radio24syv.dk/dig-og-radio24syv/securedrop/}



\end{document}
