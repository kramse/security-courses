\documentclass[Screen16to9,17pt]{foils}
%\documentclass[16pt,landscape,a4paper,footrule]{foils}
\usepackage{zencurity-slides}

% Kubernetes Security

% Vi ser på Kubernetes med sikkerhedsbrillerne på.

% Kubernetes er blevet en af de mest populære cloud teknologier både til self-hosted og hos cloud leverandører. Vi vil i dette foredrag se på infrastrukturen i cloud med Kubernetes som eksempel. Det er emner som netværksovervågning, stresstest, muligheder for beskyttelse og isolering samt forensic og opklaring af hændelser.

% Deltagerne vil efterfølgende kunne opsætte miljø tilsvarende underviserens og træne sikkerhed i et lukket miljø.

% Fokus vil være på best current practice med rigeligt med referencer til dokumenter og teknisk know-how, hvordan det kan gøres hos jer selv i praksis bagefter.

% Målgruppe: alle der er interesserede i overvågning af cloud, men primært Kubernetes.

% Varighed: 4 timer inkl pauser

% Nøgleord:
% Cloud security, DDoS protection, load balancing, logs and audit, cloud monitoring, performance testing

% Dato: Mandag 6. februar kl. 17.00-21.00
% Sted: Online. Direkte link bliver sendt pr. mail på dagen.
% Pris: Gratis for medlemmer af PROSA. 525 kr. for ikke-medlemmer



% Husk




\begin{document}

%\rm
\selectlanguage{english}

\mytitlepage
{Kubernetes Security}

\LogoOn

%\dagsplan

\slide{Plan for today}

\begin{list1}
\item Subjects
\begin{list2}
\item
\item
\item
\item
\end{list2}
\end{list1}


\slide{Goals for today}
\vskip 1 cm

\hlkimage{3cm}{dont-panic.png}
\centerline{\color{titlecolor}\LARGE Don't Panic!}


\begin{list1}
\item Introduce the term penetration testing and basic pentest methods
\item Introduce some of the basic tools in the genre of hacker tools
\item Create an understanding of Kubernetes attack surface
\item Show a hacker lab and run some tools
\item Point you towards resources, so you can get started with the fun of pentesting tools
\end{list1}

\slide{Materials -- where to start}

\begin{list2}
\item This presentation -- slides for today, start here
\item \emph{Container Security}, Liz Rice, OReilly, Apr 2020.
\end{list2}

%\centerline{We cannot go through all of them, but feel free to ask questions later}

{\bf Start a download of Kali today, if you want to play with the tools tomorrow}\\
Recommend virtual machine download 64-bit \url{https://www.kali.org/get-kali/#kali-virtual-machines}

\slide{Books and educational materials}

\begin{list2}

\item \emph{Kubernetes Security}
\item
\end{list2}



\slide{Kubernetes Attack Surface}



\slide{Sysdig Analysis of DockerHub}

\emph{Analysis on Docker Hub malicious images: Attacks through public container images}


\link{https://sysdig.com/blog/analysis-of-supply-chain-attacks-through-public-docker-images/}



\slide{Benchmarking tools}


CIS

Kube-bench is the industry-standard tool to automate checking Kubernetes compliance with the Center for Internet Security (CIS) Benchmark.

Kube-bench makes it easy for operators to check whether each node in their Kubernetes cluster is configured according to security best practices.

\link{https://info.aquasec.com/open-source}

\slide{Hacker lab setup}

\hlkimage{8cm}{hacklab-1.png}

\begin{list2}
\item Hardware: any modern laptop with CPU and virtualisation\\
Don't forget to enable it in the BIOS
\item Software: your favourite operating system Windows, Mac, Linux, ...
\item Virtualisation software: VMware, Virtual box, pick your poison
\item Hacker software: Kali as a Virtual Machine \link{https://www.kali.org/}
\item Kubernetes: Minikube -  I run this on Debian 11
\end{list2}



\slide{Conclusion Kubernetes Security}

\hlkimage{10cm}{network-layers-2022.pdf}
~
\begin{list2}
\item
\end{list2}

\myquestionspage



\end{document}
