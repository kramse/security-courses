\documentclass[Screen16to9,17pt]{foils}
\usepackage{zencurity-slides}

\selectlanguage{danish}

% Diversitet er et fælles anliggende for alle i tech v. Henrik Kramselund, Internet-, netværks- og sikkerhedskonsulent

% Radia, Dorothy, Elizabeth og Laura er fornavnene på nogle af internet-, netværks- og sikkerhedskonsulentens rollemodeller inden for IT. Lær mere om dem og vær med, når Kramselund giver et call to action og deler erfaringer med at skabe inkluderende miljøer i tech



\begin{document}

\mytitlepage{Building a Firewall}
{Show and tell}

\vskip 1cm
\centerline{\footnotesize Slides are available as PDF kramshoej@Github}



\centerline{\footnotesize Not affiliated with the OpenBSD project, but a long time and very happy user}

\slide{Goal}

We are building a new firewall, yay!

For this purpose I brought a lot of hardware! Really a lot!
\begin{list2}
\item OpenBSD as a CPE in a network with ordinary internet traffic
\item Arist router/switch in the datacenter
\item Juniper EX enterprise switch
\item We will discuss how to build a network with available devices
\item Talk about BGP, PF and service daemons
\item Mostly Juniper configuration and OpenBSD configs
\item BGP to PF Tables, firewalling/NAT based on BGP updates
\item OpenBSD niceness, why choosing OpenBSD made a lot of things easier
\item Keywords:
OpenBSD, BGP, routing, IEEE 802.1q, VLAN, IEEE802.1p, CoS/QoS, VoIP, firewalling
\end{list2}

\slide{Thank you OpenBSD}

\hlkimage{16cm}{openbsd.png}

\centerline{Donating to OpenBSD also supports OpenSSH development}

OpenBSD has a complete \emph{OpenBSD PF - User's Guide} available at:\\
\link{https://www.openbsd.org/faq/pf/index.html}

This includes a generic firewall example (4 pages if printed):\\
\link{https://www.openbsd.org/faq/pf/example1.html}

I will use some examples from a network I built with some interesting bits.

The same processes and features are available in commercial tools too

\slide{Overview}

\hlkimage{17cm}{patientsky-net-overview.png}

\slide{OpenBSD CPE: BGP, PF and service daemons}

Example network
\hlkimage{16cm}{openbsd-cpe.png}

\begin{list2}
\item OpenBSD operating system
\item Solid hardware + free operating system = reliable service
\item ISP CPE, yes some ISPs use ASR920 as CPE for 8Mbit SHDSL too\smiley
\end{list2}


\slide{Existing infrastructure, in some places}

\hlkimage{7cm}{bad-rack.jpg}

\centerline{Pretty nice heh, not always this bad, but usually not good}

\slide{Building a Network with a Firewall}

\begin{list1}
\item Vi vil nu gennemgå netværksdesign med udgangspunkt i vores setup
\item Vores setup indeholder:
\begin{list2}
\item Routere
\item Firewall
\item Wireless
\item DMZ
\item DHCPD, BGPD, OSPFD, ...
\item Name service
\end{list2}
\item Den kunne udvides med flere andre teknologier vi har til rådighed:
\begin{list2}
\item VLAN inkl VLAN trunking/distribution, WPA Enterprise
\end{list2}
\item Husk følgende slides er min mening, og fra et specifikt netværk
\item Typisk udvikler det sig over tid
\end{list1}


\slide{Principle of Economy of Mechanism}

\begin{list1}
\item {\bf Definition 14-4} The \emph{principle of economy of mechanism} states that security mechanisms should be as simple as possible.
\item Simple $->$ fewer complications $->$ fewer security errors
\item Use WPA passphrase instead of MAC address based authentication
\item
\end{list1}

\slide{Principle of Fail-Safe defaults}

\begin{list1}
\item {\bf Definition 14-3} The \emph{principle of fail-safe defaults} states that, unless a subject is given explicit access to
 an object, it should be denied access to that object.
\item Default access \emph{none}
\item In firewalls default-deny - that which is not allowed is prohibited
\item Newer devices today can come with no administrative users, while older devices often came with default admin/admin users
\item Real world example, OpenSSH config files that come with \verb+PermitRootLogin no+
\end{list1}



\slide{Important processes and components}

\begin{list2}
\item OpenBSD kernel - does routing, thank you
\item OpenBSD kernel - multiple routing tables, allow drop-in replacement in networks
\item OpenNTP - time keeping
\item OpenBGP - BGP to get NHN prefixes
\item relayd - provides failover, change default route
\item OpenSSHD - secure remote access
\item DHCPD - dhcp service to LAN
\item OpenBSD PF - awesome firewall software connecting it all nicely
\item OpenBSD PF queue - allow detailed control of bandwidth
\item OpenBSD PF prio into VLAN header - QoS/CoS of VoIP traffic
\item NLnet Labs Unbound DNS server \link{https://www.nlnetlabs.nl/projects/unbound/about/}
\end{list2}

The configurations shown in the presentation are examples from a real network

\slide{Relayd failover to 4G router}

\begin{alltt}\footnotesize
# cat /etc/relayd.conf
primary = "185.xxx.xxx.x"
secondary = "192.168.8.1"
interval 10
table <gateways> \{ $primary ip ttl 1 priority 10, $secondary ip ttl 1 priority 50 \}
router "uplinks" \{
        route 0.0.0.0/0
        forward to <gateways> check icmp
\}
\end{alltt}

Easy to understand, easy to implement

\slide{OpenBSD queue pf.conf - from 20/20Mbit customer}

\begin{alltt}\footnotesize
# Queue to fix TCP originating from our host, if we send more than
# bandwidth the shaping done by ISP cause huge backoffs
queue root on em3 bandwidth 20M max 20M

# Currently used for VoIP and PatientSky Hosting
# Note: VoIP Max 25% of bandwidth, excess dropped by provider!
queue high parent root bandwidth 5M max 5M
queue normal parent root bandwidth 20M max 20M default
queue low parent root bandwidth 15M max 15M

# {\bf Download queues} on inside interface to LAN
# by limiting this, we end up receiving less than max from outside
queue dn_parent on em2 bandwidth 20M max 20M
queue dn_high parent dn_parent bandwidth 20M max 20M
queue dn_default parent dn_parent bandwidth 15M max 15M default

# Wifi
queue guest_parent on em0 bandwidth 15M max 15M
queue guest_default parent guest_parent bandwidth 15M max 15M default
\end{alltt}

You can only limit what you send, download queues remove need for
specific queue in data center for EACH customer!

\slide{OpenBSD use queueing in rules}

\begin{alltt}\footnotesize
table <HOSTED_NETWORKS> const \{ 185.60.160.0/22 \}

# Rules start here
block all

# Normal would be 3 and patientsky higher priority
pass out set queue normal set prio 3
pass out to <HOSTED_NETWORKS> set queue high set prio 5

# High prio on all traffic originating from us and our <HOSTED_NETWORKS> address space
pass in on egress from <HOSTED_NETWORKS> to (egress:0) set queue dn_high set prio 5
\end{alltt}

\begin{list2}
\item When you limit outgoing - to the LAN, results is because of TCP it limits what you receive :-)
\item Not really doing queuing inside LAN, some switches not controlled, customer responsibility
\item If internal LAN with gigabit switches cannot handle VoIP, expect other problems
\end{list2}


\slide{OpenBGP download prefixes to PF Tables}

\begin{alltt}\footnotesize
...
# neighbors and peers
psnet="50033"
group "nhn" \{
# peering to get NHN network
        remote-as $psnet
        neighbor 185.161.xxx.xx
\}

# Our local network
network 172.22.xxx.0/27

# Filtering
allow from any
allow from AS 50033
match from group "nhn" community $psnet:56828 set pftable "NHN"
\end{alltt}

Two functions, announce our local NHN prefix, internal LAN IP

and getting a table of almost 10.000 prefixes

\slide{OpenBSD multiple routing domains are cool}

with BGP running we can use the prefixes in rules, here no-NAT rule:
\begin{alltt}\footnotesize
# towards end of pf.conf
# Routing Domain 1 used for LAN
anchor "inside" on {\bf rdomain 1} \{
    # Allow administrative access when on-site
    pass in quick on em2 inet proto tcp from any to em2 port 34
    # Internal LAN must be allowed out
    pass in on em2
    # Guest network, no access to internal LAN or NHN
    # Prio 0 in ISP is Best Effort
    pass in on em0 to \{ !(em2:network) !<NHN> \} set queue low set prio 0
    # Make sure our Hosted networks have priority and NHN traffic is sent through unharmed
    pass out quick to <HOSTED_NETWORKS> nat-to (egress:0) rtable 0 set queue high set prio 5
    pass out quick to <NHN> rtable 0 set queue normal set prio 3
    pass out to !<INSIDE_NETWORKS> nat-to (egress:0) rtable 0 set queue normal set prio 3
\}
\end{alltt}



\slide{OpenBSD priority}

\begin{alltt}\footnotesize
pass out quick to <HOSTED_NETWORKS> nat-to (egress:0) rtable 0 set queue high set prio 5

pass in proto tcp to port 25 set prio 2
pass in proto tcp to port 22 set prio (2, 5)
\end{alltt}

\begin{list1}
\item Prio is copied directly into IEEE 802.1q header, making it easy to use IEEE 802.1p
\end{list1}

\begin{quote}
If the packet is transmitted on a vlan(4) interface, the queueing priority will also be written as
the priority code point in the 802.1Q VLAN header.  If two
priorities are given, packets which have a TOS of lowdelay and
TCP ACKs with no data payload will be assigned to the second one.
\end{quote}

Hint: OpenSSH \verb+sshd_config+ using IPQoS can achieve the same

\slide{Junos MX config, show configuration class-of-service }

\begin{alltt}\footnotesize
rewrite-rules \{
    ieee-802.1 ISP \{
        forwarding-class assured-forwarding \{
            loss-priority low code-point 101;
            loss-priority high code-point 101;
            loss-priority medium-high code-point 101;
            loss-priority medium-low code-point 101;
        \}
        forwarding-class expedited-forwarding \{
            loss-priority low code-point 011;
            loss-priority high code-point 011;
            loss-priority medium-high code-point 011;
            loss-priority medium-low code-point 011;
        \}
    \}
\}
\end{alltt}

Pro tip: this requires traffic to already be classified into these classes.\\
We solved it by sending VLAN traffic with prio from OpenBSD in data center to MX

\slide{Junos outgoing, show configuration class-of-service}

\begin{alltt}\small
interfaces \{
    ae0 \{
        ...
        unit 1008 \{
            classifiers \{
                ieee-802.1 default;
            \}
            rewrite-rules \{
                ieee-802.1 ISP vlan-tag outer-and-inner;
            \}
        \}
\}
\end{alltt}

Note: We use double vlan-tags outer 1008 inner 100 in data center.\\
This ends up with VLAN 100 on ALL hosts/sites

Result: We have the same simple config on all hosts

\slide{ OpenBSD niceness}

Why choosing OpenBSD made a lot of things easier

\begin{list2}
\item Free to install routers, firewalls, where we need them, no license
\item Secure and stable, less worries, stable network yay!
\item Nifty tricks with OpenBGP makes for a very elegant PF config
\item PF integrated with IEEE 802.1p - on VLAN interfaces
\item PF has a very readable format with syntactic sugar and dynamic constructs like \verb+(em2:network)+ the network on interface em2
\item OpenBSD has stable release schedule, every 6 months
\end{list2}

TL;DR Full control with easy transparent configs

\end{document}
