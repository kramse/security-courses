\documentclass[18pt,landscape,a4paper,footrule]{foils}
%\usepackage{solido-network-slide}
\usepackage{zencurity-slides}
\usepackage[normalem]{ulem}

\usepackage{multicol}

% PatientSky is rolling out a network based on OpenBSD used as CPE routers for a health infrastructure of connected clinics in Norway. This network supports both ordinary web traffic and VoIP, so must be prioritised accordingly. PatientSky has optimised OpenBSD for this task and created their own configuration tool which from a simple config file format configures the router with BGP, PF and service daemons. This includes prefixes learned from BGP being put into PF firewall tables and multiple routing domains, allowing a drop-in of the router in existing networks. Multiple routing domains allow the use of the same IP space in front and behind the device.

% Keywords:
% OpenBSD, BGP, routing, IEEE 802.1q, VLAN, IEEE802.1p, CoS/QoS, VoIP, firewalling, JSON config

\begin{document}
\selectlanguage{english}
\mytitlepage{Drift af en infrastruktur med Ansible}


\vskip 1cm
\centerline{\footnotesize slide are available as PDF kramshoej@Github}

\slide{Goal and Agenda: Ansible and more}

PatientSky is rolling out new health infrastructure of connected clinics in Norway.

We are very few people running the systems, so we need to automate.

... but automation has other benefits.

\begin{list1}
\item Prerequisites: Python, SSH, SSH keys, sudo
\item Ansible introduction, what is this Ansible
\item Ansible targets: Linux hosts, ESXi, network devices
\item Ansible examples, and workshop
\item Keywords:
Ansible, YAML, automating boring stuff
\end{list1}

\centerline{For optimal fun, use your laptop, fetch it in next break!}


\slide{ Pasientsky.no - the environment and services}

\hlkimage{12cm}{pasientsky-no.png}

\begin{quote}
Connected Clinic from PasientSky provides modern and revolutionary solutions meeting the special communication needs in the health sector. A small and smart box provides quick and stable internet connection with integrated telephony and time book.
\end{quote}

\slide{Overview}

\hlkimage{20cm}{patientsky-net-overview.png}

Most servers are Linux, percentage is OpenBSD, running on VMware ESXi

\slide{OpenBSD CPE: BGP, PF and service daemons}

\hlkimage{23cm}{openbsd-cpe.png}

\begin{list2}
\item Soekris Net6501-50 1 Ghz CPU, 1024 Mbyte DDR2-SDRAM, 4 x 1Gbit Ethernet
\item OpenBSD operating system
\item We install new Smartboxes every week
\end{list2}


\slide{Important processes and components}

\begin{list2}
\item Setup hardware
\item Connect cables
\vskip 5mm
\item Setup development environment
\item Setup staging environment - like development
\item Setup production environment - like staging
\item Setup firewalls, security, LDAP servers
\item Setup other surrounding infrastructure
\end{list2}

\vskip 5mm
\centerline{Top parts hard to automate, bottom easier \smiley}



\slide{What is Ansible}

\begin{quote}\small
AUTOMATION FOR EVERYONE

Ansible is designed around the way people work and the way people work together.

Ansible has thousands of users, hundreds of customers and over 2,400 community contributors.

750+ Ansible modules
\end{quote}

\link{https://www.ansible.com/}

\vskip 2cm
\centerline{We have been using Ansible for about 2 years}

\vskip 2cm
{\bf Warning: we dont really use the roles in Ansible sorry}

\slide{How Ansible Works: inventory files}

\begin{alltt}\footnotesize
[all:vars]
ansible_ssh_port=34443

[office]
fw-ps-dk-01 ansible_ssh_host=192.168.1.1 ansible_ssh_port=22
ansible_python_interpreter=/usr/local/bin/python

[infrastructure]
smtp-01     ansible_ssh_host=185.60.160.37 ansible_python_interpreter=/usr/local/bin/python
vpnmon-01   ansible_ssh_host=10.50.22.18

\end{alltt}

\begin{list2}
\item Inventory files specify the hosts we work with
\item Linux and OpenBSD servers shown here
\item Real inventory for this site with development and staging approx 500 lines
\end{list2}


\slide{How Ansible Works: ad hoc parallel execution }

\begin{alltt}\footnotesize
  ansible -m ping new-server
  ansible -a "date" new-server
  ansible -m shell -a "grep a /etc/something" new-server
\end{alltt}

\begin{list2}
\item Running a command on multiple servers is easy now
\end{list2}


\slide{How Ansible Works: Playbooks}

\begin{alltt}\footnotesize
  - hosts: smartbox-*
    become: yes
    tasks:
    - name: Create a template pf.conf
      template:
        src=pf/pf.conf.j2
        dest=/etc/pf.conf owner=root group=wheel mode=0600
     notify:
        - reload pf
      tags:
        - firewall
        - pf.conf
\end{alltt}

\begin{list2}
\item Almost directly from our ansible repo
\end{list2}

\slide{How Ansible Works: typical execution}

\begin{alltt}\footnotesize
ansible-playbook -i hosts.odn1 -K infrastructure-firewalls.yml -t pf.conf --check --diff

ansible-playbook -i hosts.odn1 -K infrastructure-firewalls.yml -t pf.conf

ansible-playbook -i hosts.odn1 -K infrastructure-nagios.yml -t config-only

ansible-playbook -i smartboxes -K create-pf-conf.yml -l smartbox-xxx-01
\end{alltt}

\begin{list2}
\item Pro tip: check before you push out changes to production networks \smiley
\item Diff will show the changes about to be made
\end{list2}

\slide{How Ansible Works: atypical execution / gotchas}

\begin{alltt}\footnotesize
ansible -i ../smartboxes.osl1 --become --ask-become-pass -m shell
-a "pfctl -s rules" -l smartbox01

ansible -i ../smartboxes.osl1 --become --ask-become-pass -m shell
-a "nmap -sP 185.161.1xx.123-124 2> /dev/null| grep done" all
\end{alltt}

\begin{list2}
\item Sometimes you need a trick or persistence
\item Ansible moving from \emph{sudo} to \emph{become}
\item The normal -K did not work, but the above does for ad hoc commands
\end{list2}



\slide{Stop: discussion benefits of Ansible}

Do we even need to run the same command on multiple servers?

What are the benefits of Ansible?
\begin{list2}
\item Central configuration management - git repo
\item Same playbook - different inventory file, what happens
\item
\end{list2}



% Get started with Ansible

\slide{Up and running with Ansible}

Prequisites for Ansible:


\begin{list2}
\item python language - Ansible uses this
\item ssh keys - remote login without passwords
\item Sudo - allow regular users to do superuser tasks
\item Recommended tool: \verb+ssh-copy-id+ for getting your key on new server
\item Recommended Change: \verb+sshd_config+ - no passwords allowed, no bruteforce
\item Recommended to use: jump hosts/ProxyCommand in \verb+ssh_config+
\end{list2}


\slide{Install python on servers}

\begin{list2}
\item Ubuntu server: \verb+apt install python+
\item OpenBSD: \verb+pkg_add python+\\
Requires \verb+PKG_PATH+ set, see next slide
\end{list2}


\slide{OpenBSD python}

\verb+/root/.profile+
\begin{alltt}\footnotesize

PKG_PATH=ftp://mirror.one.com/pub/OpenBSD/`uname -r`/packages/`uname -m`
PKG_PATH=https://stable.mtier.org/updates/$(uname -r)/$(arch -s):${PKG_PATH}
export PKG_PATH
\end{alltt}

\begin{list2}
\item
\item
\item
\item
\end{list2}

\exercise{}



\slide{}

\begin{list2}
\item
\item
\item
\item
\end{list2}










\slide{Conclusion}

\begin{center}
\vskip 5mm
{\color{titlecolor}\LARGE \bf Automation is cool - use it}
\vskip 5mm

\end{center}

\end{document}
