\documentclass[Screen16to9,17pt,footrule]{foils}
\usepackage{zencurity-slides}


\externaldocument{communication-and-network-security-exercises}
\selectlanguage{english}

\begin{document}

\mytitlepage{IT Kick Off 2019}
%{White Hat Hacking, protect your network}
%{Bella Center 2019}

\centerline{Happy New Year 2019 - same problems}

\slide{Who am I}


\hlkrightpic{5cm}{0cm}{003scawebgoshindomanicon.png}{~}
\begin{list2}
\item Master in computer science from University of Copenhagen
\item Got interested in internet security around early 1990s reading the Morris Internet worm analysis
\item Began reading a lot, there was no IT-security education except a bit of cryptography
\item Today white-hat hacker, security consultant, internet samurai\\
does security testing - penetration testing
\item Teach a lot - 2019 Diploma in IT-Security KEA Kompetence, starts february\\This education is new only existed for about 3 years and is needed
\item Keywords: network and security, internet technologies, network packets
%\item {\bf We need more people in IT-security}
\end{list2}

\vskip 10mm
\centerline{\bf\Large We are all part of security}

\slide{Internet Security a Short Story}

\begin{list1}
\item Early internet before 1980 - Universities, mail was the popular \emph{app}
\item TCP/IP 1980s - got IP/TCP around 1983
\item Systems were big servers VAXEN
\item Around 60.000s servers connected on the internet by 1988
\item Security was not a high priority, research and development
\item Two examples from history
\vskip 1cm
\begin{list2}
\item Cuckoo's Egg 1986 A real spy story
\item Morris Internet Worm, On the evening of 2 November 1988\\
\emph{The Internet Worm Program: An Analysis}\\
Purdue Technical Report CSD-TR-823, Eugene H. Spafford
\end{list2}
\end{list1}


\slide{Cuckoo's Egg 1986 A real spy story}

\hlkimage{4cm}{The_Cuckoos_Egg.jpg}
\begin{list1}
\item
\emph{Cuckoo's Egg: Tracking a Spy Through the Maze of Computer
 Espionage},\\  Clifford Stoll
\item \emph{During his time at working for KGB, Hess is estimated to have broken into 400 U.S. military computers}\\
Source: \link{https://en.wikipedia.org/wiki/Markus_Hess}
\end{list1}




\slide{Morris Internet Worm - 30 years ago}

\begin{list1}
\item Used multiple vulnerabilities:
\begin{list2}
\item Sendmail Debug functionality, we have similar things and Google Hacking
\item Buffer overflow in fingerd, we still have those
\item Weak passwords/password cracking, list of 432 words and /usr/dict/words, same problem today
\item Trust between systems rsh, rexec, think Domain Admin today
\item Found new systems using /etc/hosts.equiv, .rhosts, .forward, netstat ...
\end{list2}
\item Also known as the Morris Internet Worm
\item \emph{The Internet Worm Program: An Analysis}\\
Purdue Technical Report CSD-TR-823, Eugene H. Spafford
\item Resulted in creation of the CERT, \link{http://www.cert.org}
\end{list1}

\slide{Internet Worms history repeats itself}

\begin{list1}
\item Camouflage, tried to hide, malware today hides as well
\begin{list2}
\item Program name set to 'sh', looks like a regular shell
\item Used fork() to change process ID (PID)
\item Worms in the 2000s spread quickly, like Code Red 2001 to approx 350.000 systems in a week
\item SQL Slammer "It spread rapidly, infecting most of its 75,000 victims within ten minutes."
\end{list2}
\vskip 1cm
\item New malware today can use the same strategies
\item Except a lot of malware tries to stay hidden, less noisy
\item Using a small password list of 50 words it is possible to create your own botnet with 100.000s
\end{list1}

Source: \link{https://en.wikipedia.org/wiki/Timeline_of_computer_viruses_and_worms}


\slide{Hackers don't give a shit}

\hlkrightpic{11cm}{-4cm}{kiwicon-2009-hackers-dont-give-shit.jpg}

Your system is only for testing, development, ...

Your network is a research network, under construction, \\
being phased out, ...

Try something new, stay aware

Bring all the exceptions forward, all of them, update the risk \\
analysis figures - if this happens it is about 1mill DKK

Make sure to alert someone if something is strange

if something can become a threat

May need to repeat this multiple times, until fixed


\slide{Buffer Overflows - normal programs}

How do attacks even work?

\hlkimage{15cm}{images/buffer-overflow-1.pdf}

\begin{alltt}\footnotesize
main(int argc, char **argv)
\{      char buf[200];
        strcpy(buf, argv[1]);
        printf("%s\textbackslash{}n",buf);
\}
\end{alltt}

\centerline{All programs have flaws}

\slide{Buffer Overflows - bad programs}

\hlkimage{20cm}{images/buffer-overflow-2.pdf}

\centerline{\bf\Large Using LARGE input with shell code}

Software written many years ago, and never updated - are probably even more insecure, isolate, replace, phase out

\slide{Your Privacy under Attack }

\hlkimage{18cm}{images/internet-browsing.pdf}

\begin{list2}
\item Your data travels far
\item Often crossing borders, virtually and literally
\item Many technologies are old and insecure
\end{list2}


\slide{Data found in Network data }

\begin{list1}
\item Lets take an example, DNS
\item Domain Name System DNS breadcrumbs
\begin{list2}
\item Your company domain, mailservers, vpn servers
\item Applications you use, checking for updates, sending back data
\item Web sites you visit
\item Privacy issue - how to monitor company without invading employee privacy
\end{list2}
\vskip 1cm
\item Advice show your users,ask them to participate in a experiment
\end{list1}

\emph{\bf Join this Wireless network SSID and we will show you who you are on the internet}

\vskip 1 cm
\centerline{\bf\Large Maybe use VPN more - or always!}
\vskip 1 cm

\slide{Your data has already have been owned by criminals}

\hlkimage{13cm}{pwned.png}

\begin{list1}
\item Your data is already being sold, and resold on the Internet
\item Stop reusing passwords, use a password safe to generate and remember
\item Check you own email addresses on \link{https://haveibeenpwned.com/}
\end{list1}

\centerline{Go ahead try the web site - hold up your hand if you are in those dumps}


\slide{Recommendations - Comply Everywhere, Act Anywhere}

\hlkrightpic{5cm}{1cm}{003scawebgoshindomanicon.png}
{~}

\begin{list1}
\item Follow company guidelines, be skeptical, stop and think
\item Then take control of your own security
\item {\bf Laptop storage must be encrypted}
\item Firewall must be enabled
\item Suggestions
%\begin{list2}
\begin{list2}
\item Write an email to everyone in your organisation:\\
"Hi All, we need to identify systems without full disk encryption \\
- come see us, we have christmas cookies left, Best regards IT"
%\end{list2}
\end{list2}
\vskip 5mm
\item I like your 2 Feet Principle, direct surroundings
\item Keep reporting phishing attempts, attempted breakins etc.
\end{list1}


\slide{Imagine Attacks from the Inside}

\hlkimage{6cm}{erik-odiin-568459-unsplash.jpg}

\begin{list2}
\item Now imagine you were in control of a company laptop
\item Do you have a large internal world wide network?\\
NotPetya cost Maersk about 1.9 billion DKK
%\item Try scanning everything, start in a small corner, expand
%\item Scan all you danish segments, one by one, then the nordic, then the world
%\item Yes, things may break - FINE, BREAKING is GOOD

\item entry via software update mechanism of M.E.Doc [uk]  a Ukrainian tax preparation program
\end{list2}

\centerline{\bf Better to break while we are ready to un-break}

\slide{Various key attack types, clients and employees}

\begin{list2}
\item Phishing - sending fake emails, to collect credentials
\item Spear phishing - targetted attacks
\item Person in the middle - sniffing and changing data in transit
\item Drive-by attacks - web pages infected with malware, often ad servers
\item Malware transferred via USB or email
\item Credential Stuffing, Password related, like re-use of password, see slide about being pwned
\end{list2}

\vskip 1cm
\centerline{\Large\bf If we all wait a bit, and not click links immediately}

\vskip 1cm
Hackers try to create "urgency", click this or loose money

\slide{Overlapping Security Incidents}

\hlkrightpic{12cm}{1cm}{datalaek-2019.png}

New data breaches every week, these from danish news site \link{version2.dk}

Problem, we need to receive data from others

Data from others may contain malware

Have a job posting, yes\\
- then HR will be expecting CVs sent as .doc files

\myquestionspage

\end{document}
