\documentclass[17pt,Screen16to9,footrule]{foils}

%\documentclass[20pt,landscape,a4paper,footrule]{foils}

\usepackage{pasientsky-slides}


%
% Arrangement:	Penetration testing I - basale pentest metoder og introduktion
% Mål:	Introduktion til penetrationstest.
% Forudsætninger:	Der forventes kendskab til TCP/IP på brugerniveau.
% Beskrivelse:	Denne foredragsrække består af tre uafhængige dele.

% Denne del introducerer emnet penetrationstest, hvad er det og hvad
% er værdien for dig. Emner der gennemgås er blandt andet:

% * Regler og etik for penetrationstest (ISC)² Code of Ethics
% * Informationsindsamling - aktiv og passiv
% * Portscan med nmap - TCP og UDP portscanning
% * Servicescanning - identifikation af porte og protokoller
% * Sårbarheder
% * Exploits og introduktion til buffer overflows
% * Bruteforcing online og offline værktøjer
% * Opsamling og præsentation af data

% Der vil være demonstrationer af sårbarheder på alle foredragene -
% typisk med open source programmer, således at deltagerne kan afprøve
% de selvsamme demoer hjemme.

% Note: der tages udgangspunkt i open source og Unix, men resultater og principper kan overføres til ASP og .NET teknologierne.

\begin{document}

%\slide{}

\mytitlepage
{Controlling a High Security \\Environment with Ansible}


\LogoOn

\slide{En enklere hverdag med PasientSky}

%\hlkrightimage{12cm}{intro-pasientsky.png}
%.
\ThisLRCornerWallPaper{0.8}{intro-pasientsky.png}

\begin{list1}
\item Health data
\item Doctors appointment
\item Doctors Journals
\item Medical data
\item Prescriptions
\item ...
\item Obviously this means personal data
\end{list1}


\slide{Ansible: provisioning, configuration management, security}

%\hlkrightimage{10cm}{bad-monitor.png}
\ThisLRCornerWallPaper{0.6}{bad-monitor.png}

\begin{list1}
\item Open Source \smiley
\item Simple playbooks and ad-hoc commands
\item Well supported on mainstream OSs
\item over 200 modules in the core
\item Supports almost anything which has SSH+Python
\item Currently 100s of servers
\end{list1}

\link{http://www.ansible.com/}

Note: we dont use Tower

\slide{What we learnt about Ansible}

\hlkrightimage{13cm}{mini-ansible.png}
.
\begin{list1}
\item Easy to get started - YAML playbooks
\item Easy to configure services
\item Roles sometimes suck
\item - too many files in too many directories
\item Using more flat playbooks nice
\item Long lists of settings like sysctl
\item We will continue with Ansible
\end{list1}



\slide{Why Ansible brings Higher Security}

{\small\begin{verbatim}
# VPN tunnels via customer VPN server
pass quick proto { esp, ah } from any to {{ public_ip_prefix }}.59
pass quick proto { esp, ah } from {{ public_ip_prefix }}.59 to any
\end{verbatim}}


\begin{list1}
\item We can rebuild advanced servers in 15 minutes
\item Example complete Log environment from single playbook:
\begin{list2}
\item Syslog servers, PostgreSQL database, Logstash parser, software and rules, Elasticsearch indexing servers
\item with Kibana frontend - in about 150-200 lines of playbook!
\end{list2}
\item From a base Ubuntu install with no manual steps, other than starting Ansible
\item Settings are saved in playbooks - documented and readable
\item Config files are templated and
\item testing, staging and production use {\bf the exact same playbooks/configs}
\end{list1}


\slide{What Ansible brings in a High Security Environment}

\begin{list1}
\item We can deploy a complete IDS solution in 15 minutes
\item A complete Suricata IDS environment from a single playbook,
\begin{list2}
\item Suricata IDS
\item Rulesets - configuration files the same across environments
\item Cron - jobs for updating rules
\item Elasticsearch indexing servers
\item Kibana front end
\end{list2}
\item Consistency
\item From a base Ubuntu install with no manual steps, other than starting Ansible

\item Audit servers? Run Ansible - anything changed manually? \verb$ --check --diff$
\vskip 5mm
\item Plan-Do-Check-Act process - very ISO 27001 compatible
\end{list1}


\slide{Templates}

{\small\begin{verbatim}
  jdbc \{
    # Postgres jdbc connection string to our database
    jdbc_driver_library => "/usr/share/java/postgresql-jdbc4-9.2.jar"
    jdbc_driver_class => "org.postgresql.Driver"
    jdbc_connection_string => "jdbc:postgresql://{{ private_ip_prefix }}.22.100:5432/Syslog"
\end{verbatim}}


\begin{list1}
\item We can test the SAME CONFIGS in multiple environments
\item Using variable group vars, host vars, templates
\begin{list2}
\item Site specific data,
\item RFC1918 subnets, IPs, port numbers, DNS, NTP
\item Domain names, updates servers,
\item environment: development, staging, production
\item Passwords, S3 access keys, administrative users
\item Service names: ssh (Debian), sshd (OpenBSD)
\item ...
\end{list2}
\item No untested changes brought into production
\end{list1}


\slide{Update security parameters}

\begin{alltt}\small
- lineinfile:
    dest=/etc/ssh/sshd_config state=present
    regexp='PasswordAuthentication'
    line='PasswordAuthentication no'
  notify: restart sshd
  tags:
    - sshd
\end{alltt}

Combined with:
\begin{alltt}
- name: restart sshd
    service: name={{ service_sshd }} state=restarted
\end{alltt}

\vskip 1cm
\centerline{Never forget to restart a service after changing config}

\slide{Cluster firewalls always consistent}

\ThisLRCornerWallPaper{1.0}{network-overview-cluster-firewalls.png}

\begin{alltt}\footnotesize
  - name: copy PF tables
    template:
      src=roles/infrastructure-firewall/files/pf-tables/{{ item | basename }}
      dest=/etc/pf/{{ item | basename }} owner=root group=wheel mode=0600
    with_fileglob:
        - roles/infrastructure-firewall/files/pf-tables/*.list
    notify:
      - reload pf
\end{alltt}

\begin{list1}
\item Updating firewall tables
\item Updating rulesets
\item Updating BGP import filters
\item Update all parameters and roll-out made easy
\item Skills required to update production systems, low complexity
\item Less manual steps $=>$ more reliable
\end{list1}

\slide{Golden rules}

\begin{list1}
\item Always use descriptive name: so people know why/what is being done
\item Dont use lineinfile, if changing more than a few lines, use a template
\item Dont use copy, always use a template (if syntax permits)
\item Manual changes should be banned
\item Use tags liberally, tags: pf.conf, only update THIS thing
\item Try to gather a project/feature/setup in single playbook
\item Example: logging setup with both PostgreSQL and Elasticsearch in same
\item Use versioning for your playbooks, we use Git
\end{list1}

\vskip 1cm

\centerline{\hlkbig And learn your \$EDITOR - search and replace in lots of files \smiley }

\myquestionspage

\end{document}
