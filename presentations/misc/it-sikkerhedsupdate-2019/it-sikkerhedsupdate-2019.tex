\documentclass[Screen16to9,17pt]{foils}
\usepackage{zencurity-slides}


% It-sikkerhedsupdate 2019
% Få fremtidssikret it-sikkerhedsstrategien. Arrangement for medlemmer af Forsikringsforbundet og PROSA.

% Hvad skal en ansvarlig it-sikkerhedsstrategi være for 2019. Hvilke emner er de vigtigste, og hvad er truslerne, hvis man ikke straks kommer i gang med de 10 vigtigste punkter.

% Foredraget er en gennemgang af de 10 vigtigste områder og emner, som en organisation skal have styr på i 2019, med referencer til aktuelle sager som eksempel.

% Punkterne vil inkludere de sædvanlige, kedelige, men nødvendige; backup, CMDB, brugerstyring, logging m.fl. - men med forslag til praktiske værktøjer for at understøtte dem hurtigt.

\begin{document}
\selectlanguage{danish}
\mytitlepage{It-sikkerhedsupdate}{2019}


\vskip 1cm
\centerline{\footnotesize slides are available on Github}

\slide{Formålet idag}

\hlkimage{7cm}{Shaking-hands_web.jpg}

Hvad skal en ansvarlig it-sikkerhedsstrategi være for 2019. Hvilke emner er de vigtigste, og hvad er truslerne, hvis man ikke straks kommer i gang med de 10 vigtigste punkter.


Planen for idag:
\begin{list2}
\item 4 timer, med pauser
\item Mindre præsentation, mere dialog
\end{list2}

\slide{Happy New Year 2019}

\hlkrightpic{10cm}{0cm}{happy-new-year-roven-images-601197-unsplash.jpg}
.

\begin{list2}
\item Same problems
\item Repeat last year?
\item ... or try something new!
\item 2019 will become a nightmare of break-ins and data leaks
\item GDPR is here and the snow ball is rolling
\end{list2}

\vskip 1cm
{\LARGE\bf Try not to panic, but there are lots of threats}



\slide{Hackers don't give a shit}

\hlkrightpic{11cm}{-4cm}{kiwicon-2009-hackers-dont-give-shit.jpg}

Your system is only for testing, development, ...

Your network is a research network, under construction, \\
being phased out, ...

Try something new, go to your management

Bring all the exceptions, all of them, update the risk \\
analysis figures - if this happens it is about 1mill DKK

Ask for permission to go full monty on your security

{\bf Think like attackers - don't hold back}

\slide{Fokus 2019}

\begin{list2}
\item Brugerstyring
\item Asset management
\item Laptop sikkerhed
\item VPN alle steder
\item Penetration testing
\item Firewalls og segmentering
\item TLS og VPN indstillinger
\item DNS og email
\item Syslog
\item Incident Response og reaktion
\end{list2}

\vskip 5mm
\centerline{Håber ikke I er alene om det, ellers vælg et par stykker ad gangen}

\slide{Fokus 2019: Brugerstyring}

\begin{list2}
\item Har I styr på brugerid?
\item Hvor er brugere oprettet?
\item Er det et kludetæppe - ja, mange steder er det
\end{list2}



\slide{Bruger login}

\begin{list2}
\item
\item
\end{list2}

\slide{Local administrator?}

\begin{list2}
\item Findes der systemer som er helt åbne, med lokal administrator
\item Er det stadig nødvendigt
\end{list2}


\slide{Centraliseret brugerstyring}

\begin{list1}
\item Active Directory, mange danske virksomheder bruger det
\item LDAP central brugerstyring
\item ... men brug det mere
\begin{list2}
\item Konfigurer applikationer til central styring
\item Fjern applikationer som ikke tillader central styring
\end{list2}
\item Generelt minimer brugere andre steder end i den centrale database
\end{list1}



\slide{Passwords vælges ikke tilfældigt}

\hlkimage{20cm}{50-most-used-passwords.png}

Source:
\link{https://wpengine.com/unmasked/}


\slide{Your data has already have been owned by criminals}

\hlkimage{13cm}{pwned.png}

\begin{list1}
\item Your data is already being sold, and resold on the Internet
\item Stop reusing passwords, use a password safe to generate and remember
\item Check you own email addresses on \link{https://haveibeenpwned.com/}
\end{list1}

\centerline{Go ahead try the web site - hold up your hand if you are in those dumps}


\slide{Brug mere sikre passwords}

\begin{quote}
Pwned Passwords overview\\
Pwned Passwords are more than half a billion passwords which have previously been exposed in data breaches. The service is detailed in the launch blog post then further expanded on with the release of version 2. The entire data set is both downloadable and searchable online via the Pwned Passwords page.
\end{quote}

\begin{list1}
\item I kan forhindre brugere i at vælge passwords der ALLEREDE er lækket
\item I kan bruge deres API eller download\\
{\footnotesize\link{https://www.troyhunt.com/introducing-306-million-freely-downloadable-pwned-passwords/}}
\end{list1}


\slide{Formål: sund paranoia}


\hlkimage{6cm}{password-window.png}
\centerline{Opbevaring af passwords}

\vskip 1cm
\centerline{Also problems with LastPass this week :-/}

\slide{January 2013: Github Public passwords?}

\hlkimage{20cm}{github-credentials.png}

 Sources:\\
{\footnotesize\link{https://twitter.com/brianaker/status/294228373377515522}\\
\link{http://www.webmonkey.com/2013/01/users-scramble-as-github-search-exposes-passwords-security-details/}\\
\link{http://www.leakedin.com/}\\
\link{http://www.offensive-security.com/community-projects/google-hacking-database/}
}

\vskip 5mm
\centerline{Use different passwords for different sites, yes - every site!}



\slide{Fokus 2019: Asset management}

\begin{list2}
\item Hardware
\item Software
\item Virtuelle maskiner
\item IP adresser
\end{list2}

\slide{Hardware asset management}

\begin{list2}
\item
\item
\end{list2}

\slide{Software asset management}

\begin{list2}
\item
\item
\end{list2}

\slide{IP Address Management IPAM }

\begin{list2}
\item Recommend Nipap
\item
\end{list2}

\slide{Har du styr på dependencies}

\begin{list2}
\item Skal det være helt flot så få også styr på dependencies
\item Er jeres produktion afhængig af andres moduler, biblioteker osv.
\item Tænk tilbage til Heartbleed, der gik flere år før de sidste opdateringer kom
\end{list2}



\slide{Fokus 2019: Laptop sikkerhed}

\begin{list2}
\item
\item
\end{list2}

\slide{Secure Laptops}

\hlkimage{12cm}{librem-15-v3-turns99.png}

Start with your laptops (and mobile devices if you wish)

Are they \emph{secure}, and to what extent

%{\footnotesize Laptop picture Purism team copyleft CC-by-SA 4.0 license]

\slide{Are your data secure - data at rest}

\hlkimage{15cm}{images/data-integrity-1.pdf}

\begin{list1}
\item Stolen laptop, tablet, phone - can anybody read your data?
\item Do you trust "remote wipe"
\item How do you in fact wipe data securely off devices, and SSDs?
\item Encrypt disk and storage devices before using them in the first place!
\end{list1}


\slide{Circumvent security - single user mode boot}
\begin{list1}
\item Unix systems often allows boot into singleuser mode\\
press command-s when booting Mac OS X
\item Laptops can often be booted using PXE network or CD boot
\item Mac computers can become a Firewire disk\\
hold t when booting - firewire target mode
\item Unrestricted access to un-encrypted data
\item Moving hard drive to another computer is also easy
\end{list1}
\pause
\centerline{Physical access is often - {\bf game over}}


\slide{Encrypting hard disk}

\hlkimage{14cm}{images/apple-filevault.png}

\begin{list1}
\item Becoming available in the most popular client operating systems
\begin{list2}
\item Microsoft Windows Bitlocker - requires Ultimate or Enterprise
\item Apple Mac OS X - FileVault og FileVault2
\item FreeBSD GEOM og GBDE - encryption framework
\item Linux LUKS distributions like Ubuntu ask to encrypt home dir during installation
\item PGP disk - Pretty Good Privacy - makes a virtuel krypteret disk
\item TrueCrypt - similar to PGP disk, a virtual drive with data, cross platform
\item Some vendors have BIOS passwords, or disk passwords
\end{list2}
\end{list1}



\slide{Attacks on disk encryption}

\begin{list1}
\item Firewire, DMA \& Windows, Winlockpwn via FireWire\\
Hit by a Bus: Physical Access Attacks with Firewire Ruxcon 2006
\vskip 5mm
\item Removing memory from live system - data is not immediately lost, and can be read under some circumstances\\
Lest We Remember: Cold Boot Attacks on Encryption Keys\\
\link{http://citp.princeton.edu/memory/}
\item This is very CSI or Hollywoord like - but a real threat
\item VileFault decrypts encrypted Mac OS X disk image files\\ \link{https://code.google.com/p/vilefault/}

\item  FileVault Drive Encryption (FVDE) (or FileVault2) encrypted volumes\\
\link{https://code.google.com/p/libfvde/}
\end{list1}

\centerline{So perhaps use both hard drive encryption AND turn off computer after use?}

\slide{... and deleting data}

\hlkimage{8cm}{dban-screenshot.png}

\begin{list1}
\item Getting rid of data from old devices is a pain
\item Some tools will not overwrite data, leaving it vulnerable to recovery
\item Even secure erase programs might not work on SSD - due to reallocation of blocks
\item I have used Darik's Boot and Nuke ("DBAN") \link{http://www.dban.org/}
\end{list1}


\slide{2018 attack}

\hlkimage{12cm}{ssd-attack-2018.png}
\emph{self-encrypting deception: weakness in the encryption of solid state drives (SSDs)}\\
\link{https://www.ru.nl/publish/pages/909282/draft-paper.pdf}





\slide{Recommendations - Comply Everywhere, Act Anywhere}

\hlkrightpic{5cm}{-1cm}{003scawebgoshindomanicon.png}
{~}
\begin{list1}
\item {\bf Laptop storage must be encrypted}
\item Firewall must be enabled
\item Suggestions
\begin{list2}
\item Try sniffing data from a laptop, setup Access Point/Monitor port
\item Portscan your laptopging networks - use Nmap
\item Write an email to everyone in your organisation:\\
"Hi All, we need to identify laptops without full disk encryption \\
- come see us, we have christmas cookies left, Best regards IT"
\end{list2}
\end{list1}



\slide{}

\begin{list2}
\item
\item
\end{list2}


\slide{Fokus 2019: VPN alle steder}

\begin{list2}
\item
\item
\end{list2}

\slide{Your Privacy }

\hlkimage{18cm}{images/internet-browsing.pdf}


\begin{list2}
\item Your data travels far
\item Often crossing borders, virtually and literally
\end{list2}


\slide{Data found in Network data }

\begin{list1}
\item Lets take an example, DNS
\item Domain Name System DNS breadcrumbs
\begin{list2}
\item Your company domain, mailservers, vpn servers
\item Applications you use, checking for updates, sending back data
\item Web sites you visit
\end{list2}
\vskip 1cm
\item Advice show your users,ask them to participate in a experiment
\end{list1}

\emph{\bf Join this Wireless network SSID and we will show you who you are on the internet}

\vskip 2 cm
\centerline{\bf\Large Maybe use VPN more - or always!}



\slide{Fokus 2019: Penetration testing}

\begin{list2}
\item
\item
\end{list2}

\slide{Start Attacking from the Inside}

\hlkimage{6cm}{erik-odiin-568459-unsplash.jpg}


\begin{list2}
\item Now imagine you were in control of a company laptop
\item Do you have a large internal world wide network?\\
Having a large open network may cost you 1.9 billion DKK - ref Maersk
\item Try scanning everything, start in a small corner, expand
\item Scan all you danish segments, one by one, then the nordic, then the world
\item Yes, things may break - FINE, BREAKING is GOOD
\end{list2}

\centerline{\bf Better to break while we are ready to un-break}

\slide{How to break stuff}

Think like an attacker

I sit here, but where am I connected:
\begin{alltt}\footnotesize
reading from file cisco-lldp-1.cap, link-type EN10MB (Ethernet)
16:39:43.468745 LLDP, length 328
        Chassis ID TLV (1), length 7
          Subtype MAC address (4): 70:ff:1a:01:03:02 (oui Unknown)
          0x0000:  0470 ea1a a0b3 2f
        Port ID TLV (2), length 8
          Subtype Local (7): Eth1/47
          0x0000:  0745 7468 312f 3437
        Port Description TLV (4), length 12: Ethernet1/47
          0x0000:  4574 6865 726e 6574 312f 3437
        System Description TLV (6), length 158
          Cisco Nexus Operating System (NX-OS) Software 14.0(2c) TAC support: http://www.cisco.com/tac Copyright (c) 2002-2020, Cisco Systems, Inc. All rights reserved.
\end{alltt}

\vskip 5mm
\centerline{I love LLDP, but it does reveal software version, so flaws available}

\slide{Nmap the world}

\hlkimage{19cm}{trinity-nmapscreen-hd-cropscale-418x250.jpg}

\slide{Really do Nmap your world}

\hlkimage{8cm}{nmap-zenmap.png}

\begin{list2}
\item Nmap is a port scanner, but does more
\item Finding your own infrastructure available from the guest network?
\item See your printers having all the protocols enabled AND a wireless?
\end{list2}

\slide{Hackerlab setup}

\hlkimage{11cm}{hacklab-1.png}

\begin{list2}
\item Create hacker labs, encourage hacker labs!
\item Software Host OS: Windows, Mac, Linux
\item Virtualisation software: VMware, Virtual box, HyperV pick your poison
\item Hackersoftware: Kali Virtual Machine \link{https://www.kali.org/}
\end{list2}

\slide{Hacking is not magic}

\hlkimage{11cm}{ninjas.png}

\begin{list2}
\item Hacking only requires some ninja training
\item We have been doing this since 1995 when SATAN was released
\item Listen, Plan, Act, Do hacking
\end{list2}

\slide{Book: Linux Basics for Hackers (LBhf)}

\hlkimage{6cm}{LinuxBasicsforHackers_cover-front.png}

\emph{Linux Basics for Hackers
Getting Started with Networking, Scripting, and Security in Kali}
by OccupyTheWeb
December 2018, 248 pp.
ISBN-13:
9781593278557

\link{https://nostarch.com/linuxbasicsforhackers}

\slide{Book: Kali Linux Revealed (KLR)}

\hlkimage{6cm}{kali-linux-revealed.jpg}

\emph{Kali Linux Revealed  Mastering the Penetration Testing Distribution}

\link{https://www.kali.org/download-kali-linux-revealed-book/}\\
Not curriculum but explains how to install Kali Linux

\slide{Fokus 2019: Firewalls og segmentering}

\begin{list2}
\item
\item
\end{list2}

\slide{Imagine Attacks from the Inside}

\hlkimage{6cm}{erik-odiin-568459-unsplash.jpg}

\begin{list2}
\item Now imagine you were in control of a company laptop
\item Do you have a large internal world wide network?\\
NotPetya cost Maersk about 1.9 billion DKK
%\item Try scanning everything, start in a small corner, expand
%\item Scan all you danish segments, one by one, then the nordic, then the world
%\item Yes, things may break - FINE, BREAKING is GOOD

\item entry thought to be via software update of M.E.Doc [uk] an Ukrainian tax preparation program
\item Attackers are very creative and have a large attack surface to most companies
\end{list2}



\slide{Netværk generelt}

\begin{list2}
\item Måske også på tide lige at se om der er opdateringer til switche
\item Jeg anbefaler LibreNMS
\end{list2}

\slide{Fokus 2019: TLS og VPN indstillinger}

\slide{SSL og TLS}

\hlkimage{14cm}{ehandel-https.pdf}

\begin{list1}
\item Oprindeligt udviklet af Netscape Communications Inc.
\item Secure Sockets Layer SSL er idag blevet adopteret af IETF og kaldes
derfor også for Transport Layer Security TLS
TLS er baseret på SSL Version 3.0
\item RFC-2246 The TLS Protocol Version 1.0 fra Januar 1999
\item RFC-3207 SMTP STARTTLS
\end{list1}




\begin{list1}
\item Nu vi har lært Kali og Nmap at kende
\begin{list2}
\item Find nemt alle ssl version 2 og 3\\
\verb+nmap --script ssl-enum-ciphers+
\item Brug ssllabs https://www.ssllabs.com/
\item
\item
\end{list2}
\end{list1}

\slide{SSL Labs}

\begin{list2}
\item
\item
\end{list2}


\slide{Weak DH paper}

\hlkimage{18cm}{weakdh-logjam.png}

Source: \link{https://weakdh.org/} and \\
\link{https://weakdh.org/imperfect-forward-secrecy-ccs15.pdf}

\slide{Audits}

\hlkimage{10cm}{crypto-cat.png}
\begin{list1}
\item Truecrypt audit\\
{\footnotesize\link{https://isecpartners.github.io/news/2014/04/14/iSEC-Completes-Truecrypt-Audit.html}}
\item Cryptocat audit\\
{\footnotesize\link{https://blog.crypto.cat/2013/02/cryptocat-passes-security-audit-with-flying-colors/}}
\end{list1}



\slide{VPN indstillinger}

\begin{list1}
\item PPTP, hvis du bruger det så er det godt du er kommet :-D
\item Check:
\begin{list2}
\item Certifikater/nøgler - ligesom TLS lange og rulles indimellem
\item Algoritmer DES/3DES bye bye, husk både encryption og auth algoritmer
\item DH-Group - +15 tak
\item Check både client VPN og site-2-site
\end{list2}
\end{list1}


\slide{sslscan}

\begin{alltt}\small
root@kali:~# sslscan --ssl2 web.kramse.dk
Version: 1.10.5-static
OpenSSL 1.0.2e-dev xx XXX xxxx

Testing SSL server web.kramse.dk on port 443
...
  SSL Certificate:
Signature Algorithm: sha256WithRSAEncryption
RSA Key Strength:    2048

Subject:  *.kramse.dk
Altnames: DNS:*.kramse.dk, DNS:kramse.dk
Issuer:   AlphaSSL CA - SHA256 - G2
\end{alltt}

Source:
Originally sslscan from http://www.titania.co.uk
 but use the version on Kali



\slide{}

\begin{list2}
\item
\item
\end{list2}

\slide{Fokus 2019: DNS og email}

\begin{list2}
\item
\item
\end{list2}


\slide{Various key attack types, clients and employees}

\begin{list2}
\item Phishing - sending fake emails, to collect credentials
\item Spear phishing - targetted attacks
\item Person in the middle - sniffing and changing data in transit
\item Drive-by attacks - web pages infected with malware, often ad servers
\item Malware transferred via USB or email
\item Credential Stuffing, Password related, like re-use of password, see slide about being pwned
\end{list2}

\vskip 1cm
\centerline{\Large\bf If we all wait a bit, and not click links immediately}

\vskip 1cm
Hackers try to create "urgency", click this or loose money



\slide{DNS}

\begin{list2}
\item
\item
\end{list2}


\slide{DNS query log?}

Skal vi logge ALLE DNS opslag fra klienter?


\begin{list2}
\item Uetisk?
\item Smart hvis man vil spore hvor malware kom ind
\end{list2}

\slide{DNSSEC}

\begin{list2}
\item
\item
\end{list2}


\slide{DMARC}

\begin{list2}
\item SPF
\item DKIM
\item DMARC
\end{list2}


\slide{Fokus 2019: Syslog}

\begin{list2}
\item
\item
\end{list2}

\slide{}

\begin{list2}
\item
\item
\end{list2}
\slide{}

\begin{list2}
\item
\item
\end{list2}


\slide{Fokus 2019: Incident Response og reaktion}

\begin{list2}
\item
\item
\end{list2}

\slide{Overlapping Security Incidents}

\hlkrightpic{12cm}{1cm}{datalaek-2019.png}

New data breaches nearly every week, these from danish news site \link{version2.dk}

Problem, we need to receive data from others

Data from others may contain malware

Have a job posting, yes\\
- then HR will be expecting CVs sent as .doc files

\slide{}

or the other way

{\Large\bf Attackers used a LinkedIn job ad\\
and Skype call to breach bank’s defences}
\hlkimage{12cm}{redbanc-skype-malware.png}

{\footnotesize
\link{https://nakedsecurity.sophos.com/2019/01/21/attackers-used-a-linkedin-job-ad-and-skype-call-to-breach-banks-defences/}}


\slide{}

\begin{list2}
\item
\item
\end{list2}


\slide{Spørgsmål og mere debat}


\hlkimage{7cm}{idog.jpg}

\begin{center}
\hlkbig

\myname

\end{center}




\end{document}
