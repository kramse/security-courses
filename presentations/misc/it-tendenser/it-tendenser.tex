\documentclass[Screen16to9,17pt]{foils}
\usepackage{zencurity-slides}

% PROSA på Zealand Roskilde 19/9 2024
%  *   3 x ca. 40 minutters oplæg om fremtidens it-tendenser, og hvordan det vil komme til at påvirke de studerende i fremtiden - gerne mere underholdning end fagligt
%  *   Afholdes for 30 studerende ad gangen i Zealands bygninger
%  *   Prosa faciliteter oplæggene - vi indleder og afrunder


\begin{document}
\selectlanguage{danish}
\mytitlepage{Fremtidens it-tendenser}{\the\year}


\vskip 1cm
\centerline{\footnotesize slides are also available on Github}


\slide{Plan for today}

\hlkimage{10cm}{df20030604.jpg}

\begin{list1}
\item Talk about the current world of information technology and security
\item What a crazy place we are in with a flood of vulnerabilities
\item Resource shortage -- man power, skillz etc.
\end{list1}

\slide{Every Year}

\hlkimage{10cm}{hacker-color.jpg}

\begin{list2}
\item Same problems last year? Same problems EVERY year
\item Data leaks, GDPR, ransomware, ...
\end{list2}

{\bf Try not to panic, but there are lots of threats}


\slide{Overlapping Security Incidents}

\hlkrightpic{9cm}{1cm}{datalaek-2019.png}

New data breaches nearly every week, these from danish news site \link{version2.dk}

Problem, we need to receive data from others

Data from others may contain malware

Have a job posting, yes\\
- then HR will be expecting CVs sent as .doc files

\slide{Work together}

\hlkimage{9cm}{Shaking-hands_web.jpg}

\begin{list1}
\item Learn, Team up, be curious!
\item We need to share information freely
\item We often face the same threats, so we can work on solving these together
\end{list1}



\slide{Principle of Least Privilege}

\hlkimage{10cm}{dragon-drawing-6.jpg}

\begin{list2}
\item {\bf Definition 14-1} The \emph{principle of least privilege} states that a subject should be given only those privileges that it needs in order to complete the task.
\item \emph{The Protection of Information in Computer Systems}\\
Jerome Saltzer and Michael Schroeder, 1975\\
\url{https://en.wikipedia.org/wiki/Saltzer_and_Schroeder%27s_design_principles}
\end{list2}


\slide{Fokus on the basics}

\begin{list2}
\item User management - including administrative users
\item Asset management
\item Laptop security
\item Penetration testing
\item Firewalls and segmentation
\item VPN everywhere
\item TLS and VPN settings, encryption
\item DNS and email security
\item Syslog and monitorering
\item Incident Response and response
\end{list2}

Learn technologies, read the manual -- don't trust AI will solve everything


\slide{Passwords are not random}

\hlkimage{20cm}{50-most-used-passwords.png}

Source:
\link{https://wpengine.com/unmasked/}


\slide{Your data has already have been owned by criminals}

\hlkimage{9cm}{pwned.png}

\begin{list1}
\item Your data is already being sold, and resold on the Internet
\item Stop reusing passwords, use a password safe to generate and remember
\item Check you own email addresses on \link{https://haveibeenpwned.com/}
\item They have an API you can integrate to avoid re-using already leaked passwords\\
{\footnotesize\link{https://www.troyhunt.com/introducing-306-million-freely-downloadable-pwned-passwords/}}
\end{list1}


\slide{Save the passwords}

\hlkimage{6cm}{password-window.png}

\begin{list2}
\item Use password managers! Available as cloud connected, local only, teams based
\item You will have to investigate which one to choose, but find one!
\end{list2}



\slide{Nmap the world}

\hlkimage{19cm}{trinity-nmapscreen-hd-cropscale-418x250.jpg}


\slide{Hackertools are for everyone!}

{\Large\bf Hackers work all the time trying to break stuff}

Blue teams can use hackertools, and become more effecient:
\begin{list2}
\item Nmap, Nping \link{http://nmap.org}
\item Wireshark - \link{http://www.wireshark.org/}
\item Aircrack-ng \link{http://www.aircrack-ng.org/}
\item Metasploit Framework \link{http://www.metasploit.com/}
\item Burpsuite \link{http://portswigger.net/burp/}
\item Kali Linux \link{http://www.kali.org}
\end{list2}

\vskip 5mm
\centerline{Most popular hacker tools \link{https://tools.kali.org/} and \link{http://sectools.org/}}


\slide{Hacking is not magic}

\hlkimage{11cm}{ninjas.png}

\begin{list2}
\item Hacking only requires some ninja training
\item We have been doing this since 1995 when SATAN was released
\item Listen, Plan, Act, Do hacking
\item Be curious, and honest -- let our students play with fire in special networks
\end{list2}



\slide{Vulnerabilities are everywhere!}

\hlkimage{18cm}{cve-details-new-updated.png}
Source: CVEdetails.com on 2024-09-02

\begin{list2}
\item This is crazy! \url{https://www.cvedetails.com/}
\end{list2}

\slide{Vulnerabilitiesby type \& year}

\hlkimage{17cm}{cve-details-year.png}
Source: CVEdetails.com on 2024-09-02 Graph on the web site is interactive \url{https://www.cvedetails.com/}

\slide{LG TVs 2024 -- CVE-2023-6317 up to CVE-2023-6320}

\hlkimage{10cm}{LG-shodan.png}

\begin{quote}{\large\bf
90,000+ LG TVs Vulnerable to Authorization Attacks\\
Due to WebOS Vulnerabilities}

Bitdefender Labs has revealed a critical security flaw in over 90,000 LG smart TVs running the company’s proprietary WebOS platform.

If exploited, the vulnerability could allow attackers to gain unauthorized access to the TV’s functions and potentially the user’s home network.

\end{quote}
Source: \url{https://cybersecuritynews.com/lg-tvs-vuauthorization-attacks/}


\slide{D-Link NAS devices accessible via “backdoor” account CVE-2024-3273}

%\hlkimage{}{}

\begin{quote}{\large\bf
92,000+ internet-facing D-Link NAS devices accessible via “backdoor”}

A vulnerability (CVE-2024-3273) in four old D-Link NAS models could be exploited to compromise internet-facing devices, a threat researcher has found.

The existence of the flaw was confirmed by D-Link last week, and an exploit for opening an interactive shell has popped up on GitHub.

“The vulnerability lies within the \verb+nas_sharing.cgi+ uri, which is vulnerable due to two main issues: a backdoor facilitated by hardcoded credentials, and a command injection vulnerability via the system parameter,” says the discoverer, who goes by the online handle “netsecfish”.

{\bf The “backdoor” account has messagebus as the username and doesn’t require a password.}
\end{quote}
Source: \url{https://www.helpnetsecurity.com/2024/04/08/cve-2024-3273/}



\slide{Defense in depth}

%\hlkimage{10cm}{Bartizan.png}
\hlkimage{15cm}{medieval-clipart-5}
\centerline{Picture originally from: \url{http://karenswhimsy.com/public-domain-images}}


\slide{Cloud Security is here, and needed -- Cilium overview}

\hlkimage{12cm}{cilium-overview.png}

\begin{quote}
Kubernetes provides Network Policies for controlling traffic going in and out of the pods. Cilium implements the Kubernetes Network Policies for L3/L4 level and extends with L7 policies for granular API-level security for common protocols such as HTTP, Kafka, gRPC, etc
\end{quote}
Source: picture and text from \link{https://cilium.io/blog/2018/09/19/kubernetes-network-policies/}


\slide{Security is more than blocking!}

\hlkimage{22cm}{cilium-features.png}

\begin{list2}
\item A lot of features relate to \emph{security}
\end{list2}


\slide{Expect Incidents -- train Incident Response}

\hlkimage{10cm}{margarida-csilva-121801-unsplash.jpg}

\begin{list2}
\item We know there will be security incidents
\item We know you will be tasked at handling it!
\end{list2}

Lifeguard training photo by Margarida CSilva on Unsplash


\slide{Questions}

\hlkimage{8cm}{idog.jpg}

\begin{center}
\hlkbig

\myname

\end{center}

\end{document}



\end{document}
