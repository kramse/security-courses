\documentclass[18pt,landscape,a4paper,footrule]{foils}
\usepackage{zencurity-slides}
\usepackage[normalem]{ulem}

\usepackage{multicol}

% Henrik Kramshøj: IT-sikkerhed med flere enheder
% Henrik Kramshøj gennemgår med eksempler hvordan man kan sikre dele af sit digitale liv ved at adskille data

% En gennemgang med eksempler på hvordan man kan sikre dele af sit digitale liv ved at adskille data. Foredraget vil komme ind på blandt andet. Krav til egen mailserver, fordele og ulemper. Brug af flere laptops, og ekstremet brug af Qubes OS. Hvilke data kan vi tillade os at have på mobile enheder. Kodeordshuskere, en eller flere?

% Foredraget opremser primært mine egne erfaringer som oplæg til debat og vil ikke være dybt teknisk snak.

% Henrik Kramshøj

% Henrik Kramshøj er internet-samurai, initiativtager til Bornhack, tor-exit administrator, datalog og aktivist.
% Efter oplægget fortsætter vi med åben diskussion og uformel dialog.



\begin{document}
\selectlanguage{danish}
\mytitlepage{IT-sikkerhed med flere enheder}{}


\vskip 1cm
\centerline{\footnotesize slides are available on Github}


\slide{Formålet med foredraget}
\vskip 2 cm


\hlkimage{5cm}{dont-panic.png}
\centerline{\color{titlecolor}\LARGE Don't Panic!}

\begin{list1}
\item Give eksempler på metoder for at sikre data ved brug af flere enheder
\end{list1}



\slide{Plan: Vi skal beskytte data}

\hlkimage{10cm}{cia-triad.png}

\begin{list2}
\item Nogle enheder er nemme at transportere, mobiltelefon
\item Nogle enheder har rigtigt tastatur
\end{list2}

\centerline{Næsten alle enheder idag bør have fuld kryptering af storage}

\slide{Quick Wins: Opsec Light}

\begin{list1}
\item Operations security (OpSec, OPSEC), what do you need?\\
\link{https://en.wikipedia.org/wiki/Operations_security}
\item Use multiple devices, isolate data
\item Less critical on phone, most critical on laptop with full disk encryption
\item Using different password for each service, unpossible!
\end{list1}

\slide{One time passwords}

\hlkimage{12cm}{yubikey-2018.png}

\begin{list1}
\item OTP One Time Password, sniff one and you can use it, if you have a time machine \smiley
\item I use Google Authenticator and Yubikeys
\item \link{https://en.wikipedia.org/wiki/Google_Authenticator}
\item \link{https://www.yubico.com/start/}
\end{list1}


\slide{Eksempel Mine enheder}

\begin{list2}
\item Min private telefon
\item \sout{Min arbejdstelefon} - bruger jeg ikke
\item Min primære laptop - private
\item Mine andre laptops - private
\item Min arbejdslaptop - bruger jeg minimalt til private data

\item NB: jeg er ikke religiøs - burde måske være mere kompromisløs
\item Vær passende paranoid
\end{list2}

\centerline{Vi glemmer netværket og underholdning som Chromecast m.v}




\slide{Smart Girl's Guide to Privacy}

\link{https://www.nostarch.com/smartgirlsguide}
\hlkrightimage{6cm}{sggp_frontcvr_final.png}
%{sggp_frontcvr_final.png}

\emph{Practical Tips for Staying Safe Online}
by Violet Blue

August 2015, 176 pp.
ISBN: 978-1-59327-648-5

Kan varmt anbefales! Rød, gul og grøn information


\slide{Analyse: Hvilke data}

\hlkimage{6cm}{Linus3-04041999.jpg}

\begin{list2}
\item Emails, SMS og korte tekstbeskeder, hvor er du, skal vi mødes, køb mere vodka
\item Billeder - gamle, pinlige, Coffee shop?, nøgenbilleder
\item Lokationsdata - hvor er du, hvor skal du hen, Google maps location history
\item Breve, egne noter, private dokumenter under udarbejdelse
\item Sikkerhedsinformation - inkl pentest af andres systemer
\item Projekter - egne og med andre
\item Koder og logins ... og data der skal nedarves
\end{list2}

\slide{Generelle regler for mig selv}

\begin{list1}
\item Følsomme data skal være krypterede, under transport og gemt
\item Følsomme data opbevares bag stærke kodeord, dvs ikke telefon pinkode
\item Backups skal være krypterede, Apple krypteret disk, Duplicity eller Qubes backup
\item Begræns mere data - brug mere Tor
\item Adskil applikationer og browsing mere
\item New 2018: Data der skal nedarves til min søn lægges på Dropbox, eller ukrypteret disk. Worst case får politiet eller en tyv adgang til billeder fra hans barndom
\end{list1}

\slide{Qubes OS}

A reasonably secure operating system \\
\link{https://www.qubes-os.org/}

\hlkimage{18cm}{qubes-quotes.png}

\vskip 2cm
\centerline{Qubes OS er en central del af mit setup}

\slide{Qubes OS eksempel}

\hlkimage{11cm}{qubes-vms.png}

\begin{list2}
\item Viser et mix af mine VMs, ikke alle
\item Derudover findes Disposable VMs
\item og nu tid til demo
\end{list2}

\slide{Egen mailserver}

\begin{list2}
\item Lidt bøvlet - specielt anti-spam
\item Post modtagelse Postfix \link{http://www.postfix.org/}
\item IMAPS \link{https://www.dovecot.org/}
\item Anti-spam Bogofilter \link{http://bogofilter.sourceforge.net/}
\item Hverken email eller GPG keys på telefon
\end{list2}

Men har også en Google mail konto, som bruges til
\begin{list2}
\item Sende data til folk direkte fra telefonen
\item Modtage og opbevare koncertbilletter
\item \emph{Unclassified information}
\end{list2}

\slide{Password safe}

\hlkimage{10cm}{keepassx.png}

\begin{list1}
\item Kodeordshuskere er nødvendige idag
\item Jeg har prøvet KeePassX, LastPass, 1Password, Apple Keychain
\item Jeg bruger idag primært Lastpass $\Longrightarrow$ KeePassX
\end{list1}

\slide{Værktøjer}

Følgende slides er tiltænkt som oversigt og eksempler

\slide{Tor project anonym webbrowsing}

\hlkimage{23cm}{tor-project.png}

\centerline{\link{https://www.torproject.org/}}

\vskip 2cm
\centerline{Der findes alternativer, men Tor er mest kendt}


\slide{Why use Tor?}
.
\hlkrightimage{7cm}{tor-uses.png}

\begin{list2}
\item Your public IP is {\color{red}Red Information}, often lead directly to you
\item You like to browse things, without telling your ISP, the\\
government, your teacher, ... everyone, Avoid censorship
\item You want to avoid stalkers
\item You are an investigative journalist or high school student\\
researching Al Qaeda, Daesh, ISIS for school
\item Consider getting the book \emph{The Smart Girl's Guide to Privacy}\\ \link{http://smartprivacy.tumblr.com/}
\end{list2}

Shameless plug: we are starting up danish information page\\
and more \link{https://www.torservers.dk/}

Pic from \link{https://www.torproject.org/}

\slide{Whonix}

\hlkimage{17cm}{400px-Whonix.jpg}

\begin{quote}
Whonix is an operating system focused on anonymity, privacy and security. It's based on the Tor anonymity network[5], Debian GNU/Linux[6] and security by isolation. DNS leaks are impossible, and not even malware with root privileges can find out the user's real IP.
\end{quote}

\link{https://www.whonix.org/}



\slide{Brug flere browsere}

\hlkimage{24cm}{multi-browser-strategy.png}



\slide{Fordele ved flere browsere}

\begin{list1}
\item Flere browsere giver højere sikkerhed
\item Data kan ikke flyde mellem flere browsere, cookies m.m.
\item Mit forslag:
\begin{list2}
\item En browser til \emph{sikre sites} banken, intranet
\item En browser til generel internet surfing
\item En browser med alle mulige plugins, web udvikling eksempelvis
\end{list2}
\item Installer gerne plugins til højere sikkerhed i allesammen:\\
HTTPS Everywhere, NoScript/ScriptBlock m.fl.
\end{list1}

\vskip 1cm
\centerline{Det anbefales at disse installeres og vedligeholdes fra IT-afdelingen}


\vskip 1cm
\centerline{\bf\Large Alle browsere har mange fejl!}


\slide{Generelt indstillinger for browsere}

\begin{list1}
\item Skal være indstillet på den sikre browser til generel surf
\begin{list2}
\item Slå JavaScript fra generelt med NoScript/ScriptBlock
\item Slå click-to-play til for aktivt indhold
\item Slå "Do Not Track" til
\item Slå Java helt fra, afinstaller evt. Java helt fra computeren
\item Installer en AdBlocker - jeg bruger AdBlock\\
Vigtigt: servere der viser reklamer er ofte mål for hacking
\end{list2}
\end{list1}

\slide{Hvor ændrer man indstillingerne}

\hlkimage{20cm}{firefox-settings.png}

De fleste findes under:
\begin{list2}
\item Chrome \link{chrome://settings/} og \link{chrome://extensions/}
\item Firefox Indstillingerne og for enkelte ting: \link{about:config}
\end{list2}

\centerline{Kig også gerne på Safari eller Internet Explorer indstillingerne}



\slide{HTTPS Everywhere}

\hlkimage{5cm}{HTTPS_Everywhere_new_logo.jpg}
\begin{quote}
HTTPS Everywhere is a Firefox extension produced as a collaboration between The Tor Project and the Electronic Frontier Foundation. It encrypts your communications with a number of major websites.
\end{quote}

\centerline{\link{https://www.eff.org/https-everywhere}}

Also in Chrome web store!


\slide{NoScript Firefox and ScriptBlock Chrome}

\hlkimage{18cm}{scriptblock.png}

\vskip 2cm
NoScripts for Firefox eller ScriptBlock for Chrome\\
Tillader kun JavaScript på sider hvor det er OK




\slide{Fuld Disk Kryptering: Bitlocker}

\begin{list2}
\item Microsoft tilbyder Bitlocker fuld disk kryptering
\item Åbnes med dit Windows kodeord
\item Meget transparent - data krypteres når det skrives ned
\item Nedsætter ikke hastigheden mærkbart, ofte forbedres den endda
\item Genetableringsnøgle - er slået til på FT computere\\
Giver mulighed for at IT-afd kan åbne din computer hvis du glemmer koden
\item Fungerer på både roterende diske og SSD, \\
men pas på SSD kan have data fra før kryptering slået til
\end{list2}

Kilde: mere information om Bitlocker\\
{\footnotesize \link{http://windows.microsoft.com/en-us/windows-vista/bitlocker-drive-encryption-overview}}


\slide{Microsoft Bitlocker}

\hlkimage{16cm}{bitlocker-ms.jpg}

Kilde: {\small
\link{https://technet.microsoft.com/en-us/library/cc512654.aspx}}

\slide{Apple FileVault Full Disk Encryption Mac OS X}

\hlkimage{16cm}{apple-filevault-enabled.png}

\centerline{Indbygget, gratis, stærk - slå det til når I kommer hjem}


\slide{Keeping backup duplicate your data - sample Duplicity}

\begin{quote}
{\large\bf What is it?}

Duplicity backs directories by producing encrypted tar-format volumes and uploading them to a remote or local file server. Because duplicity uses librsync, the incremental archives are space efficient and only record the parts of files that have changed since the last backup. Because duplicity uses {\bf GnuPG} to encrypt and/or sign these archives, they will be safe from spying and/or modification by the server.
\end{quote}

\link{http://duplicity.nongnu.org/} duplicity home page

\link{http://www.gnupg.org/} The GNU Privacy Guard

\vskip 2cm
\centerline{Dont forget to DELETE data also, write over or physically destroy}

\slide{Surveillance Self-Defense EFF}

\hlkimage{10cm}{ssd-eff-logo.png}

\begin{quote}
\centerline{Tips, Tools and How-tos For Safer Online Communications}

Modern technology has given the powerful new abilities to eavesdrop and collect data on innocent people. Surveillance Self-Defense is EFF's guide to defending yourself and your friends from surveillance by using secure technology and developing careful practices.
\end{quote}

Source: \link{https://ssd.eff.org/}


\slide{Opsummering }

\begin{list1}
\item Husk følgende:
\begin{list2}
\item Husk: IT-sikkerhed er ikke kun netværkssikkerhed!
\item God sikkerhed kommer fra langsigtede intiativer
\item Hvad er informationssikkerhed?
\item Data på elektronisk form, USB drev
\item Data på fysisk form, køb en makulator
\item Lav backup af data I vil gemme! Køb en ekstern USB disk til offline\\
3-2-1 backup 3 kopier i 2 programmer med 1 offline/slukket
\end{list2}
\end{list1}
\vskip 1cm
\centerline{\color{titlecolor}\LARGE Informationssikkerhed er en proces}

\myquestionspage

\end{document}
