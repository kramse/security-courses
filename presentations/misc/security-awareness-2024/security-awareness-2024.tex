\documentclass[Screen16to9,17pt]{foils}
\usepackage{kea-slides}


% It-sikkerhedsupdate
% Få fremtidssikret it-sikkerhedsstrategien. Arrangement for medlemmer af Forsikringsforbundet og PROSA.

% Hvad skal en ansvarlig it-sikkerhedsstrategi være for 2019. Hvilke emner er de vigtigste, og hvad er truslerne, hvis man ikke straks kommer i gang med de 10 vigtigste punkter.

% Foredraget er en gennemgang af de 10 vigtigste områder og emner, som en organisation skal have styr på i 2019, med referencer til aktuelle sager som eksempel.

% Punkterne vil inkludere de sædvanlige, kedelige, men nødvendige; backup, CMDB, brugerstyring, logging m.fl. - men med forslag til praktiske værktøjer for at understøtte dem hurtigt.

\begin{document}
\selectlanguage{danish}
\mytitlepage{It-sikkerhed Awareness kort}{\the\year}


\vskip 1cm
\centerline{\footnotesize slides are available on Github}

\slide{Every Year}

\hlkimage{4cm}{happy-new-year-roven-images-601197-unsplash.jpg}


\begin{list2}
\item Same problems last year? Same problems EVERY year
\item Last year was a nightmare of break-ins and data leaks
\item Data leaks, GDPR, ransomware, ...
\end{list2}

\vskip 1cm
{\LARGE\bf Try not to panic, but there are lots of threats}

Are we loosing the battle?

\slide{Har du haft snakken med din CISO?}

%\hlkimage{}{}

\begin{quote}\large
{\bf IT-sikkerhed:}\\
Vi vil gerne bede om 10 millioner til IT-sikkerhed i budget næste år

{\bf CTO/CIO/CISO:}\\
Umuligt

{\bf IT-sikkerhed:}\\
OK, fair nok. Så skal vi bare bede om {\bf 100 millioner til Ransomware}, tak.

Husk også at uddanne CFO i bitcoin transaktioner.
\end{quote}

\begin{list2}
\item Er ovenstående urealistisk?
\end{list2}

\slide{Demant 2019}

%\hlkimage{}{}

\begin{quote}
- For året 2019 rapporterede vi et tab i omsætning på {\bf 575 millioner kroner}. Det i sig selv er alvorligt. Hvad angår vores opmærksomhed på it-sikkerhedsområdet, har it-hændelsen været med til at understrege nødvendigheden af, at tage dette felt seriøst. Angreb mod it-infrastruktur er uden tvivl en af de største trusler mod en virksomhed, og det kan gå galt, hvis man ikke er i stand til at lukke ned for skaden og bruge sin back-up.

...

- På det konkrete plan har vi fået et mantra der lyder {\bf ’Active Directory is king, and backup is Queen’}. Men mere overordnet har vi også lært at Fokus skal helt op på øverste niveau i virksomheden, at man skal skaffe høj faglig indsigt i sikkerhed og trusler, og at det er et arbejde, der skal være under konstant observation og udvikling.

\end{quote}
Kilde: \link{https://dit.dk/Nyheder/2021/Demant}


\begin{list2}
\item Vi taler altså om tab i størrelsesorden tre-cifrede millionbeløb!
\end{list2}

\slide{Overlapping Security Incidents}

\hlkrightpic{9cm}{1cm}{datalaek-2019.png}

New data breaches nearly every week, these from danish news site \link{version2.dk}

Problem, we need to receive data from others

Data from others may contain malware

Have a job posting, yes\\
- then HR will be expecting CVs sent as .doc files


\slide{Paranoia defined}

\hlkimage{8cm}{paranoia-definition.png}
Source: google paranoia definition

Use appropriate paranoia, and yes, hot pink red blinking is an appropriate threat level


\slide{Hackers don't give a shit}

\hlkrightpic{11cm}{-3cm}{kiwicon-2009-hackers-dont-give-shit.jpg}

Your system is only for testing, development, ...

Your network is a research network, under construction, \\
being phased out, ...

Try something new, go to your management

Bring all the exceptions, all of them, update the risk \\
analysis figures - if this happens it is about 1mill DKK

{\bf Think like attackers - don't hold back}


\slide{Confidentiality Integrity Availability}

\hlkimage{8cm}{cia-triad-uk.pdf}

\begin{list1}
\item We want to protect something
\item Confidentiality - data kept secret
\item Integrity - no unauthorized changes to data
\item Availability - data and systems are available to authorized uses when they need them
\end{list1}


\slide{Fokus on the basics}

\begin{list2}
\item User management - including administrative users
\item Asset management
\item Laptop security
\item Penetration testing
\item Firewalls and segmentation
\item VPN everywhere
\item TLS and VPN settings, encryption
\item DNS and email security
\item Syslog and monitorering
\item Incident Response and response
\end{list2}

\vskip 5mm
\centerline{Vi skal allesammen hjælpe hinanden! Ovenstående er listen jeg giver studerende og virksomheder}



\slide{Fokus: User management}

\hlkimage{8cm}{humans2.png}

\begin{list2}
\item Relevant for alle organisationer
\item Er måden vi sikrer godkendte brugere kan udføre opgaver
\item Kodeord bruges til at forhindre uautoriseret adgang
\item Har I styr på brugerid?
\item Hvor er brugere oprettet?
\item Hvor hurtigt kan I fjerne "een bruger" eller "deaktivere en bruger" alle steder!
\item Er det et kludetæppe - ja, mange steder er det
\end{list2}


\slide{Local administrator?}

\hlkimage{10cm}{dragon-drawing-6.jpg}

\begin{list2}
\item Findes der systemer som er helt åbne, med lokal administrator
\item Er det stadig nødvendigt
\item Vi bør bruge Principle of Least privilege
\item Vi ved hvordan, for det fortalte Jerome Saltzer og Michael Schroeder i deres 1975 artikel\\ \emph{The Protection of Information in Computer Systems}\\
\url{https://en.wikipedia.org/wiki/Saltzer_and_Schroeder%27s_design_principles}
\end{list2}


\slide{Passwords vælges ikke tilfældigt}

\hlkimage{20cm}{50-most-used-passwords.png}

Source:
\link{https://wpengine.com/unmasked/}


\slide{Your data has already have been owned by criminals}

\hlkimage{9cm}{pwned.png}

\begin{list1}
\item Your data is already being sold, and resold on the Internet
\item Stop reusing passwords, use a password safe to generate and remember
\item Check you own email addresses on \link{https://haveibeenpwned.com/}
\item They have an API you can integrate to avoid re-using already leaked passwords\\
{\footnotesize\link{https://www.troyhunt.com/introducing-306-million-freely-downloadable-pwned-passwords/}}
\end{list1}


\slide{Opbevaring af passwords}

\hlkimage{6cm}{password-window.png}

\begin{list2}
\item Use password managers! Available as cloud connected, local only, teams based
\item You will have to investigate which one to choose, but find one!
\end{list2}


\slide{Fokus: Laptop sikkerhed}

\hlkimage{13cm}{kelly-sikkema-212376-unsplash.jpg}

\begin{list2}
\item Relevant for alle
\item Hvordan sikrer vi at vi ikke mister værdierne, hardware og data typisk
\end{list2}


\slide{Secure Laptops}

\hlkimage{10cm}{librem-15-v3-turns99.png}

\begin{list2}
\item Laptops (og mobile enheder)
\item Hvad kendetegner en laptop? og en telefon?
\item Hardware naturligvis, en Macbook koster officielt mere end en brugt mellemklassebil
\item - og husk brugen af laptops -- de er dyre, men indholdet er ofte mere værd!
\item Er laptops sikre, og hvad betyder det?
\end{list2}



\slide{Are your data secure - data at rest}

\hlkimage{15cm}{images/data-integrity-1.pdf}

\begin{list1}
\item Stolen laptop, tablet, phone - can anybody read your data?
\item Do you trust "remote wipe"
\item How do you in fact wipe data securely off devices, and SSDs?
\item Encrypt disk and storage devices before using them in the first place!
\end{list1}



\slide{Start Attacking from the Inside}

\hlkimage{6cm}{erik-odiin-568459-unsplash.jpg}


\begin{list2}
\item Now imagine you were in control of a company laptop
\item Do you have a large internal world wide network?\\
Having a large open network may cost you {\bf 1.9 billion DKK - ref Maersk case}
\item Try scanning everything, start in a small corner, expand
\item Scan all you danish segments, one by one, then the nordic, then the world
\item Yes, things may break - FINE, BREAKING is GOOD
\end{list2}

\centerline{\bf Better to break while we are ready to un-break}


\slide{Nmap the world}

\hlkimage{19cm}{trinity-nmapscreen-hd-cropscale-418x250.jpg}


\slide{Hackertools are for everyone!}

{\Large\bf Hackers work all the time trying to break stuff}

Blue teams can use hackertools, and become more effecient:
\begin{list2}
\item Nmap, Nping \link{http://nmap.org}
\item Wireshark - \link{http://www.wireshark.org/}
\item Aircrack-ng \link{http://www.aircrack-ng.org/}
\item Metasploit Framework \link{http://www.metasploit.com/}
\item Burpsuite \link{http://portswigger.net/burp/}
\item Kali Linux \link{http://www.kali.org}
\end{list2}

\vskip 5mm
\centerline{Most popular hacker tools \link{https://tools.kali.org/} and \link{http://sectools.org/}}


\slide{Kali Linux the pentest toolbox}

\hlkimage{14cm}{kali-linux.png}

\begin{list1}
\item  Kali \link{http://www.kali.org/}
\item 100.000s of videos on youtube alone, searching for kali and \$TOOL
\item Also versions for Raspberry Pi, mobile and other small computers
\end{list1}


\slide{Hackerlab setup}

\hlkimage{11cm}{hacklab-1.png}

\begin{list2}
\item Create hacker labs, encourage hacker labs!
\item Software Host OS: Windows, Mac, Linux
\item Virtualisation software: VMware, Virtual box, HyperV pick your poison
\item Hackersoftware: Kali Virtual Machine \link{https://www.kali.org/} ,
\end{list2}

\slide{Hacking is not magic}

\hlkimage{11cm}{ninjas.png}

\begin{list2}
\item Hacking only requires some ninja training
\item We have been doing this since 1995 when SATAN was released
\item Listen, Plan, Act, Do hacking
\item Be curious, and honest -- let our students play with fire in special networks
\end{list2}

\slide{Fokus: Firewalls og segmentering}

\hlkimage{10cm}{virksomhedens-netvaerk.pdf}

\begin{list2}
\item Hvis du har et netværk, så bør du have en firewall
\item En firewall tillader autoriseret trafik og blokerer resten
\item Hvornår har du sidst set din løsning efter?
\item Hvor lang tid tager det at se en 5.000 linier Cisco ASA config igennem?
\end{list2}

\slide{Imagine Attacks from the Inside}

\hlkimage{6cm}{erik-odiin-568459-unsplash.jpg}

\begin{list2}
\item Now imagine you were in control of a company laptop
\item Do you have a large internal world wide network?\\
NotPetya cost Maersk about 1.9 billion DKK
%\item Try scanning everything, start in a small corner, expand
%\item Scan all you danish segments, one by one, then the nordic, then the world
%\item Yes, things may break - FINE, BREAKING is GOOD

\item entry thought to be via software update of M.E.Doc [uk] an Ukrainian tax preparation program
\item Attackers are very creative and have a large attack surface to most companies
\end{list2}


\slide{LG TVs 2024 -- CVE-2023-6317 up to CVE-2023-6320}

\hlkimage{10cm}{LG-shodan.png}

\begin{quote}{\large\bf
90,000+ LG TVs Vulnerable to Authorization Attacks\\
Due to WebOS Vulnerabilities}

Bitdefender Labs has revealed a critical security flaw in over 90,000 LG smart TVs running the company’s proprietary WebOS platform.

If exploited, the vulnerability could allow attackers to gain unauthorized access to the TV’s functions and potentially the user’s home network.

\end{quote}
Source: \url{https://cybersecuritynews.com/lg-tvs-vuauthorization-attacks/}


\slide{D-Link NAS devices accessible via “backdoor” account CVE-2024-3273}

%\hlkimage{}{}

\begin{quote}{\large\bf
92,000+ internet-facing D-Link NAS devices accessible via “backdoor”}

A vulnerability (CVE-2024-3273) in four old D-Link NAS models could be exploited to compromise internet-facing devices, a threat researcher has found.

The existence of the flaw was confirmed by D-Link last week, and an exploit for opening an interactive shell has popped up on GitHub.

“The vulnerability lies within the \verb+nas_sharing.cgi+ uri, which is vulnerable due to two main issues: a backdoor facilitated by hardcoded credentials, and a command injection vulnerability via the system parameter,” says the discoverer, who goes by the online handle “netsecfish”.

{\bf The “backdoor” account has messagebus as the username and doesn’t require a password.}
\end{quote}
Source: \url{https://www.helpnetsecurity.com/2024/04/08/cve-2024-3273/}



\slide{Fokus: VPN alle steder}

\hlkimage{12cm}{ks-kyung-784757-unsplash.jpg}

\begin{list2}
\item VPN er relevant for alle der har data af værdi
\item Sikrer data der flyttes
\item Virtual Private Network dækker over klienter der kobler op, og site-2-site
\end{list2}


\slide{Fokus: DNS og email}

\hlkimage{4cm}{brian-patrick-tagalog-680954-unsplash.jpg}

\begin{list2}
\item Vi er afhængige af email, modtagelse og afsendelse
\item Når vi modtager skal det helst gå hurtigt
\item Når vi sender skal vi ikke ende i spam mappen
\item Phishing, hvem kan sende \emph{fra vores domæne}
\end{list2}


\slide{Various key attack types, clients and employees}

\begin{list2}
\item Phishing - sending fake emails, to collect credentials
\item Spear phishing - targetted attacks
\item Person in the middle - sniffing and changing data in transit
\item Drive-by attacks - web pages infected with malware, often ad servers
\item Malware transferred via USB or email
\item Credential Stuffing, Password related, like re-use of password, see slide about being pwned
\end{list2}

\vskip 1cm
\centerline{\Large\bf If we all wait a bit, and not click links immediately}

\vskip 1cm
Hackers try to create "urgency", click this or loose money


\slide{Storing query logs, old school or needed?}

\hlkimage{5cm}{bro-sample-ssl-scripts.png}

\begin{list2}
\item DNS query logs, keep it for at least a week?\\
- with DSC and PacketQ \link{https://github.com/DNS-OARC/PacketQ}
\item SSL/TLS log with Zeek/Suricata\\
{\footnotesize\link{https://www.zeek.org/sphinx-git/script-reference/scripts.html}}
\item Log with Elasticsearch?\\
{\footnotesize\link{https://www.elastic.co/guide/en/elasticsearch/guide/current/index.html}}
%\item Even netflow session logging, full 1:1 - NFSen, Suricata Flow mode?
%\item Moloch \link{https://github.com/aol/moloch}
\item  Uetisk? eller smart hvis man vil spore hvor malware kom ind
\item {\bf Vi må nok som medarbejdere acceptere mere logning, men selvfølgelig ikke som privatpersoner og borgere}
\end{list2}


\slide{Network visibility: Netflow with NFSen}

\hlkimage{15cm}{nfsen-udp-flood.png}

\centerline{An extra 100k packets per second from this netflow source (source is a router)}

Logging can show what happens/happened.


\slide{Fokus: Incident Response og reaktion}

\hlkimage{10cm}{margarida-csilva-121801-unsplash.jpg}

\begin{list2}
\item Fortsat fra logningen ... hvad så nu!
\item Hvis du har en sikkerhedshændelse skal den håndteres
\item jo hurtigere og mere effektivt det håndteres jo bedre
\end{list2}

Lifeguard training photo by Margarida CSilva on Unsplash

\slide{Øv krisesituationer}

\hlkimage{14cm}{sheldon-nunes-1226991-unsplash.jpg}

\begin{list2}
\item Lav rollespil
\item Lav tabletop exercises
\end{list2}

\slide{Spørgsmål og mere debat}

\hlkimage{7cm}{idog.jpg}

\begin{center}
\hlkbig

\myname

\end{center}

\end{document}



\end{document}
