\documentclass[20pt,landscape,a4paper,footrule]{foils}
\usepackage{solido-network-slides}
\usepackage{pdf14}
%\usepackage{ulem}

% Basic things that we need are below
\selectlanguage{danish}

%\externaldocument{unix-audit-security-oevelser}
\externaldocument{\jobname-exercises}


%\slide{Pause}
%Er det tid til en lille pause?
%\hlkimage{15cm}{300px-Fozziecurtain.JPG}


% Surveillance and hacking, protect yourself

% The internet today is extremely nice, and dangerous ... What tools are used to 
% monitor your traffic and data and how can you really protect yourself in this environment. 
% The presentation will go through examples of data being "leaked" from your devices, how to 
% investigate your devices, data, and applications. Recommendations for protecting yourself 
% will be presented, using pointers to existing materials like cryptoparty, EFF.org, and others. 
% The content will be less technical and more of a call to action.

% Henrik Kramsh�j is an internet samurai cand.scient CISSP from Denmark working mostly with internet packets and security

% Keywords: HTTP, wireshark, chaosreader, bruteforce tools, browser plugins, full disk crypto


% \hlkimage{16cm}{government-doing-nothing-wrong.jpg}



\begin{document}

\mytitlepage
{Surveillance and hacking, protect yourself}{Driving IT 2014}
\vskip 1 cm

\centerline{Go watch citizenfour \link{http://cphdox.dk/screening/citizenfour}}

{\footnotesize Slides on Github, open license: kramshoej security-courses presentations/misc/surveillance-protect-yourself}


\slide{Goals: Why?}

\hlkimage{22cm}{jacob-appelbaum.png}

Jacob Appelbaum:\\
If Everything is Under Surveillance, How Can We Have a Democracy?

{\bf Democracy:}
A free democracy must allow citizens to take decisions without constant surveillance, which are free to use cryptography to control who we allow access to our data.

\vskip 2cm
\centerline{\LARGE Crypto is a peaceful protest}



\slide{Obsessive Data Collection leads to abuse}

%\hlkimage{10cm}{homer-end-is-near.jpg}


\begin{quote}
Beware the Four Horsemen of the Information Apocalypse: terrorists, drug dealers, kidnappers, and child pornographers. Seems like you can scare any public into allowing the government to do anything with those four.
\end{quote}
Quote: Bruce Schneier 2005\\
{\small\link{https://www.schneier.com/blog/archives/2005/12/computer_crime_1.html}}

Data gathered {\bf will be abused} either for criminal purposes, commercial purposes no matter what the original intended purpose was. The gathering of data itself becomes an easily abused target.





\slide{A vulnerability can and will be abused}

Investigative organizations, like europol, FBI and other want a "golden key" which they can get with a court order. This cant happen! This wont work. See below link, and Google: clipper chip and crypto wars


\textbf{What if I told you}:

{\Large \bf Criminals will be happy to leverage backdoors created by government}

It does not matter if the crypto product has a weakness to allow investigations or the software has a backdoor to help law enforcement. Data and vulnerabilities WILL be abused and exploited.

Read more about Return of the Crypto Wars with Bruce Schneier\\
{\small \link{https://www.schneier.com/blog/archives/2014/10/iphone_encrypti_1.html}}


\slide{Hackerv�rkt�jer}
% m�ske til reference afsnit?
\hlkimage{3cm}{hackers_JOLIE+1995.jpg}

\centerline{LARGE Start hacking your devices, watch them and learn}

\begin{list2}
\item Nmap, Nping - tester porte, godt til firewall admins \link{http://nmap.org}
\item Metasploit Framework \link{http://www.metasploit.com/}
\item Wireshark avanced network analyzer - \link{http://http://www.wireshark.org/} 
%\item Paros proxy \link{http://www.parosproxy.org}
\item Burpsuite \link{http://portswigger.net/burp/}
\item Skipfish \link{http://code.google.com/p/skipfish/}
\item OpenBSD operating system extreme focus on security
 \link{http://www.openbsd.org} 
\end{list2}

Picture: Angelina Jolie, Acid Burn Hackers 1995


\slide{Wireshark - avanced network analyzer }

\hlkimage{17cm}{images/wireshark-http.png}

\centerline{\link{http://www.wireshark.org}}
\centerline{Windows and Unix}


\slide{Quick Wins: Opsec Light}

\begin{list1}
\item Operations security (OpSec, OPSEC), what do you need?\\
\link{https://en.wikipedia.org/wiki/Operations_security}

\item Great description\\
"OpSec is about attracting the right amount of attention and not to raise any suspicion."
\link{https://www.cryptoparty.at/opsec}

\item Use multiple devices, isolate data
\item less critical on phone, most critical on laptop with full disk encryption

\item Using different password for each service, unpossible!

\item OTP One Time Password, sniff one and you can use it, if you have a time machine \smiley
\end{list1}

\slide{OTP Example: Yubico Yubikey}

\hlkimage{20cm}{yubico-overview.png}
\begin{quote}
A Yubico OTP is unique sequence of characters generated every time the YubiKey button is touched. The Yubico OTP is comprised of a sequence of 32 Modhex characters representing information encrypted with a 128 bit AES-128 key
\end{quote}

\link{http://www.yubico.com/products/yubikey-hardware/}

\slide{Low tech 2-step verification }

\centerline{\Large Printing code on paper, low level pragmatic }

\hlkimage{9cm}{google-backup-codes.png}

\begin{list1}
\item Login from new devices today often requires two-factor - email sent to user
\item Google 2-factor auth. SMS with backup codes
\item Also read about S/KEY developed at Bellcore {\bf in the late 1980s}\\ \link{http://en.wikipedia.org/wiki/S/KEY}
\end{list1}

\centerline{Conclusion passwords: integrate with authentication, not reinvent}

\slide{Integrate or develop?}

\begin{list1}
\item Dont:
\begin{list2}
\item Reinvent the wheel - too many times, unless you can maintain it afterwards
\item Never invent cryptography yourself
\item No copy paste of functionality, harder to maintain in the future
\end{list2}
\item Do:
\begin{list2}
\item Integrate with existing solutions
\item Use existing well-tested code: cryptography, authentication, hashing
\item Centralize security in your code
\item Fine to hide which authentication framework is being used, easy to replace later
\end{list2}
\end{list1}

\slide{Good security}

\hlkimage{15cm}{god-sikkerhed.pdf}

\begin{list1}
\item You always have limited resources for protection - use them as best as possible
\end{list1}


\slide{First advice}

\begin{list1}
\item Use technology
\item Learn the technology - read the freaking manual
\item Think about the data you have, upload, facebook license?! WTF!
\item Think about the data you create - nude pictures taken, where will they show up?
\begin{list2}
\item Turn off features you don't use
\item Turn off network connections when not in use
\item Update software and applications
\item Turn on encryption: IMAP{\bf S}, POP3{\bf S},
  HTTP{\bf S} also for data at rest, full disk encryption, tablet encryption
\item Lock devices automatically when not used for 10 minutes
\item Dont trust fancy logins like fingerprint scanner or face recognition on cheap devices
\end{list2}
\end{list1}


\slide{Second advice use the modern operating systems}

\begin{list1}
\item Newer versions of Microsoft Windows, Mac OS X and Linux
\begin{list2}
\item Buffer overflow protection
\item Stack protection, non-executable stack
\item Heap protection, non-executable heap
\item \emph{Randomization of parameters} stack gap m.v.
\end{list2}
\item Note: these still have errors and bugs, but are better than older versions
\item OpenBSD has shown the way in many cases\\ \link{http://www.openbsd.org/papers/}
\end{list1}

\vskip 1cm

\centerline{Always try to make life worse and more costly for attackers}

I dont think we will see good security in the mobile platforms for years to come, sorry!



\slide{Safe encrypted protocols}

{\bf \LARGE Sorry, none}

\begin{quote}
The 'S' in HTTPS stands for 'secure' and the security is provided by SSL/TLS. SSL/TLS is a standard network protocol which is implemented in every browser and web server to provide confidentiality and integrity for HTTPS traffic.
\end{quote}

OpenSSL, LibreSSL, Apple SSL flaw exit exit exit!, Android SSL, certs certs cert!!!111, SSLv3, Heartbleed

\centerline{Sorry, brain overflow from SSL/TLS vulnerabilities}

Sources: see my blog posts about heartbleed for more links and tools\\ 
{\small\link{http://www.version2.dk/blog/openssl-er-doed-laenge-leve-libressl-57640}\\
\link{http://www.version2.dk/blog/opdater-openssl-og-dit-os-nu-57202}}

\slide{Heartbleed hacking}

\begin{alltt}\footnotesize
  06b0: 2D 63 61 63 68 65 0D 0A 43 61 63 68 65 2D 43 6F  -cache..Cache-Co
  06c0: 6E 74 72 6F 6C 3A 20 6E 6F 2D 63 61 63 68 65 0D  ntrol: no-cache.
  06d0: 0A 0D 0A 61 63 74 69 6F 6E 3D 67 63 5F 69 6E 73  ...action=gc_ins
  06e0: 65 72 74 5F 6F 72 64 65 72 26 62 69 6C 6C 6E 6F  ert_order&billno
  06f0: 3D 50 5A 4B 31 31 30 31 26 70 61 79 6D 65 6E 74  =PZK1101&payment
  0700: 5F 69 64 3D 31 26 63 61 72 64 5F 6E 75 6D 62 65  _id=1&{\bf card_numbe}
  0710: XX XX XX XX XX XX XX XX XX XX XX XX XX XX XX XX  {\bf r=4060xxxx413xxx}
  0720: 39 36 26 63 61 72 64 5F 65 78 70 5F 6D 6F 6E 74  {\bf 96&card_exp_mont}
  0730: 68 3D 30 32 26 63 61 72 64 5F 65 78 70 5F 79 65  {\bf h=02&card_exp_ye}
  0740: 61 72 3D 31 37 26 63 61 72 64 5F 63 76 6E 3D 31  {\bf ar=17&card_cvn=1}
  0750: 30 39 F8 6C 1B E5 72 CA 61 4D 06 4E B3 54 BC DA  {\bf 09}.l..r.aM.N.T..
\end{alltt}

\begin{list2}
\item Obtained using Heartbleed proof of concepts - Gave full credit card details
\item "can XXX be exploited" - yes, clearly! PoCs ARE needed\\
without PoCs even Akamai wouldn't have repaired completely!
\item The internet was ALMOST fooled into thinking getting private keys\\
 from Heartbleed was not possible - scary indeed.
\end{list2}

\centerline{TL;DR Fund more security audits, stop using untested/unaudited software}


\slide{Bettercrypto.org}

\hlkimage{20cm}{bettercrypto-nginx.png}
\begin{quote}
Overview

This whitepaper arose out of the need for system administrators to have an updated, solid, well researched and thought-through guide for configuring SSL, PGP, SSH and other cryptographic tools in the post-Snowden age. ... This guide is specifically written for these system administrators. 
\end{quote}

\link{https://bettercrypto.org/}

\slide{Audits}

\hlkimage{10cm}{crypto-cat.png}
\begin{list1}
\item Truecrypt audit\\
{\footnotesize\link{https://isecpartners.github.io/news/2014/04/14/iSEC-Completes-Truecrypt-Audit.html}}
\item Cryptocat audit\\
{\footnotesize\link{https://blog.crypto.cat/2013/02/cryptocat-passes-security-audit-with-flying-colors/}}
\end{list1}

\centerline{Secure products need funding! Donate to multiple including OpenBSD}


\slide{Encrypting hard disk}

%\hlkimage{6cm}{images/apple-filevault.png}

\begin{list1}
\item Becoming available in the most popular client operating systems
\begin{list2}
\item Microsoft Windows Bitlocker - requires Ultimate or Enterprise
\item Apple Mac OS X - FileVault og FileVault2
\item FreeBSD GEOM og GBDE/GELI - encryption framework
\item Linux LUKS and dm-crypt \link{https://en.wikipedia.org/wiki/Dm-crypt}
\item PGP disk - Pretty Good Privacy - makes a virtuel krypteret disk
\item TrueCrypt? \emph{Let's audit Truecrypt!} Note: truecrypt halted and insecure? who knows?\\
{\small\link{http://blog.cryptographyengineering.com/2013/10/lets-audit-truecrypt.html}}
\end{list2}
\end{list1}

Note: in closed source product you ofc trust the developer/company producing the software


\slide{Keeping backup duplicate your data - sample Duplicity}

\begin{quote}
{\large\bf What is it?}

Duplicity backs directories by producing encrypted tar-format volumes and uploading them to a remote or local file server. Because duplicity uses librsync, the incremental archives are space efficient and only record the parts of files that have changed since the last backup. Because duplicity uses {\bf GnuPG} to encrypt and/or sign these archives, they will be safe from spying and/or modification by the server.
\end{quote}

\link{http://duplicity.nongnu.org/} duplicity home page

\link{http://www.gnupg.org/} The GNU Privacy Guard

\vskip 2cm
\centerline{Dont forget to DELETE data also, write over or physically destroy}


\slide{Use VPN!}

\hlkimage{12cm}{openvpn-gui-systray.png}

\begin{list1}
\item Virtual Private Networks are {\bf useful} - or even {\bf required when traveling}
\item VPN \link{http://en.wikipedia.org/wiki/Virtual_private_network}
\item SSL/TLS VPN - Multiple incompatible vendors: OpenVPN, Cisco, Juniper, F5 Big IP
\item L2TP IPsec - easy with OpenBSD as the VPN server\\
 \link{http://undeadly.org/cgi?action=article&sid=20120427125048}
\item Note: your VPN provider may be forced to give up your identity and traffic, beware!
\end{list1}

\slide{Multiple browsers}

\hlkimage{20cm}{multi-browser-strategy.png}

\begin{list2}
\item Strict Security settings in the general browser, Firefox or Chrome?
\item More lax security settings for "trusted sites" - like home banking
\item Security plugins like HTTPS Everywhere and NoScripts for generic browsing
\end{list2}


\slide{Censurfridns.dk uncensoredDNS}

\link{http://www.censurfridns.dk}

\begin{alltt}
www.censurfridns.dk

Welcome to www.censurfridns.dk. You are welcome to use:

anycast.censurfridns.dk / 91.239.100.100 / 2001:67c:28a4::
ns1.censurfridns.dk / 89.233.43.71 / 2002:d596:2a92:1:71:53::

as a resolver to avoid DNS censorship.
Please see blog.censurfridns.dk/en for more information.
\end{alltt}

\vskip 2cm

\centerline{\Large It is unacceptable to mess with DNS!}

\vskip 1cm
\centerline{\Large Keep the DNS, change the government!}


\slide{Secure your mobile}

\hlkimage{20cm}{the-guardian-project.pdf}

\centerline{Dont forget your mobile platforms \link{https://guardianproject.info/}}


\slide{Surveillance Self-Defense EFF}

\hlkimage{10cm}{ssd-eff-logo.png}

\begin{quote}
\centerline{Tips, Tools and How-tos For Safer Online Communications}

Modern technology has given the powerful new abilities to eavesdrop and collect data on innocent people. Surveillance Self-Defense is EFF's guide to defending yourself and your friends from surveillance by using secure technology and developing careful practices.
\end{quote}

Source: \link{https://ssd.eff.org/}

\slide{Ononymous robot, formerly ONO robot}

\hlkimage{20cm}{tactical-tech-securitybox.png}
Source: \link{https://tacticaltech.org/}


\slide{Be careful - questions?}

\hlkimage{4cm}{michael-conrad.jpg}
\centerline{Hey, Lets be careful out there!}
\vskip 2 cm

\begin{center}
\myname

%\myweb
\end{center}

\vskip 2cm
Source: Michael Conrad \link{http://www.hillstreetblues.tv/}

{\footnotesize Slides on Github, open license: kramshoej security-courses presentations/misc/surveillance-protect-yourself}

\slide{Extra slides}
Sorry, didnt have room for them in 20 minutes


\slide{PROSA CTF}

\hlkimage{10cm}{DSCF3910s.jpg}

\begin{list1}
\item PROSA afholder CTF konkurrence fredag den 28. november 2014 til l�rdag
\item Capture the Flag er en mulighed for at afpr�ve sine hackerskillz
\item Distribueret CTF med hold  Sjovt og l�rerigt 
\end{list1}
Kilde: \link{http://prosa-ctf.the-playground.dk/}

\centerline{Get ready! L�r debuggere, perl, java at kende, start p� at hacke}


\slide{Tor project}

\hlkimage{21cm}{tor-project.png}

\centerline{Get the Tor browser bundle from \link{https://www.torproject.org/}}

\slide{Turkey: Erdogan bans Twitter }

\hlkimage{12cm}{twitter-turkey.png}

\slide{Censorship on the internet - lol}

\centerline{\color{titlecolor} The Net interprets censorship as damage and routes around it.}

\hlkimage{22cm}{John-Gilmore-quotes.png}

\link{http://en.wikiquote.org/wiki/John_Gilmore}\\
\link{http://en.wikipedia.org/wiki/John_Gilmore_(activist)}

\slide{Directly connection Tor Users from DK 8.000}

\hlkimage{20cm}{tor-users-dk-2014.png}

Image from \link{https://metrics.torproject.org} 

\slide{Whonix}

\hlkimage{17cm}{400px-Whonix.jpg}

\begin{quote}
Whonix is an operating system focused on anonymity, privacy and security. It's based on the Tor anonymity network[5], Debian GNU/Linux[6] and security by isolation. DNS leaks are impossible, and not even malware with root privileges can find out the user's real IP.
\end{quote}

\link{https://www.whonix.org/}



\slide{Hacker - cracker}

{\bfseries Det korte svar - drop diskussionen}

%Det lidt l�ngere svar:\\
Det havde oprindeligt en anden betydning, men medierne har taget
udtrykket til sig - og idag har det begge betydninger. 

{\color{red}\hlkbig Idag er en hacker stadig en der bryder ind i systemer!}

ref. Spafford, Cheswick, Garfinkel, Stoll, ...
- alle kendte navne indenfor sikkerhed


\begin{list2}
\item \emph{Cuckoo's Egg: Tracking a Spy Through the Maze of Computer
 Espionage},  Clifford Stoll  
\item \emph{Hackers: Heroes of the Computer Revolution},
Steven Levy
\item \emph{Practical Unix and Internet Security},
Simson Garfinkel, Gene Spafford, Alan Schwartz 
\end{list2}

\slide{Definition af hacking, oprindeligt}

\begin{quote}
Eric Raymond, der vedligeholder en ordbog over computer-slang (The Jargon File) har blandt andet f�lgende forklaringer p� ordet hacker:
\begin{list2}
\item En person, der nyder at unders�ge detaljer i programmerbare systemer og hvordan man udvider deres anvendelsesmuligheder i mods�tning til de fleste brugere, der bare l�rer det mest n�dvendige
\item En som programmerer lidenskabligt (eller enddog fanatisk) eller en der foretr�kker at programmere fremfor at teoretiserer om det
\item En ekspert i et bestemt program eller en der ofter arbejder med eller p� det; som i "en Unixhacker".
\end{list2}
\end{quote}

\begin{list1}
\item Kilde: Peter Makholm, \link{http://hacking.dk}
\item Benyttes stadig i visse sammenh�nge se \link{http://labitat.dk}
\end{list1}


\end{document}
