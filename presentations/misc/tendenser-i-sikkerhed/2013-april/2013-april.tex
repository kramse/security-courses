\documentclass[20pt,landscape,a4paper,footrule]{foils}
\usepackage{solido-network-slides}

\begin{document}
\selectlanguage{danish}
\mytitlepage{Tendenser i sikkerhed\\{\small April 2013}}


\vskip 2cm
\centerline{\footnotesize Slides are available as PDF}

\slide{Form�l}

\begin{itemize}
\item Give en update p� udviklingen indenfor internetsikkerhed og sikkerhedstrusler
\item Give input til hvad I skal fokusere p� 
\vskip 2cm
\item Jeg vil fors�ge at gennemg� ting fra 2013
\item En potpourri af sikkerhedsemner - inspiration 
\item Feedback og kommentarer modtages, dialog \smiley
\end{itemize}

%\slide{Diverse noter}
% Fra de store konferencer: black hat

% Fra de store firmaer, Arbor Networks, Cymru m.fl. hvad siger de om trafikken
% HTTP command and control botnets, har overhalet IRC botnets


% March 2011, emner

% SSL CA systemet

% Netv�rksfolk DK-NOG
% Flash, igen igen igen - flash blocker NU! Hiv laptoppen frem og bloker Flash by default, nu!
% Java aktiv kode
% NemID, kan det bruges? Er det stabilt nok?
% 
% New software releases:
% MSF 3.6
% BackTrack er d�d hil Kali.org

% Suricata
% https://github.com/chrislee35/team-cymru queries Team Cymru's ASN, Malware, and FullBogon services 
% http://www.pentest-standard.org/index.php/Main_Page
% http://www.jetmore.org/john/code/swaks/

% DNS amplification, tcp DNS, DNSSEC, IPv6 DNS

% Anbefalinger
% Oldie but goodie, NIST SP docs?




\slide{Planen idag}

\hlkimage{10cm}{Shaking-hands_web.jpg}

\begin{list1}
\item Kl 17-21 med pauser
\item Mindre foredrag mere snak
\item Mindre enetale, mere foredrag 2.0 med socialt medie, informationsdeling og interaktion

\end{list1}



\slide{Internet Security Reports}

Lots of documentation
\begin{list1}

\item 2012 Verizon Data Breach Investigations Report
\item 2013 Trustwave Global Security Report
\item 2012 Worldwide Infrastructure Security Report Volume VIII,  Arbor Networks
\item 2013 State of Software Security Report The Intractable Problem of Insecure Software, Veracode April 2013
\end{list1}


\slide{Key findings 2011}

\begin{list2}
\item Application-Layer DDoS Attacks Are Increasing in Sophistication and Operational Impact
\item Mobile/Fixed Wireless Operators Are Facing Serious Challenges to Maintaining Availability in the Face of Attacks
\item Firewalls and IPS Devices Are Falling Short on DDoS Protection
\item DNS Has Broadly Emerged as an Attack Target and Enabler
\item Lack of Visibility into and Control over IPv6 Traffic Is a Significant Challenge
\item Chronic Underfunding of Operational Security Teams
\item Operators Continue to Express Low Confidence in the Efficacy of Law Enforcement
\item Operators Have Little Confidence in Government Efforts to Protect Critical Infrastructure
\end{list2}

Kilde:
\link{http://www.arbornetworks.com/report} februar 2011 - 2011 slide repeated here without changes






\slide{DDoS udviklingen, januar 2010 rapporten}

\hlkimage{15cm}{DDoS-2010.png}

Kilde:
\link{http://www.arbornetworks.com/report} 2009 rapporten



\slide{DDoS udviklingen, februar 2011}

\hlkimage{15cm}{ddos-2010-arbornetworks.png}

Kilde:
\link{http://www.arbornetworks.com/report} 2010 rapporten




\slide{DNSSEC}

\hlkimage{20cm}{wwwdnsseckeys_02.png}

\centerline{DNSSEC - nu ogs� i Danmark}

Du kan sikre dit dom�ne med DNSSEC - wooohooo!

Det betyder en tillid til DNS som muligg�r alskens services.

Kilde:\\
\link{https://www.dk-hostmaster.dk/english/tech-notes/dnssec/}


\slide{DNS amplification}

cloudflare 300Gbit DDoS?

DNS: DNSSEC, TCP queries, IPv6 DNS, DNS reply-size testing



\slide{Flash blockers}

\hlkimage{6cm}{clicktoflash.png}

\begin{list1}
\item Safari \link{http://clicktoflash.com/}
\item Firefox Extension Flashblock
\item Chrome extension called FlashBlock
\item Internet Explorer 8: IE has the Flash block functionality built-in so you don't need to install any additional plugins to be able to block flash on IE 8.
\item FlashBlock for Opera 9 - bruger nogen Opera mere?
\item FlashBlockere til iPad? iPhone? Android? - hvorfor er det ikke default?
\end{list1}

\slide{NemID og Java}

\begin{list1}
\item Java er et krav for at bruge NemID.

\item Brug gerne flere browsere, hvor kun een har Java sl�et til
\end{list1}

\slide{IPv6 is coming}

\hlkimage{2cm}{IPv6ready.png}

\begin{list1}
\item An important consideration is that IPv6 is quite likely to be already running on the enterprise network, whether that implementation was planned or not. Some important characteristics of IPv6 include:
\begin{list2}
\item IPv6 has a mechanism to automatically assign addresses so that end systems can easily establish communications.
\item IPv6 has several mechanisms available to ease the integration of the protocol into the network.
\item Automatic tunneling mechanisms can take advantage of the underlying IPv4 network and connect it to the IPv6 Internet.
\end{list2}
\end{list1}

Kilde:\\
{\footnotesize\link{http://www.cisco.com/en/US/prod/collateral/iosswrel/ps6537/ps6553/white_paper_c11-629391.html}}


\slide{Implications}

\hlkimage{2cm}{IPv6ready.png}

\begin{list1}
\item For an IPv4 enterprise network, the existence of an IPv6 overlay network has several of implications:
\begin{list2}
\item The IPv4 firewalls can be bypassed by the IPv6 traffic, and leave the security door wide open.
\item Intrusion detection mechanisms not expecting IPv6 traffic may be confused and allow intrusion
\item In some cases (for example, with the IPv6 transition technology known as 6to4), an internal PC can communicate directly with another internal PC and evade all intrusion protection and detection systems (IPS/IDS). Botnet command and control channels are known to use these kind of tunnels.
\end{list2}
\end{list1}

Kilde:\\
{\footnotesize\link{http://www.cisco.com/en/US/prod/collateral/iosswrel/ps6537/ps6553/white_paper_c11-629391.html}}



\slide{IPv6 in the Nordic region - 2013}

\hlkimage{14cm}{ipv6-nordic-2013.png}

\link{http://v6asns.ripe.net/v/6?s=SE;s=FI;s=NO;s=DK;s=IS;s=_ALL}\\
\link{https://www.ripe.net/membership/indices/DK.html}


\slide{Hackersoftware og andre tools}

\begin{list1}
\item Software and tool releases:
\begin{list2}

\item BackTrack Kali http://www.kali.org/ \link{http://www.backtrack-linux.org}

\item Suricata \link{http://www.openinfosecfoundation.org/}

\item Nmap og Nping \link{nmap.org}
\item Metasploit Framework \link{http://www.metasploit.com/}

\item Github is also a source of great scripts and input
\end{list2}
\end{list1}



\slide{Nping check TCP socket connection}

\begin{alltt}\footnotesize
hlk@pumba:nmap-5.51$ nping -6  www.solidonetworks.com 

Starting Nping 0.5.51 ( http://nmap.org/nping ) at 2011-03-04 10:18 CET
SENT (0.0061s) Starting TCP Handshake > 2a02:9d0:10::9:80
RECV (0.0224s) Handshake with 2a02:9d0:10::9:80 completed
SENT (1.0213s) Starting TCP Handshake > 2a02:9d0:10::9:80
RECV (1.0376s) Handshake with 2a02:9d0:10::9:80 completed
SENT (2.0313s) Starting TCP Handshake > 2a02:9d0:10::9:80
RECV (2.0476s) Handshake with 2a02:9d0:10::9:80 completed
SENT (3.0413s) Starting TCP Handshake > 2a02:9d0:10::9:80
RECV (3.0576s) Handshake with 2a02:9d0:10::9:80 completed
SENT (4.0513s) Starting TCP Handshake > 2a02:9d0:10::9:80
RECV (4.0678s) Handshake with 2a02:9d0:10::9:80 completed
 
Max rtt: 16.402ms | Min rtt: 16.249ms | Avg rtt: 16.318ms
TCP connection attempts: 5 | Successful connections: 5 | Failed: 0 (0.00%)
Tx time: 4.04653s | Tx bytes/s: 98.85 | Tx pkts/s: 1.24
Rx time: 4.06292s | Rx bytes/s: 49.23 | Rx pkts/s: 1.23
Nping done: 1 IP address pinged in 4.07 seconds
\end{alltt}

\link{http://nmap.org}


\slide{Metasploit and Armitage}

\begin{list1}
\item Still rocking the internet
\item Armitage GUI fast and easy hacking for Metasploit \link{http://www.fastandeasyhacking.com/}
\end{list1}

Kilde:\\
{\small \link{http://www.metasploit.com/redmine/projects/framework/wiki/Release_Notes_360}}





\slide{it's a Unix system, I know this}


\hlkimage{24cm}{twitter-unix-security.png}

\begin{list1}
\item Skal du igang med sikkerhed?
\item Installer et netv�rk, evt. bare en VMware, Virtualbox, Parallels, Xen, GNS3, ...
\item Brug BackTrack, se evt. youtube videoer om programmerne
\end{list1}

Quote fra Jurassic Park
\link{http://www.youtube.com/watch?v=dFUlAQZB9Ng}






\slide{SSL and CA}

\begin{quote}
The 'S' in HTTPS stands for 'secure' and the security is provided by SSL/TLS. SSL/TLS is a standard network protocol which is implemented in every browser and web server to provide confidentiality and integrity for HTTPS traffic.
\end{quote}



\slide{Fokus i n�rmeste fremtid}

\begin{list1}
\item S�rg for at f� overblik over infrastrukturen
\item S�rg for at have oveblik over organisationen og leverand�rer
\item Put evt. kritiske tlfnr ind i mobilen - NOC og support hos dine ISP'er
\end{list1}

\slide{Fokus p� l�ngere sigt}

\begin{list1}
\item Hvad skal I bruge tiden p� - planl�gge fremtiden
\vskip 1cm
\item Har du beredskab til sommeren, se p� ressourcer - er der fyret medarbejdere
\vskip 1cm
\item Kast ansvar fra dig? Har du reelt ressourcer til at udf�re arbejdet forsvarligt
\vskip 1cm
\item Afd�kke afh�ngigheder - hvem er din organisation afh�ngige af
\vskip 1cm
\item Configuration Management, Patch management og automatiseret sikkerhedstest\\
Start evt. med RANCID, NeXpose Community Edition og Metasploit fra BackTrack

\end{list1}

\slide{Managed security giver mening!}

\begin{list1}
\item Trenden g�r mod komplekse infrastrukturer, mere af den og h�jere krav
\item Kunderne vil have h�j oppetid, fordi internet teknologier er forretningskritiske 
\item Kunder der ikke betragter netv�rket som forretningskritisk lider tab
\item Kunderne har ikke {\it nok} netv�rk til at have fuldtidsansatte
\item Hvad skal der til for at tilbyde Managed Security Services
\end{list1}


\slide{Typiske Managed Security Services}

\begin{quote}
In computing, managed security services (MSS) are network security services that have been outsourced to a service provider.
\end{quote}
Kilde: \link{http://en.wikipedia.org/wiki/Managed_security_service}

\begin{list1}
\item Opgaver som tidligere blev h�ndteret in-house, eller ignoreret:
\item Event opsamling og analyse, 
Email scanning, Anti-virus og spam,
\item Firewall ops�tning, drift og konfiguration
\item Audit af netv�rk l�bende, som en service - aktive pentest, paper review
\item Netv�rksops�tning internt, STP, RSTP, stacks, LACP, LLDP, ...
\item Netv�rksops�tning eksternt, BGP, LC-SC, single-mode, mono-mode, multi-mode, link-net, PI, PA, RIPE
\item Angreb DoS, DDoS m.v.
\end{list1}


\slide{Udfordringerne i MSS}

\begin{list1}
\item Definition af nye produkter - hvad f�r kunden
\item Kommunikation, b�de ved �ndringer, problemer, opf�lgning
\item Det er en omstilling for os at definere produkterne, men sundt
\item Kunderne er ikke vant til at overlade s� meget til os
\item Hvem har reelt kontrollen? kan man out-source sikkerhed?
\item Ansvar - SLA d�kker jo oppetid, hvad med brud p� sikkerheden
\item Virtualisering af sikkerhed
\end{list1}


\slide{RANCID - Really Awesome New Cisco confIg Differ}

\begin{alltt}\footnotesize
[rancid@ljh routers]$ cat router.db 
mx-lux-01:juniper:up
mx-lux-02:juniper:up
...
[rancid@ljh routers]$ crontab -l
# run config differ hourly
07 0-23/2 * * * /usr/local/rancid/bin/rancid-run
# clean out config differ logs
50 23 * * * /usr/bin/find /usr/local/rancid/var/logs -type f -mtime +2 -exec rm \{\} 
\end{alltt}

\begin{slidelist}
\item RANCID will then fetch configurations, and more, and put it into version control SVN/CVS
\item Changes are emailed to an email alias
\end{slidelist}

\slide{RANCID output}


\hlkimage{20cm}{images/rancid-email.png}




\slide{NIST Special Publication 800 series}

\begin{list1}
\item Kender I NIST special publications?
\item SP 800-119	Feb. 22, 2010	DRAFT Guidelines for the Secure Deployment of IPv6\\
God fordi den forklarer hvad IPv6 er

\item SP 800-58	Jan 2005	Security Considerations for Voice Over IP Systems\\
Giver n�sten et design der kan bruges direkte, giver svar p� sp�rgsm�l du selv har glemt at stille
\end{list1}

\link{http://csrc.nist.gov/publications/PubsSPs.html}


\slide{Near future - summer 2013}

\begin{list1}
\item DNS: DNSSEC, TCP queries, IPv6 DNS, DNS reply-size testing
\item Mere IPv6: 
\item Automatic BGP blackhole routing, perhaps based on input from Suricata
\item Conferences RIPE66 Dublin hardcore network people, OHM2013 Observe Hack Make
\end{list1}



\slide{Sources for information}

\hlkimage{18cm}{twitter-security-feed.png}

\begin{list1}
\item Twitter has replaced RSS for me
\item Email lists are still a good source of data
\item Favourite Security Diary from Internet Storm Center \link{http://isc.sans.edu/index.html}
\end{list1}


\slide{Open Mike night ...}

\vskip 3 cm

\centerline{\Large Hvad glemte jeg? Kom med dine favoritter \smiley}

\myquestionspage





\end{document}
