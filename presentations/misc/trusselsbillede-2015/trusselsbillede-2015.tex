\documentclass[20pt,landscape,a4paper,footrule]{foils}
\usepackage{solido-network-slides}

\begin{document}
\selectlanguage{danish}
\mytitlepage{IT-sikkerhed Awareness\\{\small Februar 2015}}


%\vskip 2cm
%\centerline{\footnotesize Slides are available as PDF, kramshoej@Github}

\slide{Goals of today}

\begin{list1}
\item Kl 13:00-15:00
\item Update on information security 
\item Awareness about the usual risks with examples
\item Threats to data
\vskip 2cm
\item Kl 15:00-16:00
\item Technical security related things - inspiration 
\item Kali Linux introduction, hacking examples
%\item Please give feedback and join me in discussions, dialogue \smiley
\end{list1}

\slide{Paranoia defined}

\hlkimage{15cm}{paranoia-definition.png}

Source: google paranoia definition

\slide{Face reality}

From the definition:
\begin{quote}
suspicion and mistrust of people or their actions without {\bf evidence or justification}
{\bf the global paranoia about hackers and viruses}
\end{quote}

\begin{list1}
\item It is not paranoia when:
\begin{list2}
\item Criminals sell your credit card information and identity theft
\item Trade infected computers like a commodity
\item Hackers break in and steal information
\item Governments write laws that allows them to introduce back-doors - and use these
\item Governments do blanket surveillance of their population
\item Governments implement censorship, threaten citizens and journalist
\end{list2}
\end{list1}

\vskip 1cm
\centerline{You are not paranoid when there are people actively attacking you!}

\slide{What is data?}
\hlkimage{10cm}{Linus3-04041999.jpg}

\begin{list1}
\item Personal data you dont want to loose:
\begin{list2}
\item Wedding pictures
\item Pictures of your children
\item Sextapes
\item Personal finances
\end{list2}
\end{list1}

Source: picture of my son less than 24 hours old - precious!


\slide{Hacker types anno 2008}
\hlkimage{10cm}{lisbeth-salander.jpeg}

\begin{list1}
\item Lisbeth Salander from the Stieg Larsson's award-winning Millennium series
does research about people using hacking as a method to gain access
\item How can you find information about people?
\end{list1}


\slide{Attack overview}

\hlkimage{22cm}{sicherheitstacho.png}

{\small\link{http://www.sicherheitstacho.eu/?lang=en}}


\slide{Movie:}

\hlkimage{16cm}{youtube-bic-lock.png}

\begin{list1}
\item Just search for: kryptonite lock bic pen
\item \link{https://www.youtube.com/watch?v=LahDQ2ZQ3e0}
\end{list1}


\slide{Heartbleed CVE-2014-0160}

\hlkimage{22cm}{heartbleed-com.png}

Source: \link{http://heartbleed.com/}


\slide{Heartbleed hacking}

\begin{alltt}\footnotesize
  06b0: 2D 63 61 63 68 65 0D 0A 43 61 63 68 65 2D 43 6F  -cache..Cache-Co
  06c0: 6E 74 72 6F 6C 3A 20 6E 6F 2D 63 61 63 68 65 0D  ntrol: no-cache.
  06d0: 0A 0D 0A 61 63 74 69 6F 6E 3D 67 63 5F 69 6E 73  ...action=gc_ins
  06e0: 65 72 74 5F 6F 72 64 65 72 26 62 69 6C 6C 6E 6F  ert_order&billno
  06f0: 3D 50 5A 4B 31 31 30 31 26 70 61 79 6D 65 6E 74  =PZK1101&payment
  0700: 5F 69 64 3D 31 26 63 61 72 64 5F 6E 75 6D 62 65  _id=1&{\bf card_numbe}
  0710: XX XX XX XX XX XX XX XX XX XX XX XX XX XX XX XX  {\bf r=4060xxxx413xxx}
  0720: 39 36 26 63 61 72 64 5F 65 78 70 5F 6D 6F 6E 74  {\bf 96&card_exp_mont}
  0730: 68 3D 30 32 26 63 61 72 64 5F 65 78 70 5F 79 65  {\bf h=02&card_exp_ye}
  0740: 61 72 3D 31 37 26 63 61 72 64 5F 63 76 6E 3D 31  {\bf ar=17&card_cvn=1}
  0750: 30 39 F8 6C 1B E5 72 CA 61 4D 06 4E B3 54 BC DA  {\bf 09}.l..r.aM.N.T..
\end{alltt}

\begin{list2}
\item Obtained using Heartbleed proof of concepts - Gave full credit card details
\item "can XXX be exploited" - yes, clearly! PoCs ARE needed\\
without PoCs even Akamai wouldn't have repaired completely!
\item The internet was ALMOST fooled into thinking getting private keys from Heartbleed was not possible - scary indeed.
\end{list2}

\slide{Malware charateristics}

\begin{list1}
\item Malware is advanced and sophisticated
\item Modular frameworks
\item Use strong cryptography to hide, and hide your data ransomware 
\item Use 0-day exploits - unknown to others
\item Use rootkits to stay under radar and avoid anti-virus
\item Mutate and change to avoid detection
\item In general less noisy
\end{list1}

\slide{Botnets and malware sold with support}

\hlkimage{21cm}{dagens-tilbud-trojanere.pdf}

\begin{list1}
\item Malware programmers act like software houses
\item "Buy this version with updates and support"
\item Rent a bot net with 100.000 computers
\end{list1}


\slide{Phishing - Receipt for Your Payment to mark561@bt....com}
\hlkimage{21cm}{paypal-phish.png}
\vskip 1cm
\centerline{\bf\LARGE Do you recognize Phishing?}

\slide{Risk management defined}

\hlkimage{23cm}{shon-harris-risk-management.png}

Source: Shon Harris \emph{CISSP All-in-One Exam Guide}

\vskip 2cm
\centerline{We all take risks every day - sometimes even calculated risk}

\slide{Google for it}

\hlkimage{16cm}{images/googledorks-1.pdf}

\begin{list1}
\item Google as a hacker tools?
\item Concept named googledorks when google indexes information not supposed to be public
\link{http://www.hackersforcharity.org/ghdb/}
\end{list1}


\slide{January 2013: Github Public passwords?}


\hlkimage{20cm}{github-credentials.png}

 Sources:\\
{\footnotesize\link{https://twitter.com/brianaker/status/294228373377515522}\\
\link{http://www.webmonkey.com/2013/01/users-scramble-as-github-search-exposes-passwords-security-details/}\\
\link{http://www.leakedin.com/}\\
\link{http://www.offensive-security.com/community-projects/google-hacking-database/}
}

\vskip 5mm
\centerline{Use different passwords for different sites, yes - every site!}


\slide{Sony Hack}

\begin{list1}
\item  nearly 40GB of data hacked and leaked 
\item Source Sony Pictures Entertainment's (SPE) internal computer systems.
\item Passwords - complete infrastructure and all passwords must be reset
\item Payroll information, financial information about producs and earnings
\item Information about planned products and strategy is out
\item Bad passwords allow access to critical assets
\item Leaked unreleased titles Annie, Mr. Turner, Still Alice, and To Write Love On Her Arms, as well as World War II drama Fury.
\end{list1}


Source:\\
\link{http://techcrunch.com/2014/11/30/five-sony-pictures-movie-screeners-leaked-after-hacking/}
\link{http://mashable.com/2014/12/04/sony-hack-data-details/}



\slide{Today - the network IS the computer}

\begin{list1}
\item Buzzword bingo: Cloud, SOAP, REST, XML, ...
\item Platform as a Service (PaaS) is the delivery of a computing platform and solution stack as a service.
\item Software as a service (SaaS)
\item High complexity but also computing thought as a utility like water, gas, power
\end{list1}

\vskip 2 cm
\centerline{\Large \bf Solutions are increasingly using \emph{web protocols} \& APIs}

\slide{Before}
\hlkimage{14cm}{medieval-clipart-5}
%\centerline{Picture from: http://karenswhimsy.com/public-domain-images}
\vskip 1 cm
\centerline{We are used to thinking about security as a castle with walls}

\vskip 1 cm
\centerline{Dependencies: your power, your internet, your servers, your people}


\slide{Cloud Security - different but the same}

\hlkimage{12cm}{images/cia-triad-uk.pdf}

Everything today is cloud - but did we loose security? and what is security?

Did you ask security questions? Did you even think about security when deciding to do cloud computing?


\slide{Take Charge of Your Security}

We have more customers asking about 
\begin{list2}
\item ISO/IEC 27001 - information security management system standards\\
\link{http://en.wikipedia.org/wiki/ISO/IEC_27001}
\item SSAE 16 No. 16, Reporting on Controls at a Service Organization\\ 
Statement on Standards for Attestation Engagements (SSAE) \link{http://ssae16.com/}
\item ISAE 3402 Assurance Reports on Controls at a Service Organization\\
International Standard on Assurance Engagements (ISAE)
\link{http://isae3402.com/}
\item Independent assessment from a trusted security firm - which must often also be certified
\item Physical Security, power, HVAC - trusted partners reviewing security
\item Implementing standards is also security
\end{list2}

Over the years organisations have become more mature with regards to security

Implementing security controls were easier when you owned all resources

\slide{Quick Wins: Opsec Light}

\begin{list1}
\item Operations security (OpSec, OPSEC), what do you need?\\
\link{https://en.wikipedia.org/wiki/Operations_security}

\item Great description\\
"OpSec is about attracting the right amount of attention and not to raise any suspicion."
\link{https://www.cryptoparty.at/opsec}

\item Use multiple devices, isolate data
\item less critical on phone, most critical on laptop with full disk encryption

\item Using different password for each service, unpossible!

\item Solutions:
\item Password databases - secure ones, not your Notepad txt document!
\item OTP One Time Password, sniff one and you cannot use it, unless you have a time machine \smiley
\end{list1}

\slide{Saving passwords}

\hlkimage{10cm}{password-window.png}

\vskip 5mm
\centerline{Use some kind of Password Safe program}


\slide{TwoFactor authentication: Example Duosecurity}

\hlkimage{10cm}{duosecurity-overview.png}
Video\\
\link{https://www.duosecurity.com/duo-push}

\link{https://www.duosecurity.com/}


\slide{Yubico Yubikey}

\hlkimage{20cm}{yubico-overview.png}
\begin{quote}
A Yubico OTP is unique sequence of characters generated every time the YubiKey button is touched. The Yubico OTP is comprised of a sequence of 32 Modhex characters representing information encrypted with a 128 bit AES-128 key
\end{quote}

\link{http://www.yubico.com/products/yubikey-hardware/}






\slide{Defense in depth}

\hlkimage{8cm}{security-layers-1-uk.pdf}

\centerline{\hlkbig\color{solido-orange} Defense using multiple layers is stronger!}


\slide{Cryptographic principles}

\begin{list1}
\item Known Algorithms
\item Keys are secret
\item Keys have lifetimes - change often
\item Succes in cracking an algoritm may be purely academic\\
any improvement in attacks is a success
\item New algorithms, programms, protocols etc. must be reviewed
\end{list1}

\slide{Kryptering}

Purpose of encryption

\vskip 3 cm
\centerline{\hlkbig cryptography is a way to ensure:}
\vskip 3 cm
\centerline{\hlkbig confidentiality - only authorized users can read}
\vskip 3 cm
\centerline{\hlkbig autenticity / integrity - data has not changed}

\slide{Kryptografi}

\hlkimage{18cm}{images/crypto-rot13.pdf}

\begin{list1}
\item Kryptografi er l�ren om, hvordan man kan kryptere data
\item Kryptografi benytter algoritmer som sammen med n�gler giver en
  ciffertekst - der kun kan l�ses ved hj�lp af den tilh�rende n�gle
\end{list1}

\slide{Public key kryptografi - 1}

\hlkimage{18cm}{images/crypto-public-key.pdf}

\begin{list1}
\item privat-n�gle kryptografi (eksempelvis AES) benyttes den samme
  n�gle til kryptering og dekryptering 
\item offentlig-n�gle kryptografi (eksempelvis RSA) benytter to
  separate n�gler til kryptering og dekryptering
\end{list1}

\slide{Public key kryptografi - 2}

\hlkimage{18cm}{images/crypto-public-key-2.pdf}

\begin{list1}
\item offentlig-n�gle kryptografi (eksempelvis RSA) bruger den private
  n�gle til at dekryptere
\item man kan ligeledes bruge offentlig-n�gle kryptografi til at
  signere dokumenter\\ - som s� verificeres med den offentlige n�gle
\end{list1}

\slide{Bettercrypto.org}

\hlkimage{20cm}{bettercrypto-nginx.png}
\begin{quote}
Overview

This whitepaper arose out of the need for system administrators to have an updated, solid, well researched and thought-through guide for configuring SSL, PGP, SSH and other cryptographic tools in the post-Snowden age. ... This guide is specifically written for these system administrators. 
\end{quote}

\link{https://bettercrypto.org/}

\slide{Email is insecure}

\hlkimage{20cm}{email-uden-krypterin.png}

\centerline{Email without encryption is like an open post card}


\slide{Email with encryption - sending}

\hlkimage{18cm}{email-med-kryptering.png}


\centerline{Sending a secure email is not hard}

\slide{Encrypted in transit}

\hlkimage{11cm}{modtaget-email-med-kryptering.png}

\centerline{A secure email is protected while being transported}

\slide{Good security}

\hlkimage{15cm}{god-sikkerhed.pdf}

\begin{list1}
\item You always have limited resources for protection - use them as best as possible
\end{list1}


\slide{First advice}

\begin{list1}
\item Use technology
\item Learn the technology - read the freaking manual
\item Think about the data you have, upload, facebook license?! WTF!
\begin{list2}
\item Turn off features you don't use
\item Turn off network connections when not in use
\item Update software and applications
\item Turn on encryption: IMAP{\bf S}, POP3{\bf S},
  HTTP{\bf S} also for data at rest, full disk encryption, tablet encryption
\item Lock devices automatically when not used for 10 minutes
\item Dont trust fancy logins like fingerprint scanner or face recognition on cheap devices
\end{list2}
\end{list1}

\slide{Are your data secure}

\hlkimage{15cm}{images/data-integrity-1.pdf}

\begin{list1}
\item Stolen laptop, tablet, phone - can anybody read your data?
\item Do you trust "remote wipe"
\item How do you in fact wipe data securely off devices, and SSDs?
\item Encrypt disk and storage devices before using them in the first place!
\end{list1}


\slide{Circumvent security - single user mode boot}
\begin{list1}
\item Unix systems often allows boot into singleuser mode\\
press command-s when booting Mac OS X 
\item Laptops can often be booted using PXE network or CD boot
\item Mac computers can become a Firewire disk\\
hold t when booting - firewire target mode
\item Unrestricted access to un-encrypted data
\item Moving hard drive to another computer is also easy
\end{list1}
\pause
\centerline{Physical access is often - {\bf game over}}

\slide{Mac OS X} 

\hlkimage{16cm}{macbook-black.jpg}

\centerline{Firewire target mode: Macbook disken kan tilg�s fra en anden Mac}

Press t to enter firewire target mode \smiley\\
\link{http://support.apple.com/kb/ht1661}

\slide{Encrypting hard disk}

\hlkimage{13cm}{images/apple-filevault.png}

\begin{list1}
\item Becoming available in the most popular client operating systems
\begin{list2}
\item Microsoft Windows Bitlocker - requires Ultimate or Enterprise
\item Apple Mac OS X - FileVault og FileVault2
\item FreeBSD GEOM og GBDE/GELI - encryption framework
\item Linux distributions like Ubuntu ask to encrypt home dir during installation
\item PGP disk - Pretty Good Privacy - makes a virtuel krypteret disk
\item TrueCrypt - similar to PGP disk, a virtual drive with data, cross platform
\emph{Let's audit Truecrypt!} Note: truecrypt halted and insecure? who knows?\\
{\small\link{http://blog.cryptographyengineering.com/2013/10/lets-audit-truecrypt.html}}
\end{list2}
\end{list1}


\slide{Theft - kindergarten and airports}

\begin{list1}
\item Many parents are in a hurry when they are picking up their kids
\item Many people can easily be distracted around crowds
\item Many people let their laptops stay out in the open - even at conferences
\item ... making theft likely/easy
\vskip 1 cm
\item Stolen for the value of the hardware - or for the data?
\item Industrial espionage, economic espionage or corporate espionage is real
\end{list1}

\centerline{Security breaches happens any day of the week}


\slide{Offline Backup}

\vskip 3cm
\centerline{\LARGE \bf Please do backups, and restore testing!}

\begin{list2}
\item Write a backup to DVD - most laptops today can do that
\item Save stuff in the cloud, examples Dropbox, Google Drive
\item Save data to external harddrive, cheap today
\item Use real SAN and RAID devices - mirrored disks
\end{list2}

\centerline{What happens if you loose data?}

\vskip 1cm
Sad story Mat Honan epic hacking :-(\\ {\small\link{http://www.wired.com/gadgetlab/2012/08/apple-amazon-mat-honan-hacking/all/}}


\slide{How can we protect? Generic advice}

Recommendations \hlkrightimage{8cm}{Encrypt_all_the_things.png}
\begin{list2}
\item Lock your devices, phones, tables and computers
\item Update software and apps
\item Do NOT use the same password everywhere
\item Watch out when using open wifi-networks
\item Multiple browsers: one for Facebook, and separate for home banking apps?
\item Multiple laptops? One for private data, one for work?
\item Think of the data you produce, why do people take naked pictures and SnapChat them?
\item Use pseudonyms and aliases, do not use your real name everywhere
\item Enable encryption: IMAP{\bf S}, POP3{\bf S},
  HTTP{\bf S} \\
\end{list2}


\slide{Multiple browsers}

\hlkimage{20cm}{multi-browser-strategy.png}

\begin{list2}
\item Strict Security settings in the general browser, Firefox or Chrome?
\item More lax security settings for "trusted sites" - like home banking
\item Security plugins like HTTPS Everywhere and NoScripts for generic browsing
\end{list2}

\slide{HTTPS Everywhere}

\hlkimage{5cm}{HTTPS_Everywhere_new_logo.jpg}
\begin{quote}
HTTPS Everywhere is a Firefox extension produced as a collaboration between The Tor Project and the Electronic Frontier Foundation. It encrypts your communications with a number of major websites.
\end{quote}

\centerline{\link{http://www.eff.org/https-everywhere}}



\slide{Secure your mobile}

\hlkimage{20cm}{the-guardian-project.pdf}

\centerline{Dont forget your mobile platforms \link{https://guardianproject.info/}}


\slide{Balanced security}

\hlkimage{21cm}{afbalanceret-sikkerhed.pdf}

\begin{list1}
\item Better to have the same level of security
\item If you have bad security in some part - guess where attackers will end up
\item Hackers are not required to take the hardest path into the network
\item Realize there is no such thing as 100\% security 
\end{list1}


\slide{Surveillance Self-Defense EFF}

\hlkimage{10cm}{ssd-eff-logo.png}

\begin{quote}
\centerline{Tips, Tools and How-tos For Safer Online Communications}

Modern technology has given the powerful new abilities to eavesdrop and collect data on innocent people. Surveillance Self-Defense is EFF's guide to defending yourself and your friends from surveillance by using secure technology and developing careful practices.
\end{quote}

Source: \link{https://ssd.eff.org/}


\slide{Security Sources}

\hlkimage{12cm}{twitter-security-feed.png}

\begin{list1}
\item Twitter has replaced RSS for me
\item Email lists are still a good source of data
\item Favourite Security Diary from Internet Storm Center\\
 \link{http://isc.sans.edu/index.html}\\
\link{https://isc.sans.edu/diaryarchive.html?year=2013&month=4}
\end{list1}


\myquestionspage


\end{document}
