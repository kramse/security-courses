\documentclass[Screen16to9,17pt]{foils}
%\documentclass[16pt,landscape,a4paper,footrule]{foils}
\usepackage{zencurity-slides}

% Sårbarhedsmitigering, værktøjer og processer
% Fokus på begreberne indenfor sårbarheder og på den længerevarende beskyttelse af organisationen

% Der er store problemer med sårbarheder i de fleste organisationer. Vores software og infrastruktur er bygget på sand, både reelt i form af computere, men også i overført betydning at sikkerheden skrider væk under os.

% Hver måned kommer der hundredevis af nye sårbarheder, og selvom vi ikke har al den sårbare software installeret, er det en næsten uoverkommelig opgave at holde miljøerne sikre imod indbrud og skade.

% I dette foredrag vil vi tage udgangspunkt i begreberne indenfor sårbarheder, mitigering, beskyttelse med fokus på længerevarende beskyttelse af organisationen.

% Der vil være elementer fra:
% * Sårbarheder, hvad er de, og hvad er exploits, der udnytter dem
% * Patch management, herunder asset discovery and management
% * Software security
% * Long term mitigation and protection
% * Security Metrics - hvordan måler vi, om det går fremad
% * Hvordan står vi bedst klar til hændelseshåndtering / incident response
% * Security frameworks for the organisation
%
% * Forudsætninger
% * Interesse for sikring af netværk, infrastruktur og organisationer imod sikkerhedsbrud som følge af sårbarheder.


\begin{document}
%\rm
\selectlanguage{english}

\mytitlepage
{Vulnerability Management}{Tools and Processes}

\hlkprofiluk



\slide{Goals for today}

\begin{quote}
  “A goal without a plan is just a wish.”\\
  ― Antoine de Saint-Exupéry
\end{quote}


\hlkimage{5cm}{pawel-janiak-dxFi8Ea670E-unsplash.jpg}


\begin{list1}
\item What are vulnerabilities and exploits
\item Patch management -- asset discovery and management
\item Long term mitigation and protection -- how can we achieve that
\hskip 2cm {\footnotesize Photo by Pawel Janiak on Unsplash}
\end{list1}



\slide{Malware and Worms}

\begin{list1}
\item {\bf Definition 23-1} \emph{Malicious logic}, more commonly called \emph{malware}, is a set\\
 of instructions that cause a site's security policy to be violated
 \item {\bf Definition 23-4} A \emph{computer virus} is a program that inserts (a possibly transformed version of) itself into one or more files and then performs some (possibly null) action.
\item {\bf Definition 23-2} A \emph{Trojan horse} is a program with an overt (documented or\\
known) purpose and a covert (undocumented or unexpected) purpose
\item {\bf Definition 23-14} A \emph{computer worm} is a program that copies itself from one computer to another. Computer worms has existed since research began mid-1970s
\end{list1}
Source: \emph{Computer Security: Art and Science}, 2nd edition 2019! Matt Bishop ISBN: 9780321712332\\
Virus, trojan or worm? Unless you work specifically in the computer virus industry, call it all malware


\slide{Vulnerability Analysis -- Trinity breaking in}

\hlkimage{12cm}{trinity-nmapscreen-hd-cropscale-418x250.jpg}
Very realistic: \link{https://nmap.org/movies/} and \link{https://youtu.be/51lGCTgqE_w}

\begin{list2}
\item \emph{Vulnerability} or security \emph{flaw} -- exploiting the vulnerability happens by an attacker
\item A program or script used for this is called an \emph{exploit}
\end{list2}

\slide{The Wikipedia definition}

%\hlkimage{}{}

\begin{quote}
Vulnerabilities are {\bf flaws} in a computer system that weaken the overall security of the system.

{\bf Despite intentions to achieve complete correctness, virtually all hardware and software contains bugs where the system does not behave as expected.} If the bug could enable an attacker to compromise the confidentiality, integrity, or availability of system resources, it is called a vulnerability. Insecure software development practices as well as design factors such as complexity can increase the burden of vulnerabilities. There are different types most common in different components such as hardware, operating systems, and applications.

{\bf Vulnerability management} is a process that includes identifying systems and prioritizing which are most important, scanning for vulnerabilities, and taking action to secure the system. Vulnerability management typically is a combination of remediation (fixing the vulnerability), mitigation (increasing the difficulty or reducing the danger of exploits), and accepting risks that are not economical or practical to eliminate.
\end{quote}
Source: \url{https://en.wikipedia.org/wiki/Vulnerability_(computer_security)}

Included this specifically because I agree \emph{virtually all hardware and software contains bugs}

\slide{What is an exploit?}

\begin{minted}[fontsize=\footnotesize]{ruby}
    sploit = {
      nil                   => /220.*Sendmail/,
      'DEBUG'               => /200 Debug set/,
      "MAIL FROM:<#{from}>" => /250.*Sender ok/,
      "RCPT TO:<#{to}>"     => /250.*Recipient ok/,
      'DATA'                => /354 Enter mail.*itself/,
      # Indent PATH= so it's not interpreted as a mail header
      " PATH=#{path}"       => nil,
      'export PATH'         => nil,
      payload.encoded       => nil,
      '.'                   => /250 Ok/,
      'QUIT'                => /221.*closing connection/
    }
\end{minted}
Source: {\footnotesize\url{https://github.com/rapid7/metasploit-framework/blob/master/modules/exploits/unix/smtp/morris_sendmail_debug.rb}}

\begin{list2}
    \item Command injection through debugging \emph{feature}
\end{list2}


\slide{The Internet Worm 2. nov 1988}

\begin{list1}
\item Exploited the following vulnerabilities
\begin{list2}
\item buffer overflow in fingerd - VAX code
\item Sendmail - DEBUG functionality
\item Trust between systems: rsh, rexec, ...
\item Bad passwords
\end{list2}
\item Contained camouflage!
\begin{list2}
\item Program name set to 'sh'
\item Used fork() to switch PID regularly
\item Password cracking using intern list of 432 words and /usr/dict/words
\item Found systems to infect in /etc/hosts.equiv, .rhosts, .forward, netstat ...
\end{list2}
\item Made byRobert T. Morris, Jr.
\end{list1}


\slide{Many Years ago around 1988 }

\begin{minted}[fontsize=\footnotesize]{c}
/usr/src/etc/fingerd.c from 4.3BSD:
main(argc, argv)
        char *argv[];
{
        register char *sp;
        char line[512];  // This is a fixed size buffer
        struct sockaddr_in sin;
...
        line[0] = '\0';
        gets(line);      // This line can overflow the buffer, buffer overflow vulnerability
\end{minted}

Source code link \url{https://www.tuhs.org/cgi-bin/utree.pl?file=4.3BSD/usr/src/etc/fingerd.c}

More description in the articles:\\
{\footnotesize\url{https://spaf.cerias.purdue.edu/tech-reps/823.pdf}} \emph{The Internet Worm Program: An Analysis}
Purdue Technical Report CSD-TR-823
Eugene H. Spafford\\ {\footnotesize\url{https://blog.rapid7.com/2019/01/02/the-ghost-of-exploits-past-a-deep-dive-into-the-morris-worm/}}\\ The Ghost of Exploits Past: A Deep Dive into the Morris Worm



\slide{Stuxnet}

\begin{list1}
\item Worm in 2010 intended to infect Iran nuclear program
\item Target was the uranium enrichment process
\item Infected other industrial sites
\item SCADA, and Industrial Control Systems (ICS) are becoming very important for whole countries
\item A small \emph{community} of consultants work in these \emph{isolated} networks, but can be used as infection vector - they visit multiple sites
\item More can be found in \url{https://en.wikipedia.org/wiki/Stuxnet}
\end{list1}

\centerline{Exploits are worth millions! Bug bounty is a concept, developing and selling exploits}


\slide{Reality Hits -- every month}

\hlkimage{12cm}{microsoft-patch-tuesday-oct-2024.png}

\begin{quote}
Microsoft addresses {\bf 117 CVEs} with three rated as critical and four zero-day vulnerabilities, two of which were {\bf exploited in the wild}.
\end{quote}
Source: \url{https://www.tenable.com/blog/microsoft-october-2024-patch-tuesday-addresses-117-cves-cve-2024-43572-cve-2024-43573}, originally from Microsoft October 2024 Security Updates \url{https://msrc.microsoft.com/update-guide/releaseNote/2024-Oct}

\slide{Vulnerabilities - CVE}

\begin{list1}
\item Common Vulnerabilities and Exposures (CVE):
  \begin{list2}
  \item classification
  \item identification
  \end{list2}
\item When discovered each vuln gets a CVE ID
\item CVE maintained by MITRE - not-for-profit
org for research and development in the USA.
\item National Vulnerability Database search for CVE.
\item Sources: \link{http://cve.mitre.org/} og \link{http://nvd.nist.gov}
\item also checkout OWASP Top-10 \link{http://www.owasp.org/}
\end{list1}

\slide{Vulnerabilities are everywhere!}

\hlkimage{18cm}{cve-details-new-updated.png}
Source: CVEdetails.com on 2024-09-02

\begin{list2}
\item This is crazy! \url{https://www.cvedetails.com/}
\end{list2}

\slide{Vulnerabilities by type \& year}

\hlkimage{17cm}{cve-details-year.png}
Source: CVEdetails.com on 2024-09-02 Graph on the web site is interactive \url{https://www.cvedetails.com/}

\slide{LG TVs 2024 -- CVE-2023-6317 up to CVE-2023-6320}

\hlkimage{10cm}{LG-shodan.png}

\begin{quote}{\large\bf
90,000+ LG TVs Vulnerable to Authorization Attacks\\
Due to WebOS Vulnerabilities}

Bitdefender Labs has revealed a critical security flaw in over 90,000 LG smart TVs running the company’s proprietary WebOS platform.

If exploited, the vulnerability could allow attackers to gain unauthorized access to the TV’s functions and potentially the user’s home network.

\end{quote}
Source: \url{https://cybersecuritynews.com/lg-tvs-vuauthorization-attacks/}



\slide{Sample vulnerabilities}

\begin{list1}
\item \small CVE-2000-0884\\
IIS 4.0 and 5.0 allows remote attackers to read documents outside of
the web root, and possibly execute arbitrary commands, via malformed
URLs that contain UNICODE encoded characters, aka the "Web Server
Folder Traversal" vulnerability.

\item \small CVE-2002-1182\\
IIS 5.0 and 5.1 allows remote attackers to cause a denial of service
(crash) via malformed WebDAV requests that cause a large amount of
memory to be assigned.

\item Source:\\
\link{http://cve.mitre.org/ - CVE}
\end{list1}

\centerline{And updates from vendors reference these too! A closed loop}

\slide{CWE Common Weakness Enumeration}

\hlkimage{18cm}{cwe-mitre-org.png}
\link{http://cwe.mitre.org/}

\slide{CWE/SANS Monster mitigations}

\hlkimage{13cm}{cwe-monster-mitigations.png}

Source:
\link{http://cwe.mitre.org/top25/index.html}



\slide{Reflecting on the Internet Worm at 35}

%\hlkimage{}{}

\begin{quote}
Thirty-five years ago today (November 2nd), the Internet Worm program was set loose to propagate on the Internet.
...
 All of that eventually led to a boom in add-on security measures, resulting in what is now a {\bf multi-billion dollar cybersecurity industry.}
...
The Worm provided us with an object lesson about many issues that, unfortunately, were not heeded in full to this day. {\bf That multi-billion dollar cybersecurity industry is still failing to protect far too many of our systems.}
\end{quote}
Source: \emph{Reflecting on the Internet Worm at 35}, November 02, 2023 by spaf, Eugene Spafford\\
\url{https://www.cerias.purdue.edu/site/blog/post/reflecting_on_the_internet_worm_at_35/}

\begin{list2}
\item Many of the same problems that plagued us earlier are still the same!
\item My thoughts are we should try something else than \emph{patch insanity}
\end{list2}


\slide{Case in point: SQL injection affecting airport security}

%\hlkimage{}{}

\begin{quote}
{\large Bypassing airport security via SQL injection}\\
08/29/2024\\
Like many, Sam Curry and I spend a lot of time waiting in airport security lines. If you do this enough, you might sometimes see a special lane at airport security called Known Crewmember (KCM). KCM is a TSA program that allows pilots and flight attendants to bypass security screening, even when flying on domestic personal trips.

The KCM process is fairly simple: the employee uses the dedicated lane and presents their KCM barcode or provides the TSA agent their employee number and airline. Various forms of ID need to be presented while the TSA agent’s laptop verifies the employment status with the airline. If successful, the employee can access the sterile area without any screening at all.
\end{quote}
Source: \url{https://ian.sh/tsa}

\begin{list2}
\item And SQL injection is one of the \emph{easiest} software security problems to guard against!
\end{list2}



\slide{Protection, building secure and robust networks}

\hlkimage{10cm}{sample-ip-network.pdf}


\begin{list2}
\item We should prefer security mechanisms that does NOT require us to keep patching every month
\item Can we change our infrastructure and networks to avoid this? {\bf Yes!}
\item Reduce complexity -- note adding VLANs may seem to increase, but also reduce number of systems that can interact, and each \emph{network} is afterwards easier to understand
\item Limit the attack surface -- fewer systems exposed \faArrowRight fewer vulns exposed, fewer services \faArrowRight  less code!
\end{list2}


\slide{Defense in depth}

%\hlkimage{10cm}{Bartizan.png}
\hlkimage{15cm}{medieval-clipart-5}
\centerline{Picture originally from: \url{http://karenswhimsy.com/public-domain-images}}




\slide{Goals of Security -- short version}

\hlkimage{12cm}{OODA.Boyd.png}
{\footnotesize Source: Patrick Edwin Moran - Wikipedia \link{https://en.wikipedia.org/wiki/OODA_loop}}

\begin{list2}
\item Prevention - means that an attack will fail
\item Detection - determine if attack is underway, or has occured - report it
\item Recovery - stop attack, assess damage, repair damage
\end{list2}

\slide{OSI Model and Internet Protocols}

\hlkimage{9cm,angle=90}{images/compare-osi-ip.pdf}

\centerline{I recommend securing things from the bottom and from the outside}

\slide{\faWrench\ Books and courses}

%\hlkimage{}{}

How:

I like to learn new concepts from books
\begin{list2}
\item Have a clear structure, less confusion
\item They go from a basic level towards a complete goal
\item Often have exercises available with nice progression
\item Lots of nice books available from \link{http://www.nostarch.com/} and others
\item Often you can get Humble bundles with many books for \$25
\item Some books are "free" if you give your email address, example
\item Can function as inspiration and a checklist
\end{list2}

Pro Tip: all my courses and exercise booklets are available on Github!

Humble Bundle! \url{https://www.humblebundle.com/}



\slide{Book: Defensive Security Handbook (DSH)}

\hlkimage{6cm}{defensive-security-handbook.jpg}

\emph{Defensive Security Handbook: Best Practices for Securing Infrastructure}, Lee Brotherston, Amanda Berlin, William F. Reyor ISBN: 9781098127237, 362 pages -- Note: 2nd edition updated 2024\\
{\footnotesize\link{https://learning.oreilly.com/library/view/defensive-security-handbook/9781098127237/}}

\slide{Network Security Through Data Analysis}

\hlkimage{6cm}{network-security-through-data-analysis.png}


\emph{Network Security through Data Analysis }, Michael S Collins, 2nd Edition, 2017\\
{\footnotesize\url{https://learning.oreilly.com/library/view/network-security-through/9781491962831/}}


\slide{\faWrench\ Recommended tools to learn -- DevSecOps}

\hlkimage{4cm}{005scawebiaidosezaicon.png}

\begin{list2}
\item Open Source I really love open source. There is just too much great open source software, to ignore
\item Linux/Unix knowledge is necessary
-- because a lot of the newest tools are written for Linux/Unix/BSD
\item Git and Github -- where you can find lots of tools, libraries, applications
\item Programming experience is an advantage for automating stuff\\
Python is a nice generic tool for this, PowerShell is another alternative
\item Ansible provisioning -- installing and configuring software for production
\item Elasticsearch -- how to run a \emph{service}, full fledged applications exist for Elasticsearch
\item OpenSSH -- included in Linux and Windows, allows for Rsync, Git, port forward etc.
\end{list2}

\slide{\faWrench\ Learning DevSecOps: A Practical Guide to Processes and Tools}

\hlkrightpic{5cm}{-5cm}{learning-devsecops.jpg}
\hlkimage{18cm}{secdevops-iterations.png}

\emph{Learning DevSecOps}, Steve Suehring
Released May 2024, O'Reilly, ISBN: 9781098144869

\slide{Thursday What a concept}

\hlkimage{10cm}{thursday.jpg}

\begin{quote}
Alt text: Nadia from the Russian Doll fearing that she would never see a Thursday again says, "Thursday. What a concept" while smoking a cigarette
\end{quote}

\begin{list2}
\item Mastodon bot \url{https://botsin.space/@thursday} post the same picture each thursday
\item Github Actions Workflows, se \url{https://github.com/devashishp/thursday} and \url{https://docs.github.com/en/actions/writing-workflows}

\end{list2}

\slide{\faWrench\ Github Actions Workflows }

\begin{minted}[fontsize=\small]{yaml}
jobs:
  deploy:
    runs-on: ubuntu-latest
    steps:
    - uses: actions/checkout@v3
    - name: Set up Python 3.10
      uses: actions/setup-python@v3
      with:
        python-version: "3.10"
    - name: Install dependencies
      run: |
        python -m pip install --upgrade pip
        pip install Mastodon.py
        if [ -f requirements.txt ]; then pip install -r requirements.txt; fi
    - name: Run Script
      run:
        python bot.py
\end{minted}

\slide{\faWrench\ Open Source and Python}
\hlkimage{7cm}{maltrail.png}

\begin{list2}
\item Open Source is already written *doh*
\item Can provide solutions or parts of a solution
\item Often feature-rich, mature, tested, maintained, and even when \emph{not} can be brought back to life
\item Picture from Maltrail \link{https://github.com/stamparm/maltrail}\\
Maltrail is a malicious traffic detection system, utilizing publicly available (black)lists containing malicious and/or generally suspicious trails, along with static trails compiled from various AV reports and custom user defined lists,
\end{list2}




\slide{\faWrench\ Ansible Configuration management and more!}

%\hlkimage{}{}

Platform options Ansible:
\begin{alltt}
CloudEngine OS, CNOS, Dell OS6, Dell OS9 Dell OS10, ENOS, EOS, ERIC_ECCLI, EXOS,
FRR, ICX, IOS, IOS-XR, IronWare, Junos OS, Meraki, Pluribus NETVISOR, NOS, NXOS,
RouterOS, SLX-OS, VOSS, VyOS, WeOS 4

plus routers based on Linux, OpenBSD, FreeBSD etc.
\end{alltt}

One management system with many uses, free to download and use
\begin{list2}
\item Generic configuration management -- and you end up running support systems, network near systems
\item Ansible for Network Automation\\
\link{https://docs.ansible.com/ansible/latest/network/index.html}
\item Allows you to install, configure and run your infrastructure
\item Depends on Python and SSH, or module for the network devices
\end{list2}

\slide{Python and YAML}

\hlkimage{7cm}{git-logo.png}

\begin{list2}
\item We need to store configurations of devices and systems
\item Run Ansible playbooks
\item Problem: Remember what we did, when, how
\item Solution: use git for the playbooks
\item Not the only version control system, but my preferred one
\item Git can also be used by Oxidized which I also love \link{https://github.com/ytti/oxidized}
\end{list2}



\slide{\faWrench\ Wazuh}

\hlkimage{8cm}{01-Wazuh-Security-Analytics-op.png}

\begin{quote}\small
Wazuh agents scan the monitored systems looking for malware, rootkits and suspicious anomalies. They can detect hidden files, cloaked processes or unregistered network listeners, as well as inco
nsistencies in system call responses.\\
Source: text and picture from \link{https://wazuh.com/}
\end{quote}

\begin{list2}
\item Wazuh initially a fork of the OSSEC project, and has integration with Elastic Stack
\end{list2}


\slide{Wazuh agent}

\begin{quote}\small
The Wazuh lightweight agent is designed to perform a number of tasks with the objective of detecting threats and, when necessary, trigger automatic responses. The agent core capabilities are:

The Wazuh agents run on many different platforms, including Windows, Linux, Mac OS X, AIX, Solaris and HP-UX. They can be configured and managed from the Wazuh server.\\
Source: \link{https://wazuh.com/}
\end{quote}

\begin{list2}
\item Log and events data collection
\item File and registry keys integrity monitoring
\item Inventory of running processes and installed applications
\item Monitoring of open ports and network configuration
\item Detection of rootkits or malware artifacts
\item Configuration assessment and policy monitoring
\item Execution of active responses
\end{list2}


\slide{\faWrench\ Isolation and Network Segmentation -- Virtual LAN (VLAN)}

\hlkimage{8cm}{vlan-portbased.pdf}

\begin{list1}
\item Managed switches often allow splitting into zones called virtual LANs
\item Most simple version is port based
\item Like putting ports 1-4 into one LAN and remaining in another LAN
\item Packets must traverse a router or firewall to cross between VLANs
\end{list1}

\slide{Virtual LAN (VLAN) IEEE 802.1q}

\hlkimage{15cm}{vlan-8021q.pdf}

\begin{list1}
\item Using IEEE 802.1q  VLAN tagging on Ethernet frames
\item Virtual LAN, to pass from one to another, must use a router/firewall
\item Allows separation/segmentation and protects traffic from many security issues
\item Used in most, if not all, Wi-Fi networks -- each SSID has a VLAN behind it
\end{list1}


\slide{Who are you gonna call?}

%\hlkimage{}{}

\begin{quote}
Cyberangreb kan blive en dyr omgang for SMV’erne
Et ransomware angreb koster 376.350 kr. alene i tabt omsætning fra e-handel for en virksomhed med 10-49 ansatte. I lyset af at truslen for cyberkriminalitet er på sit højeste, skal flere SMV’er have hjælp til at øge deres IT-sikkerhed. Særligt efter en hård tid under COVID-19, som har tvunget virksomhedernes fokus væk fra IT-sikkerhed.
\end{quote}
Source: SMVdanmark Marts 2022 \url{https://smvdanmark.dk/analyser/temaanalyser/cyberangreb-kan-blive-en-dyr-omgang-for-smverne}

\begin{list2}
\item You need friends!

\item Incident Response is a specialized area

\item They cost upwards of 1.500DKK / hour -- more if outside of business hours
\item Pre-arranged is recommended, agree on \emph{who can call them}, decide up front when to call them -- not for every little incident
\item Expect an incident to cost at least 100.000DKK plus time, lost hours, lost orders, etc.
\end{list2}


\slide{\faWrench\ LibreNMS Automatic discovery -- inventory management}

\hlkimage{10cm}{librenms-switches.png}

Automatically discover your entire network using CDP, FDP, LLDP, OSPF, BGP, SNMP and ARP \\
See all the versions, what do you have, what needs to be secure \url{https://www.librenms.org/}


\slide{Analysis on Docker Hub malicious images: Attacks through public container images}

\hlkimage{10cm}{sysdig-malicious-images.png}
This article is relevant, talking about malicious docker images\\
\link{https://sysdig.com/blog/analysis-of-supply-chain-attacks-through-public-docker-images/}

\slide{Keeping Container Images secure}

\begin{list2}
\item \faWrench\ \verb+anchore+ open-source project that provides a centralized service for inspection, analysis, and certification of container images
\link{https://github.com/anchore/anchore-engine}\\
"As of 2023, Anchore Engine is no longer maintained. There will be no future versions released. Users are advised to use Syft and Grype."
\item \faWrench\ \verb+Syft+ \link{https://github.com/anchore/syft} and \faWrench\ \verb+Grype+ \link{https://github.com/anchore/grype}
\item Allow direct download from the internet into your cluster, may become a problem
\item Malicious people are typosquatting popular containers!
\item Suply chain attacks in general are a problem
\end{list2}

\slide{Harden Container Images and Update Your Procedures}

\begin{list2}
\item Change the goddamn passwords!\\
Container postgresql with user postgres and password *postgres*, REALLY!!!!!!1111
\item and NO MORE ROOT! Dont run as root, we realized this was bad in the 1990s!
\item CIS Docker Benchmarking also Learn Kubernetes Security \emph{Chapter 8: Securing Kubernetes Pods}
\item Hacking Kubernetes \emph{Chapter 8: Policy} - describe things like Resource Quotas, Runtime Policies
\end{list2}

\slide{Recommendations from CIS Docker Benchmark}

\hlkimage{14cm}{cis-docker-1.png}
Summary from: \link{https://www.aquasec.com/cloud-native-academy/docker-container/docker-cis-benchmark/}

\begin{list2}
\item Latest version: CIS Docker Benchmark v1.5.0 - 12-28-2022
\end{list2}


\slide{Benchmarking tools}

\begin{quote}
\faWrench\ \verb+Kube-bench+ is the industry-standard tool to automate checking Kubernetes compliance with the Center for Internet Security (CIS) Benchmark.

Kube-bench makes it easy for operators to check whether each node in their Kubernetes cluster is configured according to security best practices.
\end{quote}
Source: \link{https://info.aquasec.com/open-source}

\begin{list2}
\item CIS Kubernetes V1.24 Benchmark v1.0.0 - 09-21-2022 -- other versions exist
\item CIS Docker Benchmark v1.5.0 - 12-28-2022
\end{list2}

%\slide{Benchmarking and keeping up to date}

% Try it out \faWrench\ \verb+kube-bench+ - make some bad changes, like LKS p95 allow anonymous authentication, show why layered defense works, and how kube-bench marks it. Another example token based access, initially we have token while installing, remember to remove!
% \faWrench\ \verb+kube-hunter+ and kube-bench

\slide{Tool example kube-bench}

%\hlkimage{}{}

\begin{alltt}\scriptsize
hlk@timon:~/bin/kube-bench/kube-bench$ {\bf kubectl logs kube-bench-gdf62}
[PASS] 1.1.7 Ensure that the etcd pod specification file permissions are set to 600 or more restrictive (Automated)
[PASS] 1.1.8 Ensure that the etcd pod specification file ownership is set to root:root (Automated)
[WARN] 1.1.9 Ensure that the Container Network Interface file permissions are set to 600 or more restrictive (Manual)
[WARN] 1.1.10 Ensure that the Container Network Interface file ownership is set to root:root (Manual)
[PASS] 1.1.11 Ensure that the etcd data directory permissions are set to 700 or more restrictive (Automated)
[FAIL] 1.1.12 Ensure that the etcd data directory ownership is set to etcd:etcd (Automated)
...
== Summary policies ==
0 checks PASS
0 checks FAIL
35 checks WARN
0 checks INFO

== Summary total ==
63 checks PASS
10 checks FAIL
58 checks WARN
0 checks INFO

\end{alltt}

\begin{list2}
\item \url{https://github.com/aquasecurity/kube-bench} also check out Lynis \url{https://cisofy.com/lynis/}
\end{list2}



\myquestionspage



\end{document}
