\documentclass[Screen16to9,17pt]{foils}
\usepackage{zencurity-slides}



% Workshop 3med fokus på cybersikkerhed i virksomheder

% Formål:Du kender sikkert problemet, dit password er mistet -igen. Men hvad sker der egentlig på IT-sikkerhedsområdet.På denne workshop arbejder vi med IT-sikkerhed struktureret ved at introducere et framework CIS kontroller og relatere dem til konkrete sikkerhedshændelser som ses i danske og internationale virksomheder.

% Indhold på dagen:På  dagen  vil  vi  gennemgå  området  startende  på  højt  niveau,  og  hurtigt  dykke  ned  i  konkrete metoder og processer for at sikre organisationer. Grundlaget er længerevarende anbefalinger til hvordan man sikrer sig selv og sin organisation mod sikkerhedshændelser.Workshoppen vil indeholde referencer til aktuelle sager i indog udland.

% Aktiviteter:Der vil være indlagt øvelser hvor de studerende skal vurdere egen sikkerhed på konkrete punkter, samt  relatere  disse  til  cyber  security  frameworks.  De  studerende  vil  således  arbejde  med interview af hinanden, samt vurdere hvilke tekniske sikkerhedsforanstaltninger de bør tage i brug i scenarier.

% Læringsmål:•Lære og genkende it-sikkerhedstrusler•Arbejde struktureret med sikkerhed, samt forstå fordele og forskellen til ad-hoc sikkerhed

\begin{document}
\selectlanguage{english}
\mytitlepage{Workshop 3: cybersikkerhed i virksomheder}
{Cyber Security CISO for a Day}

ATU hos KEA Maj, 2022

\slide{Time schedule}

\begin{list1}
\item 14:00 - 16:00 including breaks
\vskip 1cm
\item 14:00 - 14:20 -- Introduction to Security
\item 14:20 - 14:40 -- Exercises 1 + 2
\item Summary part 1 and break
\item 15:00 - 15:20 -- Exercises 3 + 4
\item 15:20 - 15:40 -- Exercises 5 + 6
\item 15:45 Summary, conclusion, last questions
\end{list1}

All slides are in english, exercises in Danish!

{\bf Hint: Don't panic if the plan breaks!}

\slide{Introduction: Attack overview}

\hlkimage{20cm}{sicherheitstacho.png}
Source: \link{http://www.sicherheitstacho.eu/}


\slide{DDoS Attacks Still a Problem}

\hlkimage{13cm}{DDos_Attack_in_2021_arbor.png}

Security attacks and DDoS is very much in the media\\
Source: link{https://www.netscout.com/threatreport/global-ddos-attack-trends/}

\slide{DDoS Attacks are HUGE}


\hlkimage{13cm}{DDos_Attack_in_2021_size_arbor.png}

Extremely hard to protect against from a small network\\
Source: link{https://www.netscout.com/threatreport/global-ddos-attack-trends/}


\slide{Ransomware Attacks are Common}


\hlkimage{13cm}{ransomware_arbor.png}

Make sure to backup your data! Test your backups!\\
Source: link{https://www.netscout.com/threatreport/global-ddos-attack-trends/}

\slide{What can we do? -- Good security}

\hlkimage{12cm}{god-sikkerhed.pdf}

\begin{list1}
\item You always have limited resources for protection - use them as best as possible
\item Good security comes from structured work
\end{list1}


\slide{Balanced security}

\hlkimage{21cm}{afbalanceret-sikkerhed.pdf}

\begin{list1}
\item Better to have the same level of security
\item If you have bad security in some part - guess where attackers will end up
\item Hackers are not required to take the hardest path into the network
\item Realize there is no such thing as 100\% security
\end{list1}



\slide{Work together}

\hlkimage{10cm}{Shaking-hands_web.jpg}

\begin{list1}
\item Team up!
\item We need to share security information freely
\item We often face the same threats, so we can work on solving these together
\end{list1}

\slide{My daily job -- Security engineering a job role}

\begin{alltt}\footnotesize
On any given day, you may be challenged to:
        Create new ways to solve existing production security issues
        Configure and install firewalls and intrusion detection systems
        Perform vulnerability testing, risk analyses and security assessments
        Develop automation scripts to handle and track incidents
        Investigate intrusion incidents, conduct forensic investigations and incident responses
        Collaborate with colleagues on authentication, authorization and encryption solutions
        Evaluate new technologies and processes that enhance security capabilities
        Test security solutions using industry standard analysis criteria
        Deliver technical reports and formal papers on test findings
        Respond to information security issues during each stage of a project’s lifecycle
        Supervise changes in software, hardware, facilities, telecommunications and user needs
        Define, implement and maintain corporate security policies
        Analyze and advise on new security technologies and program conformance
        Recommend modifications in legal, technical and regulatory areas that affect IT security
\end{alltt}

Source: \url{https://www.cyberdegrees.org/jobs/security-engineer/}\\
also
\url{https://en.wikipedia.org/wiki/Security_engineering}




\slide{Risk management defined}

\hlkimage{20cm}{shon-harris-risk-management.png}

Source: Shon Harris \emph{CISSP All-in-One Exam Guide}



\slide{Security Controls and Frameworks}

\begin{list1}
\item Multiple exist
\vskip 1cm
\begin{list2}
\item CIS controls Center for Internet Security (CIS) \link{https://www.cisecurity.org}
\item PCI Best Practices for Maintaining PCI DSS Compliance v2.0 Jan 2019
\item NIST Cybersecurity Framework (CSF)\\
Framework for Improving
Critical Infrastructure Cybersecurity\\ \link{https://www.nist.gov/cyberframework}\\
\link{https://csrc.nist.gov/publications/sp800} - SP800 series
\item National Security Agency (NSA)\\
\link{https://www.nsa.gov/Research/}
\item NSA security configuration guides\\
\link{https://apps.nsa.gov/iaarchive/library/ia-guidance/security-configuration/}
\item Information Systems Audit and Control Association (ISACA)\\
\link{http://www.isaca.org/Knowledge-Center/}
\end{list2}
\end{list1}

\slide{Center for Internet Security CIS Controls}

\begin{quote}
  “A goal without a plan is just a wish.”\\
  ― Antoine de Saint-Exupéry
\end{quote}

\begin{quote}
  The CIS ControlsTM are a prioritized set of actions that collectively form a defense-in-depth set
of best practices that mitigate the most common attacks against systems and networks. The
CIS Controls are developed by a community of IT experts who apply their first-hand experience
as cyber defenders to create these globally accepted security best practices. The experts who
develop the CIS Controls come from a wide range of sectors including retail, manufacturing,
healthcare, education, government, defense, and others.
\end{quote}

Source: \link{https://www.cisecurity.org/} CIS-Controls-Version-7-1.pdf


\slide{Kom igang med CIS}

\begin{quote}
CIS-kontrollerne består af 20 praktiske, pragmatiske kontroller, som er målbare og med direkte henvisning til, hvordan de implementeres samt forslag til, hvilke KPI’er der bør opstilles for målinger.

Forskellen på CIS-kontrollerne og fx ISO27001 er, at du ikke kan blive certificeret efter CIS, men til gengæld opdateres CIS-kontrollerne løbende, og de indeholder prioriterede lister af, hvad du i praksis skal gøre for din cybersikkerhed. Det australske forsvar har fx vist, at hvis man implementerer de første fire kontroller fuldt ud, kan man mitigere op mod 90+\% af alt malware.
\end{quote}

Dansk artikel fra Deloitte, version 7 men version 8 er ude
\link{https://www2.deloitte.com/dk/da/pages/risk/articles/vi-stiller-skarpt-pa-cis-kontroller.html}


\slide{Exercises}

CIS controls 1-6 are Basic, everyone must do them. Today I have replaced 6 with 10.

\begin{list2}
\item {\bf CIS Control 1: Inventory and Control of Hardware Assets}
\item {\bf CIS Control 2: Inventory and Control of Software Assets}
\item {\bf CIS Control 3: Continuous Vulnerability Management}
\item {\bf CIS Control 4: Controlled Use of Administrative Privileges}
\item {\bf CIS Control 5:
Secure Configuration for Hardware and Software on Mobile Devices, Laptops, Workstations and Servers}
\item {\bf CIS Control 10: Data Recovery Capabilities}
\end{list2}

We will now start the exercises, two controls at a time!

First decide, if this was a real company, how would you implement these?

1,2,3,4,5 and then 10?

\slide{Inventory and Control of Hardware Assets}

\begin{quote}
CIS Control 1:\\
Inventory and Control of Hardware Assets\\
Actively manage (inventory, track, and correct) all hardware devices on the network so that only authorized devices are given access, and unauthorized and unmanaged devices are found and prevented from gaining access.
\end{quote}

\begin{list1}
\item What is connected to our networks?
\item What firmware do we need to install on hardware?
\item Where IS the hardware we own?
\item What hardware is still supported by vendor?
\end{list1}

Source: Center for Internet Security CIS Controls 7.1 CIS-Controls-Version-7-1.pdf


\slide{Inventory and Control of Software Assets}

\begin{quote}
CIS Control 2:\\
Inventory and Control of Software Assets\\
Actively manage (inventory, track, and correct) all software on the network so that only authorized software is installed and can execute, and that all unauthorized and unmanaged software is found and prevented from installation or execution.
\end{quote}

\begin{list1}
\item What licenses do we have? Paying too much?
\item What versions of software do we depend on?
\item What software needs to be phased out, upgraded?
\item What software do our employees need to support?
\end{list1}

Source: Center for Internet Security CIS Controls 7.1 CIS-Controls-Version-7-1.pdf

\slide{Øvelse 1 + 2}

%\hlkimage{}{}


\begin{list2}
\item Nu skal I \emph{implementere} CIS kontrollerne 1 og 2
\item Det foregår ved at I prøver at tænke på dem som om I var en CISO -- Chief Information Security Officer, IT-sikkerhedschef
\item Tænk på dem med jeres viden om IT, jeres egne IT-systemer
\end{list2}

\slide{Continuous Vulnerability Management}

\begin{quote}
CIS Control 3:\\
Continuous Vulnerability Management\\
Continuously acquire, assess, and take action on new information in order to identify vulnerabilities, remediate, and minimize the window of opportunity for attackers.
\end{quote}

\begin{list1}
\item Scan for updates automatically
\item Update when vendors publish critical patches
\item Listen to news sources about software and vulnerabilities
\end{list1}

Source: Center for Internet Security CIS Controls 7.1 CIS-Controls-Version-7-1.pdf



\slide{Controlled Use of Administrative Privileges}

\begin{quote}
CIS Control 4:\\
Controlled Use of Administrative Privileges\\
The processes and tools used to track/control/prevent/correct the use, assignment, and configuration of administrative privileges on computers, networks, and applications.
\end{quote}

\begin{list1}
\item Remove local administrator from Windows workstations
\item Change default passwords
\item Use good passwords
\item Log if somebody tries to break in
\end{list1}

Source: Center for Internet Security CIS Controls 7.1 CIS-Controls-Version-7-1.pdf


\slide{Øvelse 3 + 4}

%\hlkimage{}{}


\begin{list2}
\item Nu skal I \emph{implementere} CIS kontrollerne 3 og 4
\item Det foregår ved at I prøver at tænke på dem som om I var en CISO -- Chief Information Security Officer, IT-sikkerhedschef
\item Tænk på dem med jeres viden om IT, jeres egne IT-systemer
\end{list2}



\slide{Secure Configuration for Hardware and Software}

\begin{quote}
CIS Control 5:\\
Secure Configuration for Hardware and Software on Mobile Devices, Laptops, Workstations and Servers\\
Establish, implement, and actively manage (track, report on, correct) the security configuration of mobile devices, laptops, servers, and workstations using a rigorous configuration management and change control process in order to prevent attackers from exploiting vulnerable services and settings.
\end{quote}

\begin{list1}
\item Create secure configuration -- check security settings
\item Select security mechanisms
\item Automate security settings
\end{list1}

Source: Center for Internet Security CIS Controls 7.1 CIS-Controls-Version-7-1.pdf

\slide{Data Recovery Capabilities}

\begin{quote}
CIS Control 10:\\
Data Recovery Capabilities\\
The processes and tools used to properly back up critical information with a proven methodology
for timely recovery of it
\end{quote}

\begin{list1}
\item Backup is critical
\item If we loose orders we loose money
\item Data loss, means production capacity loss
\item Separation of duty -- can one person delete both production and backup
\end{list1}

Source: Center for Internet Security CIS Controls 7.1 CIS-Controls-Version-7-1.pdf


\slide{Øvelse 5 + 10}

%\hlkimage{}{}


\begin{list2}
\item Nu skal I \emph{implementere} CIS kontrollerne 5 og 10\\
Bemærk ikke 5 og 6, nummer 10 er nemmere og vigtigere for jer
\item Det foregår ved at I prøver at tænke på dem som om I var en CISO -- Chief Information Security Officer, IT-sikkerhedschef
\item Tænk på dem med jeres viden om IT, jeres egne IT-systemer
\end{list2}



% Workshop
% CIS controls

% Workshop
% Select and implement security
% Create a security policy
% Every 10minutes interrupt with "vuln attack"

% Critical Vulnerabilities 2021
% Log4j
%


\slide{Hændelseslog og Økonomi}

Tag et stykke papir eller en computer
\begin{list2}
\item Vi er blevet afbrudt i vores vigtige arbejde med CIS kontroller
\item Vi skal udfylde en Hændelseslog og der er nogle økonomiske aspekter
\item Når der sker en sikkerhedshændelse skal den helst håndteres effektivt
\item Hvis man ikke har sikkerhedsprocedurer på plads bliver det typisk længerevarende og dyrere
\end{list2}

\slide{March 2021: ProxyLogon/ProxyShell CVE-2021-26855 CVSS:3.0 9.1 / 8.4}
\begin{quote}
In March 2021, both Microsoft and IT Professionals had a major headache in the form of an Exchange zero-day commonly known as ProxyLogon. The vulnerability, widely considered the {\bf most critical to ever hit Microsoft Exchange}, was quickly exploited in the wild by suspected state-sponsored threat actors, with US government and military systems identified as the most targeted sectors. {\bf Ransomware variants such as DoejoCrypt were soon actively exploiting unpatched Exchange instances}, attempting to monetise the vulnerability.

A follow-up exploit, dubbed ProxyShell, was evolutionary in nature and targeted on-premise Client Access Servers (CAS) in {\bf all supported versions of Exchange Server.} Due to the {\bf remotely accessible nature of Exchange CAS, any unpatched instances would be vulnerable to Remote Code Execution. High profile victims included the European Banking Authority and the Norwegian Parliament.}
\end{quote}
Source - for this description:\\
\link{https://chessict.co.uk/resources/blog/posts/2022/january/2021-top-security-vulnerabilities/}


\slide{ProxyLogon CVE-2021-26855 CVSS:3.0 9.1 / 8.4}

\begin{quote}
ProxyLogon is the formally generic name for CVE-2021-26855, a vulnerability on Microsoft Exchange Server that allows an attacker bypassing the authentication and impersonating as the admin. We have also chained this bug with another post-auth arbitrary-file-write vulnerability, CVE-2021-27065, to get code execution. All affected components are vulnerable by default!

As a result, {\bf an unauthenticated attacker can execute arbitrary commands on Microsoft Exchange Server through an only opened 443 port!}
\end{quote}

Sources: \link{https://proxylogon.com/}\\
\link{https://msrc.microsoft.com/update-guide/vulnerability/CVE-2021-26855}



\slide{Incident Handling: ProxyLogon}

Hvis jeres gruppe har implementeret CIS Control 2: Inventory and Control of Software Assets, noter følgende:
\begin{list2}
\item Hændelseslog: Marts Proxylogin - nem oprydning, ingen nedetid
\item Økonomi: Marts Proxylogin oprydning EUR 3.000
\end{list2}


Hvis jeres gruppe *IKKE* har implementeret CIS Control 2: Inventory and Control of Software Assets, noter følgende:
\begin{list2}
\item Hændelseslog: Marts Proxylogin -- mailservere inficeret 3 steder, ekstern hjælp nødvendig, nedetid 2 dage
\item Økonomi: Marts Proxylogin oprydning EUR 3.000
\item Økonomi: Marts ProxyLogon hændelseshåndtering ekstern hjælp EUR 10.000
\end{list2}



\slide{June 2021: PrintNightmare CVE-2021-34527 CVSS:3.0 8.8 / 8.2}
\begin{quote} \small
In June, Microsoft released a critical security update to address weaknesses in the Printer Spooler service on Windows desktop and server platforms. Unfortunately, it was released out-of-band outside of the standard patch Tuesdays due to the severity. Microsoft even released patches for Windows 7, an supported operating system that does not normally receive updates.

Initially categorised by Microsoft as a local privilege escalation on Windows, security researchers subsequently identified an additional {\bf Remote Code Execution (RCE)} vector resulting in an updated advisory from Microsoft. As ever, the ability to test and deploy patches in a time-sensitive manner is key to minimising the impact of such vulnerabilities.

Additionally, PrintNightmare had the additional horror factor of dropping during the {\bf summer holiday season in the northern hemisphere}. Our consultants continue to see systems vulnerable to PrintNightmare on client engagements, which can be trivially leveraged to obtain privilege escalation on unpatched Windows systems.
\end{quote}

Source - for this description:\\
\link{https://chessict.co.uk/resources/blog/posts/2022/january/2021-top-security-vulnerabilities/}

See also \link{https://msrc.microsoft.com/update-guide/vulnerability/CVE-2021-34527}


\slide{Incident Handling: PrintNightmare}

Hvis jeres gruppe har implementeret CIS Control 2: Inventory and Control of Software Assets, noter følgende:
\begin{list2}
\item Hændelseslog: Juni PrintNightmare - nem oprydning, ingen nedetid
\item Økonomi: Juni PrintNightmare oprydning EUR 3.000
\end{list2}


Hvis jeres gruppe *IKKE* har implementeret CIS Control 2: Inventory and Control of Software Assets, noter følgende:
\begin{list2}
\item Hændelseslog: Juni PrintNightmare -- servere inficeret, geninstallation nødvendig, nedetid 3 dage
\item Økonomi: Juni PrintNightmare oprydning EUR 3.000
\item Økonomi: Juni PrintNightmare hændelseshåndtering ekstern hjælp EUR 10.000
\end{list2}

Da denne skete i ferien er der desværre også brugt mere tid på at håndtere sagen, alle sætter ekstra EUR 3.000 på listen med teksten "Grundet ferie og manglende ressourcer 3.000"



\slide{September 2021: ForcedEntry}
\begin{quote}\small
{\bf Apple didn’t escape the wrath of critical zero-day vulnerabilities in 2021}, with ForcedEntry made public in September. The concern was not just that it could escape in-built sandbox controls and be leveraged against {\bf almost all iOS versions at the time}, but also that it was in the form of a {\bf one-click exploit meaning that no user interaction was needed}. A threat actor would simply require the target victim’s phone number or email address to send a weaponised GIF. {\bf Furthermore, iMessage was affected on macOS and watchOS, giving the exploit a significant attack surface of well over a billion devices.}

An analysis released at the end of 2021 confirmed a highly complex exploit which is believed to have been created by the NSO Group, creators of the Pegasus platform, albeit with the sophistication of nation-state actors. Given the nature of the attack and the level of complexity, high profile individuals are likely to be the intended targets of such exploits, only used sparingly against targeted victims.
\end{quote}

Source - for this description:\\
\link{https://chessict.co.uk/resources/blog/posts/2022/january/2021-top-security-vulnerabilities/}

See also
\link{https://en.wikipedia.org/wiki/FORCEDENTRY}

\slide{Incident Handling: ForcedEntry}

Hvis jeres gruppe har implementeret CIS Control 1+2, og ingen Apple telefoner har noter følgende:
\begin{list2}
\item Hændelseslog: September ForcedEntry - nem oprydning, ingen nedetid
\item Økonomi: September ForcedEntry oprydning EUR 1.000
\end{list2}

Hvorfor er det ikke EUR 0?

Hvis jeres gruppe har Apple telefoner noter følgende:
\begin{list2}
\item Hændelseslog: September ForcedEntry -- Apple telefoner inficeret, ekstern hjælp nødvendig
\item Økonomi: September ForcedEntry oprydning EUR 3.000
\item Økonomi: September ForcedEntry hændelseshåndtering ekstern hjælp EUR 10.000
\end{list2}





\slide{November 2021: Log4Shell}
\begin{quote}\small
It would not be possible to discuss 2021 in the context of vulnerabilities without the mention of Log4Shell. {\bf A widely used Java-based logging library caused headaches for Security professionals worldwide}. Many scrambled to quantify their use of Log4j within their estates.

A zero-day exploit quickly followed, confirming the worst - {\bf Remote Code Execution (RCE) was indeed possible.} However, what made the nature of the vulnerability even more challenging was the ability to exploit a backend logging system from an unaffected front end host. For example, an attacker can craft a weaponised log entry on a mobile app or webserver not running Log4j. The attacker could make their way through to backend middleware itself running Log4j, which significantly extends the attack surface of the vulnerability.

The NCSC even took the step of recommending the update was immediately applied, whether or not Log4Shell was known to be in use. As is commonly the case with critical vulnerabilities, two successive Log4j patches were subsequently released in the week following the original addressing Denial of Service (DoS) and a further RCE. This further increased workloads of Security and IT teams just as they thought the worst of 2021 had been and gone.
\end{quote}
Source - for this description:\\
\link{https://chessict.co.uk/resources/blog/posts/2022/january/2021-top-security-vulnerabilities/}

See also \link{https://en.wikipedia.org/wiki/Log4Shell}



\slide{Incident Handling: Log4Shell}

Hvis jeres gruppe har implementeret CIS Control 1+2 og ingen Java har, noter følgende:
\begin{list2}
\item Hændelseslog: November Log4Shell - nem oprydning, ingen nedetid
\item Økonomi: November Log4Shell oprydning EUR 3.000
\end{list2}


Hvis jeres gruppe *IKKE* har implementeret CIS kontroller, noter følgende:
\begin{list2}
\item Hændelseslog: November Log4Shell -- mailservere inficeret 3 steder, ekstern hjælp nødvendig, nedetid 2 dage
\item Økonomi: November Log4Shell oprydning EUR 3.000
\item Økonomi: November Log4Shell hændelseshåndtering ekstern hjælp EUR 10.000
\end{list2}



\slide{March 2022: Dirty pipe Linux CVE-2022-0847}

%\hlkimage{}{}

\begin{quote}
This is the story of CVE-2022-0847, a vulnerability in the {\bf Linux kernel since 5.8} which allows overwriting data in arbitrary read-only files. This leads to {\bf privilege escalation because unprivileged processes can inject code into root processes.}

It is similar to CVE-2016-5195 “Dirty Cow” but is easier to exploit.

The vulnerability was fixed in Linux 5.16.11, 5.15.25 and 5.10.102.
\end{quote}
Sources:
\link{https://dirtypipe.cm4all.com/}
\link{https://thestack.technology/dirty-pipe-exploited-linux-vulnerability-cve-2022-0847/}
\link{https://access.redhat.com/security/cve/CVE-2022-0847}


\slide{Incident Handling: Dirty pipe}

Hvis jeres gruppe har implementeret CIS Control 1+2 og ingen Linux har, noter følgende:
\begin{list2}
\item Hændelseslog: Marts Dirty pipe - nem oprydning, ingen nedetid
\item Økonomi: Marts Dirty pipe oprydning EUR 3.000
\end{list2}


Hvis jeres gruppe har Linux , noter følgende:
\begin{list2}
\item Hændelseslog: Marts Dirty pipe -- servere inficeret, ekstern hjælp nødvendig, nedetid 2 dage
\item Økonomi: Marts Dirty pipe oprydning EUR 3.000
\item Økonomi: Marts Dirty pipe hændelseshåndtering ekstern hjælp EUR 10.000
\end{list2}




\slide{April 2022: Lenovo UEFI }
\begin{quote}
ESET researchers have discovered and analyzed three vulnerabilities affecting various Lenovo consumer laptop models. The first two of these vulnerabilities – CVE-2021-3971, CVE-2021-3972 – affect UEFI firmware drivers originally meant to be used only during the manufacturing process of Lenovo consumer notebooks. Unfortunately, they were mistakenly included also in the production BIOS images without being properly deactivated. These affected firmware drivers can be activated by attacker to directly disable SPI flash protections (BIOS Control Register bits and Protected Range registers) or the UEFI Secure Boot feature from a privileged user-mode process during OS runtime. It means that exploitation of these vulnerabilities would allow attackers to deploy and successfully execute SPI flash or ESP implants, like LoJax or our latest UEFI malware discovery ESPecter, on the affected devices.
\end{quote}

Source:

also:
\link{https://www.welivesecurity.com/2022/04/19/when-secure-isnt-secure-uefi-vulnerabilities-lenovo-consumer-laptops/}

See also:
\link{https://www.bleepingcomputer.com/news/security/lenovo-uefi-firmware-driver-bugs-affect-over-100-laptop-models/}


\slide{Incident Handling: Lenovo UEFI}

Hvis jeres gruppe har implementeret CIS Control 1+2, og ingen Lenovo computere har noter følgende:
\begin{list2}
\item Hændelseslog: Marts Lenovo UEFI - ingen Lenovo computere, ingen nedetid
\item Økonomi: Marts Lenovo UEFI oprydning EUR 0
\end{list2}


Hvis jeres gruppe har Lenovo computere, noter følgende:
\begin{list2}
\item Hændelseslog: Marts Lenovo UEFI -- Lenovo computere udskiftet hos 10 medarbejdere, nedetid 1 dag for hver 10 medarbejdere
\item Økonomi: Marts Lenovo UEFI oprydning EUR 5.000
\item Økonomi: Marts Lenovo UEFI hændelseshåndtering udskiftning af udstyr EUR 10.000
\end{list2}





\slide{Konklusion: Kaos og panik}

\hlkimage{5cm}{dont-panic.png}
\begin{list2}
\item Vi startede godt, struktureret arbejde!
\item Vi blev afbrudt ... og det sker tit
\item Vi blev ikke færdige! Det bliver man sjældent i virkeligheden
\item Microsoft alene frigiver opdateringer for mere end 100 sårbarheder om måneden
\item Al software har sikkerhedsproblemer, og skal opdateres!
\end{list2}



\hlkprofiluk

\end{document}
