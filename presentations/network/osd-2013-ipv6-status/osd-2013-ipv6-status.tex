\documentclass[28pt,landscape,a4paper,footrule]{foils}
\usepackage{solido-network-slides}





%It is safe to bet that the absolute volumes of IPv6 traffic and IPv4 traffic will fit this growth pattern with the proportion of IPv6 traffic growing at a slow but steady pace. The proportion of inquiries in IPv6 in the Google ranking, currently at 1.1% should reach the 2% mark twelve months from now. Less than that would be a bit disappointing.


%http://www.google.com/ipv6/statistics.html

% Husk:
% RIPE LIR readyness


\begin{document}
\selectlanguage{english}
\mytitlepage{IPv6 status in Denmark\\\vskip 3mm get moving!}


\vskip 2cm
\centerline{\footnotesize Slides are available as PDF}



\slide{Goal}
\hlkimage{6cm}{kame-noanime-small.png}

\begin{list1}
\item Introduce IPv6 - facts and features
\item IPv6 Status Denmark
\item Enabled providers and sites
\item How to get your site on IPv6
\item Why I think we should prioritize IPv6
\end{list1}


\slide{Internetworking: history}

\begin{list2}  
\item[1960s]  L. Kleinrock, MIT packet-switching theory,  J. C. R. Licklider, MIT - notes 
  Paul Baran: On Distributed Communications
\item[1969]  ARPANET 4 nodes
\item[1971]  14 nodes
\item[1974]  TCP/IP: Cerf/Kahn: A protocol for Packet
        Network Interconnection
\item[1983]  {\bf Switching from NCP to IP/TCP}
\item[1983]  EUUG $\rightarrow$ DKUUG/DIKU forbindelse
\item[2010] IANA reserved blocks 7\% (Maj 2010) - \link{http://www.potaroo.net/tools/ipv4/}
\item[2011] February 3 IANA pool ran out - last 5 /8 allocated to RIRs
\item[2011] April 19 APNIC ran into their last /8 and started a more restrictive policy
\item[2012] Sept 14 RIPE NCC ran into their last /8 and started a more restrictive policy
\end{list2}

\slide{Internetworking: future of IPv4}

\hlkimage{14cm}{plotend-ipv4.png}
\begin{list2}  
\item Projected Exhaustion Date
\item[2014] June ARIN - September LACNIC
\item[2020] AFRINIC - has about 3.8 /8s Source \link{http://www.potaroo.net/tools/ipv4/}
\end{list2}


\slide{How to use IPv6}

\begin{center}
\vskip 3 cm
\hlkbig
www.solidonetworks.com

hlk@kramse.org 

(hlk@solido.net is not IPv6 enabled - ooops!)

\end{center}

\slide{Really how to use IPv6?}

\begin{list1}
\item Get IPv6 address and routing
\item Add AAAA (quad A) records to your DNS
\item Done
\end{list1}
\vskip 1cm
\centerline{\Large www.solidonetworks.com}

\begin{alltt}
\LARGE
www     IN	A       91.102.95.20
        IN	AAAA    2a02:9d0:10::9
\end{alltt}





\slide{IPv6 Status Denmark}

\begin{list1}
\item Unofficial IPv6 task force at \link{http://www.ipv6tf.dk/}
\item Major ISPs are ready in core networks
\item Major ISP deliver IPv6 to business customers
\item No major providers deliver IPv6 to consumers
\item Some smaller internet Providers are working on IPv6 to homes
\item A large percentage of the LIRs servicing Denmark has IPv6!
\vskip 1cm 
\item Many hosting and content providers are ready in core networks
\item Many hosting and content providers can deliver IPv6 to business customers
\end{list1}

\slide{IPv6 in the Nordic region - 2011}

\hlkimage{14cm}{ipv6-nordic-2011.png}

\link{http://v6asns.ripe.net/v/6?s=_ALL;s=DK;s=SE;s=NO;s=NL}


\slide{IPv6 in the Nordic region - 2013}

\hlkimage{14cm}{ipv6-nordic-2013.png}

\link{http://v6asns.ripe.net/v/6?s=SE;s=FI;s=NO;s=DK;s=IS;s=_ALL}\\
\link{https://www.ripe.net/membership/indices/DK.html}

\slide{RIPE NCC getting IPv4 in Europe}

\begin{quote}
On 14th September 2012, the RIPE NCC distributed the final IPv4 address space
before reaching the last /8. This means that section 5.6 of IPv4 Address
Allocation and Assignment Policies for the RIPE NCC Service Region is now in
effect.

This section states that, once the RIPE NCC begins to allocate address space
from the last /8, {\bf an LIR may receive only a /22 (1,024 IPv4 addresses)}, even
if they can justify a larger allocation. This /22 allocation will only be
made to LIRs {\bf if they have already received an IPv6 allocation} from an
upstream LIR or the RIPE NCC.

\vskip 1cm
\verb+<blink>+No new IPv4 Provider Independent (PI) space will be assigned.\verb+</blink>+
\end{quote}

\vskip 15mm

\centerline{\color{titlecolor}\bf \LARGE IPv6 or GTFO}


\slide{The price of IPv4}

\begin{quote}
Hi,

thank you for the interesting, sorry for the late reply, but I have got
20-30 interests/ day. I have got two unused ip range 91.135.112.0/21 and
91.135.120.0/22 I'm waiting for offers and the highest offer will become
it. I can give it to rent or I can sell it. The best offer for rent is 3,5
eur / month / ip and for buy is 30 eur / ip.

Best regards,
Zoltan
\end{quote}

Let me calculate that for you {\bf /22 is 1,024 IPs each 30 EUR = 30,720 EUR}

How many do you want?

\vskip 2cm 
Source: private email communication asking about an IPv4 listing on RIPE portal

\slide{Current status Denmark}

\hlkimage{12cm}{demotivational-poster-Lazy.jpg}

\begin{list1}
\item Too little interest (upgraded from "no interest" in earlier presentations)
\item Some providers are doing testing, Thanks Bolig:net for the native IPv6 in my home!
\item Perceived NO NEEED - this is a problem - {\bf WTF people, get real!}
\end{list1}


\slide{Carrier Grade NAT - an IPv4 solution}


\hlkimage{12cm}{ingenuity-720642.jpg}

CGN sucks and RFC6598 IANA-Reserved IPv4 Prefix for Shared Address Space extends the life - double NAT whammy - no problems *jediwave*
\vskip 8mm
\centerline{Join me everyone - {\bf NAT IS BAD}}


\slide{Every day is IPv6 day}

\hlkimage{6cm}{ipv6-launch-flag.png}

\begin{list1}
\item Free, a major French ISP rolled-out IPv6 at end of year 2007
\item XS4All As of August 2010 native IPv6 DSL connections became available to almost all their customers.
\item Denmark are frontrunners in IT, *sigh*
\end{list1}

Source: \link{http://en.wikipedia.org/wiki/IPv6_deployment}

\slide{Danish ISPs and infrastructure}

\begin{list1}
\item \link{http://www.tdc.dk} Large ISP
\item \link{http://globalconnect.dk} Fiber and internet provider
\item \link{http://netgroup.dk} Internet provider
\item \link{http://nianet.dk} Internet provider
\item \link{http://bolignet.dk/} Internet provider
\item \link{http://zensystems.dk} Internet provider
\item \link{http://fiberby.dk/} Internet provider
\item  \link{http://siminn.dk} Internet provider (IPv6 RSN \smiley )
\item \link{http://www.dk-hostmaster.dk} 
\end{list1}

\slide{IPv6 enabled sites}

\begin{list1}
\item \link{http://www.lynero.dk} \link{www.feriebolig-spanien.dk} 
\item \link{http://mirrors.dotsrc.org} {\bf \link{http://www.herning.dk}}
\item \link{http://www.computerworld.dk}  \link{http://www.version2.dk} 
\item \link{http://www.information.dk} \link{http://kiaklub.dk/}
\item \link{http://xstream.dk}  \link{http://ssl.isecurity.dk}
\item \link{https://bitbureauet.dk/} \link{http://Ugenr.dk}
\item \link{http://bingo.wenneberg.net/} \link{http://mirror.dk.freebsd.org}
\item \link{http://Pixolink.com} \link{http://coolsms.com}
\end{list1}

\slide{Personal and other web sites}

\begin{list1}
\item \link{http://flemmingriis.com}
\item \link{http://graffen.dk}
\item \link{http://iboserup.dk}
\item \link{http://blog.andersen.nu/}
\item \link{http://iptv-analyzer.org}
\item \link{http://web.gratisdns.dk} Larsen Data\\
+ approx 51313 hosts via URL forwarder service!
\end{list1}
\vskip 1cm 

\centerline{\Large \color{titlecolor}Quick conclusion - there is danish content on IPv6}
\vskip 15mm 
See more at: \link{http://world-ipv6-day.dk/danske-ipv6-sites}

\slide{Telenor}

\hlkimage{22cm}{telenor-ipv6-2013.png}

\slide{TDC Yousee}

\hlkimage{12cm}{yousee-ipv6-2013.png}

aha, de har ellers tidligere meldt mere aggressivt ud.

\slide{3.dk}

\hlkimage{20cm}{3-dk-ipv6-2013.png}

Mkay, kort tid ...



\slide{Non-IPv6 sites in Denmark}

\hlkimage{13cm}{lazy-cat.jpg}

but no DR.dk, folketinget.dk, ministerier, offentlige myndigheder, no KMD, no web sites with CSC, IBM, ...? 
\vskip 10mm
\centerline{\LARGE it's not hard - get moving}


\slide{How to get your site on IPv6}

Practical information for your network

\begin{list1}
\item Strategy and actions points
\begin{list2}
\item Collect information about IPv6 
\item Collect information about your network
\item Collect information about your hosts and services
\item Ask your providers for IPv6 plans
\item Experiment with IPv6 - today
\item Implement small proof of concept, in production!
\item Expand coverage
\end{list2}
\end{list1}

\vskip 2cm
\centerline{Process for LIRs: apply for IPv6 space, announce with BGP - 2 days work!}


\slide{Ripeness}

\begin{list1}
\item First star: IPv6 address space
\item One more star: route6 object
\item Another star: reverse DNS
\item Yet another star: prefix visible in RIS
\end{list1}

\link{http://ipv6ripeness.ripe.net/4star/DK.html}


\slide{Security Implications - take control}

\hlkimage{2cm}{IPv6ready.png}

\begin{list1}
\item For an IPv4 enterprise network, the existence of an IPv6 overlay network has several of implications:
\begin{list2}
\item The IPv4 firewalls can be bypassed by the IPv6 traffic, and leave the security door wide open.
\item Intrusion detection mechanisms not expecting IPv6 traffic may be confused and allow intrusion
\item In some cases (for example, with the IPv6 transition technology known as 6to4), an internal PC can communicate directly with another internal PC and evade all intrusion protection and detection systems (IPS/IDS). Botnet command and control channels are known to use these kind of tunnels.
\end{list2}
\end{list1}

Kilde:\\
{\footnotesize\link{http://www.cisco.com/en/US/prod/collateral/iosswrel/ps6537/ps6553/white_paper_c11-629391.html}}


\slide{Collect information about IPv6}

\begin{list1}
\item \emph{Guidelines for the Secure Deployment of IPv6}, SP800-119, NIST\\
\link{http://csrc.nist.gov/publications/nistpubs/800-119/sp800-119.pdf}
\item \emph{The Second Internet: Reinventing Computer Networks with IPv6}, Lawrence E. Hughes, October 2010,\\ \link{http://www.secondinternet.org/}
\item \emph{IPv6 Network Administration}
af David Malone og Niall Richard Murphy
\item \link{http://www.ripe.net}
\item This presentation \smiley
\end{list1}

\slide{Allocating IPv6 addresses} 

\begin{list1}
\item You have plenty!
\item Providers and LIRs will typically get /32
\item Providers will typically give organisations /48 or /56
\item Your /48 can be used for:
\begin{list2}
\item 65536 subnets - all host subnets are /64
\item Each subnet has $2^{64}$ addresses
\end{list2}
\end{list1}

\slide{Preparing an IPv6
Addressing Plan}

\hlkimage{20cm}{ipv6-address-plan-ripe.png}

{\footnotesize \link{http://www.ripe.net/training/material/IPv6-for-LIRs-Training-Course/IPv6_addr_plan4.pdf}}

\slide{Example adress plan input}

\hlkimage{22cm}{ipv6-linked-to-ipv4.png}

\centerline{Easy and coupled with VLAN IDs it will work \smiley}

\slide{Run IPv6 in production}

\begin{list1}
\item Make sure you establish IPv6 in {\bf production}
\item Enabling service on IPv6 without production - bad experience for users
\item Start by enabling your DNS servers for IPv6 - and DNSSEC - and DNS over TCP\\
Remember that your firewall might have problems with large DNS packets
\item Add a production IPv6 router - hardware device or generic server
\item Tunnels are OK, and SixXS consider their service production
\end{list1}

\slide{Up and running with IPv6}

\begin{list1}
%\item Join the fun - join the wireless network
\item \link{https://www.sixxs.net/} Apply for IPv6 tunnel 
\item \link{http://www.tunnelbroker.net/} Apply for IPv6 tunnel 
\item Use ping/ping6 and traceroute to test connectivity - Done enjoy \smiley
\item Try in your browser:
\begin{list2}
\item \link{http://www.kame.net} Dancing turtle
\item \link{http://www.ripe.net} RIPE, look for address up right corner 
\item \link{http://loopsofzen.co.uk/} Play a game
\item \link{https://www.sixxs.net/} Apply for IPv6 tunnel 
\item \link{http://ipv6.he.net/certification/} Join the Hurricane Electric IPv6 Certification Project
\end{list2}
\end{list1}


\slide{F5 load balancer example}

\hlkimage{\linewidth-3cm}{f5-load-balancer.png}


\slide{Why prioritize IPv6 IPv6 - the business case}


\begin{list2}
\item An almost unlimited scalability with a very large IPv6 address space ($2^{128}$ addresses), enabling IP addresses to each and every device.

\item Address self-configuration mechanisms, easing the deployment. Router advertisements are simple

\item Improved security and authentication features, such as mandatory IPSec capacities and direct connections

\item Peer-to-peer connectivity, solving the NAT barrier with specific and permanent IP addresses for any device and/or user of the Internet.

\item Mobility features, enabling a seamless connexion when moving from one access point to another access point on the Internet.

\item Multi cast and any cast functionalities.

\item IPv6 will provide an easier remote interaction with each and every device with a {\bfseries direct integration to the Internet.} In other words, IPv6 will make possible to move from a network of servers, to a network of things.

\end{list2}

\centerline{ Business case for IPv6 is {\bf continuity}}


{\footnotesize Partially inspired by \link{http://www.smartipv6building.org/index.php/en/ipv6-potential}}



\slide{Conclusion}

\begin{center}
\vskip 5mm
{\color{titlecolor}\LARGE \bf IPv6 is here already - use it}
\vskip 5mm

\hlkimage{7cm}{taskforce-logo.jpg}

\link{http://www.ipv6tf.dk}
\vskip 1cm
\centerline{ Danish IPv6 task force - unofficial - not very active}

\end{center}



\myquestionspage


\slide{Ressources}

\begin{list1}
\item \emph{Guidelines for the Secure Deployment of IPv6}, SP800-119, NIST\\
\link{http://csrc.nist.gov/publications/nistpubs/800-119/sp800-119.pdf}
\item \emph{The Second Internet: Reinventing Computer Networks with IPv6}, Lawrence E. Hughes, October 2010,\\ \link{http://www.secondinternet.org/}
\end{list1}



\slide{VikingScan.org - free portscanning}

\hlkimage{18cm}{vikingscan.png}
%\vskip 1cm 
%\centerline{\link{http://www.vikingscan.org}}


\hlkprofiluk

\end{document}
