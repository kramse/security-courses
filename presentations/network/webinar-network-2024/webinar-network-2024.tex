\documentclass[Screen16to9,17pt]{foils}
%\documentclass[16pt,landscape,a4paper,footrule]{foils}
%\usepackage{zencurity-slides-troopers}
\usepackage{kea-slides}



\addbibresource{~/projects/security-courses/texfiles/firewall-refs.bib}

\begin{document}

%\rm
\selectlanguage{english}
\mytitlepage{Webinar: Diplomuddannelsen i IT-sikkerhed}{KEA 2024}


\hlkprofil

\slide{Diplomuddannelsen i IT-sikkerhed}

\hlkimage{18cm}{kompetence-menukort.png }

Læs mere på \link{https://kompetence.kea.dk/uddannelser/it/diplom-i-it-sikkerhed}


\slide{Course Description}
\begin{quote}\small
{\bf OB1 Netværks- og kommunikationssikkerhed (10 ECTS)}

Indhold:

Elementet går ud på at forstå og håndtere netværkssikkerhedstrusler samt implementere og
konfigurere udstyr til samme.

Elementet omhandler forskellig sikkerhedsudstyr (IDS) til monitorering. Derudover vurdering
af sikkerheden i et netværk, udarbejdelse af plan til at lukke eventuelle sårbarheder i
netværket samt gennemgang af forskellige VPN teknologier.
\end{quote}

STUDIEORDNING Diplomuddannelse i it-sikkerhed\\
{\footnotesize
\url{https://kompetence.kea.dk/studieordninger/Studieordning_Diplom_IT-sikkerhed_2022_03.pdf}}

\slide{Viden}

%\hlkimage{}{}

Den studerende har viden om og forståelse for:
\begin{list2}
\item Netværkstrusler
\item Trådløs sikkerhed
\item Sikkerhed i TCP/IP
\item Adressering i de forskellige lag
\item Dybdegående kendskab til flere af de mest anvendte internet protokoller (ssl)
\item Hvilke enheder, der anvender hvilke protokoller
\item Forskellige sniffing strategier og teknikker
\item Netværk management (overvågning/logning, snmp)
\item Forskellige VPN setups
\item Gængse netværksenheder der bruges ifm. sikkerhed (firewall, IDS/IPS, honeypot,
DPI).
\end{list2}

\slide{Færdigheder}

Den studerende kan:
\begin{list2}
\item Overvåge netværk samt netværkskomponenter, (f.eks. IDS eller IPS, honeypot)
\item Teste netværk for angreb rettet mod de mest anvendte protokoller
\item Identificere sårbarheder som et netværk kan have.
\end{list2}

\slide{Kompetencer}

Den studerende kan:

Håndtere udviklingsorienterede situationer herunder:
\begin{list2}
\item designe, konstruere og implementere samt teste et sikkert netværk
\item monitorere og administrere et netværks komponenter mht. it-sikkerhed
\item Udfærdige en rapport om de sårbarheder et netværk eventuelt skulle have (red
team report)
\item Opsætte og konfigurere et IDS eller IPS.
\end{list2}

Kan håndtere relevante krypteringstiltag til sikring af netværkskommunikation


\slide{Book: Applied Network Security Monitoring (ANSM)}

\hlkimage{5cm}{ansm-book.png}

\emph{Applied Network Security Monitoring: Collection, Detection, and Analysis}
1st Edition

Chris Sanders, Jason Smith
eBook ISBN: 9780124172166
Paperback ISBN: 9780124172081 496 pp.
Imprint: Syngress, December 2013

{\footnotesize\link{https://www.elsevier.com/books/applied-network-security-monitoring/unknown/978-0-12-417208-1}}

\slide{Book: Practical Packet Analysis (PPA)}
\hlkimage{6cm}{PracticalPacketAnalysis3E_cover.png}

\emph{Practical Packet Analysis,
Using Wireshark to Solve Real-World Network Problems}
by Chris Sanders, 3rd Edition
April 2017, 368 pp.
ISBN-13:
978-1-59327-802-1

\link{https://nostarch.com/packetanalysis3}



\slide{Supporting literature books}
\begin{list2}
\item \emph{Linux Basics for Hackers Getting Started with Networking, Scripting, and Security in Kali}\\
OccupyTheWeb, December 2018, 248 pp. ISBN-13: 978-1-59327-855-7 - shortened LBfH
\item \emph{The Debian Administrator’s Handbook}, Raphaël Hertzog and Roland Mas\\
\url{https://debian-handbook.info/} - shortened DEB
\item \emph{Kali Linux Revealed  Mastering the Penetration Testing Distribution}\\
Raphaël Hertzog, Jim O'Gorman - shortened KLR
\end{list2}


\myquestionspage


\end{document}
