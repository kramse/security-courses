\documentclass[20pt,landscape,a4paper,footrule]{foils}
%\usepackage{solido-network-slides}
\usepackage{zencurity-slides}

%\externaldocument{unix-audit-security-oevelser}
\externaldocument{\jobname-exercises}

% Kom og hack IT-systemer

% Har du indimellem læst artiklerne om IT-sikkerhedsproblemer og tænkt hvordan foregår hacking?
% Så er denne session til dig der vil prøve at være hacker i et par timer.

% Vi vil introducere Kali 2.0 Linux hackerplatformen og med eksempler som stiger i sværhedsgrad præsentere hacking af websystemer og sårbare virtuelle systemer.

% Du vil få mulighed for at prøve:
% * hacking af websystemer gennem et sårbart website
% * portscanning med Nmap sikkerhedsværktøjet
% * hacking af en sårbar virtuel maskine med Metasploit

% Formålet med sessionen er at demonstrere hvordan hackere arbejder, og hvordan samme teknikker kan bruges til at verificere sikkerheden på netværk og systemer.

% Der vil være to sessioner med samme indhold
% Kl 11 – 13 - session 1
% Kl 14 – 16 - session 2

% Medbring PC
% Der vil være opgaver hvor det forventes at deltagerene er aktive for at få det fulde udbytte. Så medbring din PC.
% De indledende opgaver kan udføres med en almindelig browser, som Firefox der bør være installeret før dagen. Til de mest tekniske vil det største udbytte opnås med en virtuel maskine hvor Kali Linux kan installeres. Dette kan gøres med virtualiseringsprogrammer som VMware, Virtual Box - men spørg lige IT-afdelingen hvad de anbefaler.

% Vi anbefaler at der arbejdes i hold på 2 personer, men opgaverne kan udføres alene. Der behøves kun virtualisering på een PC per hold.

% IT-sikkerhed
% Der vil være et lukket netværk hvor opgaverne udføres mod specielt opsatte sårbare systemer. Der vil blive arbejdet med sikkerhedsprogrammer og hackerprogrammer, men ikke malware og virus! PC kan således trygt medbringes.

% Basic things that we need are below
\begin{document}
\selectlanguage{danish}

\mytitlepage
{Kom og Hack IT-systemer}{}


\slide{Formålet }

\vskip 2 cm

\hlkimage{2cm}{dont-panic.png}
\centerline{\color{titlecolor}\LARGE Don't Panic!}

\begin{list1}
\item Hvordan foregår hacking?
\item Vi vil introducere Kali 2.0 Linux hackerplatformen og med eksempler som stiger i sværhedsgrad præsentere hacking af websystemer og sårbare virtuelle systemer.
\item Skabe forståelse for hackerværktøjer
  samt penetrationstest metoder
\item  Du vil få mulighed for at prøve:
\begin{list2}
\item hacking af websystemer gennem et sårbart website
\item portscanning med Nmap sikkerhedsværktøjet
\item hacking af en sårbar virtuel maskine med Metasploit
\end{list2}

\end{list1}

\slide{Hackerværktøjer}

\begin{list1}
\item \emph{Improving the Security of Your Site by Breaking Into it} af
Dan Farmer og Wietse Venema i 1993
\item De udgav i 1995 så en softwarepakke med navnet SATAN
\emph{Security Administrator Tool for Analyzing Networks}
\item De forårsagede en del panik og furore, alle kan hacke, verden bryder sammen

\vskip 1cm
\begin{quote}
We realize that SATAN is a two-edged sword -- like
many tools, it can be used for good and for evil
purposes. We also realize that intruders (including
wannabees) have much more capable (read intrusive)
tools than offered with SATAN.
\end{quote}
\end{list1}

\vskip 1cm
Kilde:
\link{http://www.fish2.com/security/admin-guide-to-cracking.html}


\slide{Brug hackerværktøjer!}

\begin{list1}
\item Hackerværktøjer -- bruger I dem? -- efter dette kursus gør I
\item Portscannere kan afsløre huller i forsvaret
\item Webtestværktøjer som crawler igennem et website og finder alle
  forms kan hjælpe
\item I vil kunne finde mange potentielle problemer proaktivt ved
  regelmæssig brug af disse værktøjer -- også potentielle driftsproblemer
\end{list1}



\slide{Hacking er magi}

\hlkimage{7cm}{wizard_in_blue_hat.png}

\vskip 1 cm

\centerline{Hacking ligner indimellem  magi}


\slide{Hacking er ikke magi}

\hlkimage{17cm}{ninjas.png}

\vskip 1 cm
\centerline{Hacking kræver blot lidt ninja-træning}

\slide{Movie:Kryptonite lock - old}

\hlkimage{16cm}{youtube-bic-lock.png}

\begin{list1}
\item Just search for: kryptonite lock bic pen
\item \link{https://www.youtube.com/watch?v=LahDQ2ZQ3e0}
\end{list1}



\slide{Hacking eksempel - det er ikke magi}

\begin{list1}
\item MAC filtrering på trådløse netværk
\item Alle netkort har en MAC adresse - BRÆNDT ind i kortet fra fabrikken
\item Mange trådløse Access Points kan filtrere MAC adresser
\item Kun kort som er på listen over godkendte adresser tillades adgang til netværket
\pause
\item Det virker dog ikke \smiley
\item De fleste netkort tillader at man overskriver denne adresse midlertidigt
\item Derudover har der ofte været fejl i implementeringen af MAC filtrering
\end{list1}


\slide{MAC filtrering}

\hlkimage{15cm}{stupid-security.jpg}


\slide{Heartbleed CVE-2014-0160}

\hlkimage{22cm}{heartbleed-com.png}

Source: \link{http://heartbleed.com/}


\slide{Heartbleed hacking}

\begin{alltt}\footnotesize
  06b0: 2D 63 61 63 68 65 0D 0A 43 61 63 68 65 2D 43 6F  -cache..Cache-Co
  06c0: 6E 74 72 6F 6C 3A 20 6E 6F 2D 63 61 63 68 65 0D  ntrol: no-cache.
  06d0: 0A 0D 0A 61 63 74 69 6F 6E 3D 67 63 5F 69 6E 73  ...action=gc_ins
  06e0: 65 72 74 5F 6F 72 64 65 72 26 62 69 6C 6C 6E 6F  ert_order&billno
  06f0: 3D 50 5A 4B 31 31 30 31 26 70 61 79 6D 65 6E 74  =PZK1101&payment
  0700: 5F 69 64 3D 31 26 63 61 72 64 5F 6E 75 6D 62 65  _id=1&{\bf card_numbe}
  0710: XX XX XX XX XX XX XX XX XX XX XX XX XX XX XX XX  {\bf r=4060xxxx413xxx}
  0720: 39 36 26 63 61 72 64 5F 65 78 70 5F 6D 6F 6E 74  {\bf 96&card_exp_mont}
  0730: 68 3D 30 32 26 63 61 72 64 5F 65 78 70 5F 79 65  {\bf h=02&card_exp_ye}
  0740: 61 72 3D 31 37 26 63 61 72 64 5F 63 76 6E 3D 31  {\bf ar=17&card_cvn=1}
  0750: 30 39 F8 6C 1B E5 72 CA 61 4D 06 4E B3 54 BC DA  {\bf 09}.l..r.aM.N.T..
\end{alltt}

\begin{list2}
\item Obtained using Heartbleed proof of concepts - Gave full credit card details
\item "can XXX be exploited" - yes, clearly! PoCs ARE needed\\
without PoCs even Akamai wouldn't have repaired completely!
\item \link{https://github.com/rapid7/metasploit-framework/blob/master/modules/auxiliary/scanner/ssl/openssl_heartbleed.rb}
\end{list2}


\slide{Most vulnerable operating systems in 2014}

\hlkimage{18cm}{GFI-vulns-2014-OS-chart.jpg}

\begin{quote}
An average of 19 vulnerabilities per day were reported in 2014, according to the data from the National Vulnerability Database (NVD).
\end{quote}

Source:\\
{\footnotesize
\link{http://www.gfi.com/blog/most-vulnerable-operating-systems-and-applications-in-2014/}}


\slide{Most vulnerable applications in 2014}

\hlkimage{16cm}{gfi-vulns-application-chart.jpg}

\begin{quote}\small
Not surprisingly at all, web browsers continue to have the most security vulnerabilities because they are a popular gateway to access a server and to spread malware on the clients.
\end{quote}

Source:\\
{\footnotesize
\link{http://www.gfi.com/blog/most-vulnerable-operating-systems-and-applications-in-2014/}}



\slide{OSI og Internet modellerne}

\hlkimage{14cm,angle=90}{images/compare-osi-ip.pdf}


\slide{Wireshark - grafisk pakkesniffer}

\hlkimage{20cm}{images/wireshark-website.png}

\centerline{\link{http://www.wireshark.org}}
\vskip 5mm
\centerline{både til Windows og UNIX}

\slide{Wireshark usage}
\hlkimage{16cm}{wireshark-http.png}

Wireshark: Filters, hexdump, protocol dissection, overview, coloring, advanced features

\demo{Wireshark}


\slide{Network mapping}

\hlkimage{23cm}{images/network-example.pdf}

\begin{list1}
\item Ved brug af traceroute og tilsvarende programmer kan man ofte
  udlede topologien i det netværk man undersøger
\end{list1}


\slide{Portscan med Zenmap GUI}

\hlkimage{16cm}{nmap-zenmap.png}





\slide{Cracking passwords}

\begin{list2}
\item Hashcat is the world's fastest CPU-based password recovery tool.
\item oclHashcat-plus is a GPGPU-based multi-hash cracker using a brute-force attack (implemented as mask attack), combinator attack, dictionary attack, hybrid attack, mask attack, and rule-based attack.
\item oclHashcat-lite is a GPGPU cracker that is optimized for cracking performance. Therefore, it is limited to only doing single-hash cracking using Markov attack, Brute-Force attack and Mask attack.
\item John the Ripper password cracker old skool men stadig nyttig
\end{list2}

Source:\\
\link{http://hashcat.net/wiki/}\\
\link{http://www.openwall.com/john/}

\slide{Parallella John}

\hlkimage{20cm}{parallella-john.png}

\link{https://twitter.com/solardiz/status/492037995080712192}

Warning: FPGA hacking - not finished part of presentation

\slide{Stacking Parallella boards}
\hlkimage{16cm}{4BoardStack.jpg}

\link{http://www.parallella.org/power-supply/}


\slide{Hackerværktøjer}
% måske til reference afsnit?
\hlkimage{3cm}{hackers_JOLIE+1995.jpg}

\begin{list2}
\item Alle bruger nogenlunde de samme værktøjer, se også \link{http://www.sectools.org/}
\item Portscanner Nmap, Nping -- tester porte, godt til firewall admins \link{https://nmap.org}
\item Generel sårbarhedsscanner Metasploit Framework \link{https://www.metasploit.com/}
\item Specialscannere, eksempelvis web sårbarhedsscanner -- eksempelvis Nikto, Skipfish
\item Specielle scannere -- wifi Aircrack-ng, web Burpsuite \link{http://portswigger.net/burp/}
\item Wireshark avanceret netværkssniffer -- \link{https://www.wireshark.org/}
\item og scripting, PowerShell, Unix shell, Perl, Python, Ruby, \ldots
\end{list2}

Billedet: Angelina Jolie fra Hackers 1995


\slide{Hvad skal der ske?}

\begin{list1}
\item Tænk som en hacker
\item Rekognoscering
\begin{list2}
\item ping sweep, port scan
\item OS detection -- TCP/IP eller banner grab
\item Servicescan -- rpcinfo, netbios, ...
\item telnet/netcat interaktion med services
\end{list2}
\item Udnyttelse/afprøvning: Metasploit, Nikto, exploit programs
\item Oprydning/hærdning vises måske ikke, men I bør i praksis:
\begin{list2}
\item Lav en rapport
\item Ændre, forbedre og hærde systemer
\item Gennemgå rapporten, registrer ændringer
\item Opdater programmer, konfigurationer, arkitektur, osv.
\end{list2}
\item I skal jo også VISE andre at I gør noget ved sikkerheden.
\end{list1}


\slide{Hackerlab opsætning}

\hlkimage{10cm}{hacklab-1.png}

\begin{list2}
\item Hardware: en moderne laptop med CPU der kan bruge virtualiseting\\
Husk at slå virtualisering til i BIOS
\item Software: dit favoritoperativsystem, Windows, Mac, Linux
\item Virtualiseringssoftware: VMware, Virtual box, vælg selv
\item Hackersoftware: Kali som Virtual Machine \link{https://www.kali.org/}
\item Soft targets: Metasploitable, Windows 2000, Windows XP, ...
\end{list2}


\slide{Kali Linux the new backtrack}

\hlkimage{20cm}{kali-linux.png}

\begin{list1}
\item BackTrack -- \link{http://www.backtrack-linux.org}
\item Kali -- \link{https://www.kali.org/} version 2.0 netop udkommet!
\item Wireshark -- \link{https://www.wireshark.org} avanceret netværkssniffer
\end{list1}


\slide{Demo: Metasploit Armitage }

\hlkimage{16cm}{armitage-overview.png}


\slide{Metasploit and Armitage Still rocking the internet}


\hlkimage{14cm}{metasploit-about.png}

\begin{list1}
\item Udviklingsværktøjerne til exploits er i dag meget raffinerede!
\item \link{http://www.metasploit.com/}
\item Armitage GUI fast and easy hacking for Metasploit\\
\link{http://www.fastandeasyhacking.com/}
\item Kursus Metasploit Unleashed\\
\link{http://www.offensive-security.com/metasploit-unleashed/Main_Page}
\item Bog: Metasploit: The Penetration Tester's Guide, No Starch Press\\
ISBN-10: 159327288X
\end{list1}

\slide{ Kali øvelser}

Vi kan ikke nå alverden men prøv at gentage det jeg viste
\begin{list2}
\item Wireshark
\item Wireshark med FTP
\item Nmap med Zenmap
\item Armitage og Metasploit, husk \verb+service postgresql start+
\item Prøv også OWASP Web Goat som jeg har startet
\end{list2}


\myquestionspage

\end{document}
