\documentclass[Screen16to9,17pt]{foils}
\usepackage{zencurity-slides}


% Beskrivelse: Det kunne f.eks. handle om hvordan man undgår at afsløre sine brugernavne og adgangskoder når man sidder på et offentligt netværk, og evt. andre ting man skal være opmærksom på som internet-bruger. Herunder en demonstration af hvor nemt det er at sniffe på netværket.
% Type: Foredrag
% Forslagstiller: Jette Derriche
% Foredragsholder: Henrik Kramshøj (hlkv6), Flemming Jacobsen (F3), Thomas Rasmussen (Tykling)
% Grundet sidste års diskussion omkring snifning af data gøres alle opmærksomme på risikoen ved at bruge usikre protokoller som FTP, SMTP, HTTP (uden S) osv.
% Der opsættes testnetværk som sender traffik der kan sniffes og der snakkes om programmer og metoder til opsamling af hemmeligheder.
% Det anbefales at alle der konfigurerer egen webmail osv. spørger om hjælp til opsætning af SSL ;-)
% Det er meningen at vi her under kontrollerede former og venlige sjæle gør alle opmærksomme på problemer - så vi alle fremover, ude i den store verden benytter bedre protokoller.



\begin{document}
\selectlanguage{english}
\mytitlepage{Hacking - protect yourself}


\slide{Goal of this presentation}
\vskip 2 cm

%{\hlkbig En 3 dages workshop, hvor du lærer at angribe dit netværk!}

\hlkimage{5cm}{dont-panic.png}
\centerline{\color{titlecolor}\LARGE Don't Panic!}

\begin{list1}
\item Introduce hacking and a couple of hacker tools
\item List some tools that can be used to protect your computer and data
\item PS Sorry about the many TLAs ... og danglish
\end{list1}

\slide{Introduktion til hacking}

\hlkimage{7cm}{hackers_JOLIE+1995.jpg}

\centerline{\link{http://www.imdb.com/title/tt0113243/} Hackers (1995)}

\slide{Internet today}


\hlkimage{8cm}{images/server-client.pdf}

\begin{list1}
\item Clients and servers
\item Rooted in academic networks
\item Protocols which are more than 20 years old, moved to TCP/IP in 1981
\item Trying to migrate to IPv6 - a lot of hacking opportunities here
\end{list1}

\slide{Hackers don't give a shit}

\hlkrightpic{11cm}{-3cm}{kiwicon-2009-hackers-dont-give-shit.jpg}

Your system is only for testing, development, ...

Your network is a research network, under construction, \\
being phased out, ...

Try something new, go to your management

Bring all the exceptions, all of them, update the risk \\
analysis figures - if this happens it is about 1mill DKK

Ask for permission to go full monty on your security

{\bf Think like attackers - don't hold back}

\slide{Hacking seems like magic}

\hlkimage{6cm}{wizard_in_blue_hat.png}

\vskip 1 cm

{\Large Hacking looks like magic}


\slide{Hacking is not magic}

\hlkimage{11cm}{ninjas.png}

\begin{list2}
\item Hacking only requires some ninja training
\item We have been doing this since at least 1995 when SATAN was released
\item Listen, Plan, Act, Do hacking
\end{list2}

\slide{The Internet Worm 2. nov 1988}

\begin{list1}
\item Udnyttede følgende sårbarheder
\begin{list2}
\item buffer overflow i fingerd - VAX kode
\item  Sendmail - DEBUG
\item Tillid mellem systemer: rsh, rexec, ...
\item dårlige passwords
\end{list2}
\item Avanceret + camouflage!
\begin{list2}
\item Programnavnet sat til 'sh'
\item Brugte fork() til at skifte PID jævnligt
\item password cracking med intern liste med 432 ord og /usr/dict/words
\item Fandt systemer i /etc/hosts.equiv, .rhosts, .forward, netstat ...
\end{list2}
\item Lavet af Robert T. Morris, Jr.
\item Medførte dannelsen af CERT, \link{http://www.cert.org}
\end{list1}


\slide{Teknisk hvad er hacking}

\hlkimage{12cm}{buffer-overflow-3.pdf}

\slide{buffer overflows et C problem}

\begin{list1}
\item {\bfseries Et buffer overflow}
er det der sker når man skriver flere data end der er afsat plads til
i en buffer, et dataområde. Typisk vil programmet gå ned, men i visse
tilfælde kan en angriber overskrive returadresser for funktionskald og
overtage kontrollen.
\item {\bfseries Stack protection}
er et udtryk for de systemer der ved hjælp af operativsystemer,
programbiblioteker og lign. beskytter stakken med returadresser og
andre variable mod overskrivning gennem buffer overflows. StackGuard
og Propolice er nogle af de mest kendte.
\end{list1}

\slide{Buffer og stacks}

\hlkimage{17cm}{buffer-overflow-1.pdf}

\begin{alltt}\small
main(int argc, char **argv)
\{      char buf[200];
        strcpy(buf, argv[1]);
        printf("%s\textbackslash{}n",buf);
\}
\end{alltt}


\slide{Overflow - segmentation fault }

\hlkimage{17cm}{buffer-overflow-2.pdf}


\begin{list1}
\item Bad function overwrites return value!
\item Control return address
\item Run shellcode from buffer, or from other place
\end{list1}


\slide{Exploits}

\vskip 1 cm

\begin{alltt}\small
$buffer = "";
$null = "\textbackslash{}x00"; \pause
$nop = "\textbackslash{}x90";
$nopsize = 1; \pause
$len = 201; // what is needed to overflow, maybe 201, maybe more!
$the_shell_pointer = 0xdeadbeef; // address where shellcode is
# Fill buffer
for ($i = 1; $i < $len;$i += $nopsize) \{
    $buffer .= $nop;
\}\pause
$address = pack('l', $the_shell_pointer);
$buffer .= $address;\pause
exec "$program", "$buffer";
\end{alltt}
\vskip 1 cm
\centerline{Demo exploit in Perl}

\slide{Hvordan finder man buffer overflow, og andre fejl}

\begin{list1}
\item Black box testing
\item Closed source reverse engineering
\item White box testing
\item Open source betyder man kan læse og analysere koden
\item Source code review - automatisk eller manuelt
\item Fejl kan findes ved at prøve sig frem - fuzzing
\item Exploits virker typisk mod specifikke versioner af software
\end{list1}

\slide{The Exploit Database - \link{http://www.exploit-db.com/}}

\hlkimage{15cm}{exploit-db.png}

\slide{Metasploit}

\hlkimage{15cm}{metasploit-about.png}

\begin{list1}
\item Idag findes der samlinger af exploits
\item Udviklingsværktøjerne til exploits er idag meget raffinerede!
\item \link{http://www.metasploit.com/}
\item \link{http://www.fastandeasyhacking.com/} Armitage GUI til Metasploit
\item \link{http://www.offensive-security.com/metasploit-unleashed/}
\end{list1}


\slide{Forudsætninger}

\begin{list1}
\item Bemærk: alle angreb har forudsætninger for at virke
\item Et angreb mod Telnet virker kun hvis du bruger Telnet
\item Et angreb mod Apache HTTPD virker ikke mod Microsoft IIS
\item Kan du bryde kæden af forudsætninger har du vundet!
\end{list1}

\slide{Eksempler på forudsætninger}


\begin{list1}
\item Computeren skal være tændt
\item Funktionen der misbruges skal være slået til
\item Executable stack
\item Executable heap
\item Fejl i programmet
\end{list1}
\pause
\vskip 2cm

\centerline{\color{titlecolor}\LARGE \bf alle programmer har fejl}


\slide{Kali Linux the pentest toolbox}

\hlkimage{14cm}{kali-linux.png}

\begin{list1}
\item  Kali \link{http://www.kali.org/}
\item 100.000s of videos on youtube alone, searching for kali and \$TOOL
\item Also versions for Raspberry Pi, mobile and other small computers
\end{list1}

\slide{Trinity breaking in}

\hlkimage{16cm}{trinity-nmapscreen-hd-cropscale-418x250.jpg}
\link{http://nmap.org/movies.html}\\
Meget realistisk \link{http://www.youtube.com/watch?v=51lGCTgqE_w}



\slide{Really do Nmap your world}

\hlkimage{8cm}{nmap-zenmap.png}

\begin{list2}
\item Nmap is a port scanner, but does more
\item Finding your own infrastructure available from the guest network?
\item See your printers having all the protocols enabled AND a wireless?
\end{list2}

\slide{Hackertools are for everyone!}

\hlkimage{2cm}{hackers_JOLIE+1995.jpg}

\begin{list2}
\item Hackers work all the time to break stuff, Use hackertools:
\item Nmap, Nping \link{http://nmap.org}
\item Wireshark - \link{http://www.wireshark.org/}
\item Aircrack-ng \link{http://www.aircrack-ng.org/}
\item Metasploit Framework \link{http://www.metasploit.com/}
\item Burpsuite \link{http://portswigger.net/burp/}
\item Kali Linux \link{http://www.kali.org}
\end{list2}

\vskip 5mm
\centerline{Most popular hacker tools \link{http://sectools.org/}}


\slide{Hackerlab setup}

\hlkimage{11cm}{hacklab-1.png}

\begin{list2}
\item Create hacker labs, encourage hacker labs!
\item Software Host OS: Windows, Mac, Linux
\item Virtualisation software: VMware, Virtual box, HyperV pick your poison
\item Hackersoftware: Kali Virtual Machine \link{https://www.kali.org/}
\end{list2}


\slide{Book: Linux Basics for Hackers (LBfH)}

\hlkimage{6cm}{LinuxBasicsforHackers_cover-front.png}

\emph{Linux Basics for Hackers
Getting Started with Networking, Scripting, and Security in Kali}
by OccupyTheWeb
December 2018, 248 pp.
ISBN-13:
9781593278557

\link{https://nostarch.com/linuxbasicsforhackers}

\slide{Book: Kali Linux Revealed (KLR)}

\hlkimage{6cm}{kali-linux-revealed.jpg}

\emph{Kali Linux Revealed  Mastering the Penetration Testing Distribution}

\link{https://www.kali.org/download-kali-linux-revealed-book/}\\
explains how to install Kali Linux

\slide{Fokus 2019: Firewalls og segmentering}

\hlkimage{10cm}{virksomhedens-netvaerk.pdf}

\begin{list2}
\item Hvis du har et netværk, så bør du have en firewall
\item En firewall tillader autoriseret trafik og blokerer resten
\item Hvornår har du sidst set din løsning efter?
\item Hvor lang tid tager det at se en 5.000 linier Cisco ASA config igennem?
\end{list2}

\slide{Imagine Attacks from the Inside}

\hlkimage{6cm}{erik-odiin-568459-unsplash.jpg}

\begin{list2}
\item Now imagine you were in control of a company laptop
\item Do you have a large internal world wide network?\\
NotPetya cost Maersk about 1.9 billion DKK
%\item Try scanning everything, start in a small corner, expand
%\item Scan all you danish segments, one by one, then the nordic, then the world
%\item Yes, things may break - FINE, BREAKING is GOOD

\item entry thought to be via software update of M.E.Doc [uk] an Ukrainian tax preparation program
\item Attackers are very creative and have a large attack surface to most companies
\end{list2}

\slide{Big firewalls}

\hlkimage{15cm}{network-layers-1.png}

\centerline{Big firewalls are not a single device}



\slide{Enhance and secure runtime environment}

%\hlkimage{10cm}{Bartizan.png}
\hlkimage{17cm}{medieval-clipart-5}
%\centerline{Picture from: http://karenswhimsy.com/public-domain-images}

\slide{Gode operativsystemer}

\begin{list1}
\item Nyere versioner af Microsoft Windows, Mac OS X og Linux distributionerne inkluderer:
\begin{list2}
\item Buffer overflow protection
\item Stack protection, non-executable stack
\item Heap protection, non-executable heap
\item \emph{Randomization of parameters} stack gap m.v.
\end{list2}
\item Vælg derfor hellere:
\begin{list2}
\item Windows 10, end Windows Xp
\item Mac OS X nyeste versioner
\item Linux sikkerhedsopdateringer, sig ja når de kommer
\end{list2}
\item Det samme gælder for serveroperativsystemer
\item NB: meget få embedded systemer har beskyttelse!
\end{list1}




\slide{Hackertyper anno 1995}

\hlkimage{5cm}{hackers_JOLIE+1995.jpg}

\centerline{Lad os lige gå tilbage til hackerne}

\slide{Hackertyper anno 2008}
\hlkimage{10cm}{lisbeth-salander.jpeg}

\begin{list1}
\item Lisbeth laver PU, personundersøgelser ved hjælp af hacking
\item Hvordan finder man information om andre
\end{list1}

\slide{Fra mønstre til person}

\begin{list1}
\item Først kan man finde nogle mønstre
\item Derefter søge med de mønstre
\item Nogle giver direkte information
\item Andre giver baggrundsinformation
\item Hvad er offentligt og hvad er privat? (googledorks!)
\begin{list2}
\item Navn, fulde navn, fornavne, efternavne, alias'es
\item Diverse idnumre, som CPR - tør du søge på dit CPR nr?
\item Computerrelaterede informationer: IP, Whois, Handles
\item Øgenavne, kendenavne
\item Skrivestil, ordbrug mv.
\item Tiden på din computer?
\item Tænk kreativt \smiley
\end{list2}
\end{list1}

\slide{Google for it}

\hlkimage{13cm}{images/googledorks-1.pdf}

\begin{list1}
\item Google som hacker værktøj?
\item Googledorks
\link{http://johnny.ihackstuff.com/}
\end{list1}


\myquestionspage


\slide{Reklamer: kursusafholdelse}

\begin{list1}
\item Følgende kurser afholdes med mig som underviser
\begin{list2}
\item Nmap workshop - 4 timer\\
Nmap for blue team, forsvaret.
En workshop med Nmap pakken af værktøjer som gør dig istand til at beskytte dit netværk bedre, fordi du effektivt kan undersøge det.
Vi gennemgår almindelig portscanning, som sættes i system, samt andre værktøjer som Nping til verifikation af forbindelser mellem enheder og igennem filtre og firewalls.
\vskip 1cm
\item Suricata, Bro og DNS opsamling - 4 timer\\
Workshop: Suricata, Zeek og DNS opsamling
Netværk idag er ofte blevet uoverskuelige, men kritiske for vores IT-brug. Denne workshop ser på applikationer til automatisk at afkode netværkstrafik med øvelser. Vi laver øvelser med Suricata, Zeek (tidligere Bro) og passiv DNS opsamling som eksempler på at kunne logge og gå tilbage i tiden. Niveauet er introduktion til værktøjerne og øvelserne bliver udført på Linux.
\end{list2}
\end{list1}

\end{document}
