\documentclass[Screen16to9,17pt]{foils}
%\documentclass[16pt,landscape,a4paper,footrule]{foils}
\usepackage{zencurity-slides}

% Pentest-cases
% For alle der er interesserede i almindelige problemer i produktionsnetværk og servere

% Denne aften er en gennemgang af diverse pentest sager, anonymiseret.

% Hvordan blev systemerne fundet, undersøgt og hacket. Vi vil se på almindelige fejl som observeret gennem nogle årtier. Dernæst vil vi gennemgå blue team pentest værktøjer som deltagerne selv kan benytte for at undgå lignende sager i egne organisationer.

% Nogle af sagerne vil kunne beskrives således:

% * Hacking af Tomcat server
% * Hvor er backuppen
% * Er core switchen sikker
% * Kan vi måske (KVM) ikke producere mere interne (IPMI) computer systemer som andre ikke kan styre
% * Vokseværk - hvorfor står Exchange serveren stadig på client LAN




\begin{document}

%\rm
\selectlanguage{english}

\mytitlepage
{Pentesting Cases}
%{Pentest introduction - greatest hits}


\slide{Goals for today}
\vskip 1 cm

\hlkimage{3cm}{dont-panic.png}
\centerline{\color{titlecolor}\LARGE Don't Panic!}


\begin{list1}
\item Introduce the term penetration testing and talk about pentest cases
\item Talk about things I have seen in real life pentesting
\item Discuss how we can avoid them in your future environments
\end{list1}

\slide{Materials -- where to start}

\begin{list2}
\item This presentation -- slides for today, start here
\item Nmap Workshop exercises\\{\footnotesize
\link{https://github.com/kramse/security-courses/blob/master/courses/pentest/nmap-workshop/nmap-workshop-exercises.pdf}}
\item KEA Pentest course exercises\\{\footnotesize
\link{https://github.com/kramse/security-courses/blob/master/courses/pentest/kea-pentest/kea-pentest-exercises.pdf}}
\item Setup instructions for creating a Kali virtual machine:\\
\link{https://github.com/kramse/kramse-labs}
\item Also the Simulated DDoS Workshop is available:\\{\footnotesize
\link{https://github.com/kramse/security-courses/tree/master/presentations/pentest/simulated-ddos-workshop}}
\end{list2}

%\centerline{We cannot go through all of them, but feel free to ask questions later}

{\bf Start a download of Kali today, if you want to play with the tools tomorrow}\\
Recommend virtual machine download 64-bit \url{https://www.kali.org/get-kali/#kali-virtual-machines}


\hlkprofiluk

\slide{Internet Today}

\hlkimage{10cm}{images/server-client.pdf}

\begin{list1}
\item Clients and servers, roots in the academic world
\item Protocols are old, some more than 20 years
\item Very little is encrypted, mostly HTTPS
\end{list1}


\slide{Hacker tools}

\begin{list1}
\item \emph{Improving the Security of Your Site by Breaking Into it}\\ by
Dan Farmer and Wietse Venema in 1993
\item Later in 1995 released the software SATAN\\
\emph{Security Administrator Tool for Analyzing Networks}
\item Caused some commotion, panic and discussions, every script kiddie can hack, the internet will melt down!
\vskip 5mm
\begin{quote}
We realize that SATAN is a two-edged sword -- like
many tools, it can be used for good and for evil
purposes. We also realize that intruders (including
wannabees) have much more capable (read intrusive)
tools than offered with SATAN.
\end{quote}
\end{list1}

\vskip 1cm
Source:
\link{http://www.fish2.com/security/admin-guide-to-cracking.html}




\slide{Hacker -- cracker}

{\bfseries Short answer -- dont discuss this}

%Det lidt længere svar:\\
Yes, originally there was another meaning to hacker, but the media has perverted it and today, and since early 1990s it has meant breaking into stuff for the public

{\color{red}\hlkbig Today a hacker breaks into systems!}

Reference. Spafford, Cheswick, Garfinkel, Stoll, \ldots
- wrote about this and it was lost

Story is interesting and the old meaning is ALSO used in smaller communities, like hacker spaces full of hackers - doing fun and interesting stuff
\begin{list2}
\item \emph{Cuckoo's Egg: Tracking a Spy Through the Maze of Computer
 Espionage},  Clifford Stoll
\item \emph{Hackers: Heroes of the Computer Revolution},
Steven Levy
\item \emph{Practical Unix and Internet Security},
Simson Garfinkel, Gene Spafford, Alan Schwartz
\end{list2}

\slide{Agreements for testing networks}

\begin{quote}\small
Danish Criminal Code\\
Straffelovens paragraf 263 Stk. 2. Med bøde eller fængsel indtil 1 år og 6 måneder straffes den, der uberettiget skaffer sig adgang til en andens oplysninger eller programmer, der er bestemt til at bruges i et informationssystem.
\end{quote}

Hacking can result in:
\begin{list2}
\item Getting your devices confiscated by the police
\item Paying damages to persons or businesses
\item If older getting a fine and a record -- even jail perhaps
\item Getting a criminal record, making it hard to travel to some countries and working in security
\item Fear of terror has increased the focus -- so dont step over bounds!
\end{list2}

Asking for permission and getting an OK before doing invasive tests, always!


\slide{Why even do security testing?}

\begin{list1}
\item Lots of security problems
\item Pentesting may be a requirement from external partners -- example VISA PCI standard
\end{list1}

\begin{list2}
\item Boss asking: should we do a security test?
\item CIO: hmm, okay
\item IT Admins: *sigh* -- I know the security sucks in places!
\item Its not your systems -- dont take the criticism personal, but as an opportunity to get things improved
\vskip 1cm
\item Pentest tools are great resources for doing discovery of assets, evaluating the security of large installations quickly -- in short using pentest tools makes you more efficient!
\end{list2}

\centerline{\Large Many see the benefits after doing a pentest, so try it!}


\slide{Benefits of having a planned security test done}

\begin{quote}
Goal of testing is to reduce risk for the systems and secure the organisation\\ from unexpected loss of data, image and increased costs.
\end{quote}

\begin{list1}
\item Intended audience:
\begin{list2}
\item IT-department and technical personnel
\item Management and board
\item External auditors, government, financial control VISA/PCI, the public
\end{list2}
\item Output from testing:
\begin{list2}
\item Reports with technical content and recommendations
\item Executive summary
\end{list2}
\end{list1}

Goal is not to find a scape goat to blame -- management allocates resources

If security is below in places more resources may be needed.


\slide{Shit Happened}


\begin{alltt}
Login: admin
Password: admin
\end{alltt}

\begin{list2}
\item Sometimes you buy something, power it, and forget about it
\item We all do
\item It is still a problem if it happens at work
\item Good thing, often easier to find with scanning tools
\end{list2}


\slide{Malicious Configuration and Negligence}

%\hlkimage{}{im-a-hacker.jpg}

\begin{quote}
Negligence (Lat. negligentia)[1] is a failure to exercise appropriate and/or ethical ruled care expected to be exercised amongst specified circumstances.[2] The area of tort law known as negligence involves harm caused by failing to act as a form of carelessness possibly with extenuating circumstances.
\end{quote}
Source: \url{https://en.wikipedia.org/wiki/Negligence}


\begin{list2}
\item When I find something which is NOT default, but had to be configured
\item It contains a default user like admin with password admin
\item Or some network configuration which is equally bad
\end{list2}

I consider this malicious, somebody \emph{on purpose} configured something {\bf badly}

\slide{On-site pentesting}
% Flere sager, men eksampelvis hos Forsikringsselskab med admin/admin

%\hlkimage{}{}

So doing pentest can be remote or on-site, at the customer site. I was at this insurance company, very professional, very nice, doing pentesting on the network.

I scanned the network and found the usual stuff, which often includes
\begin{list2}
\item Dirty server room -- they all rely on the devices in this room, great
\item No UPS -- power cables are a mess
\item Printers with default settings -- we can have fun reconfiguring them
\item Server administration -- more about that later
\end{list2}

The usual stuff ...


\slide{Nice Rack you got there!}

\hlkimage{12cm}{rackskab.jpg}

\begin{list2}
\item Really?! Who does this!
\end{list2}



\slide{Core Switch Administration}

Then I also found

\hlkimage{14cm}{edgemax-admin-admin.png}

\begin{list2}
\item Is this the main switch for the whole office?! Yes - unfortunately
\end{list2}

\slide{Upgrade your firmware (2020)}

%\hlkimage{}{}

This was seen in 2020:
\begin{alltt}
Cisco ASA Version 9.8(2) - Released: August 28, 2017
Cisco ASDM 7.8(2)Cisco
\end{alltt}

\begin{list2}
\item We can talk about firmware quality, but if you haven't \emph{upgraded} it in 3+ years ...
\item Firmware updates fix known problems and security issues -- install them
\item Create a process to review and update once in a while
\end{list2}

\slide{}

%\hlkimage{}{}

\begin{quote}

\end{quote}

\begin{list2}
    \item
\end{list2}




\slide{Network Faults }

%\hlkimage{}{}

\begin{quote}

\end{quote}

\begin{list2}
    \item
\end{list2}

\slide{SNMP problems}

\begin{quote}
5.5 Simple Network Management Protocol
The Simple Network Management Protocol (SNMP) [37] has recently been defined to aid in network
management. Clearly, access to such a resource must be heavily protected. The RFC states this, but
also allows for a null authentication service; this is a bad idea. Even a ‘‘read-only’’ mode is dangerous;
it may expose the target host to netstat-type attacks if the particular Management Information Base
(MIB) [38] used includes sequence numbers. (T
\end{quote}
Source: The paper \emph{Security Problems in the TCP/IP Protocol Suite} was originally published in Computer Communication Review, Vol. 19, No. 2, in April, 1989, Steven M. Bellovin\\
\url{https://www.cs.columbia.edu/~smb/papers/ipext.pdf}

An update was published in 2004
\emph{A Look Back at “Security Problems in the TCP/IP Protocol Suite”}, Steven M. Bellovin \\
\url{https://www.cs.columbia.edu/~smb/papers/acsac-ipext.pdf}

\slide{SNMP Public (2013)}
% Bank! SRX
%\hlkimage{}{}
So if this was common knowledge in the 1990s and onwards, why do we still see systems with SNMP \verb+public+

The situation:
\begin{list2}
\item I was put into this office at a bank -- scan the network, here are the prefixes
\item Found NOTHING, really slow, getting frustrated
\item Then I found SNMP available on the core router, a firewall type of devices SRX3400
\item Turned out they had configured
\end{list2}



Problems found with SNMP:
\begin{list2}
\item SNMP uptime often correlates to last firmware update, so 1800 days uptime -- no firmware installed for a long time
\item Network mapping - can show network infrastructure information
\end{list2}



\slide{local networks}

\begin{quote}
6.1 Vulnerability of the Local Network
Some local-area networks, notably the Ethernet networks, are extremely vulnerable to eavesdropping and
host-spoofing. If such networks are used, physical access must be strictly controlled. It is also unwise
to trust any hosts on such networks if any machine on the network is accessible to untrusted personnel,
unless authentication servers are used.
If the local network uses the Address Resolution Protocol (ARP) [42] more subtle forms of host-spoofing
are possible. In particular, it becomes trivial to intercept, modify, and forward packets, rather than just
taking over the host’s role or simply spying on all traffic.
\end{quote}

Today we can send VXLAN spoofed packets across the internet layer 3 and inject ARP behind firewalls, in some cloud infrastructure cases ...

\slide{Why talk about VXLAN RFC7348 2014}

\quote{\small
Virtual Extensible LAN (VXLAN) is a network virtualization technology ... uses a VLAN-like encapsulation technique to {\bf encapsulate
OSI layer 2 Ethernet frames} within {\bf layer 4 UDP datagrams}, ... VXLAN endpoints, which terminate VXLAN tunnels and may be either v
irtual or physical switch ports, are known as {\bf VXLAN tunnel endpoints (VTEPs)}.[2][3]

\vskip 5mm

The VXLAN specification was originally created by {\bf VMware, Arista Networks and Cisco}.[5][6] Other backers of the VXLAN technology
include {\bf Huawei,[7] Broadcom, Citrix, Pica8, Cumulus Networks, Dell EMC, Mellanox,[8] FreeBSD,[9] OpenBSD,[10] Red Hat,[11] Joyent,
 and Juniper Networks.}
}

\vskip 10mm
\centerline{\Large Already in production use}

\vskip 1cm
Security Considerations\\
   TBD.

   Source for quote:\\
   \url{https://en.wikipedia.org/wiki/Virtual_Extensible_LAN}

\slide{Overview VXLAN RFC7348 2014}

\hlkimage{21cm}{vxlan-basic.png}

How does it work?

\begin{list2}
\item Router 1 takes Layer 2 traffic, encapsulates with IP+UDP port 4789, routes
\item Router 2 receives IP+UDP+data, decapsulates, forward/switches layer 2 onto VLAN
\item Hosts 10.0.0.10 can talk to 10.0.0.20 as if they where next to each other in switch
\item Most often VLAN IEEE 802.1q involved too, but not shown
%\item Lets only consider two routers
\end{list2}

\slide{VXLAN injection}

\hlkimage{19cm}{vxlan-basic-injection.png}

I tested using my pentest server in one AS, sending across an internet exchange into a production network, towards Arista testing devices - no problems, it's just routed layer 3 IP+UDP packets

\slide{Example: Send UDP DNS reqs to inside server}

\hlkimage{22cm}{vxlan-basic-injection-dns.pdf}

%One interesting attack is injecting UDP packets to allow DNS\\
%requests to inside server which might not even have public IP

%\begin{enumerate}
%\item Select target: internal server, 10.0.0.10 and DNS service 53/UDP
%\item Create VXLAN packet(s): DNS request dst 10.0.0.10 UDP dport 53
%\item Source for this probe is your external pentest server
%\item Make sure inside packet has Ethernet destination that reaches server
%\item Send spoofed VXLAN packet across internet
%\item After VXLAN decap this packet is sent to the server
%\item Server process DNS request, send back response
%\item Attacker waiting for the UDP DNS reply, gets it
%\end{enumerate}

{\small Attacker can send UDP DNS request to inside server on RFC1918 destination\\
Note: server has no external IP or incoming ports forwarded.\\
Tested working with Clavister with DNS UDP probes/requests, no inspection }


\slide{VXLAN also used a lot in Cisco ACI}

%\hlkimage{}{}

\begin{quote}\footnotesize
Cisco Application Centric Infrastructure (ACI), {\bf the industry’s most secure, open, and comprehensive software-defined networking (SDN) solution}, enables automation that accelerates infrastructure deployment and governance, simplifies management to easily move workloads across a multifabric and multicloud frameworks, and proactively secures against risk arising from anywhere. It radically simplifies, optimizes, and expedites the application deployment lifecycle.
\end{quote}
Source:\\ {\scriptsize\url{https://www.cisco.com/c/en/us/solutions/collateral/data-center-virtualization/application-centric-infrastructure/solution-overview-c22-741487.html}}

\begin{list2}
    \item
\end{list2}

\slide{Cisco ACI (2019)}

%\hlkimage{}{}

\begin{quote}
Vulnerability Analysis
\begin{list2}
\item Remote Code Execution on Leaf Switches over IPv6 via Local SSH Server (CVE-2019-1836, CVE2019-1803, and CVE-2019-1804) -- SSH access with specific source port, private key left on firmware image, and on all switches
\item Cisco Nexus 9000 Series Fabric Switches ACI Mode Fabric Infrastructure VLAN Unauthorized Access
Vulnerability (CVE-2019-1890)
\item Cisco Nexus 9000 Series Fabric Switches Application Centric Infrastructure Mode Link Layer Discovery
Protocol Buffer Overflow Vulnerability (CVE-2019-1901)
\item Cisco Application Policy Infrastructure Controller REST API Privilege Escalation Vulnerability (CVE2019-1889)
\end{list2}
\end{quote}
Source: \url{https://static.ernw.de/whitepaper/ERNW_Whitepaper68_Vulnerability_Assessment_Cisco_ACI_signed.pdf}

Further all processes run as root user -- good job Cisco

\slide{Secure Shell access to core devices }

%\hlkimage{}{}
This reminds me of a case where I was doing audit of a network with Juniper devices, core routers.

They all had router protection filters only certain IP ranges could access \emph{management} -- highly recommended.

Like this one for my router:
\begin{alltt}
ip access-list ssh-acl
   10 permit tcp 10.123.44.0/24 any
!
\end{alltt}

\begin{list2}
    \item
\end{list2}


% * Kan vi måske (KVM) ikke producere mere interne (IPMI) computer systemer som andre ikke kan styre
\slide{}


\begin{list2}
\item ILO Compaq/HP
\item DRAC Dell Remote
\item IPMI -- generic
\end{list2}

Common problems found:
\begin{list2}
\item Default settings, default passwords -- often direct access to server administration
\item Not upgraded, firmware has known vulnerabilities
\end{list2}

I have seen this for many many years, and still see this in networks after 2020

\slide{}

%\hlkimage{}{}

\begin{quote}
{\LARGE Insert example Metasploit run from custmer report }

\end{quote}

\begin{list2}
    \item
\end{list2}



% * Hacking af Tomcat server
\slide{}

%\hlkimage{}{}

\begin{quote}

\end{quote}

\begin{list2}
    \item
\end{list2}


\slide{Where my backup dude}
% To Solaris med NFS world export
%\hlkimage{}{}

\begin{quote}

\end{quote}

\begin{list2}
    \item
\end{list2}

% * Vokseværk - hvorfor står Exchange serveren stadig på client LAN
\slide{In 2022 Don't keep your Exchange server on the LAN!}

%\hlkimage{}{}

\begin{quote}

\end{quote}

\begin{list2}
    \item
\end{list2}


\slide{My Own Faults}
% Commit full-synchronize
% Shutdown AIX server with 100 developers

%\hlkimage{}{}

\begin{quote}

\end{quote}

\begin{list2}
    \item
\end{list2}


\slide{What you don't know can hurt you}

keaviden.dk DMARC sagen

%\hlkimage{}{}

\begin{quote}

\end{quote}

\begin{list2}
    \item
\end{list2}


\slide{What you don't know can hurt you -- part II}

Problem: You send personal data -- GDPR\\
You want to have it ALL encrypted, but SMTP does NOT require encryption -- or does it?!
%\hlkimage{}{}

\begin{quote}\footnotesize
The SMTP protocol isn’t secure and wasn’t designed to be. Email sent in the early days of the Internet were the digital equivalent of sending a postcard through the postal system. Eventually, Transport Layer Security (TLS) encryption was added to protect SMTP communications. But to maintain backward compatibility, it was never made compulsory and even today it’s used only opportunistically by senders.

...

The SMTP MTA Strict Transport Security (MTA-STS) standard was developed to ensure that TLS is always used, and to provide a way to for sending servers to refuse to deliver messages to servers that don’t support TLS and have a trusted certificate. The MTA-STS standard was developed by several email industry companies brought together by the Messaging, Malware and Mobile Anti-Abuse Working Group (M3AAWG). We have been validating our implementation and are now pleased to announce support for MTA-STS for all outgoing messages from Exchange Online.
\end{quote}

Sources:
\begin{list2}
\item \url{https://en.wikipedia.org/wiki/Opportunistic_encryption}
\item \url{https://techcommunity.microsoft.com/t5/exchange-team-blog/introducing-mta-sts-for-exchange-online/ba-p/3106386}
\end{list2}

\slide{Keep Up to Date with technologies you use}

%\hlkimage{}{}

insert picture of Homer saying duh

\begin{quote}

\end{quote}

\begin{list2}
\item Make an effort
\item Be a professional
\end{list2}


\slide{Being a Professional}

%\hlkimage{}{}
Many definitions:
\begin{quote}
A {\bf professional} is a member of a profession or any person who works in a specified professional activity. The term also describes the {\bf standards of education and training} that prepare members of the profession with the {\bf particular knowledge and skills} necessary to perform their {\bf specific role} within that profession. In addition, most professionals are subject to {\bf strict codes of conduct}, enshrining {\bf rigorous ethical and moral obligations}.[1] Professional standards of practice and {\bf ethics} for a particular field are typically agreed upon and maintained through widely recognized professional associations, such as the IEEE.[2]
\end{quote}
Source: \url{https://en.wikipedia.org/wiki/Professional}

\begin{list2}
\item The field of IT has a lot of amateurs
\item Sorry if this sound elitist, but we should take responsebility not only for ourselves but for our communities
\end{list2}


\myquestionspage


\slide{Advice for enterprise networks}


\begin{list2}
\item Portscanning - start using portscans in your networks, verify how far malware and hackers can travel, and identify soft systems needing updates or isolation
\item Have separation -- anywhere, starting with organisation units, management networks, server networks, customers, guests, LAN, WAN, Mail, web, ...
\item Use Web proxies - do not allow HTTP directly except for a short allow list, \\
do not allow traffic to and from any new TLD
\item Use only your own DNS servers, create a pair of Unbound servers, \\
point your internal DNS running on Windows to these\\
Create filtering, logging, restrictions on these Unbound DNS servers\\
\link{https://www.nlnetlabs.nl/projects/unbound/about/} and also \link{https://pi-hole.net/}
\item Only allow SMTP via your own mail servers, create a simple forwarder if you must
\end{list2}

Allow lists are better than block list, even if it takes some time to do it

\slide{Books and educational materials}

\begin{list2}
\item \emph{The Linux Command Line: A Complete Introduction}, 2nd Edition\\
 by William Shotts, internet edition \link{https://sourceforge.net/projects/linuxcommand}
\item \emph{Linux Basics for Hackers Getting Started with Networking, Scripting, and Security in Kali}. OccupyTheWeb, December 2018, 248 pp. ISBN-13: 978-1-59327-855-7
\item \emph{Gray Hat Hacking: The Ethical Hacker's Handbook}, 5. ed. Allen Harper and others ISBN: 978-1-260-10841-5
\item \emph{Web Application Security}, Andrew Hoffman, 2020, ISBN: 9781492053118
\item \emph{Practical Packet Analysis, Using Wireshark to Solve Real-World Network Problems}
by Chris Sanders, 3rd ed, ISBN: 978-1-59327-802-1
\item \emph{Hacking, 2nd Edition: The Art of Exploitation}, Jon Erickson, February 2008, ISBN-13: 9781593271442
\item \emph{Kali Linux Revealed Mastering the Penetration Testing Distribution}\\
\link{https://www.kali.org/}
\end{list2}


We teach using these books and others! Diploma in IT-security at KEA Kompetence\\
 \link{https://zencurity.gitbook.io/}


\slide{Capture data and logs!}


\begin{list2}
\item Run DNS query logs -- when client1 is infected with malware from domain malwareexample.com, then search for more clients i
nfected
\item Run Zeek and gather information about all HTTPS sessions -- captures certificates by default, and we can again search for
certificate related to malwareexample.com
\item Run network logging -- session logs in enterprise networks are GREAT \\
(country wide illegal logging is of course NOT)
\end{list2}

Make sure to check with employees, inform them!

\slide{DROP SOME TRAFFIC NOW}

\begin{list2}
\item Drop some traffic on the border of everything
\item Seriously do NOT allow Windows RPC across borders
\item Border here may be from regional country office back to HQ
\item Border may be from internet to internal networks
\item Block Windows RPC ports, 135, 137, 139, 445
\item Block DNS directly to internet, do not allow clients to use any DNS, fake 8.8.8.8 if you must internally
\item Block SMTP directly to internet
\item Create allow list for internal networks, client networks should not contact other client networks but only relevant server networks
\end{list2}

You DONT need to allow direct DNS towards internet, except from your own recursive DNS servers

If you get hacked by Windows RPC in 2022, you probably deserve it, sorry for being blunt

Best would be to analyze traffic and create allow lists, some internal networks to not need internet at all


\slide{Default permit}

%\hlkimage{}{}

One of the early implementers of firewalls Marcus J. Ranum summarized in 2005 The Six Dumbest Ideas in Computer Security \link{https://www.ranum.com/security/computer_security/editorials/dumb/} which includes the always appropriate discussion about default permit versus default deny.

\begin{quote}\small {\bf
\#1) Default Permit}\\
This dumb idea crops up in a lot of different forms; it’s incredibly persistent and difficult to eradicate. Why? Because it’s so attractive. Systems based on ”Default Permit” are the computer security equivalent of empty calories: tasty, yet fattening.

The most recognizable form in which the ”Default Permit” dumb idea manifests itself is in firewall rules. Back in the very early days of computer security, network managers would set up an internet connection and decide to secure it by turning off incoming telnet, incoming rlogin, and incoming FTP. Everything else was allowed through, hence the name ”Default Permit.” This put the security practitioner in an endless arms-race with the hackers.
\end{quote}


\begin{list2}
\item Allow all current networks today on all ports for all protocols \emph{is} an allow list \\
Which tomorrow can be split into one for TCP, UDP and remaining, and measured upon
\item Measure, improve, repeat
\end{list2}



\slide{We cannot do X}

\begin{quote}
We cannot block SMTP from internal networks, since we do not know for sure if vendor X equipment needs to send the MOST important email alert at some unspecific time in the future
\end{quote}

Cool, then we can do an allow list starting today on our border firewall:
\begin{alltt}
table <smtp-exchange> \{ $exchange1 $exchange2 $exchange3 \}
table <smtp-unknown> persist file "/firewall/mail/smtp-internal-unknown.txt"
# Regular use, allowed
pass out on egress inet proto tcp from smtp-echange to any port 25/tcp
# Unknown, remove when phased out
pass out on egress inet proto tcp from smtp-internal to any port 25/tcp
\end{alltt}

Year 0 the unknown list may be 100\% of all internal networks, but new networks added to infrastructure are NOT added, so list will shrink -- evaluate the list, and compare to network logs, did networks send ANY SMTP for 1,2,3 years?

\slide{Zeek is a framework and platform}

\hlkimage{12cm}{zeek-ids.png}

\begin{quote}
While focusing on network security monitoring, Zeek provides a comprehensive platform for more general network traffic analysis as well. Well grounded in more than 15 years of research, Zeek has successfully bridged the traditional gap between academia and operations since its inception.
\end{quote}

\link{https://www.Zeek.org/}
Does useful things out of the box using more than 10.000 script lines

\slide{Suricata IDS/IPS/NSM}
\hlkimage{6cm}{suricata.png}

\begin{quote}
Suricata is a high performance Network IDS, IPS and Network Security Monitoring engine.
\end{quote}

 \link{http://suricata-ids.org/}
 \link{http://openinfosecfoundation.org}

Suricata, Zeek og DNS Capture -- it a nice world, use it!\\
{\small\link{https://github.com/kramse/security-courses/tree/master/courses/networking/suricatazeek-workshop}}



\slide{Firewall -- Another definition}

% Remove?
I am also fond of this longer and technical definition from RFC4949:
\begin{quote}
\$ firewall

      1. (I) {\bf An internetwork gateway that restricts data communication
      traffic to and from one of the connected networks} (the one said to
      be "inside" the firewall) and thus protects that network's system
      resources against threats from the other network (the one that is
      said to be "outside" the firewall). (See: guard, security
      gateway.)

      2. (O) {\bf A device or system that controls the flow of traffic
      between networks using differing security postures.} Wack, J. et al (NIST), "Guidelines on Firewalls and Firewall Policy", Special Publication 800-41,
      January 2002.

      Tutorial: A firewall typically protects a smaller, secure network
      (such as a corporate LAN, or even just one host) from a larger
      network (such as the Internet). The firewall is installed at the
      point where the networks connect, and the firewall applies policy
      rules to control traffic that flows in and out of the protected
      network.
\end{quote}

\slide{Firewall -- Another definition}
% Remove?
\begin{quote}
\$ firewall, continued

      {\bf A firewall is not always a single computer.} For example, a
      firewall may consist of a pair of filtering routers and one or
      more proxy servers running on one or more bastion hosts, all
      connected to a small, dedicated LAN (see: buffer zone) between the
      two routers.

      The external router blocks attacks that use IP to
      break security (IP address spoofing, source routing, packet
      fragments), while proxy servers block attacks that would exploit a
      vulnerability in a higher-layer protocol or service. The internal
      router blocks traffic from leaving the protected network except
      through the proxy servers.

      The difficult part is defining criteria by which packets are denied passage through the firewall, because
      a firewall not only needs to keep unauthorized traffic (i.e., intruders) out, but usually also needs to let authorized traffic
      pass both in and out.
\end{quote}


\slide{Routing Security}


\begin{list2}
\item Use MD5 passwords or better authentication for routing protocols {\myalert}
\item TTL Security -- avoid routed packets
\item Max prefix -- of course, only allow expected networks
\item Prefix filtering -- only parts of IPv6 space is used
\item TCP Authentication Option [RFC 5925] replaces TCP MD5 [RFC 2385]
\item Turn ON RPKI for both IPv4 and IPv6 prefixes, {\myalert} \\
\link{https://nlnetlabs.nl/projects/rpki/about/}
\item Drop bogons on IPv4 and IPv6, article with multiple references YMMV\\
\link{https://theinternetprotocolblog.wordpress.com/2020/01/15/some-notes-on-ipv6-bogon-filtering/}
\end{list2}


\slide{Mutually Agreed Norms for Routing Security (MANRS)}

%\hlkimage{2cm}{MANRS_square.png}

\begin{quote}
  Mutually Agreed Norms for Routing Security (MANRS) is a global initiative, supported by the Internet Society, that provides crucial fixes to reduce the most common routing threats. 
\end{quote}
Source: {\small\link{https://www.manrs.org/wp-content/uploads/2018/09/MANRS_PDF_Sep2016.pdf}}

\begin{list2}
\item Problems related to incorrect routing information
\item Problems related to traffic with spoofed source IP addresses
\item Problems related to coordination and collaboration between network operators
\item Also BCP38 RFC2827 \emph{Network Ingress Filtering: Defeating Denial of Service Attacks
which employ IP Source Address Spoofing}
\end{list2}

You should all ask your internet providers if they know about MANRS, and follow it. We should ask our government and institutions to support and follow MANRS and good practices for network on the Internet



\end{document}
